\documentclass[12pt,preprint]{aastex}
\usepackage{natbib,amsmath}

\special{papersize=8.5in,11in}
\begin{document}

\def\N#1{{N({\rm #1})}}
\newcommand{\tmax}{$\tau_v^{\rm IGM, max}$}
\newcommand{\mtmax}{\tau_v^{\rm IGM, max}}
\newcommand{\mtauv}{\tau_v}
\newcommand{\tauv}{$\tau_v$}
\newcommand{\tigm}{$\tau_v^{\rm IGM}$}
\newcommand{\mtigm}{\tau_v^{\rm IGM}}
\newcommand{\mmc}{m_{\rm c}}
\def\mrvir{r_{\rm vir}}
\def\rvir{$r_{\rm vir}$}
\def \msun {M_\odot}
\def\wlya{$W^{\rm Ly\alpha}$}
\def\mwlya{W^{\rm Ly\alpha}}
\newcommand{\tpk}{$\tau_{\rm pk}$}
\newcommand{\mtpk}{\tau_{\rm pk}}
\newcommand{\mdv}{\delta v}
\newcommand{\ewabs}{$W_{\rm a}$}
\newcommand{\ewe}{$W_{\rm e}$}
\newcommand{\rtt}{$r_{\tau = 0.2}$}
\newcommand{\mrtt}{r_{\tau = 0.2}}
\newcommand{\bturb}{$b_{\rm turb}$}
\newcommand{\mbturb}{b_{\rm turb}}
\newcommand{\taud}{$\tau_{\rm dust}$}
\newcommand{\mtaud}{\tau_{\rm dust}}
\newcommand{\maconfig}{2p$^6$3s}
\newcommand{\mbconfig}{2p$^6$3p}
\newcommand{\aconfig}{a~$^6$D$^0$}
\newcommand{\zconfig}{z~$^6$D$^0$}
\newcommand{\mvr}{v_{\rm r}}
\newcommand{\naid}{\ion{Na}{1}~$\lambda\lambda 5891, 5897$}
\newcommand{\mgii}{\ion{Mg}{2}}
\newcommand{\mgiid}{\ion{Mg}{2}~$\lambda\lambda 2796, 2803$}
\newcommand{\mgiia}{\ion{Mg}{2}~$\lambda 2796$}
\newcommand{\mgiib}{\ion{Mg}{2}~$\lambda 2803$}
\newcommand{\feiia}{\ion{Fe}{2}~$\lambda 2586$}
\newcommand{\feiib}{\ion{Fe}{2}~$\lambda 2600$}
\newcommand{\feiic}{\ion{Fe}{2}$* \; \lambda 2612$}
\newcommand{\feiie}{\ion{Fe}{2}$* \; \lambda 2626$}
\newcommand{\feiid}{\ion{Fe}{2}~$\lambda\lambda 2586, 2600$}
\newcommand{\feiis}{\ion{Fe}{2}$*$}
\newcommand{\nmg}{$n_{\rm Mg^+}$}
\newcommand{\mnmg}{n_{\rm Mg^+}}
\newcommand{\nfe}{$n_{\rm Fe^+}$}
\newcommand{\mnfe}{n_{\rm Fe^+}}
\def\hub{h_{72}^{-1}}
\def\umfp{{\hub \, \rm Mpc}}
\def\mzq{z_q}
\def\zabs{$z_{\rm abs}$}
\def\mzabs{z_{\rm abs}}
\def\intl{\int\limits}
\def\cmma{\;\;\; ,}
\def\perd{\;\;\; .}
\def\ltk{\left [ \,}
\def\ltp{\left ( \,}
\def\ltb{\left \{ \,}
\def\rtk{\, \right  ] }
\def\rtp{\, \right  ) }
\def\rtb{\, \right \} }
\def\sci#1{{\; \times \; 10^{#1}}}
\def \rAA {\rm \AA}
\def \zem {$z_{\rm em}$}
\def \mzem {z_{\rm em}}
\def\smm{\sum\limits}
\def \cmm  {cm$^{-2}$}
\def \cmmm {cm$^{-3}$}
\def \kms  {km~s$^{-1}$}
\def \mkms  {\, {\rm km~s^{-1}}}
\def \lyaf {Ly$\alpha$ forest}
\def \Lya  {Ly$\alpha$}
\def \lya  {Ly$\alpha$}
\def \mlya  {Ly\alpha}
\def \Lyb  {Ly$\beta$}
\def \lyb  {Ly$\beta$}
\def \lyg  {Ly$\gamma$}
\def \ly5  {Ly-5}
\def \ly6  {Ly-6}
\def \ly7  {Ly-7}
\def \nhi  {$N_{\rm HI}$}
\def \mnhi  {N_{\rm HI}}
\def \lnhi {$\log N_{HI}$}
\def \mlnhi {\log N_{HI}}
\def \etal {\textit{et al.}}
\def \lyaf {Lyman--$\alpha$ forest}
\def \mnmin {\mnhi^{\rm min}}
\def \nmin {$\mnhi^{\rm min}$}
\def \O {${\mathcal O}(N,X)$}
\newcommand{\cm}[1]{\, {\rm cm^{#1}}}
\def \snrlim {SNR$_{lim}$}

\title{The Emperor Has No Clothes}

\author{
J. Xavier Prochaska\altaffilmark{1}, 
Others
%John M. O'Meara\altaffilmark{2}, 
%Gabor Worseck\altaffilmark{1} 
%\& Scott Burles\altaffilmark{3}
}
\altaffiltext{1}{Department of Astronomy and Astrophysics, UCO/Lick Observatory, University of California, 1156 High Street, Santa Cruz, CA 95064}
%\altaffiltext{2}{Department of Chemistry and Physics, Saint Michael's College.
%One Winooski Park, Colchester, VT 05439}

\begin{abstract}
We consider the observed \lya\ emission from Lyman Break Galaxies
(LBGs) and their surrounding environment in the context of the LBG
clump model proposed by \cite{steidel10}.  This model does not
reproduce the observations: the systematic redshift of \lya\ is not a
natural prediction of a clumpy wind and the clump model has too low
covering fraction and velocity graident to yield significant scattered
emission at large impact parameter.  We also consider the mass, mass
flux, energy, and power of this wind model and find that one requires
extreme quantities.  [Mention Lya emission here?]
[Mention most of the mass in the wind accounts for very little EW in
the down-the-barell observations]
We argue that this model is an inaccurate description of the CGM
surrounding high-$z$ star-forming galaxies, and that the observed
outflow is likely confined to $r \lesssim 10$\,kpc.
\end{abstract}

\keywords{absorption lines -- intergalactic medium -- Lyman limit systems -- SDSS}

\section{Introduction}

Spectroscopic observations of star-forming galaxies reveal that 
systems at all redshifts experience outflows of cool, diffuse gas
\citep[e.g.][]{rupke,weiner,shapley03,rubin}.  The indisputable
signature is the preponderence of significant low-ion absorption
(e.g.\ \ion{Na}{1}, \ion{Mg}{2}, \ion{Si}{2}) that is
blueshifted with respect to the galaxy's stars and nebular regions.
These data characterize the kinematics of the outflow and place
constraints on the optical depth and/or covering fraction of the
gas.  These ``down-the-barrel'' observations, however, provide
very weak constraints on the distance of the flow from the galaxy.
Therefore, the mass, energy, and power of the wind and its subsequent
effects on the circumgalactic medium (CGM) of the galaxy and/or the
intergalctic medium (IGM) beyond it remain
open and compelling questions.

At $z\sim 3$, the most extensively observed population of star-forming
galaxies is the Lyman Break Galaxies (LBGs), named for their
characteristic dip in flux due to \lya\ absorption by the IGM.
For the past $\sim 15$ years, the group led by C. Steidel has
amassed the largest spectroscopic and imaging datasets on these
galaxies \citep{steidel96,shapley03,reddyXX}.  Their early works
demonstrated nearly ubiquitous absorption from low and high-ion
transitions that is systematically blue-shifted relative to the
galaxy's stars and nebular emission.  These same spectra also reveal
a diversity of \lya\ emission lines whose properties correlate with galaxy photometric
and spectroscopic properties \citep{shapley08,cooke08}.  
A nearly generic characteristic of this emission is that
the line is 
shifted redward of systemic.  This has been interpted by numerous
authors as yet another signature a galactic-scale outflow
\citep[e.g.][]{pettini,verhamme}. 

Recently, the Steidel
collaboration has published two papers that offer new insight into the
CGM surrounding LBGs.   In one experiment \citep[][hereafter
S+10]{steidel+10}, the authors have stacked low S/N, moderate
resolution ($R \sim 1000$) spectra of background galaxies at the
rest-wavelengths of foreground LBGs having impact parameters $\rho \sim
10-100$\,kpc.  This analysis revealed that LBGs are surrounded by a
CGM to nearly $\rho \approx 100$\,kpc 
that gives rise to \ion{H}{1}, low-ion (e.g.\ \ion{C}{2}~1334,
\ion{Si}{2}), and high-ion (e.g.\ \ion{C}{4}) absorption. 
S+10 modeled the blue-shifted absorption
observed down-the-barrel with this extended CGM absorption 
as a clumpy, galactic-scale wind with speeds of
$v \sim 800 \, \mkms$ and radial extent $r \sim 100$\,kpc.  Although
this wind model is not a unique interpretation of the extend
\ion{H}{1} absorption \citep{fpk+11}, it apparently satisifies the key
absorption observables.

Complementing the absorption-line observations and analysis,
Steidel et al.\ (2011; hereafter S+11) have explored the spatial
extent of \lya\ emission from the CGM surrounding LBGs.  In a stacked,
narrow-band image of $\approx 100$ LBGs at $z \approx 2.5$ they
measure a `halo' of low surface brightness, \lya\ emission with an
approximately exponential profile $\mu(\rho) \sim \exp[-\rho/b]$ from
$\rho \approx 10-80$\,kpc with $b \approx 20$\,kpc.  These authors
interpreted the signal as the scattering of \lya\ photons generated
within the central LBG by an extended CGM surrounding each galaxy.
They further reported that the same LBG clumpy wind model that
reproduces the stacked absorption spectra (S+10) can reproduce the
kinematics of \lya\ emitted by the galaxy and also the extended \lya\
emission.  At the same time, they allowed that a proper treatment
of \lya\ emission that includes a full radiative transfer calculation was warranted.  

In a recent publication \citep[][hereafter PKR11]{pkr11}, 
we explored the absorption and emission from
a series of simple wind models with emphasis on the
\mgiid\ doublet and \ion{Fe}{2}~UV1 multiplet.  In addition to a
series of idealized models, we also considered two variations of the
LBG wind model proposed by S+10.  Although we argued that the results
for \mgii\ should act as an analog for \lya\ (because each is a
resonantly trapped doublet), we return in this paper to consider
\lya\ directly ($\S$~\ref{sec:lya}).  We also expand upon our discussion on the required
energetics and mass flow for such a wind scenario
($\S$~\ref{sec:energy}).  Unless otherwise indicated, all distances
are physical under the assumption of a $\Lambda$CDM universe with
$\Lambda = 0.74$, $\Omega_m = 0.26$ and 
$H_0 = 72 h_{72} \, \mkms/{\rm Mpc}$ \citep{wmap05}.

\section{\lya\ Emission/Absorption from the LBG Clump Model}
\label{sec:lya}

The LBG clump model proposed by S+10 envisions an isotropic outflow of
dense clumps, each optically thick to \lya\ and metal-line
transitions, but which as an ensemble have a partial covering fraction to
the LBG and background sources.  We discuss in PKR11 our radiative
transfer treatment for this wind in the context of \mgiid.  In the
following, we adopt the same algorithms but focus on
\lya\footnote{Following standard practice, we treat \lya\ as a single
  transition instead of a double because the line separation is much less
  than the astrophysical velocities of interest here.}
and modify the parameters of the LBG clump model as appropriate.

In contrast to PKR11, we include two additional features in the
model.  First, we have explored two intrinsic emission models for the
galaxy near the \lya\ line: 
(i) a flat continuum, assumed to be generated by O and B stars within the
star-forming galaxy; and
(ii) a continuum plus \lya\ emission-line model parameterized by a
line-width $\sigma_{\rm Ly\alpha}$ and a \lya\ equivalent width \wlya.
The \lya\ emission is presumed to be generated by \ion{H}{2} regions within the
galaxy.  We emphasize that neither of these simple models are likely
to describe the complex emission of a star-forming galaxy at \lya.
Nevertheless, they help capture some of the salient features of the
LBG clump wind model.

Second, we have modeled the attenuation of photons near \lya\ by the
IGM surrounding LBGs on scales of $\sim 100\,
{\rm kpc} - 10$\,Mpc.  Our approach follows the method of
\cite{santos04}, using the \cite{barkana04} model for infall onto
galactic halos at $z \approx 3$.  We assume a galaxy with dark matter
halo mass $M_{\rm DM} \approx 10^{12} \msun$, virial radius $\mrvir 
= 80$\,kpc, circular velocity $v_{\rm c} = 220 \mkms$, and an
IGM with $T = 10^4$\,K.  THis amterial is bathed in a uniform
extragalactic UV
background (EUVB) radiation field with ionization rate $\Gamma =
10^{-12.5} \, \rm s^{-1}$ \citep{fpl+08}.  With this parameterization,
we can calculate the density and velocity profiles of gas at $r >
\mrvir$ for the LBG (and assume the gas at $r\le \mrvir$ is dominated by
the outflow) and apply the Sobolev approximation to calculate the
optical depth \tigm\ to \lya\ as a function of velocity.
We caution that
this $\exp(-\mtigm)$ prescription does not capture the complexity of
gas on these scales.  Indeed, \cite{zheng10a} have argued that a more
proper treatment tends to yield even higher average opacity.
[INCLUDE THE BOOST? Yes, and justify to match $D_A$]
[CHECK against Djisktra]
Therefore, one may wish to consider our estimation of \tigm\ as a
lower limit.

Figure~\ref{fig:1D} (top panels) presents the 
calculated 1D \lya\ spectrum for the
two intrinsic emission models of the galaxy.
For the line-emission, we have adopted $\sigma_{\rm Ly\alpha} =
50\mkms$ and $\mwlya = 100$\AA, the latter of which is suggested for
actively star-forming galaxies \citep{lya_emit}.
The second panels show the predicted results when one includes
radiative transfer through the LBG clumpy wind model.  Similar to the \mgii\
results of PKR11, we observe significant absorption by the outflow for
$v < 0\mkms$.  
Although the opacity of the outflow peaks at $v \approx 0 \mkms$, the
minimum flux is actually observed to be closer to $v \approx -300
\mkms$ because
scattered emission `fills-in' the absorption at $v \lesssim 0
\mkms$.  Most of these scattered photons were
emitted transverse to our sightline and then scattered once to
Earth.  The tail of emission to positive velocities ($v
\approx +500\mkms$) in contrast, are associated with photons that
scattered off the far-side of this isotropic outflow.  
We emphasize that this emission is weak in comparison with the
flux at $v \approx 0 \mkms$, simply because of geometrical projection.

[Emphasize that in this clump model, photons don't diffuse in
frequency much] 


The third pair of panels includes the effects of IGM scattering.  For
the DM halo and IGM model assumed, we find a peak optical depth
$\mtmax \approx 7$ at $v \approx v_{\rm c}$ that rapidly declines to
$\mtauv < 0.5$ at $v=0\mkms$.  This implies complete attenuation of
the \lya\ emission at $v \approx v_{\rm c}$ but only modest
attenuation at lower velocities.  Although the IGM effects for our
parameterization are relatively modest compared to the outflow, it is evident that
the IGM cannot be ignored at this redshift.  If the IGM were just $5
\times$ more neutral than modeled here, the \lya\ line would be
completely ``extinguished'' at $v \gtrsim 0 \mkms$, as predicted for
the $z \sim 6$ universe \citep{zheng+10a}.  By the same token, if the
galaxy emits ionizing photons then \tmax\ may fall below unity and the
IGM would play a minor role.  Clearly, a full treatment 
is warranted for the $z \sim 3$ universe.

In the bottom panels we present the full model, convolved with an
ideal spectrometer having FWHM~$\approx 250\mkms$, compared against the
average \lya\ emission profile observed for a sample of $z \sim 3$ 
LBGs \citep{shapley03}. 
[Match our resolution to Shapley??]
At this spectral resolution, which is
typical of LBG spectroscopic observations, the weak emission at $v
\approx 0 \mkms$ that was
apparent for the continuum model is lost and one predicts a simple
absorption profile that extends to positive velocity. 
This model is a very poor description of the data.  
For the line+continuum intrinsic model, in contrast, it is difficult
to detect the outflow and the IGM attenuation is completely lost at
this spectral resolution within the \lya\ emission profile.
Neither model predicts significant line-emission 
at several hundred \kms\ redward of systemic.  
Such emission is {\it not a natural prediction of a clumpy outflow
  model} where the photons tend to escape because of partial covering
not photon diffusion.  In fact, reviewing the
results on \mgii\ from PKR11, such shifts are rare in
outflow models that we have considered.  The only models in PKR11
with significant shifts to positive velocity was when there was large 
opacity at $v \approx 0 \mkms$ near the source, e.g.\ models that
included an ISM component. In essence, this gas modifies the intrinsic
line profile of \lya, yielding the standard `double hump' emission and
then the blue hump is strongly attenuated by the outflow.
[Test this with Lya!] 
It is also worth noting that a strict infall scenario on Mpc scales
also naturaly predicts a redshifted \lya\ line.  The IGM model of
\cite{zheng+10}, for example, predicts \lya\ emission will be detectd
at $v \approx +XX \mkms$ from star-forming galaxies.

In contrast to our assertions,
other authors have modeled the \lya\ profiles of LBGs and have
concluded the redshifted emission is a `clear signature' of an
outflow.
The [most extensive] modeling of \lya\ emissoin by LBGas to date has
been performed by A. Verhomme and collaborators
\citep{verhommea,verhommeb}.  Their preferred model has been a thin
expanding shell characterized by the expansion speed $v_{\rm exp}$ and
total surface density of \ion{H}{1} gas in the shell.  Although the
shell's dynamics play a role in the \lya\ emission kinematics, the
[dominant] aspect for the emergent spectrum is the \ion{H}{1} column
density.  Indeed, they only predict a dominant emission peak at $v \gg
0 \mkms$ if $\mnhi > 2\sci{20} \cm{-2}$.  For lower values, the
emission is preferentially at $v \lesssim +100 \mkms$.  The
explanation is similar to the ISM model described above;  large \nhi\
values force diffusion of \lya\ in frequency.  Furthermore, their
models predict substantial \lya\ absorption at $v < 0 \mkms$.  
Strong absorption is observed in a handful of individual spectra
\citep[e.g.][]{cb58,dessauges}, but is not evident in the stacked
spectra of \cite{shapley93} or \cite{steidel+10}.  
[Why is there sometimes weak \lya\ emission at $v < 0 \mkms$ in LAEs
at $z \sim 3$?]
[Read Shapley et al 2009 paper]
[Redshfited, asymmetric emission is suggestive of an outflow]


Now consider the spatial distribution of \lya\ emission.
Figure~\ref{fig:SB} shows the predicted surface brightness profile
vs.\ projected radius for the flat-continuum intrinsic model and with
no IGM attenuation.  The signal is integrated over velocities $v =
[-100,400]\mkms$, which corresponds to the majority of predicted line-emission.
Overplotted on the model, at an arbirtrary
normalization, is the fitted profile of S+11 to their observations,
$\mu(\rho) \propto \exp[-\rho/b]$ with $b=20$\,kpc.  
Although this model appears to be tending toward such an exponential
profile at very large impact parameter ($\rho > 50$\,kpc), the model
predict greater flux and a steeper fall-off at small radii.
In strong constrast to
the observations, the clump model predicts that 95\%\ of the
integrated flux is emitted at $\rho < 10$\,kpc.  
It is simply impossible to reconcile the LBG clump model with the
observed profile, and we further emphasize that including the IGM
attenutation and/or an intrinsically bright \lya\ emission line only
worsens the agreement.

The explanation for the model's failure is twofold:  
(1) the covering fraction of the clumps decrease with increasing radius to
$f_{\rm c} < 0.2$ at $r > 50$\,kpc; and
(2) the velocity gradient is very shallow beyond the inner few kpc. Therefore,
the outflow absorbs very few photons at large radius compared to the
inner regions.  In terms of the Sobolev approximation, we estimate
$\tau_{\rm S}(\rho = 60 \, {\rm kpc}) = 10^{-4} \tau_{\rm S}(\rho=1\, {\rm
  kpc})$.  [Figure?]
We further emphasize that the inclusion of an ISM
component at small radii would not qualitatively alter any of these conclusions.
Instead, one requires additional sources of \lya\ emission at large
radii or additional opacity in the CGM far beyond that predicgted by
the LBG clump model.  We return to these issues in
$\S$~\ref{sec:later}.


\section{On the Energetics of the LBG Clump Model}
\label{sec:energy}

In the previous section, we showed that the LBG clump model
of S+10 cannot on its own reproduce the observed \lya\ emission associated with
LBGs.   We also emphasize, however, that the radiative transfer of
\lya\ is sensitive to the intrinsic emission of the galaxy ($r \sim
10$\,kpc), the physics of the gas in the CGM ($r \sim 100$\,kpc), and
also the properties of the IGM on larger scales ($r \sim 1-10$\,Mpc).
It is possible, therefore, that the observed \lya\ emission is
dominated by components other than a galactic-scale wind.
In this respect, the LBG clump
model cannot be ruled out because of its discordance with the \lya\
observations.

In PKR11, we provided estimates for the energy and mass flux of the
LBG wind model, as approximated by a Sobolev solution.  We now
examine the same quantities for the clump model.  Consider a shell of
thickness $\Delta r$ at radius $r$ from an LBG galaxy.  We may relate
the number density of clumps $n(r)$ in this shell to the covering
fraction of the clumps $f_{\rm c}(r)$:

\begin{equation}
n(r) = \frac{f_{\rm c}(r)}{\sigma} \frac{1}{\Delta r}
\label{eqn:density}
\end{equation}
with $\sigma$ the cross-section of the clump.  If each clump has a
mass $m_{\rm c}$ then the mass within the shell is

\begin{equation}
M_{\Delta r}(r) = 4\pi r^2 f_{\rm c}(r) \, \frac{m_{\rm c}}{\sigma}
\perd
\label{eqn:mshell}
\end{equation}
Following the Sobolev formalism, we set the shell thickness at a given
radius according to the velocity gradient of the flow,

\begin{equation}
\Delta r = \Delta v_{\rm c} \ltp \frac{dv}{dr} \rtp^{-1} \cmma
\end{equation}
where $\Delta v_{\rm c}$ characterizes the velocity dispersion of the
clumps.  This velocity dispersion may result from any combination of internal or external motions within
the shell.  In the following we adopt $\Delta v_{\rm c} = 10 \mkms$.

Lastly, we constrain the surface density of the clumps $m_{\rm
  c}/\sigma$ by demanding each clump be optically thick to the
metal-line transitions observed by S+10, e.g.\ \ion{Si}{2}~1526.  
This implies a minimum column density of Si$^+$ ions;\footnote{For this
  calculation we adopt the LBG clump model parameters from S+10 that
  apply to Si$^+$.  In this respect, the estimates may be a lower
  limit given that the Hydrogen gas is predicted to extend to larger
  radii.} 
we adopt
$\N{Si^+} > 10^{14} \cm{-2}$.  In turn, this yields a lower limit to
$m_{\rm c}/\sigma$.  For a solar abundance $n_{\rm H} = 10^{-4.5}
n_{\rm Si}$ and assuming no ionization correction, we find $N_{\rm H}
\ge 10^{18.5} \cm{-2}$ and therefore $m_{\rm c} / \sigma \ge 3.3
\sci{-2} \msun \, \rm pc^{-2}$ when one accounts for helium.

Adopting the minimum value for $\mmc/\sigma$ from above,  we derive the mass,
mass flux, energy, and power profiles shown in
Figure~\ref{fig:mass_energy}.
[Note that the clump flux ($nvr^2$) is nearly constant (increases by a
factor of 2 somehow!) at $10^{-2}$ clumps/yr].
Integrating the mass profile of the clumpy wind, we calculate 
$M_{\rm TOT} \approx 10^{10} \msun$ [check] with nearly all of this
matter laying at radii $r>20$\,kpc.  Ironically, this material accounts for very little
of the down-the-barrel absorption\footnote{The only indisputable
  outflow component.} in the model: XX\%\ of the mass in the flow
corresponds to XX\%\ of the absorption.  
[Can momentum driven winds get this much energy/mass flux?]
This massive outflow was instead motivated by the significant
absorption toward background sources.

Both the energy and power of this outflow are predicted to achieve
very large quantities and, again, most of this occurs at large radii.
[Discuss \lya\ emission here]

\acknowledgments

J.X.P and K.R. are partially supported
by an NSF CAREER grant (AST--0548180), and 
by NSF grant AST-0908910.

\clearpage

%\bibliographystyle{/u/xavier/NSF/SASIR/SASIR-ATI/prop2009/Text/nsfati}
%\bibliography{/u/xavier/NSF/SASIR/SASIR-ATI/prop2009/Text/nsfati09}
\bibliographystyle{/u/xavier/paper/Bibli/apj}
\bibliography{/u/xavier/paper/Bibli/allrefs}

\clearpage

%\begin{deluxetable}{lcccccc}
\tabletypesize{\footnotesize}
\tablecolumns{11}
\tablecaption{Observed Transitions and Limits \label{tab:atomic}}
\tablewidth{0pt}
\tablehead{\colhead{} & \colhead{$\rm E_{high}$} & \colhead{$\rm E_{low}$} & \colhead{$J_{\rm high}$} & \colhead{$J_{\rm low}$} & \colhead{$\lambda$} & \colhead{$A$} \\
 & \colhead{($\rm cm^{-1}$)} & \colhead{($\rm cm^{-1}$)} &&& \colhead{(\AA)} & \colhead{($\rm s^{-1}$)} } 
\startdata
\ion{Fe}{2} UV1 & 38458.98 &     0.00 &   9/2 & 9/2 & 2600.173 & 2.36E08  \\
           & 38458.98 &   384.79 &   9/2 & 7/2 & 2626.451 & 3.41E+07 \\
           & 38660.04 &     0.00 &   7/2 & 9/2 & 2586.650 & 8.61E+07 \\
           & 38660.04 &   384.79 &   7/2 & 7/2 & 2612.654 & 1.23E+08 \\
           & 38660.04 &   667.68 &   7/2 & 5/2 & 2632.108 & 6.21E+07 \\
           & 38858.96 &   667.68 &   5/2 & 5/2 & 2618.399 & 4.91E+07 \\
           & 38858.96 &   862.62 &   5/2 & 3/2 & 2631.832 & 8.39E+07 \\
           & 39013.21 &   667.68 &   3/2 & 5/2 & 2607.866 & 1.74E+08 \\
           & 39013.21 &   862.61 &   3/2 & 3/2 & 2621.191 & 3.81E+06 \\
           & 39013.21 &   977.05 &   3/2 & 1/2 & 2629.078 & 8.35E+07 \\
           & 39109.31 &   862.61 &   1/2 & 3/2 & 2614.605 & 2.11E+08 \\
           & 39109.31 &   977.05 &   1/2 & 1/2 & 2622.452 & 5.43E+07 \\
\tableline \\ [-1.5ex]
\ion{Mg}{2}& 35760.89 &     0.00 &   3/2 &   0 & 2796.351 & 2.63E+08\\
           & 35669.34 &     0.00 &   1/2 &   0 & 2803.528 & 2.60E+08\\
\enddata
\tablecomments{Atomic data was obtained from \citet{Morton2003} unless otherwise indicated.}
\end{deluxetable}


\begin{figure}
\epsscale{0.95}
\plotone{Figures/fig_lbg_cover_1d.ps}
\caption{
1D spectra
}
\label{fig:1D}
\end{figure}

\begin{figure}
\epsscale{0.8}
\plotone{Figures/fig_lbg_sb.ps}
\caption{
Surface brightness
}
\label{fig:SB}
\end{figure}

\begin{figure}
\includegraphics[scale=0.6,angle=90]{Figures/fig_lbg_mass_etc.ps}
\caption{
Mass, energy
}
\label{fig:mass_energy}
\end{figure}

\end{document}
