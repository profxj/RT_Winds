\documentclass[12pt,preprint]{aastex}
\usepackage{natbib,amsmath}

\special{papersize=8.5in,11in}
\begin{document}

\newcommand{\tpk}{$\tau_{\rm pk}$}
\newcommand{\mtpk}{\tau_{\rm pk}}
\newcommand{\mdv}{\delta v}
\newcommand{\ewabs}{$W_{\rm a}$}
\newcommand{\ewe}{$W_{\rm e}$}
\newcommand{\rtt}{$r_{\tau = 0.2}$}
\newcommand{\mrtt}{r_{\tau = 0.2}}
\newcommand{\bturb}{$b_{\rm turb}$}
\newcommand{\mbturb}{b_{\rm turb}}
\newcommand{\taud}{$\tau_{\rm dust}$}
\newcommand{\mtaud}{\tau_{\rm dust}}
\newcommand{\maconfig}{2p$^6$3s}
\newcommand{\mbconfig}{2p$^6$3p}
\newcommand{\aconfig}{a~$^6$D$^0$}
\newcommand{\zconfig}{z~$^6$D$^0$}
\newcommand{\mvr}{v_{\rm r}}
\newcommand{\naid}{\ion{Na}{1}~$\lambda\lambda 5891, 5897$}
\newcommand{\mgiid}{\ion{Mg}{2}~$\lambda\lambda 2796, 2803$}
\newcommand{\mgiia}{\ion{Mg}{2}~$\lambda 2796$}
\newcommand{\mgiib}{\ion{Mg}{2}~$\lambda 2803$}
\newcommand{\feiia}{\ion{Fe}{2}~$\lambda 2586$}
\newcommand{\feiib}{\ion{Fe}{2}~$\lambda 2600$}
\newcommand{\feiic}{\ion{Fe}{2}$* \; \lambda 2612$}
\newcommand{\feiie}{\ion{Fe}{2}$* \; \lambda 2626$}
\newcommand{\feiid}{\ion{Fe}{2}~$\lambda\lambda 2586, 2600$}
\newcommand{\feiis}{\ion{Fe}{2}$*$}
\newcommand{\nmg}{$n_{\rm Mg^+}$}
\newcommand{\mnmg}{n_{\rm Mg^+}}
\newcommand{\nfe}{$n_{\rm Fe^+}$}
\newcommand{\mnfe}{n_{\rm Fe^+}}
\def\hub{h_{72}^{-1}}
\def\umfp{{\hub \, \rm Mpc}}
\def\mzq{z_q}
\def\zabs{$z_{\rm abs}$}
\def\mzabs{z_{\rm abs}}
\def\intl{\int\limits}
\def\cmma{\;\;\; ,}
\def\perd{\;\;\; .}
\def\ltk{\left [ \,}
\def\ltp{\left ( \,}
\def\ltb{\left \{ \,}
\def\rtk{\, \right  ] }
\def\rtp{\, \right  ) }
\def\rtb{\, \right \} }
\def\sci#1{{\; \times \; 10^{#1}}}
\def \rAA {\rm \AA}
\def \zem {$z_{\rm em}$}
\def \mzem {z_{\rm em}}
\def\smm{\sum\limits}
\def \cmm  {cm$^{-2}$}
\def \cmmm {cm$^{-3}$}
\def \kms  {km~s$^{-1}$}
\def \mkms  {\, {\rm km~s^{-1}}}
\def \lyaf {Ly$\alpha$ forest}
\def \Lya  {Ly$\alpha$}
\def \lya  {Ly$\alpha$}
\def \mlya  {Ly\alpha}
\def \Lyb  {Ly$\beta$}
\def \lyb  {Ly$\beta$}
\def \lyg  {Ly$\gamma$}
\def \ly5  {Ly-5}
\def \ly6  {Ly-6}
\def \ly7  {Ly-7}
\def \nhi  {$N_{\rm HI}$}
\def \mnhi  {N_{\rm HI}}
\def \lnhi {$\log N_{HI}$}
\def \mlnhi {\log N_{HI}}
\def \etal {\textit{et al.}}
\def \lyaf {Lyman--$\alpha$ forest}
\def \mnmin {\mnhi^{\rm min}}
\def \nmin {$\mnhi^{\rm min}$}
\def \O {${\mathcal O}(N,X)$}
\newcommand{\cm}[1]{\, {\rm cm^{#1}}}
\def \snrlim {SNR$_{lim}$}

\title{The Emperor Has No Clothes}

\author{
J. Xavier Prochaska\altaffilmark{1}, 
Others
%John M. O'Meara\altaffilmark{2}, 
%Gabor Worseck\altaffilmark{1} 
%\& Scott Burles\altaffilmark{3}
}
\altaffiltext{1}{Department of Astronomy and Astrophysics, UCO/Lick Observatory, University of California, 1156 High Street, Santa Cruz, CA 95064}
%\altaffiltext{2}{Department of Chemistry and Physics, Saint Michael's College.
%One Winooski Park, Colchester, VT 05439}

\begin{abstract}
We consider the observed \lya\ emission from Lyman Break Galaxies
(LBGs) and their surrounding environment in the context of the LBG
clump model proposed by \cite{steidel10}.  This model does not
reproduce the observations: the systematic redshift of \lya\ is not a
natural prediction of a clumpy wind and the clump model has too low
covering fraction and velocity graident to yield significant scattered
emission at large impact parameter.  We also consider the mass, mass
flux, energy, and power of this wind model and find that one requires
extreme quantities.  [Mention Lya emission here?]
[Mention most of the mass in the wind accounts for very little EW in
the down-the-barell observations]
We argue that this model is an inaccurate description of the CGM
surrounding high-$z$ star-forming galaxies, and that the observed
outflow is likely confined to $r \lesssim 10$\,kpc.
\end{abstract}

\keywords{absorption lines -- intergalactic medium -- Lyman limit systems -- SDSS}

\section{Introduction}

Spectroscopic observations of star-forming galaxies reveal that 
systems at all redshifts experience outflows of cool, diffuse gas
\citep[e.g.][]{rupke,weiner,shapley03,rubin}.  The indisputable
signature is the preponderence of significant low-ion absorption
(e.g.\ \ion{Na}{1}, \ion{Mg}{2}, \ion{Si}{2}) that is
blueshifted with respect to the galaxy's stars and nebular regions.
These data characterize the kinematics of the outflow and may place
constraints on the optical depth and/or covering fraction of the
material.  These ``down-the-barrel'' observations, however, provide
very weak constraints on the distance of the gas from the galaxy.
Therefore, the mass, energy, and power of the wind and its subsequent
affect on the so-called circumgalactic medium of the galaxy remain
open and compelling questions.

At $z\sim 3$, the most extensively observed poplution of star-forming
galaxies is the Lyman Break Galaxies (LBGs), named for their
characteristic dip in flux from absorption by the intergalactic medium
(IGM).  For the past $\sim 15$ years, the group led by C. Steidel has
amassed the largest spectroscopic and imaging datasets on these
galaxies \citep{steidel96,shapley03,reddyXX}.  These earlier works
demonstrated nearly ubiquitous absorption from low and high-ion
transitions that is systematically blue-shifted relative to the
galaxy's stars and nebular emission.  These same spectra also revealed
a diversity of \lya\ emission lines whose properties correlate with galaxy photometric
and spectroscopic properties \citep{shapley08,cooke08}.  Irrespective
of the \lya\ equivalent widths \wlya\, however, the line is uniformly
shifted redward of systemic.  This has been interpted by numerous
authors as yet another signature of the outflow
\citep[e.g.][]{pettini,verhamme}. 

Recently, their
collaboration has published two papers that offer new insight into the
CGM surrounding LBGs.   In one experiment \citep[][hereafter
S+10]{steidel+10}, the authors have stacked low S/N, moderate
resolution ($R \sim XX$) spectra of background galaxies at the
rest-wavelengths of foreground LBGs with impact parameters $\rho \sim
10-100$\,kpc.  This analysis revealed that LBGs are surrounded by a
CGM that gives rise to \ion{H}{1}, low-ion (e.g.\ \ion{C}{2}~1334,
\ion{Si}{2}), and high-ion (e.g.\ \ion{C}{4}) absorption from $\rho
\approx 0-100$\,kpc.  S+10 modeled the blue-shifted absorption
observed down-the-barrel with the extended absorption detected to
large impact parameter as a clumpy, galactic-scale wind with speeds of
$v \sim 800 \, \mkms$ and radial extend $r \sim 100$\,kpc.  Although
this wind model is not a unique interpretation of the extend
\ion{H}{1} absorption \citep{fpk+11}, it appears to satisify the key
absorption observations.

Complementing the absorption-line observations and analysis,
Steidel et al.\ (2011; hereafter S+11) have explored the spatial
extent of \lya\ emission from the CGM surrounding LBGs.  In a stacked,
narrow-band image of $\approx 100$ LBGs at $z \approx 2.5$ they
measure a `halo' of low surface brightness \lya\ emission with an
approximately expoentinal profile $\mu(\rho) \sim \exp{-\rho/b}$ from
$\rho \approx 10-80$\,kpc with $b \approx 20$\,kpc.  These authors
interpreted the signal as the scattering of \lya\ photons generated
within the central LBG by an extended CGM surrounding each galaxy.
They further reported that the same LBG clumpy wind model that
reproduces the stacked absorption spectra (S+10) can reproduce the
kinematics of \lya\ emitted by the galaxy and the extended \lya\
emission.  At the same time, they allowed that a proper treatment
including full radiative transfer was highly warranted.





\acknowledgments

J.X.P and K.R. are partially supported
by an NSF CAREER grant (AST--0548180), and 
by NSF grant AST-0908910.

\clearpage

%\bibliographystyle{/u/xavier/NSF/SASIR/SASIR-ATI/prop2009/Text/nsfati}
%\bibliography{/u/xavier/NSF/SASIR/SASIR-ATI/prop2009/Text/nsfati09}
\bibliographystyle{/u/xavier/paper/Bibli/apj}
\bibliography{/u/xavier/paper/Bibli/allrefs}

\clearpage

%\begin{deluxetable}{lcccccc}
\tabletypesize{\footnotesize}
\tablecolumns{11}
\tablecaption{Observed Transitions and Limits \label{tab:atomic}}
\tablewidth{0pt}
\tablehead{\colhead{} & \colhead{$\rm E_{high}$} & \colhead{$\rm E_{low}$} & \colhead{$J_{\rm high}$} & \colhead{$J_{\rm low}$} & \colhead{$\lambda$} & \colhead{$A$} \\
 & \colhead{($\rm cm^{-1}$)} & \colhead{($\rm cm^{-1}$)} &&& \colhead{(\AA)} & \colhead{($\rm s^{-1}$)} } 
\startdata
\ion{Fe}{2} UV1 & 38458.98 &     0.00 &   9/2 & 9/2 & 2600.173 & 2.36E08  \\
           & 38458.98 &   384.79 &   9/2 & 7/2 & 2626.451 & 3.41E+07 \\
           & 38660.04 &     0.00 &   7/2 & 9/2 & 2586.650 & 8.61E+07 \\
           & 38660.04 &   384.79 &   7/2 & 7/2 & 2612.654 & 1.23E+08 \\
           & 38660.04 &   667.68 &   7/2 & 5/2 & 2632.108 & 6.21E+07 \\
           & 38858.96 &   667.68 &   5/2 & 5/2 & 2618.399 & 4.91E+07 \\
           & 38858.96 &   862.62 &   5/2 & 3/2 & 2631.832 & 8.39E+07 \\
           & 39013.21 &   667.68 &   3/2 & 5/2 & 2607.866 & 1.74E+08 \\
           & 39013.21 &   862.61 &   3/2 & 3/2 & 2621.191 & 3.81E+06 \\
           & 39013.21 &   977.05 &   3/2 & 1/2 & 2629.078 & 8.35E+07 \\
           & 39109.31 &   862.61 &   1/2 & 3/2 & 2614.605 & 2.11E+08 \\
           & 39109.31 &   977.05 &   1/2 & 1/2 & 2622.452 & 5.43E+07 \\
\tableline \\ [-1.5ex]
\ion{Mg}{2}& 35760.89 &     0.00 &   3/2 &   0 & 2796.351 & 2.63E+08\\
           & 35669.34 &     0.00 &   1/2 &   0 & 2803.528 & 2.60E+08\\
\enddata
\tablecomments{Atomic data was obtained from \citet{Morton2003} unless otherwise indicated.}
\end{deluxetable}


\begin{figure}
\epsscale{0.95}
\plotone{Figures/fig_lbg_cover_1d.ps}
\caption{
1D spectra
}
\label{fig:1D}
\end{figure}

\begin{figure}
\epsscale{0.8}
\plotone{Figures/fig_lbg_sb.ps}
\caption{
Surface brightness
}
\label{fig:SB}
\end{figure}

\begin{figure}
%\includegraphics[scale=0.6,angle=90]{Figures/mass_energy.ps}
\caption{
Mass, energy
}
\label{fig:mass_energy}
\end{figure}

\end{document}
