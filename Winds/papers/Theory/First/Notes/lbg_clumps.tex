\documentclass[11pt,letterpaper]{article}

%\font\bfit = cmbxti10 scaled \magstephalf

\usepackage{amsmath}
\usepackage{latexsym}
\usepackage{graphicx}
\usepackage{amssymb}
\usepackage{ulem}
\usepackage{fancyheadings}
\usepackage[usenames]{color}
 
\pretolerance=10000
\textwidth=6.7in
\textheight=8.9in
\topmargin=-0.6in
\headheight=.15in
\hoffset = -0.2in
\headsep=.35in
\oddsidemargin=0in
\evensidemargin=0in
\parindent=2em
\parskip=0.2ex

\input{/u/xavier/bin/defs}
\input{/u/xavier/bin/latex}

\newcommand{\nc}{$\mathcal{N}$}
\newcommand{\mnc}{\mathcal{N}}
\newcommand{\mvr}{v_{\rm r}}

\newenvironment{my_enumerate}{
\begin{enumerate}
  \setlength{\itemsep}{1pt}
  \setlength{\parskip}{0pt}
  \setlength{\parsep}{0pt}}{\end{enumerate}
}

\newenvironment{my_itemize}{
\begin{itemize}
  \setlength{\itemsep}{1pt}
  \setlength{\parskip}{0pt}
  \setlength{\parsep}{0pt}}{\end{itemize}
}


\pagestyle{empty}

\special{papersize=8.5in,11in}

\begin{document}

\begin{center}
{\Large Simple Exploration of the LBG Clump Model (Steidel+10): v2}
\end{center}

Here are some definitions of key quantities: 

\begin{my_itemize}
\item \nc: Number of clumps in a given shell of width $\Delta r$
\item $\Delta v_c$: Internal velocity dispersion of the clump (e.g.\
  10 km/s)
\item $\Delta r$: Width of the shell. This will relate to the internal
  velocity dispersion of the clumps.
\item $n$: Number density of clumps at radius $r$
\item $d_c$: Size of the clump (assumed to be cubes)
\item $\sigma$: Cross-section of the clump ($d_c^2$)
\item $n_c$: Number density of a clump (Hydrogen)
\item $m_c$: Mass of a clump, i.e. $m_p n_c d_c^3$
%\item $A$: Total area of clumps in $\Delta r$, i.e. $\mnc \sigma$
\item $f_c$: Covering fraction of clumps at radius $r$
\end{my_itemize}

In the S10 model, the clumps have covering fraction

\begin{equation}
f_c(r) = f_{c,max} \ltp \frac{r}{r_{\rm inner}} \rtp^{-\gamma} \cmma
\label{eqn:covering}
\end{equation}
with $\gamma \approx 0.5$ and velocity

\begin{equation}
\mvr = \ltp \frac{A_{\rm LBG}}{1-\alpha} \rtp^{1/2} \ltk r_{\rm
  inner}^{1-\alpha} - r^{1-\alpha} \rtk^{1/2} \perd
\label{eqn:LBG_vlaw}
\end{equation}
with $\alpha \approx 1.3$.

Consider a shell of width $\Delta r$ at radius $r$.   The number of
clumps in this shell is:

\begin{equation}
\mnc = n \cdot \Delta r \cdot 4 \pi r^2
\end{equation}
And the covering fraction will be

\begin{equation}
f_c = \frac{\sigma \mnc}{4 \pi r^2} = n \sigma \Delta r
\end{equation}
Lastly, the thickness of a given shell relates to the overlap of the
clumps in velocity space.  Given an internal velocity dispersion $\Delta
v_c$ for each clump, we have:

\begin{equation}
\Delta r = \Delta v_c \ltp \frac{dv}{dr} \rtp^{-1}  \perd
\end{equation}
This starts to look like the Sobolev equations, but not quite.  The
key issue here is that $\Delta r$ varies with radius.  Because the LBG
velocity curve flattens, $\Delta r$ increases quite rapidly with $r$.

If we like, we can rewrite these equations in terms of the density of
the clumps

\begin{equation}
n(r) = \frac{f_c(r)}{\sigma} \frac{1}{\Delta r} \cmma
\end{equation}
the mass of a shell of $\Delta r$

\begin{equation}
M_{\Delta r} = \frac{f}{\sigma} \cdot 4 \pi r^2 \cdot m_c
\end{equation}
We see (as in v1 of this write-up) that the mass in each shell goes as
$r^{3/2}$ given the functional form of the covering fraction.  But the
mistake I had made is that the thickness of the shells ($\Delta r$) is
increasing with radius.  So the cumulative mass doesn't increase as
rapidly.   

Now consider an estimation of the absolute value for the mass.  I'll
begin by assuming a size for the clump of $d_c = 100$\,pc.  If we
desire this clump to be optically thick for SiII~1526, then we need a
Si$^+$ number density $n_{\rm Si^+} = 10^{14} \cm{-2} / d_c$.  Lastly,
I will assume solar abundance, i.e. $n_H = 10^{4.5} n_{\rm Si^+}$.
This gives a number density $n_H \approx 0.01 \cm{-3}$.  I think it
might be hard to have much lower number density. Each clump,
then, has a mass $m_c = 253 \msol$. 

With this clump size and mass, we get the density, mass, and flux
profiles shown in Figure~\ref{fig:aclump}.  Each of these curves start
at $r = 2$\,kpc because at smaller radii the clumps overlap.  Again,
the mass approaches $10^{10} \msol$ and most of the mass lies at the
outer radii.  The clump density drops very rapidly with radius such
that the clump flux is nearly constant.  I think this is by design
given the shells have thickness $\Delta r \propto (dv/dr)^{-1}$, but
I'm not sure.

\begin{figure}[ht]
\begin{center}
\includegraphics[width=5.5in]{../Figures/fig_lbg_aclump.pdf}
\end{center}
  \caption{$\Delta r$, density, mass, and flux profiles for the LBG
    wind model assuming $\Delta v_c = 10 \mkms$ and that the clump has
    $d_c = 100$\,pc and $m_c = 250 \msol$.
}
  \label{fig:aclump}
\end{figure}

So, is this at all plausible?  And, if not, can one monkey with the
size or $\Delta v_c$ of my clumps etc.\ to make a plausible scenario?
It is evident that $M_{\rm TOT} \propto m_c / (\sigma \Delta v_c)$
where the $\Delta v_c$ comes in because that sets the size of each
shell.  I had worried that $(m_c / \sigma) \propto d_c$ but we are
demanding that each cloud is optically thick, i.e.\ $n_c d_c = {\rm
  Const}$.  Therefore, $m_c / \sigma = {\rm Const}$.  So all the
dependence is in $\Delta v_c$ and I don't see raising that by more
than a factor of a few.


\end{document}

IDL> c = x_constants()
IDL> dc = 100*c.pc
IDL> nSi = 1d14 / dc
IDL> nH = 10.^(4.5)*nSi
IDL> print, nH
     0.010248243
