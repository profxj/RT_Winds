\documentclass[12pt,letterpaper]{article}

%\font\bfit = cmbxti10 scaled \magstephalf

\usepackage{amsmath}
\usepackage{latexsym}
\usepackage{graphicx}
\usepackage{amssymb}
\usepackage{ulem}
\usepackage{fancyheadings}
\usepackage[usenames]{color}
 
\pretolerance=10000
\textwidth=6.7in
\textheight=8.9in
\topmargin=-0.6in
\headheight=.15in
\hoffset = -0.2in
\headsep=.35in
\oddsidemargin=0in
\evensidemargin=0in
\parindent=2em
\parskip=0.2ex

\input{/u/xavier/bin/defs}
\input{/u/xavier/bin/latex}

\newcommand{\nc}{$\mathcal{N}$}
\newcommand{\mnc}{\mathcal{N}}
\newcommand{\mvr}{v_{\rm r}}

\newenvironment{my_enumerate}{
\begin{enumerate}
  \setlength{\itemsep}{1pt}
  \setlength{\parskip}{0pt}
  \setlength{\parsep}{0pt}}{\end{enumerate}
}

\newenvironment{my_itemize}{
\begin{itemize}
  \setlength{\itemsep}{1pt}
  \setlength{\parskip}{0pt}
  \setlength{\parsep}{0pt}}{\end{itemize}
}


\pagestyle{empty}

\special{papersize=8.5in,11in}

\begin{document}

\begin{center}
{\Large Simple Exploration of the LBG Clump Model (Steidel+10)}
\end{center}

Here are some definitions of key quantities: 

\begin{my_itemize}
\item \nc: Number of clumps in a given shell of width $\Delta r$
\item $d_c$: Size of the clump (assumed to be cubes)
\item $\sigma$: Cross-section of the clump ($d_c^2$)
\item $n_c$: Number density of a clump (Hydrogen)
\item $m_c$: Mass of a clump, i.e. $m_p n_c d_c^3$
\item $A$: Total area of clumps, i.e. $\mnc \sigma$
\item $f_c$: Covering fraction of clumps at radius $r$
\end{my_itemize}

In the S10 model, the clumps have covering fraction

\begin{equation}
f_c(r) = f_{c,max} \ltp \frac{r}{r_{\rm inner}} \rtp^{-\gamma} \cmma
\label{eqn:covering}
\end{equation}

with $\gamma \approx 0.5$ and velocity

\begin{equation}
\mvr = \ltp \frac{A_{\rm LBG}}{1-\alpha} \rtp^{1/2} \ltk r_{\rm
  inner}^{1-\alpha} - r^{1-\alpha} \rtk^{1/2} \perd
\label{eqn:LBG_vlaw}
\end{equation}
with $\alpha \approx 1.3$.

Consider a shell of width $\Delta r$ at radius $r$.  The surface area
of this shell is $4\pi r^2$.  To achieve a covering fraction $f_c$,
one requires

\begin{equation}
\mnc(r) = \frac{4 \pi r^2}{\sigma} \cdot f_c(r) = 
\frac{4 \pi f_{c,max}}{\sigma \, (\rm kpc^{1/2})} \cdot r^{3/2}
\end{equation}
clumps within the shell.  It is evident that the number of clumps
increases rather steeply with radius.  [Note: we could allow for
varying clump size by doing the calculation in terms of $A$ but I
don't think it would matter much.]

Now consider the mass profile of this clumpy wind.  Again, assuming
identical clumps at all radii, we have the mass per shell of 

\begin{equation}
M_{\Delta r} = m_c \mnc 
\end{equation}
Normalizing to the mass in the first shell ($r = 1 - 2$\,kpc), we get
the cumulative mass profile shown in Figure~\ref{fig:mass}.
We find that there is 10000$\times$ more mass in the wind beyond that
in the innermost shell!  

\clearpage

\begin{figure}[ht]
\begin{center}
\includegraphics[width=4.5in,angle=180]{../Figures/fig_lbg_clump_mass.pdf}
\end{center}
  \caption{Cumulative mass profile, normalized to the mass within
    $r=1-2$\,kpc, for identical clumps.  There is over 10000$\times$
    more mass beyond 2\,kpc than within that first shell. 
}
  \label{fig:mass}
\end{figure}

\begin{figure}[ht]
\begin{center}
\includegraphics[width=4.5in,angle=180]{../Figures/fig_lbg_clump_eandp.pdf}
\end{center}
  \caption{Cumulative energy and momentum profiles.
}
  \label{fig:eandp}
\end{figure}


Figure~\ref{fig:eandp} shows the energy and momentum
profiles, again relative to that in the inner most shell assuming
$E_{\Delta r} = M_{\Delta r} v^2$ and $P = M_{\Delta r} v$.  The
results are similar to the mass profile because the velocity curve
rises so steeply at small radii.

\vskip 0.2in

Now consider an estimation of the absolute value for the mass.  I'll
begin by assuming a size for the clump of $d_c = 100$\,pc.  If we
desire this clump to be optically thick for SiII~1526, then we need a
Si$^+$ number density $n_{\rm Si^+} = 10^{14} \cm{-2} / d_c$.  Lastly,
I will assume solar abundance, i.e. $n_H = 10^{4.5} n_{\rm Si^+}$.
This gives a number density $n_H \approx 0.01 \cm{-3}$.  I think it
might be hard to have much lower number density. Each clump,
then, has a mass $m_c = 253 \msol$. To
get a 80\%\ covering fraction at 2\,kpc, we will need $\mnc \approx
4000$ clumps or a total of $\approx 10^6 \msol$.  Finally, we estimate
that the total mass in the wind is $10^{10} \msol$ with $\sim 90\%$ of
that mass beyond 30\,kpc.  All of this material is travelling at
800\,\kms.  

\vskip 0.1in

So, is this at all plausible?  And, if not, can one monkey with the
size of my clumps etc.\ to make a plausible scenario?

\end{document}

IDL> c = x_constants()
IDL> dc = 100*c.pc
IDL> nSi = 1d14 / dc
IDL> nH = 10.^(4.5)*nSi
IDL> print, nH
     0.010248243
