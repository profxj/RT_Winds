\documentclass[12pt,preprint]{aastex}
\usepackage{natbib,amsmath}

\special{papersize=8.5in,11in}
\begin{document}

\newcommand{\taud}{$\tau_{\rm dust}$}
\newcommand{\mtaud}{\tau_{\rm dust}}
\newcommand{\maconfig}{2p$^6$3s}
\newcommand{\mbconfig}{2p$^6$3p}
\newcommand{\aconfig}{a~$^6$D$^0$}
\newcommand{\zconfig}{z~$^6$D$^0$}
\newcommand{\mvr}{v_{\rm r}}
\newcommand{\naid}{\ion{Na}{1}~$\lambda\lambda 5891, 5897$}
\newcommand{\mgiid}{\ion{Mg}{2}~$\lambda\lambda 2796, 2803$}
\newcommand{\mgiia}{\ion{Mg}{2}~$\lambda 2796$}
\newcommand{\mgiib}{\ion{Mg}{2}~$\lambda 2803$}
\newcommand{\feiia}{\ion{Fe}{2}~$\lambda 2586$}
\newcommand{\feiib}{\ion{Fe}{2}~$\lambda 2600$}
\newcommand{\feiic}{\ion{Fe}{2}$^* \; \lambda 2612$}
\newcommand{\feiid}{\ion{Fe}{2}~$\lambda\lambda 2586, 2600$}
\newcommand{\nmg}{$n_{\rm Mg^+}$}
\newcommand{\mnmg}{n_{\rm Mg^+}}
\newcommand{\nfe}{$n_{\rm Fe^+}$}
\newcommand{\mnfe}{n_{\rm Fe^+}}
\def\hub{h_{72}^{-1}}
\def\umfp{{\hub \, \rm Mpc}}
\def\mzq{z_q}
\def\zabs{$z_{\rm abs}$}
\def\mzabs{z_{\rm abs}}
\def\intl{\int\limits}
\def\cmma{\;\;\; ,}
\def\perd{\;\;\; .}
\def\ltk{\left [ \,}
\def\ltp{\left ( \,}
\def\ltb{\left \{ \,}
\def\rtk{\, \right  ] }
\def\rtp{\, \right  ) }
\def\rtb{\, \right \} }
\def\sci#1{{\; \times \; 10^{#1}}}
\def \rAA {\rm \AA}
\def \zem {$z_{\rm em}$}
\def \mzem {z_{\rm em}}
\def\smm{\sum\limits}
\def \cmm  {cm$^{-2}$}
\def \cmmm {cm$^{-3}$}
\def \kms  {km~s$^{-1}$}
\def \mkms  {{\rm km~s^{-1}}}
\def \lyaf {Ly$\alpha$ forest}
\def \Lya  {Ly$\alpha$}
\def \lya  {Ly$\alpha$}
\def \mlya  {Ly\alpha}
\def \Lyb  {Ly$\beta$}
\def \lyb  {Ly$\beta$}
\def \lyg  {Ly$\gamma$}
\def \ly5  {Ly-5}
\def \ly6  {Ly-6}
\def \ly7  {Ly-7}
\def \nhi  {$N_{\rm HI}$}
\def \mnhi  {N_{\rm HI}}
\def \lnhi {$\log N_{HI}$}
\def \mlnhi {\log N_{HI}}
\def \etal {\textit{et al.}}
\def \lyaf {Lyman--$\alpha$ forest}
\def \mnmin {\mnhi^{\rm min}}
\def \nmin {$\mnhi^{\rm min}$}
\def \O {${\mathcal O}(N,X)$}
\newcommand{\cm}[1]{\, {\rm cm^{#1}}}
\def \snrlim {SNR$_{lim}$}

\title{Coding the LBG Wind (in 1D)}

\begin{abstract}
These notes summarize my thoughts on how to code the LBG wind
introduced by \cite{steidel+10} in 1D.   Am certain it can be done,
but I'd like a reasonably fast and somewhat malleable algorithm.
\end{abstract}

\author{
J. Xavier Prochaska
}

\section{LBG Model}

\cite{steidel+10} parameterized their wind
in terms of radius with two quantities the wind speed, $v_{\rm wind}(r)$,
and the covering fraction of the gas $f_c(r)$.  For the wind speed,
they use this functional form

\begin{equation}
v_{\rm wind}(r) = \ltp \frac{A}{1-\alpha} \rtp^{1/2} 
\ltp r_{\rm min}^{1-\alpha} - r^{1-\alpha} \rtp^{1/2}
\end{equation}
where $r_{\rm min} = 1$\,kpc, $\alpha \approx 1.3$ and $A$ is tuned to
give a speed of $\approx 750\mkms$ at $r \gg r_{\rm min}$.  They
parameterize the covering fraction as a simple power-law:

\begin{equation}
f_{\rm c}(r) =  f_{\rm c,max} r^{-\gamma}
\end{equation}
where $\gamma = 0.5$ and the maximum covering fraction $f_{\rm
  c,max}=0.6$ is assumed to occur at $r=r_{\rm min}$.

\section{General Idea}

\noindent These are the steps as I see them

\begin{enumerate}
\item Generate a photon at $r=r_{\rm min}$ with wavelength $\lambda =
  \lambda_\gamma$
\item Propogate it to the radius where its velocity $v_{\gamma} =
  c(\lambda_\gamma-\lambda_0)/\lambda_0$ matches the wind
\item Calculate $f_{\rm c}$ at that radius.
\item Determine if the photon was scattered ($P=f_{\rm c}$.  If not, it escapes.  
\item Decide whether the scattered photon has a chance to interact
  with the wind again (often it will not).
\item If it can, there should be two radii where it could.  
\item Calculate $f_{\rm c}$ at the first radius and see if it is
  scattered. If so, proceed as in the last few steps.  If not, carry
  it to the next radius and repeat.
\item Propogate the photon until is out.  
\end{enumerate}

It would be nice to allow for the possibilty of dust extinction as
well.

\section{Some Minutiae}

As I see it, the only challenging part is to determine whether a
scattered photon has a non-zero probability of interacting with the
wind again.  Here is one way of formalizing the problem (in 2D,
i.e. $x,y$ or $r,\theta$).

First, assume the photon has scattered at a position
$x_\gamma^0,y_\gamma^0$.  We will assume it scatters into a direction
$\hat \phi$.  Parameterizing its path by the variable $s$, we have:

\begin{align}
x_\gamma(s) &= x^0_\gamma + s \, \cos\phi \\
y_\gamma(s) &= y^0_\gamma + s \, \sin\phi
\end{align}
which translates into
\begin{align}
\theta_\gamma &= \tan^{-1}(y_\gamma/x_\gamma) \\
r_\gamma &= \ltk x_\gamma^2 + y_\gamma^2 \rtk^{1/2}
\end{align}

The idea, then, is to solve for the two values of $s$ where:

\begin{equation}
v_{\rm wind}(s) [\hat r(s) \times \hat \phi] = v_\gamma
\end{equation}
where the stuff in square brackets captures the direction of the
photon relative to the radial direction.

I had thought initially that this would be a simple quadratic to
solve, but the velocity expression for the wind is sufficiently
scary-looking that I doubt it.  Looks like the job for a fast root-finder.

\bibliographystyle{/u/xavier/paper/Bibli/apj}
\bibliography{/u/xavier/paper/Bibli/allrefs}

%\begin{figure}
%\epsscale{0.8}
%\plotone{Figures/fig_dust_spec.ps}
%\caption{
%altogether.
%}
%\label{fig:dust}
%\end{figure}


\end{document}
