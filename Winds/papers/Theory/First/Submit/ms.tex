\documentclass[12pt,preprint]{aastex}
\usepackage{natbib,amsmath}

\special{papersize=8.5in,11in}
\begin{document}

\newcommand{\nhn}{$n_{\rm H}^0$}
\newcommand{\mnhn}{n_{\rm H}^0}
\newcommand{\mnhf}{n_{\rm H}^{0,f}}
\newcommand{\msun}{M_\odot}
\newcommand{\tpk}{$\tau_{\rm pk}$}
\newcommand{\mtpk}{\tau_{\rm pk}}
\newcommand{\mdv}{\delta v}
\newcommand{\ewabs}{$W_{\rm a}$}
\newcommand{\ewe}{$W_{\rm e}$}
\newcommand{\rtt}{$r_{\tau = 0.2}$}
\newcommand{\mrtt}{r_{\tau = 0.2}}
\newcommand{\bturb}{$b_{\rm turb}$}
\newcommand{\mbturb}{b_{\rm turb}}
\newcommand{\taud}{$\tau_{\rm dust}$}
\newcommand{\mtaud}{\tau_{\rm dust}}
\newcommand{\maconfig}{2p$^6$3s}
\newcommand{\mbconfig}{2p$^6$3p}
\newcommand{\aconfig}{a~$^6$D$^0$}
\newcommand{\zconfig}{z~$^6$D$^0$}
\newcommand{\mvr}{v_{\rm r}}
\newcommand{\naid}{\ion{Na}{1}~$\lambda\lambda 5891, 5897$}
\newcommand{\mgiid}{\ion{Mg}{2}~$\lambda\lambda 2796, 2803$}
\newcommand{\mgiia}{\ion{Mg}{2}~$\lambda 2796$}
\newcommand{\mgiib}{\ion{Mg}{2}~$\lambda 2803$}
\newcommand{\feiia}{\ion{Fe}{2}~$\lambda 2586$}
\newcommand{\feiib}{\ion{Fe}{2}~$\lambda 2600$}
\newcommand{\feiic}{\ion{Fe}{2}*$ \; \lambda 2612$}
\newcommand{\feiie}{\ion{Fe}{2}*$ \; \lambda 2626$}
\newcommand{\feiid}{\ion{Fe}{2}~$\lambda\lambda 2586, 2600$}
\newcommand{\feiis}{\ion{Fe}{2}*}
\newcommand{\nmg}{$n_{\rm Mg^+}$}
\newcommand{\mnmg}{n_{\rm Mg^+}}
\newcommand{\nfe}{$n_{\rm Fe^+}$}
\newcommand{\mnfe}{n_{\rm Fe^+}}
\def\hub{h_{72}^{-1}}
\def\umfp{{\hub \, \rm Mpc}}
\def\mzq{z_q}
\def\zabs{$z_{\rm abs}$}
\def\mzabs{z_{\rm abs}}
\def\intl{\int\limits}
\def\cmma{\;\;\; ,}
\def\perd{\;\;\; .}
\def\ltk{\left [ \,}
\def\ltp{\left ( \,}
\def\ltb{\left \{ \,}
\def\rtk{\, \right  ] }
\def\rtp{\, \right  ) }
\def\rtb{\, \right \} }
\def\sci#1{{\; \times \; 10^{#1}}}
\def \rAA {\rm \AA}
\def \zem {$z_{\rm em}$}
\def \mzem {z_{\rm em}}
\def\smm{\sum\limits}
\def \cmm  {cm$^{-2}$}
\def \cmmm {cm$^{-3}$}
\def \kms  {km~s$^{-1}$}
\def \mkms  {\, {\rm km~s^{-1}}}
\def \lyaf {Ly$\alpha$ forest}
\def \Lya  {Ly$\alpha$}
\def \lya  {Ly$\alpha$}
\def \mlya  {Ly\alpha}
\def \Lyb  {Ly$\beta$}
\def \lyb  {Ly$\beta$}
\def \lyg  {Ly$\gamma$}
\def \ly5  {Ly-5}
\def \ly6  {Ly-6}
\def \ly7  {Ly-7}
\def \nhi  {$N_{\rm HI}$}
\def \mnhi  {N_{\rm HI}}
\def \lnhi {$\log N_{HI}$}
\def \mlnhi {\log N_{HI}}
\def \etal {\textit{et al.}}
\def \lyaf {Lyman--$\alpha$ forest}
\def \mnmin {\mnhi^{\rm min}}
\def \nmin {$\mnhi^{\rm min}$}
\def \O {${\mathcal O}(N,X)$}
\newcommand{\cm}[1]{\, {\rm cm^{#1}}}
\def \snrlim {SNR$_{lim}$}

\title{Idealized Wind Models of Cool Gas Outflows}

\author{
J. Xavier Prochaska\altaffilmark{1}, 
Others
%John M. O'Meara\altaffilmark{2}, 
%Gabor Worseck\altaffilmark{1} 
%\& Scott Burles\altaffilmark{3}
}
\altaffiltext{1}{Department of Astronomy and Astrophysics, UCO/Lick Observatory, University of California, 1156 High Street, Santa Cruz, CA 95064}
%\altaffiltext{2}{Department of Chemistry and Physics, Saint Michael's College.
%One Winooski Park, Colchester, VT 05439}

\begin{abstract}
\begin{itemize}
\item Emission is a generic feature of (nearly) isotropic winds.  This
  emission fills in absorption at $v \approx 0$, significantly
  complicating the absorption-line analysis, especially of an ISM
  component. 
\item The relative strengths of the emission
lines, therefore, is sensitive to both the opacity and velocity
extent of the wind.
\end{itemize}
\end{abstract}

\keywords{absorption lines -- intergalactic medium -- Lyman limit systems -- SDSS}

\section{Introduction}

Nearly all gaseous objects that shine are also
observed to generate gaseous flows.  This includes the jets of protostars, the
stellar winds of massive O and B stars, the gentle Solar wind
of our Sun, the associated absorption of bright quasars, and the
spectacular jets of radio-loud AGN.   These gaseous outflows moderate
the accretion of new material onto the object, inject energy and
momentum into gas on large scales, and ...
Developing a comprehensive model for these flows is critical to
understanding the evolution of the source and its impact on the
surrounding environment.

Star-burst galaxies, whose luminosity is dominated by \ion{H}{2} regions
and massive stars, are also observed to drive gaseous outflows.  These
flows are generally expected (and sometimes observed) to have multiple
phases, for example a hot and diffuse phase traced by X-ray emission
together with a cool, denser phase traced by H$\alpha$ emission 
\citep[e.g.][]{shc+04,km10}. 
Several spectacular
examples in the local universe demonstrate that flows can extend to
tens of kpc from the galaxy \citep{M87} carrying significant speed
to escape from the gravitational potential well of the galaxy's dark
matter halo \citep{wind_escape}.

Beyond $z \sim 0$, galactic outflows are 
revealed by UV and optical absorption lines, e.g.\ \ion{Na}{1},
\ion{Mg}{2}, \ion{Si}{2} and \ion{C}{4} transitions.  With the galaxy
as a backlight, one observes gas that is predominantly
blue-shifted which indicates a flow toward
Earth and away from the galaxy.  These transitions are sensitive to
the cool (\ion{Mg}{2}; $T \sim 10^4$K) and warm (\ion{C}{4}; $T \sim
10^5$K) phases of the flow.  The incidence of cool gas outflows is
nearly universal in star-forming galaxies;  this includes systems at $z \sim 0$
which exhibit \ion{Na}{1} and \ion{Mg}{2} absorption
\citep{rvs05a,martin05,smn+09}, $z \sim 1$ star-forming galaxies traced by
\ion{Fe}{2} and \ion{Mg}{2} transitions \citep{wcp+09,rwk+10}, and
$z>2$ Lyman break galaxies (LBGs) that show blue-shifted \ion{Si}{2},
\ion{C}{2}, and \ion{O}{1} transitions \citep{sgp+96,lkg+97,shapley03}.

The observation of metal-line absorption is now a well established
means of identifying outflows. Very little research,
however,  has been directed
to comparing the observations against (even idealized) wind
models.  Instead, researchers have gleaned what limited information
is afforded by direct analysis of the absorption lines.  The data do yield
robust measurements of the speed of the gas yet they poorly constrain the
optical depth, covering fraction, density, temperature, and distance
of the flow from the galaxy.   In turn, constraints related to the
mass, energetics, and momentum of the flow suffer from
orders of magnitude uncertainty.  Both the origin and impact of
galactic-scale winds, therefore, remain open matters of debate
\citep{debate}.

Recent studies of $z \sim 1$ star-forming galaxies have revealed that
the cool gas often exhibits significant resonant-line emission (e.g.\
\ion{Mg}{2}) in
tandem with the nearly ubiquitous blue-shifted absorption
\citep{wcp+09,rwk+10}.  The resultant spectra resemble the P-Cygni
profile characteristic of stellar winds.
\cite{rubin+10c} have also detected line
emission from non-resonant \ion{Fe}{2}$^*$ transitions 
Their interpretation is that the emission is the result of photons
scattered off the wind into our sightline, analogous to P-Cygni.
\cite{rubin+10c} also observed spatially extended \ion{Mg}{2} emission
which they use to estimate the size of the outflow.  They infer that
the wind extends to at least 5\,kpc from the galaxy.  Line emission is
also observed for $z \sim 3$ LBGs in the resonant \lya\ transition
and non-resonant \ion{Si}{2}$^*$ transitions \citep{prs+02,shapley03}.
A comprehensive analysis of the emission lines
related to galactic-scale outflows
(e.g.\ via deep integral-field-unit observations) may offer unique
diagnostics on the morphology and density of the outflow, eventually
setting tighter constraints on the energetics of the flow.  

[Phillips first reported?]
Bright \ion{Mg}{2} and \ion{Fe}{2}* line emission has also been
reported for individual galaxies at $z \sim 0.7$ by \cite{rubin+10c}
and Rubin et al.\ 2011 (in prep).
\cite{wcp+09}, who
studied \ion{Mg}{2} absorption in $z \sim 1.4$ galaxies, reported
\ion{Mg}{2} emission in a small subset of the individual galaxy
spectra of their large sample.  These were excluded from the full
analysis on concerns that the emission was related to AGN activity.
The stacked spectra of the remaining galaxies, however, also indicated
\ion{Mg}{2} emisison, both directly and when the authors modeled and
`removed' the $v \approx 0\mkms$ absorption component.  Although the
authors suggested the emission could be related to back-scattered
light in the wind, they also allowed that it could be related to weak
AGN activity.   Very similar \ion{Mg}{2} emission was observed by
\cite{rwk+10} who repeated the analysis of \cite{wcp+09} on a set
of lower redshift galaxies. 
[Did Kate see FeII* emission in her stacks? If not, why not?]

Although astronomers are rapidly producing a wealth of observational
datasets on galactic-scale winds, a key ingredient to a proper
analysis is absent.
[Draw analogy to exploring SN lightcurves and spectra]

In this paper, we take the first steps toward modeling the absorption
and emission properties of cool gas outflows.  Using Monte Carlo
radiative transfer techniques, we study the nature of \ion{Mg}{2} and
\ion{Fe}{2} absorption and emission for winds with a range of
properties.  Although the winds are idealized, the results frequently
contradict our intuition and 
challenge the straightforward conversion of observables to (even crude) physical
constraints.  [Add another line or two]

The paper is organized as follows.  In $\S$~\ref{sec:method}, we
describe the methodology of our radiative transfer algorithms.  These
are applied to a fiducial wind model in $\S$~\ref{sec:fiducial} and
variations of this model in $\S$~\ref{sec:variants}.  In
$\S$~\ref{sec:alternate}, we explore wind models with a broader range
of density and velocity laws.  Consideration of key observables is
given in $\S$~\ref{sec:obs} and we discuss the principal results
in $\S$~\ref{sec:discuss}.  A brief summary is given in
$\S$~\ref{sec:summary}.

\section{Methodology}
\label{sec:method}

This section describes our methodology for generating
emission/absorption profiles from simple wind models.

\subsection{The Radiative Transitions}

In this paper, we focus on two sets of radiative transitions
arising from Fe$^+$ and Mg$^+$ ions
(Table~\ref{tab:atomic}, Figure~\ref{fig:energy}).
This is a necessarily limited
set, but the two ions and their transitions do have characteristics
shared by the majority of low-ion transitions
observed in cool-gas outflows. Therefore, many
of the results that follow may be generalized to observational studies that
consider other atoms and ions of cool gas.

The Mg$^+$ ion, with a single 3s electron in the ground-state,
exhibits an alkali doublet of transitions at $\lambda \approx
2800$\AA\ analogous to the
\lya\ doublet of neutral hydrogen.  Figure~\ref{fig:energy}
presents the energy level diagram for this 
\mgiid\ doublet.  In non-relativistic quantum
mechanics, the 2p$^6$3p energy level is said to be split by spin-orbit
coupling giving the observed line doublet.  These are the only
\ion{Mg}{2} electric-dipole transitions 
with wavelengths near 2800\AA\ and the transition connecting
the $\rm {}^2P_{3/2}$ and $\rm {}^2P_{1/2}$ states is forbidden by several
selection rules.  Therefore, an absorption from
\maconfig~$\to$~\mbconfig\
is followed $\approx 100\%$ of the time by a spontaneous decay
($t_{\rm decay} \approx 4\sci{-9}$s) to the
ground state. Our treatment will ignore any other possibilities
(e.g.\ absorption by a second photon when the electron is at the \mbconfig\ level).

In terms of radiative transfer, the 
\mgiid\ doublet is very similar to that for \ion{H}{1}
\lya, the \naid\ doublet, and many other doublets commonly
studied in the interstellar medium (ISM) of distant galaxies.  
Each of these has the ground-state connected to a pair of electric
dipole transitions with nearly identical energy.
The doublets differ only in 
their rest wavelengths and the energy of the doublet separation. 
For \ion{H}{1} \lya, the
separation is sufficiently small ($\Delta v = c \Delta E / E \approx
1.3 \mkms$) that most radiative transfer treatments actually ignore it
is a doublet.
This is generally justifiable for \lya\ because 
most astrophysical processes have turbulent motions that
significantly exceed the doublet's velocity separation and effectively mix the
two transitions.  For \ion{Mg}{2} ($\Delta v \approx 770 \mkms$),  
\ion{Na}{1} ($\Delta v \approx 304 \mkms$), and most of the other doublets
commonly observed, the separation is large and the transitions
must be treated separately.  

Iron exhibits the most complex set of energy levels for elements
frequently studied in astrophysics.  The Fe$^+$ ion alone has over
10,000 energy levels recorded \citep{iron}, and even this is an
incomplete list.  
One reason for iron's complexity is
that the majority of its configurations exhibit fine-structure splitting.
This includes the ground-state configuration (\aconfig) which is split
into 5 levels, 
labeled by the total angular momentum $J$, 
with excitation energies $T_{\rm ex} \equiv \Delta E / k$ ranging from
$T_{\rm ex} \approx 500-1500$\,K (Figure~\ref{fig:energy}).  
Transitions between these fine-structure levels are 
forbidden (magnetic-dipole) and have spontaneous decay times of several hours.  

In this paper, we examine transitions between the ground-state
configuration and the energy levels of the \zconfig\
configuration.  This set of transitions (named the
UV1 multiplet) have wavelengths near 2600\AA.
There are two resonance-line transitions\footnote{We adopt the
  standard convention that a ``resonance line'' is an electric-dipole
  transition connected to the ground-state.  We also label
  non-resonant transitions with an asterisk, e.g.\ \feiic.} 
associated with this multiplet (\feiid)
corresponding to $\Delta J = 0, -1$; these are indicated by upward (black) arrows
in Figure~\ref{fig:energy}. The solid (green) downward
arrows in Figure~\ref{fig:energy} mark the non-resonant \feiis\
transitions that are connected to
the upper energy levels of the resonance lines.  These transitions may
occur following the absorption of a single photon by Fe$^+$ in its
ground-state.   This process may also be referred to as flourescence.
Note that two of these transitions (\feiis$\; \lambda\lambda 2626, 2632$) are
close enough in energy that their line profiles can overlap.

The Figure also shows (as dashed, downward arrows) two of the
\feiis\ transitions that connect to higher energy levels of the \zconfig\
configuration.  Ignoring collisions and recombinations, these
transitions may only occur after the absorption
of two photons: one to raise the electron from the ground-state to an
excited state and another to raise the electron from the excited state
to one of the \zconfig\ levels with $J \le 5/2$.  The excitation of
fine-structure levels by 
the absorption of UV photons is termed indirect UV pumping
\citep[e.g][]{silva02,pcb06} and requires the ion to lie
near an intense source of UV photons.  
Even a bright, star-forming galaxy emits too few photons at $\lambda
\approx 2500$\AA\ to UV-pump Fe$^+$ ions that are farther than $\sim
100$\,pc from the stars.
In the
following, we will assume that emission from this process is
negligible.

Our calculations also ignore collisional
processes\footnote{Recombination is also ignored.}, i.e.\ collisional
excitation and de-excitation of the various levels.  For the
fine-structure levels of the \aconfig\ configuration, the excitation
energies are modest ($T_{\rm ex} \sim 1000$\,K) but the critical
density $n_c$ is large.  For the \aconfig$_{9/2} \to
$\aconfig$_{7/2}$ transition, the critical density $n_e^C \approx 4
\sci{5} \cm{-3}$.  At these densities, one would predict  
detectable quantities of \ion{Fe}{1} which has not yet been observed
in galactic-scale outflows. Furthermore, observations rarely show
{\it absorption} from the
fine-structure levels of the \aconfig\ configuration and that material
is not significantly blue-shifted (Rubin et al., in prep). 
In the following, we assume that electrons only occupy the
ground-state, i.e.\ the gas has zero opacity to the non-resonant
lines.  If collisional excitation is insignificant 
then one may also neglect collisional de-excitation.  
Regarding the \ion{Mg}{2} doublet, its
excitation energy is significantly higher implying 
negligible collisional processes at essentially any density.

\subsection{The Source}

Nearly all of the absorption studies of galactic-scale outflows 
have focused on intensely, star-forming galaxies.  The
intrinsic emission of these galaxies is a complex combination of
stars and \ion{H}{2} regions that is then modulated by dust and gas
within the interstellar medium (ISM).  For the spectral regions studied
here, most stars show a featureless continuum but a few spectral
types do show significant \ion{Mg}{2} and \ion{Fe}{2} absorption.
{\bf [Kate: Any emission??]}  \ion{H}{2} regions, meanwhile, are observed to emit
at the \mgiid\ doublet, primarily due to recombinations in the outer
layers {\bf [Kate: How about FeII?]}.  It is beyond the scope of this paper to
properly model the 
stellar absorption and \ion{H}{2} region emission, but the reader
should be aware that they can complicate the observed spectrum,
independent of the outflow, especially at velocities $v \approx 0 \mkms$.
In the following, we assume a simple flat continuum 
normalized to unit value.  The size of the emitting
region $r_{\rm source}$ is a free parameter, but we restrict its value
to be smaller than the minimum radial extent of any gaseous
component.   Lastly, the source does not absorb any scattered or
emitted photons.


\subsection{Monte Carlo Algorithm}

[Describe 1D and 3D algorithms]

[Put tests in an Appendix?]

\subsection{Dust}
\label{sec:dust_method}

For the majority of models studied in this paper, we assume the gas
contains no dust.
This is an invalid assumption, especially for material associated with
the ISM of a galaxy.  
Extraplanar material likely associated with a galactic-scale 
outflow has been observed to emit IR radiation characteristic of dust \citep[e.g.][]{M87_dust}.
Essentially all astrophysical environments that contain both cool gas
and metals also show signatures of dust depletion and extinction.  This includes the
ISM of star-forming and \ion{H}{1}-selected galaxies
\citep[e.g.][]{ss96,pw01,pcd+07}, strong \ion{Mg}{2} metal-line
absorption systems \citep{ykv+06,mnt+08}, and the galactic winds traced
by low-ion transitions \citep{prs+02,rvs05b}.  Although the galactic winds
traced specifically by \ion{Mg}{2} and \ion{Fe}{2} transitions have not (yet) been
demonstrated to contain dust, it is reasonable to consider its
effects.
In addition, \cite{msf+10} have argued from a statistical analysis
that dust is distributed to many tens kpc from $z \sim 0.1$ galaxies
and have suggested it was transported from the galaxies by
galactic-scale winds.  One is motivated, therefore, to consider
models that include the effects of dust extinction.

For analysis on normalized spectra, the effects of dust are largely minimized; 
dust has a nearly constant opacity over small spectral regions and all
features are simply scaled together.
For scattered and resonantly trapped photons, however, the relative effect of dust
extinction can be much greater.  
These photons travel a much longer distance to escape the medium and
may experience a much higher integrated opacity from dust.
Indeed, dust is frequently invoked to
explain the weak (or absent) \lya\ emission from star-forming galaxies
\citep[e.g.][]{shapley03}.  Although the transitions studied here have
much lower opacity than \lya, dust could still play an important role
in the predicted profiles.

In a few models, we include absorption by dust under the following
assumptions:
(i) the dust opacity scales with the density of the gas (i.e.\ we
adopt a fixed dust-to-gas ratio);
(ii) the opacity is independent of wavelength, a reasonable
approximation given the small spectral range analyzed;
(iii) the photons absorbed by dust are re-emitted at IR wavelengths
and are `lost' from the system.  The dust absorption is normalized by \taud, 
the integrated opacity of dust from the center of the system to
infinity.  The ambient ISM of a star-forming galaxy may be expected to exhibit
\taud\ values of one to a few at $\lambda \sim 3000$\AA\ \citep[e.g.][]{dust_again}.

\subsection{The Sobolev Approximation}

For a wind with a steep velocity law (i.e.\ a large gradient $dv/dr$)
and/or a a narrow intrinsic profile (i.e.\ a small Doppler parameter),
\citet{sobolev60} introduced a formalism that defines the opacity of
the wind at a given radius $\tau(r)$ in terms of the density and
velocity gradient of the flow at that radius;

\begin{equation}
\tau^S(r) = \kappa^\ell \rho(r) \, \lambda^\ell \ltp \frac{dv}{dr} 
\rtp^{-1}  \cmma
\label{eqn:Sobolev}
\end{equation}
where $\kappa^\ell$ is the opacity of transition $\ell$.  This opacity
applies to a photon with wavelength $\lambda^S
\equiv \lambda^\ell (1-v(r)/c)$.
This is, of course, not the true opacity;
a photon with wavelength $\lambda^S$ that travels through the wind 
experiences a total opacity 
$\tau_\nu^\ell = \intl_0^\infty \rho(r) \kappa_\nu^\ell dr$.  The Sobolev
approximation is valid when the opacity the photon experiences is
domianted by gas in a localized portion of the flow.
We find that it applies to nearly all of the
models presented in this paper and therefore provides a convenient
approach to estimating $\tau^\ell_\nu$.



\section{The Fiducial Wind Model}
\label{sec:fiducial}

In this section, we study a simple yet illustrative wind model for
a galactic-scale outflow.  The properties of this wind were tuned, in
part, to yield a \ion{Mg}{2} absorption profile 
similar to those observed for $z \sim 1$, star-forming galaxies
\citep{wcp+09,rubin+10c}.  We emphasize, however, that we do not
favor this fiducial model over any other wind scenario nor do its
properties have special physical motivation.
Its role is to establish a baseline
for discussion.

The fiducial wind is isotropic, dust-free, and extends from an inner wind
radius $r_{\rm inner}$ to an outer wind radius $r_{\rm outer}$.  
It follows a (mass-flux conserving) density law,

\begin{equation}
n_{\rm H} (r) = \frac{n^0_{\rm H}}{r^2} \;\;\; , 
\label{eqn:density}
\end{equation}
and a velocity law with a purely radial flow

\begin{equation}
\vec v = v_r (r) \hat r = v_0 r \hat r  \perd
\label{eqn:vel}
\end{equation}
Turbulent motions are
characterized by a Doppler parameter\footnote{
  The adopted Doppler parameter
  ($\mbturb = 15 \mkms$) has a minor impact on the results.
  The absorption profiles are insensitive to its value and varying \bturb\
  only tends to modify the widths and modestly
  shift the centroids of emission lines.} 
, $b_{\rm turb}$.  
We convert the hydrogen density $n_{\rm H}$ to the number densities of
Mg$^+$ and Fe$^+$ ions by assuming solar relative abundances with an
absolute metallicity of 1/2 solar and depletion factors of 1/10 and
1/20 for Mg and Fe respectively, i.e.\  $\mnmg = 10^{-5.47} n_{\rm H}$ 
and \nfe=\nmg/2. At the center of the wind is a homogeneous source of
continuum photons with size $r_{\rm source}$. The parameters for the
fiducial wind model are summarized in Table~\ref{tab:fiducial}.   

In Figure~\ref{fig:fiducial_nvt} we plot the density and velocity
laws against radius;  
their simple power-law expressions are evident.  The figure also
shows the optical depth profile for the \mgiia\ transition
($\tau_{2796}$), estimated from the Sobolev approximation\footnote{We
  have verified this approxmiation holds by calculating 
  $\tau_{2796}$ first
  as a function of velocity by summing the opacity for a series of
  discrete and small radial intervals
  between $r_{\rm inner}$  and $r_{\rm outer}$.   We then mapped
  $\tau_{2796}$ onto radius using the velocity law
  (Equation~\ref{eqn:vel}). This optical depth profile is shown as the
black dotted line in Figure~\ref{fig:fiducial_nvt}.}
(Equation~\ref{eqn:Sobolev}). 

The $\tau_{2796}$ profile peaks with $\tau_{2796}^{\rm max} \approx 30$
at a velocity $\mvr \approx 65 \mkms$ corresponding to $r \approx 1.3
r_{\rm inner}$.  The optical depth profile for the \mgiib\ transition
(not plotted) is scaled down by the $f\lambda$ ratio but is otherwise identical.  Similarly,
the optical depth profiles for the \feiid\ transitions are
scaled down by $f \lambda$ and the \nfe/\nmg\ ratio.  

Because the results are dictated by frequency dependence of the
$\tau_{2796}$ profile, 
the actual dimensions and density of the wind are
essentially unimportant provided they scale together to give nearly the same
optical depth. Therefore, one may consider the choices for
$r_{\rm inner}, r_{\rm outer}, n_{\rm H}^0$ as arbitrary.
Nevertheless, we adopted values for this fiducial model with
some astrophysical motivation,  e.g., values that correspond to
galactic dimensions and a normalization that gives $\tau_{2796}^{\rm
  max} \sim 10$.


Using the methodology described in $\S$~\ref{sec:method}, we
propogated photons from the source and through the outflow to an
`observer' at $r \gg r_{\rm outer}$ that views the entire wind+source
complex.  Figure~\ref{fig:fiducial_1d} presents the 1D spectrum
that this observer would record, with the unattenuated flux
normalized to unit value.   The \ion{Mg}{2} doublet
shows the canonical `P-Cygni profile' that characterizes a continuum
source embedded within an outflow.  Strong absorption is evident at
$\delta v  \equiv (\lambda/\lambda_0 - 1) < -50 \mkms$ in both transitions (equivalent widths, $W^{2796} =
3.0$\AA\ and $W^{2803} = 1.3$\AA) and each shows emission at
positive velocities.  For an isotropic and dust-free model, the
total equivalent width of the doublet must be zero,
i.e.\ every photon
absorbed eventually escapes the system, typically at lower
energy.  The wind simply shuffles the photons in frequency space.
A simple summation of the absorption and emission equivalent widths
(Table~\ref{tab:fiducial_EW}) confirms this expectation.

Focusing further on the \ion{Mg}{2} absorption, one notes that the profiles lie
well above zero intensity and have similar depth even though their $f\lambda$
values differ by a factor of two.  In standard absorption line
analysis, this is 
the tell-tale signature of a `cloud' that has a high optical depth (i.e.\
saturated) which only partially covers the emitting source
\citep[e.g.][]{hammann+10}.  Our fiducial wind model, however, 
{\it entirely covers the source}; the apparent partial covering must
be related to a different effect.
Figure~\ref{fig:noemiss} further emphasizes this point by comparing the 
absorption profiles from Figure~\ref{fig:fiducial_1d} against an
artificial model where no absorbed photons are 
re-emitted.   As expected from the
$\tau_{2796}$ profile (Figure~\ref{fig:fiducial_nvt}), this
`intrinsic' model
produces a strong \mgiid\ doublet that absorbs all photons at
$\delta v \approx -100 \mkms$, i.e.\ $I_{2796}^{\rm min} = \exp(-\tau^{\rm
  max}_{2796}) \approx 0$.
The true model, in contrast, has been `filled in' at $\delta v \approx -100
\mkms$ by photons scattered in the wind.  An
absorption-line analysis that ignores these effects
would (i) systematically underestimate the true optical
depth and (ii) falsely conclude that the wind partially covers the
source.  We will find that this
[behaviour] is a generic result, even for wind models that include
dust and are not fully isotropic.

Turning to the emission profiles of the \mgiid\ doublet, one notes
that they are also similar with comparable equivalent widths The
flux of the \mgiib\ transition even exceeds that for \mgiia\ giving a
line ratio that is far below the $2:1$ ratio that one may have naively
expected. 
[Does recombination predict 2:1??]
The emission profiles are very similar because the gas is optically
thick yielding comparable total absorption. 
Furthermore, the flux from \mgiib\ actually exceeds the
\mgiia\ emission because the wind speed is greater than the velocity separation
of the doublet, $|\mvr|_{\rm max} > (\Delta v)_{\rm MgII}$.
Therefore, the red wing of the
\mgiia\ emission profile is partially absorbed by \mgiib\ and
re-emitted at lower frequency.  We conclude that the relative
strengths of the emission lines is sensitive to both the opacity and
velocity extent of the wind.   

Now consider the \ion{Fe}{2} transitions:
the bottom left panel of Figure~\ref{fig:fiducial_1d} covers the
majority of the \ion{Fe}{2} UV1 transitions and several are
shown in the velocity plot.  The line
profile for \feiib\ is very similar to the \mgiid\ doublet;
one observes strong absorption to negative velocities and strong
emission at $\mdv > 0 \mkms$ producing a characteristic P-Cygni profile. 
Splitting
the profile at $\mdv = -50 \mkms$, we measure an equivalent width
$W^{2600}_{\rm a} = 1.16$\AA\ in absorption and $W^{2600}_{\rm e} =
-0.94$\AA\ in emission (Table~\ref{tab:fiducial_EW}) for a total
equivalent width of $W^{2600}_{\rm TOT} \approx 0.22$\AA.  
In contrast, the \feiia\ resonance line shows much weaker emission and
a much higher total equivalent width ($W^{2586}_{\rm TOT} = 0.49$\AA),
even though the line has a $2 \times$ lower $f\lambda$ value.
These differences between the \ion{Fe}{2} resonance lines (and with the
\ion{Mg}{2} doublet) occur because of the complex of non-resonant
\feiis\ transitions that are coupled to the resonance lines
(Figure~\ref{fig:energy}).  Specifically, 
resonance photons absorbed at \feiid\ have a finite probability of
being re-emitted as a non-resonant photon which then escapes the system
without further interaction.  The principal effects are to reduce the
line emission of \feiid\ and to produce non-resonant line-emission (e.g.\
\feiic).

The reduced \feiia\ emission relative to \feiib\ is related to
two factors:
(i) there is an additional downward transition from the
\zconfig$_{7/2}$ level and 
(ii) the Einstein A
coefficients of the non-resonant lines coupled to \feiia\ are comparable to and even
exceed the Einstein A coefficient of
the resonant transition.  In contrast, 
the \feiie\ transition (associated with \feiib)
has an approximately  $4\times$ smaller A coefficient than the
corresponding
resonance line.  Therefore, the majority of photons absorbed at
$\lambda \approx 2600$\AA\ are re-emitted as \feiib\ photons whereas 
the majority of photons absorbed at $\lambda \approx 2586$\AA\ are re-emitted 
at longer wavelengths (\feiic\ or $\lambda 2632$).
If we were to increase $\tau_{2600}$ (and especially if we include
gas with $\mvr \approx 0 \mkms$) then the \feiib\ emission is
significantly suppressed (e.g.\ $\S$~\ref{sec:ISM}).
The total equivalent width, however, of the three lines connected to the
\zconfig$_{7/2}$ upper level must still vanish (photons are conserved
in this isotropic, dust-free model).


The preceding discussion emphasizes the filling-in of resonance
absorption at $\mdv
\lesssim -50 \mkms$ and the generation of emission lines at $\mdv \approx
0 \mkms$ by photons scattered in the wind.  To study the spatial
extent of this emission, we mapped the emission of the fiducial
model (see $\S$~\ref{sec:monte} for a description of the algorithm).
The output is a set of surface-brightness maps in a series of 
frequency channels yielding a
dataset analogous to integral-field-unit (IFU) observations.  In
Figure~\ref{fig:fiducial_ifu_mgii}, we present the output 
at several velocities relative to the \mgiia\
transition. At $\mdv = -250 \mkms$, where the wind has an optical
depth $\tau_{2796} < 1$ (Figure~\ref{fig:fiducial_nvt}),
the source contributes roughly half of the observed flux.  
At $\mdv=-100 \mkms$, however, the
wind absorbs all photons from the source and the observed emission is
entirely from photons scattered by the wind.  This scattered emission
actually exceeds the source+wind emission at 
$\mdv = -250 \mkms$ such that the absorption profile is
negatively offset from the velocity where $\tau_{2796}$ is maximal
(Figure~\ref{fig:fiducial_nvt}).
The net result is
weaker \ion{Mg}{2} absorption that peaks blueward of the actual peak in the
optical depth profile.  
Table~\ref{tab:line_diag} reports several kinematic measurements of
the absorption and emission features.
Clearly, these effects complicate estimates for the
speed, covering fraction, and total column density of the wind.  At
$\mdv = 0 \mkms$, the wind and source have comparable total flux with the
latter dominating at higher velocities.  

Similar results are observed for the \ion{Fe}{2} resonance
transitions (Figure~\ref{fig:fiducial_ifu_feii}).
For transitions to fine-structure levels of the \aconfig, the source
is unattenuated but there is a significant contribution from photons
generated in the wind. 
The results presented in Figures~\ref{fig:fiducial_ifu_mgii} and
\ref{fig:fiducial_ifu_feii} are sensitive to the radial extent,
morphology, density and velocity profiles of this galactic-scale
wind.  Consequently, IFU observations of line emission from low-ion
transitions may offer the most direct constraints on galactic-scale
wind properties. 

At all velocities, the majority of light comes from the inner regions
of the wind. 
The majority of
\ion{Mg}{2} emission occurs within the inner few kpc, e.g.\ $50-60\%$
of the light at $\mdv=-100$ to $+100 \mkms$
comes from $|r| < 3$\,kpc.
The emission
is even more centrally concentrated for the \ion{Fe}{2} transitions.
A proper treatment of these
distributions is critical to interpret observations
acquired through a slit, i.e.\ where the aperture has a limited extent
in one or more dimensions.  A standard longslit on 10m-class
telescopes, for example, subtends $\approx 1''$ corresponding to
$5-10$\,kpc for $z \sim 1$.    We return to this issue in
$\S$~\ref{sec:discuss}. 


%%%%%%%%%%%%%%%%%%%%%%%%%%%%%%%%%%%%%%%%%%%%%%%%%%%%%%%%%%
\section{Variations to the Fiducial Model}
\label{sec:variants}

In this section, we investigate a series of more complex wind
scenarios
through modifications to the fiducial model.  These include relaxing
the assumption of isotropy, introducing dust, adding an ISM
component within $r_{\rm inner}$, and varying the normalization of the
optical depth profiles.

\subsection{Anisotropic Winds}
\label{sec:anisotropic}

The fiducial model assumes an
isotropic wind with only radial variations in velocity and density. 
Angular isotropy is obviously an idealized case, but
it is frequently assumed in studies of galactic-scale outflows
\citep[e.g.][]{steidel+10}.   There are several reasons, however, to
consider anisotropic winds.  Firstly, galaxies are not spherically
symmetric;  the sources driving the
wind (e.g.\ supernovae, AGN) are very unlikely to be isotropically distributed
within the galaxy.  
Secondly, the galactic ISM frequently has a disk-like morphology
which will suppress the wind preferentially at low galactic latitudes,
perhaps yielding a bi-conic morphology \citep[e.g.][]{M87}.
Lastly, the galaxy could be surrounded by an
anisotropic gaseous halo whose interaction would produce an irregular 
outflow.

With these considerations in mind, we reanalyzed the fiducial model
with the 3D algorithm after departing from isotropy.  It is beyond the
scope of this paper to explore a full suite of anisotropic profiles.
We consider two simple examples: (i) half the fiducial model, where 
the density is set to zero for $2\pi$ steradians.  This model is
viewed from $\phi = 0^\circ$ (source uncovered) 
to $\phi = 180^\circ$ (source covered); and 
(ii) a bi-conical wind where the density is set to zero for angles
$|\theta| < \theta_b$ that and is viewed along the plane of symmetry.

The resulting \ion{Mg}{2} and
\ion{Fe}{2} profiles for the half wind are compared against the fiducial model
(isotropic wind) in Figure~\ref{fig:anisotropic}.  
Examining the \mgiid\ doublet, 
the $\phi = 0^\circ$ model only shows
line-emission from photons scattered
off the back side.  These photons, by definition, have $\mdv \gtrsim 0 \mkms$
relative to line-center (a subset have $\mdv \lesssim 0 \mkms$ because
of turbulent motions in the wind). 
When viewed from the opposite direction ($\phi = 180^\circ$), the
absorption lines dominate but there is still significant
line-emission, at $\mdv \approx 0 \mkms$ and at $\mdv < 0 \mkms$ which fills
in the absorption, from photons that scatter through the wind.  The
key difference from the isotropic wind is the absence of photons
scattered to $\mdv > 100 \mkms$;  this also implies deeper 
\mgiib\ absorption at $\mdv \approx -100 \mkms$. The 
shift in velocity centroid and asymmetry of the emission lines
serve to diagnose the degree of wind isotropy, especially in
conjunction with analysis of the absorption profiles. 
The results are similar for the \feiid\ resonance lines.  The
\ion{Fe}{2}$^* \; \lambda 2612$ line, meanwhile, shows most clearly the
offset in velocity between the source unobscured ($\phi = 0^\circ$)
and source covered ($\phi = 180^\circ$) cases.  The offset of the
\feiis\ lines is the most significant 
difference from the fully isotropic wind.

[Add a paragraph on the bi-conical wind]

%%%%%%%%%%%%%%%%%%%%%%%%%%%%%
\subsection{Dust}
\label{sec:dust}

As described in $\S$~\ref{sec:dust_method}, 
one generally expects dust in astrophysical environments that contain
cool gas and metals.  This dust 
modifies the observed wind profiles in two manners. 
First, it is a source of opacity for all of 
the photons.  This suppresses the flux at all
wavelengths by $\approx \exp(-\mtaud)$ but because we re-normalize the
profiles, this effect is essentially ignored.  Second, photons that are
scattered by the wind must travel a greater
distance to escape and therefore suffer from greater extinction.  A photon that is
trapped for many scatterings may have a very high probability of being absorbed
by dust.  
Section~\ref{sec:dust_method} describes the details on our treatment of dust; we 
remind the reader here that we assume a constant dust-to-gas ratio 
that is normalized by the total optical
depth \taud\ a photon would experience if it traveled from the
source to infinity without scattering. 

In Figure~\ref{fig:dust}, we show the \ion{Mg}{2} and \ion{Fe}{2}
profiles of the fiducial model ($\mtaud = 0$) against a series of
models with $\mtaud > 0$.  For the \ion{Mg}{2} transitions, the
dominant effect is the suppression of line emission at $\mdv \ge 0
\mkms$.  These `red' photons have scattered off the
backside of the wind and must travel a longer path than other
photons.  Dust leads to a differential reddening that increases with 
velocity relative to line-center. This is a natural consequence of dust
extinction and is most evident in the \feiic\ 
emission profile which is symmetrically distributed around
$\mdv= 0 \mkms$ in the $\mtaud=0$ model.   
The degree of suppression of the line-emission is relatively modest,
however.  Specifically, we find that the flux is reduced by a factor 
of the order of (1+\taud)$^{-1}$ (Figure~\ref{fig:dust_tau},
instead of the factor
exp(--\taud) that one may have naively predicted. 
Although the \ion{Mg}{2} photons are resonantly
In terms of absorption, the profiles are
nearly identical for $\mtaud \le 3$.  One requires very high
extinction to eventually produce a deepening of the profiles at 
$\mdv \approx -100 \mkms$.

% The following could go in the Discussion section
We conclude that dust has only a modest influence on this fiducial model and,
by inference, models with moderate peak optical depth and
significant velocity gradients with radius (i.e.\ scenarios where the
photons scatter one to a few times before exiting).
For qualitative changes, one requires an extreme level of
extinction ($\mtaud = 10$).  In this case, the source would be
extinguished by 15\,magnitudes and could never be observed. 
Even $\mtaud = 3$ is larger than typically inferred for the
star-forming galaxies that drive outflows \citep[e.g.][]{dust}.
For the emission lines,
the dominant effect is a reduction in the flux 
with a greater extinction at higher velocities relative to line-center.
In these respects, dust extinction crudely mimics the behavior of the
anisotropic
wind described in Section~\ref{sec:anisotropic}. [Is there a
distinguishing factor, e.g. flux of 2612/2600??]


\subsection{ISM}
\label{sec:ISM}

The fiducial model does not include gas associated with
the interstellar medium of the galaxy, i.e.\ material at $r \sim
0$\,kpc with $v_{\rm r} \approx 0 \mkms$.  This allowed us to focus on
results related solely to a wind component.  The decision to ignore the ISM
was also motivated by the general absence of significant absorption at
$\mdv \approx 0 \mkms$ in galaxies that exhibit outflows 
\citep[e.g.][]{wcp+09,rwk+10,steidel+10}.
On the other hand, the stars that comprise the
source are very likely embedded within and fueled by gas
of the ISM.  
Consider, then, a modification to the fiducial model that 
includes an ISM. 
Specifically, we assume the ISM component has density $n_{\rm H} = 1 \cm{-3}$ for
$r_{\rm ISM} \le r < r_{\rm inner}$ with $r_{\rm ISM} = 0.5$\,kpc, 
an average velocity of $\mvr = 0 \mkms$, and a larger turbulent velocity $b_{\rm ISM} = 40 \mkms$.
%Figure~\ref{fig:ISM} summarizes the model.
The resultant optical depth profile $\tau_{2796}$ is identical to the
fiducial model for $r > 2$\,kpc, has a slightly higher opacity at
$r=1-2$\,kpc, and has a very large opacity at $r = 0.5-1$\,kpc.

In Figure~\ref{fig:ISM_spec}, the solid curves show the \ion{Mg}{2} and
\ion{Fe}{2} profiles for the ISM+wind and fiducial 
models. In comparison, the
dotted curve shows the intrinsic absorption profile for the ISM+wind
model corresponding to the (unphysical) case
where none of the absorbed photons are scattered or re-emitted.   Focus first on the
\ion{Mg}{2} doublet.  As expected, the dotted curve shows strong
absorption at $\mdv \approx 0 \mkms$ and blueward.  The full models,
in contrast, show non-zero flux at these velocities and even a
normalized flux exceeding unity at $\mdv \approx 0 \mkms$.  In
fact, the ISM+wind model is nearly identical to the fiducial model;
the only quantitative difference is that the velocity centroid of
the emission lines are shifted redward by $\approx +100 \mkms$.
We have also examined the spatial emission of this model and note
qualitatively similar results to the fiducial wind model
(Figures~\ref{fig:fiducial_ifu_mgii} and \ref{fig:fiducial_ifu_feii}).

There are, however, several qualitative differences 
for the \ion{Fe}{2} transitions. 
First, the \feiia\ transition in the ISM+wind model
shows much stronger absorption at $\mdv \approx 0$
to $-100 \mkms$.  In contrast to the \ion{Mg}{2} doublet,
the profile is not filled in by scattered photons. Instead, 
the majority of \feiia\ photons that are absorbed are re-emitted as
\feiis$\lambda\lambda 2612, 2632$ photons.  In fact, the
\feiia\ profile very nearly matches the profile without re-emission 
(compare to the dotted lines); this transition provides a
very good description of the intrinsic ISM+wind optical depth profile.  
We conclude that resonant transitions that are coupled to (multiple)
non-resonant, electric dipole transitions offer the best
diagnosis of ISM absorption.

The differences in the \ion{Fe}{2} absorption profiles are reflected
in the much higher strengths ($3-10\times$) of emission from
transitions to the excited states of \aconfig.   This occurs because:
(1) there is greater absorption by the \feiid\
resonance lines; (2) the high opacity of the ISM component leads to
an enhanced conversion of resonance photons with $\mdv \approx 0\mkms$
into \feiis\ photons.  This is especially notable for the
\feiib\ transition whose coupled \feiis\ transition shows an equivalent width nearly
$10\times$ stronger than for the fiducial model.  The relative
strengths of the \feiib\ and \feiie\ lines provide a direct
diagnostic on the degree to which the resonance lines are trapped.
[Comment:  this is a product of $\tau_{2600}$ and the velocity
gradient]

Because of the high degree of photon trapping within the ISM
component, this model does suffer more from dust extinction than the
fiducial model.  We have studied the ISM+wind model including dust
with $\mtaud=1$ (normalized to include the ISM gas).  All of the 
emission lines are significantly reduced.  The \ion{Mg}{2} emission is
affect most because these lines are resonantly trapped.  The \feiis\
emission is also reduced relative to the dust-free model, but the
absolute flux still exceeds the fiducial model
(Table~\ref{tab:line_diag}).
[Should I show a figure?] 

\subsection{Varying $n_{\rm H}^0$}

The final modification to the fiducial model considered was to
uniformly vary the normalization of the optical depth profiles.
Specifically, we ran a series of additional models with $n_{\rm H}^0$ at
1/3, 3, and 10 times the fiducial value of $n_{\rm H}^{0,f} = 0.1 \cm{-3}$.
The resulting \ion{Mg}{2} and \ion{Fe}{2} profiles are compared
against the fiducial model in Figure~\ref{fig:norm}.  Inspecting
\ion{Mg}{2}, one notes that the models with 1/3 and $3 n_{\rm H}^{0,f}$
behave as expected.  Higher/lower optical depths lead to
greater/weaker \mgiia\ absorption and stronger/weaker \mgiib\ emission
In the extreme case of
$n_{\rm H}^0 = 10 n_{\rm H}^{0,f}$, the \mgiib\ absorption and \mgiia\
emission have nearly disappeared and one primarily observes very
strong \mgiia\ absorption and \mgiib\ emission.
In essence, this wind has converted all of the photons absorbed by the
\ion{Mg}{2} doublet into \mgiib\ emission.

Similar behaviour is observed for the \ion{Fe}{2} transitions.
Interestingly, the \feiia\ transiions never show significant emission,
only stronger absorption with increasing $n_{\rm H}^0$.  In fact, for $n_{\rm H}^0
> 2 n_{\rm H}^{0,f}$ the peak optical depth of \feiia\ actually exceeds that
for \feiib\ because the latter remains significantly filled-in by
scattered photons.  By the same token, the flux of the \feiic\
emission increases with \nhn.

\subsection{Summary Table}

Table~\ref{tab:line_diag} presents a series of quantitative measures
for the \ion{Mg}{2} and \ion{Fe}{2} absorption and emission lines for
the fiducial model ($\S$~\ref{sec:fiducial}) and a subset of the models
presented in this section.  Listed are the absorption and emission
equivalent widths ($W_{\rm a}, W_{\rm e}$), the peak optical depth
for the absorption $\tau_{\rm pk}$, the
velocity where the optical depth peaks $v_\tau$, the optical
depth-weighted velocity centroid $v_{\bar \tau} \equiv \int dv \, v
\ln[I(v)] / \int dv \ln[I(v)]$, the peak flux $f_{\rm pk}$ in
emission, the velocity where the flux peaks $v_f$, and the
flux-weighted velocity centroid of the emission line $v_{\bar f}$.
We discuss several of these measures in $\S$~\ref{sec:discuss}.


%%%%%%%%%%%%%%%%%%%%%%%%%%%%%%%%%%%%%%%%%%%%%%%%%%%%%%%%%%%%%%%%%
\section{Alternate Wind Models}
\label{sec:alternate}

This section presents several additional wind scenarios that differ
significantly from the fiducial model explored in the previous
sections.  In each case, we maintain the simple assumption of
isotropy.

\subsection{The Lyman Break Galaxy Model}
\label{sec:lbg}

The Lyman break galaxies (LBGs), UV color-selected galaxies at $z \sim 3$,
exhibit cool gas outflows in \ion{Si}{2}, \ion{C}{2},
etc.\ transitions with speeds up to 1000\,\kms\
\citep[e.g.][]{lkg+97,pks+98}.
Researchers have invoked these winds to explain enrichment of
the intergalactic medium \citep[e.g.][]{ahs+01,spa+06}, the origin of the
damped \lya\ systems \citep{nbf98,schaye01a}, and the formation of
`red and dead' galaxies \citep[e.g.][]{redgal}.  Although the
presence of these outflows were established over a decade ago,
the processes that drive them remain
unidentified.  Similarly,  current estimates of the mass and energetics of the
outflow suffer from orders of magnitude uncertainty.

Recently, \citet[][; hereafter S10]{steidel+10} introduced a model to
explain jointly the average absorption they observed
along the sightlines to several hundred LBGs and the average absorption in gas
observed transverse to these galaxies.  
Their wind model is defined by two
expressions: (i) a radial velocity law $\mvr(r)$ and (ii) the covering
fraction of optically thick `clouds' $f_c(r)$.  For the latter, S10
envision an 
ensemble of small, optically thick $(\tau \gg 1)$ clouds that only
partially cover the galaxies.
For the velocity law, they adopted the following functional
form

\begin{equation}
\mvr = \ltp \frac{A_{\rm LBG}}{1-\alpha} \rtp^{1/2} \ltk r_{\rm
  inner}^{1-\alpha} - r^{1-\alpha} \rtk^{1/2}
\label{eqn:LBG_vlaw}
\end{equation}
with $A_{\rm LBG}$ the constant that sets the terminal speed,
$r_{\rm inner}$ the inner radius of the wind (taken to be 1\,kpc), and
$\alpha$ the power-law exponent that describes how steeply the velocity curve rises.  Their
analysis of the LBG absorption profiles implied
a very steeply rising curve with $\alpha \approx 1.3$.
This velocity expression is shown as a dotted line in 
Figure~\ref{fig:LBG_Sobolev}a.  

The covering fraction of optically thick clouds, meanwhile, was assumed to have
the functional form

\begin{equation}
f_c(r) = f_{c,max} \ltp \frac{r}{r_{\rm inner}} \rtp^{-\gamma} \cmma
\label{eqn:covering}
\end{equation}
with $\gamma \approx 0.5$ and $f_{c,max}$ the maximum covering
fraction.  From this expression and the velocity law, one can recover
an absorption profile $I_{\rm LBG}(v) = 1 - f_c[r(v)]$, written
explicitly as

\begin{equation}
I_{\rm LBG}(v) = 1 - f_{c,max} \ltk r_{\rm inner}^{1-\alpha} - \ltp
\frac{1-\alpha}{A_{\rm LBG}} \rtp v^2 \rtk^{\gamma/(\alpha-1)}
\perd
\label{eqn:LBG_I}
\end{equation}
The resulting profile for $f_{c,max} = 0.6$, $\gamma=0.5$,
$\alpha=1.3$, and $A_{\rm LBG} = -192,000 \, \rm km^2 \, s^{-2} \, kpc^{-2}$ 
is displayed in Figure~\ref{fig:LBG_Sobolev}b.  

In the following, we consider two methods to analyze the LBG wind.
Both approaches assume an isotropic wind and adopt the velocity law given by
Equation~\ref{eqn:LBG_vlaw}.  In one model, we treat the cool gas as a
diffuse medium with unit covering fraction and a radial density
profile determined from the Sobolev approximation.  We then apply 
the Monte Carlo methodology used for the other wind models to predict
\ion{Mg}{2} and \ion{Fe}{2} line profiles.  In the other LBG model,
we modify our algorithms to more precisely mimic the concept of an
ensemble of optically thick clouds with a partial covering fraction on
galactic scales.

%%%%%%%%%%%%%%%%%%%%%%%%%%%%%%%%%%%%%%%%%%%%%%%%%%
\subsubsection{Sobolev approximation}
\label{sec:Sobolev}

As demonstrated in Figure~\ref{fig:LBG_Sobolev}, the wind velocity for
the LBG model rises very steeply with increasing radius before
flattening at large radii.  Under these conditions, the Sobolev
approximation provies an accurate relation between the optical depth
of the flowing medium to the
density and velocity gradient ($d\mvr/dr$) of the gas.  In our case, we 
generate the optical depth profile in velocity space using Equation~\ref{eqn:LBG_I}, 

\begin{equation}
\tau_{\rm LBG}(v) = -\ln \ltk I_{\rm LBG}(v) \rtk \perd
\label{eqn:tauLBG}
\end{equation}
We then invert the Sobolev approximation (Equation~\ref{eqn:Sobolev})
to derive the radial density profile:

\begin{equation}
n_{\rm LBG}(r) = \frac{\tau_{\rm LBG}(r) \; d\mvr/dr}{\kappa^\ell
  \lambda^\ell} \cmma
\label{eqn:nSobolev}
\end{equation}
with $\lambda^\ell$ the rest wavelength and $\kappa^\ell \equiv f^\ell
\pi^2 e^2/m_e c$.  

The solid curve in Figure~\ref{fig:LBG_Sobolev}a shows the
resultant density profile for Mg$^+$ assuming that the \mgiia\ line
follows the intensity profile drawn in Figure~\ref{fig:LBG_Sobolev}b. 
This is a relatively extreme density profile.  From the inner radius
of 1\,kpc to 2\,kpc, the density drops by over 2 orders of
magnitude including nearly one order of magnitude over the first
10\,pc.  Beyond 2\,kpc, the density drops even more rapidly, falling
orders of magnitude from 2 to 100\,kpc.

We verified that the Sobolev-derived density profile shown in
Figure~\ref{fig:LBG_Sobolev}a
reproduces the proper
absorption profile by discretizing the gas into a series of layers
and calculating the integrated absorption profile.  This
calculation is shown as a dotted red curve in
Figure~\ref{fig:LBG_Sobolev}b; it is an excellent
match to the desired profile (black curve).
To calculate the optical depth profiles for the other transitions, we
assume $\mnfe = \mnmg/2$ and scale $\tau$ by the $f\lambda$ product.

We generated \ion{Mg}{2} and \ion{Fe}{2} profiles for this wind
using the 1D algorithm with no dust extinction; these are shown as
red curves in Figure~\ref{fig:LBG_spec}.   For this analysis, one
should focus on the \mgiia\ transition.  The dotted line in the Figure
shows the intensity profile when one ignores re-emission
of absorbed photons.  By construction, it is the same profile\footnote{Note that
  one should not make this comparison for the other transitions in
  the Sobolev model because those are scaled down by $f\lambda$ and
  for \ion{Fe}{2} the reduced Fe$^+$ abundance.} 
described by Equation~\ref{eqn:LBG_I} and plotted in
Figure~\ref{fig:LBG_Sobolev}b.   In comparison, the full model (solid,
red curve)
shows much weaker absorption, especially at $v = 0$ to $-300 \mkms$
because scattered photons fill in the absorption profile.
In this respect, our LBG-Sobolev model is an
inaccurate description of the observations. The model
also predicts significant emission in the \ion{Mg}{2} lines and several of
the \ion{Fe}{2}$^*$ transitions.   Emission associated with cool gas
has been observed for \ion{Si}{2}$^*$
transitions in LBGs \citep{prs+02,shapley03}, but
the \ion{Fe}{2}$^*$ transitions
modeled here lie in the near-IR and have not yet been investigated.
On the other hand, there have been no 
reported detections of significant line-emission related to low-ion resonance
transitions (e.g.\ \ion {Mg}{2}) in LBGs, only $z \le 1$ star-forming galaxies
\citep{wcp+09,rwk+10}.  
The principal result of the LBG-Sobolev model is that the scattering and re-emission of
absorbed photons significantly alters the predicted absorption profiles
for the inputted model.  This is, of course, an unavoidable
consequence of an isotropic, dust-free model with unit covering
fraction.

Because we have explicit velocity and densities profiles for the LBG-Sobolev
model, it is straightforward to calculate the radial
distributions of mass, energy, and momentum of
the wind.  These are shown in cumulative form in
Figure~\ref{fig:LBG_cumul}.  Before discussing the results, we offer
two cautionary comments: (i) the conversion of \nmg\ to $n_{\rm H}$
assumes a very poorly constrained scalar factor of $10^{5.47}$.  One
should give minimal weight to the absolute values for any of the
quantities;
(ii) the Sobolev approximation is not a proper description of the S10
LBG wind model (see the following sub-section).
The primary result expressed by Figure~\ref{fig:LBG_cumul} is that the
majority of energy, mass, and momentum in the wind is transported by
the outer layers ($r > 20$\,kpc).  This is surprising given that the
density is $\approx 5$~orders of magnitude lower at these radii than
at $r = 1$\,pc.  


%%%%%%%%%%%%%%%%%%%%%%%%%%%%%%%%%%%%%%%%%%%%%%%%%%
\subsubsection{Partial Covering Fraction}
\label{sec:Covering}

In the previous sub-section, we described a Sobolev solution that
reproduces the average absorption profile of LBGs in cool gas
transitions when scattered photons are ignored.  A proper analysis
that includes scattered photons,
however, predicts line-profiles that are qualitatively
different from the observations because scattered photons fill-in
absorption and generate significant line-emission (similar to the
fiducial wind model; $\S$~\ref{sec:fiducial}).
This LBG-Sobolev model, however, differs from
the one proposed by S10:  those authors proposed an ensemble of optically
thick clouds with a partial covering fraction described by
Equation~\ref{eqn:covering}.  In contrast, the Sobolev model assumes
a diffuse medium with a declining density profile but a unit covering
fraction.  At face value, one may question whether this difference
leads to the contradictory results of the LBG-Sobolev model. 

To more properly model the LBG wind described in S10, we 
performed the following Monte Carlo calculation.  First, we propogate
a photon from the source until its velocity relative to line-center
resonates with the wind (the photon escapes if this never occurs).
The photon then has a probability $P = f_c(r)$ of scattering.  If it
scatters, we track the photon until it comes into
resonance again\footnote{Because this wind has a monotonically
  increasing velocity law, the photon may only interact again if a
  longer wavelength transition is available, e.g.\ \mgiib\ for a
  scattered \mgiia\ photon.} or escapes the system.  In this model, all of the
resonance transitions are assumed to have identical (infinite) optical
depth at line center. 

The results of the full calculation (absorption plus scattering) are
shown as the black curve in Figure~\ref{fig:LBG_spec}.  Remarkably, the results
are very similar to the LBG-Sobolev calculation; scattered photons fill-in
the absorption profiles at $v \approx 0 \mkms$ and yield significant
emission lines at $v \gtrsim 0 \mkms$ in the resonance
lines. Furthermore, significant emission is observed 
for the \feiis\ transitions, centered at $\mdv \approx 0 \mkms$.  
In contrast to the \ion{Mg}{2} profiles, the predicted \ion{Fe}{2}
lines (especially \feiia) much more closely resemble the intrinsic
profiles (dotted curves).  This occurs because many of these
absorbed photons flouresce into \feiis\ emission.
The equivalent width of \feiia\ even exceeds that for \feiib, an
inversion that, if observed, would strongly support this model.

A robust conclusion of our
analysis is that these simple models cannot reproduce the
observed profiles of resonantly trapped lines like the \ion{Mg}{2}
doublet because of the emission from scattered photons.  In order to
achieve a significant opacity at $\mdv \approx 0 \mkms$, one must
either suppress the line-emission (e.g.\ with severe dust extinction) 
or include a substantial ISM component that absorbs light at $\mdv
\gtrsim 0 \mkms$ (e.g.\ $\S$~\ref{sec:ISM}).  As we discuss in the
following section, it is difficult (if not impossible) to fully
suppress the line-emission,  especially for scenarios that invoke
partial covering.  We conclude, therefore, that the LBG model
introduced by S10 is not a valid description of the data.

Lastly, we caution that resonant line-emission observed for LBGs
\citep[e.g.\ \ion{C}{4};][]{prs+02} may result from the galactic-scale outflow, not
only the expected stellar winds of massive stars.
A proper treatment of these radiative transfer effects is required to
quantitatively analyze these features.

%%%%%%%%%%%%%%%%%%%%%%%
\subsection{Radiation Pressure}
\label{sec:radiative}

[Kate: Write the details of velocity + density law]

\begin{equation}
\mvr(r) = 2\sigma \sqrt{R_g \ltp \frac{1}{R_0} - \frac{1}{r} \rtp
   + \ln\ltp R_0/r \rtp }
\end{equation}

\begin{equation}
n(r) = \frac{dM_{\rm wind}/dt}{r^2 v(r)}
\end{equation}
Similar to the fiducial wind model, this wind has a decreasing density
law with $n_{\rm H}$ roughly proportional to $r^{-2}$. 
To produce a wind whose \mgiia\ optical depth profile peaks at 
$\tau_{2796}^{\rm max} \gtrsim 10$, we set $R_g = r_{\rm inner} =
1\,\rm kpc$, $R_0=20\,\rm kpc$, and $\sigma = 125 \mkms$.  The density
and velocity peak at $r=r_{\rm inner}$ where $n_{\rm H}^0 = 0.07
\cm{-3}$ and $\mvr = 380 \mkms$.  This corresponds to a mass flow of
$dM_{\rm wind}/dt = 0.63 \msun\; \rm yr^{-1}$.  

Figure~\ref{fig:rad_spec} presents the \ion{Mg}{2} and \ion{Fe}{2}
profiles for this radiation-driven wind model compared against the
fiducial wind model.
In contrast to the fiducial model, the
velocity of the radiation-driven model decreases with increasing
radius such that the optical depth peaks at large velocity (here $\mdv
\approx -400 \mkms$).  At these velocities, the absorption is not
filled-in by scattered photons such that one recovers absorption lines
that very nearly match the intrinsic optical depth profile (dotted
curve).  Although the \ion{Mg}{2} absorption profiles are very
different from the fiducial model, the \ion{Mg}{2} line-emission has
similar peak flux and velocity centroids (see also
Table~\ref{tab:plaw_diag}).  The \feiis\ emisison also has a simiar
equivalent width, yet the lines of the radiation-driven model are much
broader. This reflects the fact that the peak optical
depth occurs at $\mdv \approx -400 \mkms$.
[Last point]


%%%%%%%%%%%%%%%%%%%%
\subsection{Power-Law Models}
\label{sec:power}

The fiducial model assumed power-law descriptions for both the density and velocity
laws (Equations~\ref{eqn:density},\ref{eqn:vel}).
The power-law exponents were arbitrarily chosen, i.e.\ with little physical
motivation.  In this sub-section, we explore the results for a series
of other power-law expressions, also arbitarily defined.
We consider three different density laws ($n_{\rm H} \propto
r^{-3}, r^{0}, r^2$) and three different velocity laws ($\mvr
\propto r^{-2}, r^{-1}, r^{0.5}$) for 9 new wind models
(Table~\ref{tab:plaw_parm}, Figure~\ref{fig:plaws}a).  
Each of these winds extends over the same inner and outer radii as the
fiducial model.
The density and velocity normalizations $n_{\rm H}^0, v^0$ have been
modified to yield an \mgiia\ optical depth profile that peaks at
$\tau_{2796}^{\rm max} \approx 10-10^3$ and then usually decreases to
$\tau_{2796} < 1$ (Figure~\ref{fig:plaws}b). 
All models assume a Doppler parameter
$b=15\mkms$, full isotropy, and a dust-free environment.
Lastly, the abundance of Mg$^+$ and Fe$^+$ ions scale with hydrogen as
in the fiducial model ($\S$~\ref{sec:fiducial}).

Figure~\ref{fig:plaws_spec} presents the \ion{Mg}{2} and \ion{Fe}{2}
profiles for the full suite of power-law models.  Although these
models differ qualitatively from the fiducial model in their density
and velocity laws, the resultant profiles share many of the same
features.  Each shows significant absorption at $\mdv < 0 \mkms$ with
the profiles extending to the velocity where $\tau_{2796} < 0.1$
(Figure~\ref{fig:plaws}b).  All of the models also exhibit strong
line-emission, primarily at velocities $\mdv \sim 0 \mkms$.  
This emission fills-in the \ion{Mg}{2} absorption at $\mdv \gtrsim
-100 \mkms$ such that the profiles rarely achieve a relative flux less
than $\approx 0.1$ at these velocities. 
Similar to the radiative driven wind ($\S$~\ref{sec:radiate}),
however, the power-law models that have significant opacity at $\mdv <
-300 \mkms$ have larger peak optical depths.  Another commonality is
the weak or absent line-emission at \feiia; one instead notes strong
\feiic\ emission that generally exceeds the equivalent width of the
fiducial model.  

One of the few obvious distinctions of these models from the fiducial wind is
the higher peak optical depth of absorption in the \feiib\
transition.  This is predominantly due to the 
higher peak optical depth of this transition. 
Table~\ref{tab:plaw_diag} compares measures of the absorption and
emission profiles against the fiducial model.
In detail, the line profiles do differ in their peak optical depths,
the velocity centroids of absorption, and the total equivalent widths.
We find that most of these differences are driven by differences in
the velocity laws, i.e.\ the kinematics of the outflow.
[Last point]

%%%%%%%%%%%%%%%%%%%%%%%%%%%%%%%%%%%%%%%%%%%%%%%%%%%%%%%%
%%%%%%%%%%%%%%%%%%%%%%%%%%%%%%%%%%%%%%%%%%%%%%%%%%%%%%%%
%%%%%%%%%%%%%%%%%%%%%%%%%%%%%%%%%%%%%%%%%%%%%%%%%%%%%%%%
\section{Discussion}
\label{sec:discuss}

We now discuss the principal results of our analysis and comment on
the observational consequences and implications. 

The previous sections presented idealized wind models for
cool gas outflows, and explored the absorption/emission profiles of
the \mgiid\ doublet and the \ion{Fe}{2} UV1 multiplet.  In addition to
the ubiquitous presence of blue-shifted absorption, the wind models
also predict strong line-emission in both resonance
and non-resonance transitions.  For isotropic and dust-free scenarios,
this is a simple conservation of photons: every photon emitted by
the source eventually escapes the system maintaining 
zero total equivalent width. A principal result of this paper,
therefore, is that cool gas outflows should generate detectable 
line-emission.
 
Indeed, line-emission from low-ion transitions has been reported from 
star-forming galaxies that exhibit cool gas outflows.  
This includes emission related to \ion{Na}{1} \citep{phillips,chen},
\ion{Mg}{2} emission \citep{wcp+09,rwk+10}, 
\ion{Si}{2}* emission from LBG galaxies \citep{shapley03},
and, most recently, significant \feiis\ emission \citep{rubin+10c}. 
A variety of origins have been proposed for this
line-emission including AGN activity, recombination in \ion{H}{2}
regions, and back-scattering off the galactic-scale wind.
[Are there aspects of those works that are now understood, e.g. the
difference in FeII 2586 and FeII 2600?]
Our results suggest that the majority of this line-emission is 
from scattered photons in the cool gas outflows of these star-forming
galaxies.  Indeed, our results indicate that line emission could and
should be produced by all of galaxies driving a wind.

On the other hand,
there are many examples of galaxies where blue-shifted absorption is
detected yet the authors report no significant line emission.  This
includes \ion{Na}{1} flows \cite{rupke,martin,sato}, 
\ion{Mg}{2} and \ion{Fe}{2} absorption in
ULIRGS \citep{mb09}, and the extreme \ion{Mg}{2} outflows identified
by \cite{tmd07}.  Similarly, there have been no reports of
resonance line-emission from low-ion transitions in the
LBGs\footnote{\citet{psa+00}
  reported and analyzed \ion{C}{4} line-emission from the lensed LBG
  cb58, attributing this flux to stellar winds.  We caution, however,
  that the galactic-scale wind proposed to exist for cb58 should also
  have contributed to the \ion{C}{4} line-emission.}, and many
$z \lesssim 1$ galaxies show no detectable \ion{Mg}{2} or \ion{Fe}{2}*
emission despite significant blue-shifted absorption (Rubin et al., in prep). 
These non-detections appear to contradict a primary conclusion of 
this paper.   We are motivated, therefore, to
reassess several of the effects that can reduce the line-emission
and consider whether these could preclude its detection for many
star-forming galaxies.  [If MgII emission is 'common' for HII regions,
why don't we see it *always* when we see OII, Hbeta, etc? Is it
trapped and then extincted? Same for FeII*]

%% STOPPED HERE %%

Dust is frequently invoked to explain the suppression of line emission
related to resonantly trapped transitions (e.g.\ \lya).  Indeed, a photon
that is trapped for many scatterings within a dusty medium will 
preferentially be extincted relative to a non-resonant photon.  In
$\S$~\ref{sec:dust}, we examined the effects of dust extinction on 
\ion{Mg}{2} and \ion{Fe}{2} emission.  The general result
(Figure~\ref{fig:dust_tau}) is the modest
suppression of line flux that scales with (1+\taud)$^{-1}$ instead of
exp(--\taud). Although the \ion{Mg}{2} photons are resonantly
trapped, they require only one to a few scatterings
to escape the wind model and dust has only a
modest effect.  This reflects the moderate opacity of the \ion{Mg}{2}
doublet (e.g.\ relative to \lya) and also the velocity law of the
fiducial wind model.
As we have discussed, in scenarios where the \ion{Mg}{2} photons are
more effectively trapped, dust further suppresses
the emission (e.g.\ the ISM+wind model from $\S$~\ref{sec:ISM}).
In contrast to \ion{Mg}{2}, 
the \ion{Fe}{2} resonance photons may be
converted to non-resonant photons that freely escape the wind.
Therefore, a wind model that trapped \ion{Mg}{2}
photons for many scatterings would not similarly trap 
\feiid\ photons.  The effects of dust are much 
reduced and, by inference, the same holds for any other set of
transitions that are coupled to a fine-structure level (e.g.\
\ion{Si}{2}/\ion{Si}{2}*).  In summary,
dust does reduce the line flux relative to the continuum, but 
often has only a modest effect on the predicted \ion{Mg}{2}
emission and always a minor effect on \feiis\ emission.

Another factor that may reduce the line-emission is a
non-isotropic wind model. The flux is lower, for example, if one
eliminates the backside to the wind (e.g.\ the source itself could
shadow a significant fraction of the backside).  To zeroth order, the
line-emission scales 
roughly as $\Omega/4\pi$ where $\Omega$ is the angular extent of the
wind\footnote{There is also a first order dependence on the actual
  orientation of the wind, with maximal emission when the backside is
  fully viewed.}.  Because we require that the wind 
points toward us, it is difficult to reduce $\Omega$ much
below $2 \pi$ and, therefore, anisotropic winds reduce the
emission by a factor of order unity.  
Similarly, an anisotropic source may alter the predicted
line-emission.  For example, if the backside of the galaxy were
brighter/fainter then would one predict brighter/fainter emission
relative to the observed continuum.
To reduce the emission, the brightest regions
of the galaxy would need to be oriented toward Earth.  Although this
hould not be a
generic occurance, most spectroscopic samples are magnitude-limited
and biased towards detecting galaxies when observed from their
brightest viewpoint.  In principle, this might imply a further
reduction of order unity.

Together, dust and non-isotropic wind models may reduce the line-emission
of \ion{Mg}{2} and \ion{Fe}{2}
by a factor of order unity.  This would generally
not be sufficient, however, to reduce the emission below the typical detection
limits. 

There is another (more subtle) effect that could
greatly reduce the observed line emission: slit-loss.  As
described in Figure~\ref{fig:fiducial_ifu_mgii}, 
emission from the wind is spatially extended with
a non-zero surface brightness predicted at large radii. This implies 
a non-negligible luminosity emitted well beyond the galaxy.  
The majority of observations of star-forming galaxies to date have
been taken through a spectroscopic slit designed to cover
the brightest continuum regions of the galaxy. Slits with angular
extents of $1-1.5''$ subtend roughly $8-15$\,kpc for galaxies at $z>0.5$.
Therefore, a $1''$ slit covering a galaxy with our fiducial wind would cover
less than half the wind.  The result is reduced line-emission,
when compared to the galaxy continuum.

We explore the effects of slit-loss as follows.
We model the slit as a perfect, infinitely-long rectangle centered on the
galaxy.  We then tabulate the equivalent width of the line-emission
through slits with a range of widths, parameterized by twice the radius
where the wind has a Sobolev optical depth of $\tau^S = 0.2$ (\rtt).
The line-emission will be weak beyond
this radius because the photons have only a low probability of
interacting with the wind.  For the fiducial wind model, 
the \mgiia\ transition has $\mrtt \approx 15$\,kpc 
(Figure~\ref{fig:fiducial_nvt}).  

The predicted equivalent width in emission \ewe\ relative to the absorption
equivalent width \ewabs\ is presented in
Figure~\ref{fig:obs_slit} for a series of transitions for the fiducial
wind model (black curves).
[Add LBG, radiative models]
The \ewe/\ewabs\ curves rise very steeply with increasing slit width and then
plateau when the slit width reaches $\approx \mrtt$.  
The results are comparable for other wind models considered
in this paper (colored curves in Figure~\ref{fig:obs_slit}).
All of the curves rise so steeply that 
the effects of slit-loss is of order unity
unless the slit-width is very small.   
Nevertheless, the results do motivate
extended aperture observations, e.g.\ integral field unit (IFU)
instrumentation that would map the wind both spatially and spectrally.


Although several effects can reduce the line-emission
relative to the absorption of the outflow, our analysis
indicates that detectable line-emission should occur frequently. 
Furthermore, the line-emission could be suppressed so as not to exceed
the galaxy continum yet still (partially) fill-in the absorption
profiles. Indeed, dust and anisotropic winds preferentially suppress
line-emission at $\mdv > 0 \mkms$
(Figures~\ref{fig:dust},\ref{fig:anisotropic}).
The emission would still 
modify the observed absorption profiles (e.g.\
Figure~\ref{fig:noemiss}). This may lead to erroneous conclusions
regarding several characteristics of the flow.  {\it An analysis of cool gas outflows
that entirely ignores line-emission may incorrectly conclude that the gas is
partially covered, has a significantly lower peak optical depth,
and/or that a $\mdv \sim 0 \mkms$ component is
absent.}  
We now consider several quantitative effects of the line-emission.

Figure~\ref{fig:obs_ew} demonstrates one observational
consequence: reduced measurements for the absorption
equivalent width \ewabs\ of the flow.  In the case of \ion{Mg}{2},
which has the most affected transitions, the \ewabs\ is reduced by
$30-50\%$ from the intrinsic absorption.  In turn, one may derive 
a systematically lower optical depth or velocity extent for the wind,
and therefore a lower total mass and kinetic energy.  
The effects are most pronounced for wind
scenarios where the peak optical depth occurs near $\mdv = 0 \mkms$.
Geometric projection limits the majority of scattered photon emission to
have $|\mdv| < 200 \mkms$; therefore,  the absorption profiles
are filled-in primarily at these velocities.  

Another (related) consequence is the reduction of peak depth of the
absorption.
In Figure~\ref{fig:obs_peaktau}, we plot the observed peak optical depth
$\tau_{\rm pk}$ for \mgiia\ 
versus the velocity where the profile has greatest depth 
for the various wind models (i.e.\
$\tau_{\rm pk}$ vs.\ $v_\tau$ from Tables~\ref{tab:line_diag} and
\ref{tab:plaw_diag}).   In every one of the models, the true peak
optical depth $\tau_{\rm pk}^{\rm true} > 10$.  For the
majority of cases with $v_\tau > -300\mkms$, one observes $\tau_{\rm pk} <
2$ and would infer the wind is not even optically thick!
This occurs because photons scattered by the wind have `filled-in' the
absorption at velocity $\mdv \approx 0 \mkms$.  In contrast, wind
models with $v_\tau < -400 \mkms$ all yield $\tau_{\rm pk} > 3$.  The
results are similar for the \mgiib\ profile.  In fact, one infers a
very similar $\tau_{\rm pk}$ for the two lines which would generally
lead to the false conclusion of partial covering. 
[Last point]

Tf these negative effects for \ion{Mg}{2} absorption are reduced
for the \ion{Fe}{2} absorption profiles, especially for \feiia.  This
is because a signficant fraction (even a majority) of the
\ion{Fe}{2} emission is florescent \ion{Fe}{2}* emission at longer
wavelengths, which has no effect on the absorption profiles. 
Therefore, the \ion{Fe}{2} absorption equivalent widths (\ewabs) more
closely follow the intrinsic value, one derives more accurate peak
optical depths, and the absorption kinematics more faithfully reflect
the motions of the flow.  Regarding the last point, one also
predicts a velocity offset between the \ion{Fe}{2} and \ion{Mg}{2}
centroids (Figure~\ref{fig:obs_vtau}) because scattered photons
preferentially fill-in
absorption at $v \sim 0 \mkms$ for the \ion{Mg}{2}
transitions.  This affects gas related to the ISM of the galaxy and
also material infalling at a modest speed.  One must take careful
consideration of line emission, therefore,  before drawing robust
conclusions about the presence/absence of gas with small velocity
offsets from systemic. Figure~\ref{fig:obs_vtau} also emphasizes that
the \ion{Mg}{2} profiles may misrepresent the kinematics of the bulk of
the gas.  Analysis of these lines, without consideration of
line-emission, would lead to incorrect conclusions on the energetics
and mass flux of the wind.

We emphasize that all of these effects are heightened by the
relatively low spectral resolution and S/N characteristic of the data
obtained on $z>0$ star-forming galaxies.  Figure~\ref{fig:obs_lris}a
shows one realization of the \mgiid\ doublet for the ISM+wind model
convolved with the line-spread-function of the Keck/LRIS spectrometer
(a Gaussian with FWHM=250\kms) and an assumed signal-to-noise of
S/N=7 per (1 \AA) pixel.  Both the absorption
and emission are well detected, but it would be
difficult to resolve the issues discussed above (e.g.\ partial
covering, peak optical depth) with these data.  
One also notes several sysmetaic effects of the lower spectral
resolution, e.g.\ reduced peak flux in the line-emission and a
systematics shift of \mgiia\ absorption to more negative velocity.
We have also modelled an observation of the fiducial model as a
stacked of the galaxy spectra (Figure~\ref{fig:obs_lris}b) [word].
Specifically, we
averaged 100 identical \ion{Mg}{2} profiles from the fiducial model
degraded to a S/N=2\,pix$^{-1}$ and assuming a random redshift error
$\sigma(z) = 100\mkms$.  
This treatment is meant to illustrate the
implications of stacking galaxy spectra to sutdy outflows
\citep[e.g.][]{wcp+09,sato+,rkc+10}.   The main difference from the
single galaxy observation shown in panel (a) is the smearing of
line-emission and absorption that reduces the height/depth of each.
The effects would be even more pronounced if one studied spectra with
a diversity of \ion{Mg}{2} profiles.    [{\bf Kate}: try stacking your set
of observed LRIS data]

[Should galaxies with strong \lya\ emission be expected to show
low-ion emission? Perhaps not if \lya\ is produced by ionizing photons
from SF regions.]


The previous few paragraphs sound a cautionary and largely negative
perspective on the implications for absorption-line analysis of
galactic-scale outflows in the presence of (expected) significant
line-emission.   While this is a necessary complication, analysis of
detected line-emission does offer new and unique
constraints on the charcteristics of the outflow.  And when coupled
with the absorption-line data, the two sets of constraints may break
various degeneracies on the physical characteristics of the outflow.

The most direct measure provided by the line emission is an estimate
of the size and morphology of the outflow \citep]{rubin+10c}.  
Line emission is predicted to extend to radii where the Sobolev
optical depth exceeds a few tenths [is that right?].  The principle
challenge is to achieve sufficient sensitivity to detect the low
surface brightness emission.  
As Figure~\ref{fig:obs_sb}
demonstrates, the surface brightness at the inner wind radius of our
fiducial wind exceeds that at the outer radius by several orders of magnitude.
Nevertheless, an instrument that sampled the entire wind (e.g.\ a
large format IFU or narrow band imager) may detect the emission in
azimuthally averaged apertures. [Give estimate]
[Add FeII* to the SB figure]

Of the transitions considered in this paper, \ion{Mg}{2} emission is
preferred for [this analysis] because 
(i) it has the largest equivalent width in absorption and
(ii) there are fewer emission channels per absorption line than for
\ion{Fe}{2}.  This implies the highest peak and integrated fluxes
(Figure~\ref{fig:obs_sb}).  With an IFU observation covering all of
the \ion{Fe}{2} transitions, one could sum the together to enhance the
signal. 

The kinematics of the line emission also offer insight into physical
characteristics of the wind.

[Relative strengths of MgII 2796 and 2803 speak to $\tau$ and
$v_r^{max}$]

%%%%%%%%%%%%%%%%%

3)  Emission can be used for good!   This is largely an extension of
your emission paper, of course.

   (i)  Morphology [MgII is likely best]
  (ii)  Kinematics  --  FWHM, centroid $=>$  Wind speed, ISM
 (iii)  Flux ratios?  e.g.   f\_2796/f\_2803 and/or f\_2626/f\_2600

[Does recombination predict 2:1 for MgII??]
[We also emphasize that departures from a 2:1
ratio indicate that processes other than simple recombination are
active. ]

[Profiles are modified by {\it many} factors => Difficult to derive
robust constraints from absorption alone.]

[Is one safe if there is no emission detected?]

[What does MgI tell us about winds?  Why no P-Cygni?]

\section{Summary}
\label{sec:conclude}

1. More complicated winds needed

2. Line-profiles can be explained by both increasing and decreasing
velocity laws.

\acknowledgments

J.X.P and K.R. are partially supported
by an NSF CAREER grant (AST--0548180), and 
by NSF grant AST-0908910.

\clearpage

%\bibliographystyle{/u/xavier/NSF/SASIR/SASIR-ATI/prop2009/Text/nsfati}
%\bibliography{/u/xavier/NSF/SASIR/SASIR-ATI/prop2009/Text/nsfati09}
\bibliographystyle{/u/xavier/paper/Bibli/apj}
\bibliography{/u/xavier/paper/Bibli/allrefs}

\clearpage

\begin{deluxetable}{lcccccc}
\tabletypesize{\footnotesize}
\tablecolumns{11}
\tablecaption{Observed Transitions and Limits \label{tab:atomic}}
\tablewidth{0pt}
\tablehead{\colhead{} & \colhead{$\rm E_{high}$} & \colhead{$\rm E_{low}$} & \colhead{$J_{\rm high}$} & \colhead{$J_{\rm low}$} & \colhead{$\lambda$} & \colhead{$A$} \\
 & \colhead{($\rm cm^{-1}$)} & \colhead{($\rm cm^{-1}$)} &&& \colhead{(\AA)} & \colhead{($\rm s^{-1}$)} } 
\startdata
\ion{Fe}{2} UV1 & 38458.98 &     0.00 &   9/2 & 9/2 & 2600.173 & 2.36E08  \\
           & 38458.98 &   384.79 &   9/2 & 7/2 & 2626.451 & 3.41E+07 \\
           & 38660.04 &     0.00 &   7/2 & 9/2 & 2586.650 & 8.61E+07 \\
           & 38660.04 &   384.79 &   7/2 & 7/2 & 2612.654 & 1.23E+08 \\
           & 38660.04 &   667.68 &   7/2 & 5/2 & 2632.108 & 6.21E+07 \\
           & 38858.96 &   667.68 &   5/2 & 5/2 & 2618.399 & 4.91E+07 \\
           & 38858.96 &   862.62 &   5/2 & 3/2 & 2631.832 & 8.39E+07 \\
           & 39013.21 &   667.68 &   3/2 & 5/2 & 2607.866 & 1.74E+08 \\
           & 39013.21 &   862.61 &   3/2 & 3/2 & 2621.191 & 3.81E+06 \\
           & 39013.21 &   977.05 &   3/2 & 1/2 & 2629.078 & 8.35E+07 \\
           & 39109.31 &   862.61 &   1/2 & 3/2 & 2614.605 & 2.11E+08 \\
           & 39109.31 &   977.05 &   1/2 & 1/2 & 2622.452 & 5.43E+07 \\
\tableline \\ [-1.5ex]
\ion{Mg}{2}& 35760.89 &     0.00 &   3/2 &   0 & 2796.351 & 2.63E+08\\
           & 35669.34 &     0.00 &   1/2 &   0 & 2803.528 & 2.60E+08\\
\enddata
\tablecomments{Atomic data was obtained from \citet{Morton2003} unless otherwise indicated.}
\end{deluxetable}

 
 
\begin{deluxetable}{ccl}
\tablewidth{0pc}
\tablecaption{Wind Parameters: Fiducial Model\label{tab:fiducial}}
\tabletypesize{\footnotesize}
\tablehead{\colhead{Property} & \colhead{Parameter} & \colhead{Value} } 
\startdata
Density law  & $n(r)$ & $\propto r^{-2}$ \\
Velocity law  & $v_r$ & $ \propto r$ \\
Inner Radius & $r_{\rm inner}$ & 1\,kpc \\
Outer Radius & $r_{\rm outer}$ & 20\,kpc \\
Source size  & $r_{\rm source}$ & 0.5\,kpc \\
Density Normalization & $n^0_{\rm H}$ & $0.1\cm{-3}$ at $r_{\rm inner}$ \\
Velocity Normalization & $v^0$ & 50\kms at $r_{\rm inner}$ \\
Turbulence   & $b_{\rm turb}$  & 15 \kms \\
Mg$^+$ Normalization & \nmg\ & $10^{-5.47} n_{\rm H}$ \\
Fe$^+$ Normalization & \nfe\ & \nmg/2 \\
\enddata
%\tablecomments{Unless specified otherwise, all quantities refer to the \sna=2 threshold.  The cosmology assumed has $\Omega_\Lambda = 0.7, \Omega_m = 0.3$, and $H_0 = 72 \mkms \rm Mpc^{-1}$.}
%\tablenotetext{a}{Total redshift survey path for the \sna=2 criterion.}

\end{deluxetable}

\input{../Tables/tab_fiducial_ew.tex}
 
 
\begin{deluxetable}{ccrccccccccccc}
\rotate
\tablewidth{0pc}
\tablecaption{Line Diagnostics for the Fiducial Model and Variants
\label{tab:line_diag}}
\tabletypesize{\footnotesize}
\tablehead{\colhead{Transition} & \colhead{Model} & \colhead{$v_{\rm int}^a$} & \colhead{$W_{\rm i}$} & \colhead{$W_{\rm a}$} & \colhead{$\tau_{\rm pk}$} & \colhead{$v_\tau$} 
& \colhead{$v_{\bar \tau}$}
& \colhead{$v_{\rm int}^b$} & \colhead{$W_{\rm e}$} & \colhead{$f_{\rm pk}$} & \colhead{$v_f$} 
& \colhead{$v_{\bar f}$}
\\
&& (\kms) & (\AA) & (\AA) && (\kms) & (\kms) & (\kms) & (\AA) & & (\kms) & (\kms)}
\startdata
  MgII 2796  \\
&Fiducial&[$-1009,-43$]& 4.78& 2.83&0.94&$ -215$&$ -372$&[$-32,311$]&$-1.77$& 2.48&$   32$&$  117$\\
&$\phi=0^\circ$&$\dots$&$\dots$&$\dots$&$\dots$&$\dots$&$\dots$&$\dots$&$\dots$&$\dots$&$\dots$&$\dots$&$\dots$\\
&$\phi=180^\circ$&[$-1030,-65$]& 4.78& 2.98&1.03&$ -199$&$ -369$&[$-65,70$]&$-0.40$& 1.78&$  -11$&$    6$\\
&\taud=1&[$-1009,-32$]& 4.77& 2.94&0.95&$ -215$&$ -360$&[$-32,257$]&$-0.92$& 1.90&$   32$&$  101$\\
&\taud=3&[$-998,-32$]& 4.78& 3.07&1.03&$ -193$&$ -342$&[$-22,182$]&$-0.39$& 1.48&$   43$&$   76$\\
&ISM&[$-1009,-65$]& 6.36& 2.67&0.90&$ -204$&$ -390$&[$-54,311$]&$-1.60$& 2.36&$  118$&$  125$\\
&ISM+dust&[$-998,150$]& 6.42& 4.06&1.06&$ -193$&$ -270$&[$590,601$]&$ 0.12$& 0.40&$  590$&$  595$\\
  MgII 2803  \\
&Fiducial&[$-437,-41$]& 3.29& 1.19&0.76&$ -148$&$ -193$&[$-41,676$]&$-2.22$& 2.55&$   34$&$  269$\\
&$\phi=0^\circ$&$\dots$&$\dots$&$\dots$&$\dots$&$\dots$&$\dots$&$\dots$&$\dots$&$\dots$&$\dots$&$\dots$&$\dots$\\
&$\phi=180^\circ$&[$-699,-57$]& 3.29& 1.98&0.97&$ -137$&$ -257$&[$-57,104$]&$-0.50$& 1.81&$   -3$&$   22$\\
&\taud=1&[$-479,-41$]& 3.28& 1.41&0.83&$ -137$&$ -195$&[$-41,591$]&$-1.27$& 1.95&$   34$&$  247$\\
&\taud=3&[$-533,-30$]& 3.26& 1.67&0.94&$ -116$&$ -195$&[$-30,484$]&$-0.64$& 1.50&$   34$&$  213$\\
&ISM&[$-426,-62$]& 6.49& 1.03&0.67&$ -169$&$ -206$&[$-51,655$]&$-2.09$& 2.57&$   98$&$  265$\\
&ISM+dust&[$-1774,130$]& 6.51& 6.73&1.06&$ -961$&$ -686$&[$173,195$]&$-0.01$& 1.03&$  184$&$  184$\\
  FeII 2586  \\
&Fiducial&[$-348,-35$]& 0.82& 0.61&1.01&$  -70$&$ -119$&[$-35,128$]&$-0.08$& 1.11&$   35$&$   47$\\
&$\phi=0^\circ$&$\dots$&$\dots$&$\dots$&$\dots$&$\dots$&$\dots$&$\dots$&$\dots$&$\dots$&$\dots$&$\dots$&$\dots$\\
&$\phi=180^\circ$&[$-365,-46$]& 0.82& 0.60&1.01&$  -75$&$ -133$&[$-46,70$]&$-0.01$& 1.06&$  -17$&$   12$\\
&\taud=1&[$-348,-35$]& 0.82& 0.61&1.04&$  -70$&$ -118$&[$-35,116$]&$-0.04$& 1.06&$   23$&$   41$\\
&\taud=3&[$-313,-35$]& 0.94& 0.60&1.06&$  -70$&$ -113$&[$-23,46$]&$-0.02$& 1.04&$  -12$&$   12$\\
&ISM&[$-348,104$]& 1.90& 1.60&2.75&$    0$&$  -29$&[$1670,1948$]&$-0.19$& 1.15&$ 1682$&$ 1806$\\
&ISM+dust&[$-313,116$]& 1.84& 1.63&2.96&$   23$&$  -25$&[$278,290$]&$-0.00$& 1.03&$  278$&$  284$\\
  FeII 2600  \\
&Fiducial&[$-580,-37$]& 1.87& 1.18&1.08&$  -83$&$ -181$&[$-37,459$]&$-0.82$& 1.70&$   32$&$  191$\\
&$\phi=0^\circ$&$\dots$&$\dots$&$\dots$&$\dots$&$\dots$&$\dots$&$\dots$&$\dots$&$\dots$&$\dots$&$\dots$&$\dots$\\
&$\phi=180^\circ$&[$-597,-49$]& 1.87& 1.18&1.31&$  -78$&$ -188$&[$-49,95$]&$-0.23$& 1.39&$  -20$&$   22$\\
&\taud=1&[$-591,-37$]& 1.87& 1.20&1.14&$  -83$&$ -176$&[$-37,332$]&$-0.48$& 1.45&$   32$&$  138$\\
&\taud=3&[$-580,-37$]& 1.95& 1.25&1.31&$  -72$&$ -169$&[$-37,228$]&$-0.20$& 1.22&$   44$&$   93$\\
&ISM&[$-568,90$]& 3.12& 1.88&1.19&$  -72$&$  -99$&[$101,378$]&$-0.19$& 1.15&$  113$&$  237$\\
&ISM+dust&[$-603,101$]& 3.06& 2.12&1.40&$   44$&$  -93$&[$182,194$]&$-0.00$& 1.02&$  182$&$  188$\\
  FeII* 2612 \\
&Fiducial&&&&&&&$\dots$&$\dots$&$\dots$&$\dots$&$\dots$&$\dots$\\
&$\phi=0^\circ$&&&&&&&$\dots$&$\dots$&$\dots$&$\dots$&$\dots$&$\dots$\\
&$\phi=180^\circ$&&&&&&&$\dots$&$\dots$&$\dots$&$\dots$&$\dots$&$\dots$\\
&\taud=1&&&&&&&$\dots$&$\dots$&$\dots$&$\dots$&$\dots$&$\dots$\\
&\taud=3&&&&&&&$\dots$&$\dots$&$\dots$&$\dots$&$\dots$&$\dots$\\
&ISM&&&&&&&$\dots$&$\dots$&$\dots$&$\dots$&$\dots$&$\dots$\\
&ISM+dust&&&&&&&$\dots$&$\dots$&$\dots$&$\dots$&$\dots$&$\dots$\\
  FeII* 2626 \\
&Fiducial&&&&&&&$\dots$&$\dots$&$\dots$&$\dots$&$\dots$&$\dots$\\
&$\phi=0^\circ$&&&&&&&$\dots$&$\dots$&$\dots$&$\dots$&$\dots$&$\dots$\\
&$\phi=180^\circ$&&&&&&&$\dots$&$\dots$&$\dots$&$\dots$&$\dots$&$\dots$\\
&\taud=1&&&&&&&$\dots$&$\dots$&$\dots$&$\dots$&$\dots$&$\dots$\\
&\taud=3&&&&&&&$\dots$&$\dots$&$\dots$&$\dots$&$\dots$&$\dots$\\
&ISM&&&&&&&$\dots$&$\dots$&$\dots$&$\dots$&$\dots$&$\dots$\\
&ISM+dust&&&&&&&$\dots$&$\dots$&$\dots$&$\dots$&$\dots$&$\dots$\\
  FeII* 2632 \\
&Fiducial&&&&&&&$\dots$&$\dots$&$\dots$&$\dots$&$\dots$&$\dots$\\
&$\phi=0^\circ$&&&&&&&$\dots$&$\dots$&$\dots$&$\dots$&$\dots$&$\dots$\\
&$\phi=180^\circ$&&&&&&&$\dots$&$\dots$&$\dots$&$\dots$&$\dots$&$\dots$\\
&\taud=1&&&&&&&$\dots$&$\dots$&$\dots$&$\dots$&$\dots$&$\dots$\\
&\taud=3&&&&&&&$\dots$&$\dots$&$\dots$&$\dots$&$\dots$&$\dots$\\
&ISM&&&&&&&$\dots$&$\dots$&$\dots$&$\dots$&$\dots$&$\dots$\\
&ISM+dust&&&&&&&$\dots$&$\dots$&$\dots$&$\dots$&$\dots$&$\dots$\\
\enddata
\tablecomments{{L}isted are the equivalent widths (intrinsic, absorption, and emission), the peak optical depth for the absorption
$\tau_{\rm pk} \equiv -\ln(I_{\rm min})$, the velocity where the optical depth peaks $v_\tau$, the optical depth-weighted velocity centroid 
$v_{\bar \tau} \equiv \int dv \, v \ln[I(v)] / \int dv \ln[I(v)]$, the peak flux $f_{\rm pk}$ in emission, the velocity where the flux peaks 
$v_f$, and the flux-weighted velocity centroid of the emission line $v_{\bar f}$.}
\end{deluxetable}

 
 
\begin{deluxetable}{ccl}
\tablewidth{0pc}
\tablecaption{Wind Parameters: Power-law Models\label{tab:plaw_parm}}
\tabletypesize{\footnotesize}
\tablehead{\colhead{Label} & \colhead{$n(r)$} & \colhead{$n_{\rm
      H}^0$} & \colhead{$v(r)$} & \colhead{$v^0$} &
  \colhead{$\log \tau_{2796}^{\rm max}$} }
\startdata
A & $r^{-3}$& 0.4000 & $r^{-2.0}$&   2.0 & 0.8 \\ 
B & $r^{-3}$& 0.5000 & $r^{-1.0}$&  50.0 & 1.1 \\ 
C & $r^{-3}$& 0.3000 & $r^{ 0.5}$& 100.0 & 1.8 \\ 
D & $r^{ 0}$& 0.0100 & $r^{-2.0}$&   2.0 & 2.1 \\ 
E & $r^{ 0}$& 0.0100 & $r^{-1.0}$&  50.0 & 1.7 \\ 
F & $r^{ 0}$& 0.0200 & $r^{ 0.5}$& 100.0 & 1.6 \\ 
G & $r^{ 2}$& 0.0100 & $r^{-2.0}$&   2.0 & 4.4 \\ 
H & $r^{ 2}$& 0.0010 & $r^{-1.0}$&  50.0 & 3.2 \\ 
I & $r^{ 2}$& 0.0001 & $r^{ 0.5}$& 100.0 & 1.8 \\ 
\enddata
%\tablecomments{Unless specified otherwise, all quantities refer to the \sna=2 threshold.  The cosmology assumed has $\Omega_\Lambda = 0.7, \Omega_m = 0.3$, and $H_0 = 72 \mkms \rm Mpc^{-1}$.}
%\tablenotetext{a}{Total redshift survey path for the \sna=2 criterion.}

\end{deluxetable}

\input{../Tables/tab_meas_plaws.tex}

\begin{figure}
\epsscale{0.95}
\plotone{../Figures/energy_levels.ps}
\caption{
Energy level diagrams for the \mgiid\ doublet and the UV1
multiplet of \ion{Fe}{2} transitions   
(based on Figure~7 from \cite{hmt+99}).
Each transition shown is
labeled by its rest wavelength (\AA) and Einstein A-coefficient
(s$^{-1}$). Black upward arrows
indicate the resonance-line transitions, i.e.\ those connected to the ground
state.  The 2p$^6$3p configuration of Mg$^+$ is split into
two energy levels that give rise to the \mgiid\ doublet.  
Both the 3d$^6$4s ground state and 3d$^6$4p upper level of Fe$^+$
exhibit fine-structure splitting that gives rise to a series of
electric-dipole transitions. 
The downward (green) arrows show the \feiis\ transitions that are connected to the
resonance-line transitions (i.e.\ they share the same upper energy
levels).  We also show a pair of levels (\ion{Fe}{2}$^* \lambda\lambda
2618,2631$) that arise from higher levels in the \zconfig\
configuration.  These transitions have not yet been observed in
galactic-scale outflows and are not considered in our analysis.
}
\label{fig:energy}
\end{figure}

\begin{figure}
\epsscale{0.8}
\plotone{../Figures/fig_nvtau_vs_r.ps}
\caption{
Density (dashed; red), radial velocity (dotted; blue), and
\mgiia\ optical depth profiles (solid; black) for the fiducial
wind model (see Table~\ref{tab:fiducial} for details).
The density and velocity laws are simple $r^{-2}$ and $r^1$
power-laws.  The optical depth profile was calculated by summing
the opacity at small and discrete radial intervals in velocity space
and was then converted to radius with the velocity law.  One notes that
the wind is optically thick at the inner radius ($r_{\rm inner} =
1$\,kpc) and becomes optically thin at the outer radius ($r_{\rm
  outer} = 20$\,kpc).
Note that the density and velocity curves have been scaled for plotting
convenience.  
}
\label{fig:fiducial_nvt}
\end{figure}

\begin{figure}
\includegraphics[scale=0.6,angle=90]{../Figures/fig_fiducial_1d.ps}
\caption{
{\it Left} -- (Upper) \mgiid\ profiles for the fiducial wind model
described in Table~\ref{tab:fiducial} and
Figure~\ref{fig:fiducial_nvt}.  The doublet shows the P-Cygni profiles
characteristic of an outflow with significant absorption blueward of
line-center (dashed vertical lines) extending to $v = -1000\mkms$
and significant emission redward of line-center.  Note
that even though the peak optical depth of the \ion{Mg}{2} transitions
is $\tau_{2796}^{\rm max} \approx 30$ at $v \approx -70 \mkms$,
photons scattered off the outflow fill in the absorption.
(Lower) \ion{Fe}{2} absorption and emission profiles for the UV1
multiplet at $\lambda \approx 2600$\AA.  The \feiid\ resonance lines 
show weaker absorption due to the smaller Fe$^+$ number density and
lower $f\lambda$ values.  Each also shows a P-Cygni profile, although
the emission for \feiia\ is significantly weaker than that of the
\feiib\ and \mgiid\ transitions.  This is because a majority of the
absorbed \feiia\ photons are converted into
\ion{Fe}{2}$^*~\lambda\lambda 2612, 2632$ photons.
{\it Right} -- A subset of the transitions displayed in a velocity
plot.
}
\label{fig:fiducial_1d}
\end{figure}

\begin{figure}
\epsscale{0.8}
\plotone{../Figures/fig_noemiss.ps}
\caption{
The solid curves show the line profiles (absorption and emission) of
the \ion{Mg}{2} and \ion{Fe}{2} resonance lines for the fiducial wind
model.  These include the effects of scattered photons and show the
canonical P-Cygni profiles of a source 
embedded within an outflow.  Overplotted (dotted line) on each transition is
the predicted absorption profile under the constraint that every
absorbed photon is lost from the system, i.e.\ no scattering or
re-emission occurs.   The true profiles have been `filled
in' significantly by light scattered in the wind.  Ignoring this
process, one would model the absorption lines with a systematically lower
optical depth and (incorrectly) conclude that the source is partly covered by the
gas.  
}
\label{fig:noemiss}
\end{figure}

\begin{figure}
\epsscale{0.8}
\plotone{../Figures/fig_fiducial_ifu_mgii.ps}
\caption{
Surface-brightness emission maps around the \mgiia\ transition for the
source+wind complex of the fiducial model.  The middle panel shows the
1D spectrum with $v=0 
\mkms$ corresponding to $\lambda = 2796.35$\AA\ and the dotted vertical
curves indicate the velocity slices for the emission maps.  The
source has a size $r_{\rm source} = 0.5$\,kpc, traced by a few
pixels at the center.   At $v=-250 \mkms$, the wind has an optical
depth of $\tau_{2796} \approx 1$ and the source contributes
roughly half of the observed flux.  At $v=-100 \mkms$ the
wind absorbs all photons from the source and the observed emission is
entirely due to photons scattered by the wind.  Impressively, this
emission exceeds the integrated flux at $v = -250\mkms$ such that the
line center is offset from the peak optical depth.  At $v \ge 0
\mkms$,  both the source and wind contribute to the observed emission.
At all velocities, the majority of emission comes from the inner
$\approx 5$\,kpc.
}
\label{fig:fiducial_ifu_mgii}
\end{figure}

\begin{figure}
\epsscale{0.8}
\plotone{../Figures/fig_feii_ifu.ps}
\caption{
(Upper) Surface-brightness emission maps around the \feiib\ transition for the
source+wind complex.  The middle panel shows the 1D spectrum with $v=0
\mkms$ corresponding to $\lambda = 2600.173$\AA\ and the dotted vertical
curves indicating the velocity slices for the emission maps.  
The results are very similar to those observed for the \mgiid\ doublet
(Figure~\ref{fig:fiducial_ifu_mgii}).
(Lower) Surface-brightness emission maps around the \ion{Fe}{2}~$\lambda
2612$ transition for the 
source+wind complex.  The middle panel shows the 1D spectrum with $v=0
\mkms$ corresponding to $\lambda = 2612.654$\AA\ and the dotted vertical
curves indicating the velocity slices for the emission maps.  
In this case, the source is unattenuated yet scattered photons from
the wind also have a significant contribution. 
}
\label{fig:fiducial_ifu_feii}
\end{figure}

%\begin{figure}
%\epsscale{0.8}
%\plotone{Figures/fig_fiducial_cut.ps}
%\caption{
%The panels show the relative emission from the wind as a function of
%radius for (a) \mgiia\ and (b) \feiib\ in a series of velocity
%channels.  With the exception of $v = -100 \mkms$ for \mgiia, the wind
%has finite transmission such that one observes the source at $r =
%0$\,kpc.  In all cases,  the majority of emission occurs at $r
%\lesssim 2.5$\,kpc, even if one ignores the source emission.
%}
%\label{fig:fiducial_cuts}
%\end{figure}

\begin{figure}
\epsscale{0.8}
\plotone{../Figures/fig_asymm_spec.ps}
\caption{
Profiles of the \ion{Fe}{2} and \ion{Mg}{2} profiles for the fiducial
case (black lines) compared against an anisotropic wind blowing into
only $2\pi$ steradians as viewed from $\phi = 0^\circ$ (source
uncovered) to $\phi = 180^\circ$ (source covered).  One detects
significant emission for all orientations of this wind but significant
absorption only for $\phi \ge 90^\circ$.
The velocity centroid of the 
emission shifts from positive to negative velocities as $\phi$
increases and one transitions from viewing the wind to be behind the
source to in front.  The velocity centroid of emission, therefore, may
give a robust measure of wind isotropy.
}
\label{fig:anisotropic}
\end{figure}

\begin{figure}
\epsscale{0.8}
\plotone{../Figures/fig_dust_spec.ps}
\caption{
Profiles of the \ion{Fe}{2} and \ion{Mg}{2} profiles for the fiducial
model (black) against a series of models that include 
dust extinction.  The primary effect is suppression of
the line-emission relative to the continuum. 
A more subtle but important effect is that the redder photons in the
emission lines (corresponding to higher velocity relative to
line-center) suffer greater extinction.  This is most evident in the
\feiic\ emission line; it occurs because these photons must
travel farther to scatter off the backside of the wind.  Note that
the absorption lines are nearly unmodified until $\mtaud = 10$, a
level of extinction that would preclude observing the source
altogether.
}
\label{fig:dust}
\end{figure}

\begin{figure}
\epsscale{0.8}
\plotone{../Figures/fig_taud_vs_we.ps}
\caption{
Attenuation by dust
}
\label{fig:dust_tau}
\end{figure}

\begin{figure}
\epsscale{0.7}
\plotone{../Figures/fig_ism_spec.ps}
\caption{
Profiles of the \ion{Fe}{2} and \ion{Mg}{2} profiles for the ISM+wind
model (red) compared against the fiducial wind model (black). 
The dotted line traces the predicted absorption profile in the absence
of re-emission and scattering.
Regarding the \ion{Mg}{2}~doublet, the primary difference between the
ISM+wind and the fiducial models is the shift of $\approx 100 \mkms$
in the emission lines from $v \approx 0 \mkms$ for the fiducial
model to $v \approx +100\mkms$ for the ISM+wind model. 
For the ISM+wind model, 
the \ion{Fe}{2} profiles show qualitative differences. 
the ISM model.  The \feiid\ resonance transitions each exhibit much
greater absorption at $v \approx 0 \mkms$ than the fiducial model.
The resonant line-emission is also substantially
reduced, implying much higher fluxes for the non-resonant lines (e.g.\
\feiic).  The lack of scattered \feiia\ photons results in an
absorption profile that very nearly matches the input opacity profile.
We conclude that resonance transitions that are strongly coupled to
non-resonant lines offer the best
characterization of an ISM component.
}
\label{fig:ISM_spec}
\end{figure}

\begin{figure}
\epsscale{0.8}
\plotone{../Figures/fig_norm_spec.ps}
\caption{
\ion{Mg}{2} and \ion{Fe}{2} profiles for the fiducial model with
varying normalization, parameterized by \nhn.  As expected, the
strength of absorption increases with increasing \nhn\ which also
results in stronger line-emission.  Note that the \feiia\ emission is
always weak; only its absorption varies with \nhn\ actually exceeding
the depth of \feiib\ for $\mnhn > \mnhf$.  The depth of the
\ion{Mg}{2} doublet, meanwhile, always falls below a relative flux of
0.3 while the emission rises steadily with \nhn.
}
\label{fig:norm}
\end{figure}

\begin{figure}
\epsscale{0.8}
\plotone{../Figures/fig_lbg_sobolev.ps}
\caption{
(a) The dotted curve shows the velocity law (values are labeled on the
right side) for the LBG model
of S10.  Note how rapidly the velocity rises from $r =
1$ to 2\,kpc.  The solid curve shows the Sobolev solution for the
Mg$^+$ gas
density derived from 
the average absorption profile of LBG galaxies (see below).
The density drops off as $(r/{\rm kpc}-1)^{-0.6}$ initially and then
steepens to $(r/{\rm kpc}-1)^{-1.8}$.
(b) Average absorption profile for LBG cool gas absorption as
measured and defined by S10
(black solid curve).  Overplotted on this curve is a (red) dotted line that
shows the \mgiia\ absorption profile derived from the density (and velocity)
law shown in the upper panel.  The excellent agreement confirms the
validity of the Sobolev approximation.
}
\label{fig:LBG_Sobolev}
\end{figure}

\clearpage

\begin{figure}
\epsscale{0.8}
\plotone{../Figures/fig_lbg_spec.ps}
\caption{
Profiles of the \ion{Fe}{2} and \ion{Mg}{2} profiles for the two
LBG models considered: (red) the profiles from the 
Sobolev approximation 
and (black) a model constructed to faithfully
correspond to the wind model advocated by S10.  The dotted line
shows the absorption profiles of the latter model if one were
to neglect scattered and re-emitted photons.  Similar to the fiducial
model (Figure~\ref{fig:fiducial_1d}), scattered and re-emitted photons
significantly modify the absorption profiles (especially \ion{Mg}{2})
and produce significant emission lines.  
The result, for \ion{Mg}{2} especially, is a set of profiles that do
not match the average observed LBG absorption profile.
Note the excellent agreement
between the \mgiia\ profile for the two models. [Cast doubt on S10
here?]
}
\label{fig:LBG_spec}
\end{figure}

\begin{figure}
\includegraphics[scale=0.6,angle=90]{../Figures/fig_lbg_cumul.ps}
\caption{
Profiles of the \ion{Fe}{2} and \ion{Mg}{2} profiles for the fiducial
}
\label{fig:LBG_cumul}
\end{figure}

%%%% RADIATIVE %%%%%%
%\begin{figure}
%\epsscale{0.8}
%\plotone{Figures/fig_radiation_nvt.ps}
%\caption{
%Radiation diagnosed
%}
%\label{fig:rad_nvt}
%\end{figure}


\begin{figure}
\epsscale{0.8}
\plotone{../Figures/fig_rad_spec.ps}
\caption{
\ion{Mg}{2} and \ion{Fe}{2}
profiles for a radiation-driven wind model (red) compared against the
fiducial wind model (black).  In contrast to the fiducial model, the
velocity of the radiation-driven model decreases with increasing
radius such that the optical depth peaks at large velocity (here $\mdv
\approx -400 \mkms$).  At these velocities, the absorption is not
filled-in by scattered photons such that one recovers absorption lines
that very nearly match the intrinsic optical depth profile (dotted
curve).  Although the \ion{Mg}{2} absorption profiles are very
different from the fiducial model, the \ion{Mg}{2} line-emission has
similar peak flux and velocity centroids (see also
Table~\ref{tab:plaw_diag}).  The \feiis\ emisison also has a simiar
equivalent width, yet the lines are much broader for the
radiative-driven model. This reflects the fact that the peak optical
depth occurs at $\mdv \approx -400 \mkms$.
}
\label{fig:rad_spec}
\end{figure}



%%%% POWER LAWS %%%%
\begin{figure}
\epsscale{0.8}
\plotone{../Figures/fig_vary_profiles.ps}
\caption{
Power-law profiles
}
\label{fig:plaws}
\end{figure}

\begin{figure}
\includegraphics[scale=0.6,angle=90]{../Figures/fig_plaw_spec.ps}
\caption{
\ion{Mg}{2} and \ion{Fe}{2} profiles for a series of power-law wind
models described in Table~\ref{tab:plaw_parm}.  These panels reveal
the diversity of absorption and emission that result from a set of
intrinsic optical depth profiles that have relatively modest
differences (Figure~\ref{fig:plaws}b).  As designed, all of the models
show significant absorption blueward of systemic velocity.  As
important, each shows strong line-emission at $\mdv \sim 0 \mkms$,
with the flux proportional to the degree of absorption.
}
\label{fig:plaws_spec}
\end{figure}


\begin{figure}
\includegraphics[scale=0.6,angle=90]{../Figures/fig_obs_slit.ps}
\caption{
Emission equivalent width ($W_{\rm e}$) relative to the observed
absorption equivalent width ($W_{\rm a}$) for a series of transitions:
(solid -- \mgiia; dotted -- \mgiib; dashed -- \feiib).  The
\ewe/\ewabs\ ratio is plottd as a function of slit width relative to
twice the radius \rtt, defined to be where the Sobolev optical depth 
$\tau^S = 0.2$.  The black curves correspond to the fiducial wind
model ($\S$~\ref{sec:fiducial}), the red curves are for the
LBG-partial covering scenario ($\S$~\ref{sec:Covering}), and the
radiative-driven wind ($\S$~\ref{sec:radiative}) has blue curves.
For all of the wind models, the \ewe/\ewabs\ ratio rises very steeply
with slit width and plataeus at $\approx \mrtt$.  [Add?]
}
\label{fig:obs_slit}
\end{figure}

\begin{figure}
\includegraphics[scale=0.6,angle=90]{../Figures/fig_obs_ew.ps}
\caption{
Difference between the `observed' absorption equivalent width \ewabs\
which includes the flux of scattered photons and the intrinsic
equivalent width $W_{\rm i}$ that ignores photon scattering.  The
dashed (dotted) curves trace a 50\%\ (10\%) reduction in \ewabs\
relative to $W_{\rm i}$.  One notes a reduction in \ewabs by $\approx
30-50\%$ for the \mgiia\ transition (the effect is generally larger
for \mgiib).  The effects of scattered photons are reduced for the
\ion{Fe}{2} transitions because a fraction (in fact a majority for
\feiia) of the absorbed photons flourecse as \feiis\ emission at
longer wavelengths and do not `fill-in' the absorption profiles.
}
\label{fig:obs_ew}
\end{figure}

\begin{figure}
\includegraphics[scale=0.6,angle=90]{../Figures/fig_obs_peaktau.ps}
\caption{
The observed peak optical depth $\tau_{\rm pk} \equiv -\ln [I_{\rm
  min}]$ for \mgiia\ profiles of the wind models studied in this paper
(Tables~\ref{tab:line_diag}, \ref{tab:plaw_diag}) with $I_{\rm min}$
the minimum normalized intensity of the absorption profile.  Cases
where \tpk\ exceeds 3 are presented as lower limits.  The \tpk\ values
are plotted against the velocity $v_\tau$ where $I(v) = I_{\rm min}$.  In all
of the models, the true peak optical depth $\mtpk^{\rm true} > 10$.
The much lower `observed' values occur because scattered \mgiia\
photons have filled-in the absorption profiels at velocities $\mdv
\gtrsim -400 \mkms$.  Therefore, the effects are greates for wind
models where the optical depth peaks near $\mdv \sim 0 \mkms$.
Indeed, models with $v_\tau < -400 \mkms$ all show $\mtpk > 3$.
}
\label{fig:obs_peaktau}
\end{figure}

\begin{figure}
\includegraphics[scale=0.6,angle=90]{../Figures/fig_obs_vtau.ps}
\caption{
Comparison of the velocities where the \feiia\ and \mgiia\ absorption
profiles have observed peak optical depth.  Results for all the models
studied in this paper are presented (Tables~\ref{tab:line_diag},
\ref{tab:plaw_diag}).  The dashsed curve traces the one-to-one line.
It is evident that for wind models have a peak optical depth near the
systemic velocity, the \mgiib\ absorption profile is shifted blueward
by one to several hundred \kms.  One may therefore falesly conclude
that (i) the majority of mass in the wind is travelling at a higher
velocity; and (ii) there is no gas with $\mdv \sim 0 \mkms$.
}
\label{fig:obs_vtau}
\end{figure}

\begin{figure}
\includegraphics[scale=0.6,angle=90]{../Figures/fig_obs_lris.ps}
\caption{
`Observations' of the \ion{Mg}{2} profiles for the fiducial wind
model.  
(a) The true profile (dotted lines) have been convolved with a
Gaussian line-spread-function (FWHM~$= 250 \mkms$) and noise has been
added to give a S/N=7 per pixel.  Note the suppressed peak in the
line-emission; one may even infer the flux of \mgiia\ exceeds that of
\mgiia.  One also finds that the \mgiia\ absorption is shifted to
shorted wavelength (a greater velocity offset from systemic).  
(b) The solid curve shows a stack of 100 spectra of the fiducial
profile, each degraded to a S/N=2 per pixel, and offset from systemic
by a normal deviate of $\sigma = 100 \mkms$ to mimic uncercainty in
the redshift of the galaxy.  
This treatment is meant to illustrate the
implications of stacking galaxy spectra to sutdy outflows
\citep[e.g.][]{wcp+09,sato+,rkc+10}.   The main difference from the
single galaxy observation shown in panel (a) is the smearing of
line-emission and absorption that reduces the height/depth of each.
The effects would be even more pronounced if one studied spectra with
a diversity of \ion{Mg}{2} profiles.    
}
\label{fig:obs_lris}
\end{figure}

\begin{figure}
\includegraphics[scale=0.6,angle=90]{../Figures/fig_obs_sb.ps}
\caption{
Surface brightness
}
\label{fig:obs_sb}
\end{figure}



\end{document}
