\documentclass[12pt,preprint]{aastex}
\usepackage{natbib,amsmath}

\special{papersize=8.5in,11in}
\begin{document}

\newcommand{\taud}{$\tau_{\rm dust}$}
\newcommand{\mtaud}{\tau_{\rm dust}}
\newcommand{\maconfig}{2p$^6$3s}
\newcommand{\mbconfig}{2p$^6$3p}
\newcommand{\aconfig}{a~$^6$D$^0$}
\newcommand{\zconfig}{z~$^6$D$^0$}
\newcommand{\mvr}{v_{\rm r}}
\newcommand{\naid}{\ion{Na}{1}~$\lambda\lambda 5891, 5897$}
\newcommand{\mgiid}{\ion{Mg}{2}~$\lambda\lambda 2796, 2803$}
\newcommand{\mgiia}{\ion{Mg}{2}~$\lambda 2796$}
\newcommand{\mgiib}{\ion{Mg}{2}~$\lambda 2803$}
\newcommand{\feiia}{\ion{Fe}{2}~$\lambda 2586$}
\newcommand{\feiib}{\ion{Fe}{2}~$\lambda 2600$}
\newcommand{\feiic}{\ion{Fe}{2}$^* \; \lambda 2612$}
\newcommand{\feiid}{\ion{Fe}{2}~$\lambda\lambda 2586, 2600$}
\newcommand{\nmg}{$n_{\rm Mg^+}$}
\newcommand{\mnmg}{n_{\rm Mg^+}}
\newcommand{\nfe}{$n_{\rm Fe^+}$}
\newcommand{\mnfe}{n_{\rm Fe^+}}
\def\hub{h_{72}^{-1}}
\def\umfp{{\hub \, \rm Mpc}}
\def\mzq{z_q}
\def\zabs{$z_{\rm abs}$}
\def\mzabs{z_{\rm abs}}
\def\intl{\int\limits}
\def\cmma{\;\;\; ,}
\def\perd{\;\;\; .}
\def\ltk{\left [ \,}
\def\ltp{\left ( \,}
\def\ltb{\left \{ \,}
\def\rtk{\, \right  ] }
\def\rtp{\, \right  ) }
\def\rtb{\, \right \} }
\def\sci#1{{\; \times \; 10^{#1}}}
\def \rAA {\rm \AA}
\def \zem {$z_{\rm em}$}
\def \mzem {z_{\rm em}}
\def\smm{\sum\limits}
\def \cmm  {cm$^{-2}$}
\def \cmmm {cm$^{-3}$}
\def \kms  {km~s$^{-1}$}
\def \mkms  {{\rm km~s^{-1}}}
\def \lyaf {Ly$\alpha$ forest}
\def \Lya  {Ly$\alpha$}
\def \lya  {Ly$\alpha$}
\def \mlya  {Ly\alpha}
\def \Lyb  {Ly$\beta$}
\def \lyb  {Ly$\beta$}
\def \lyg  {Ly$\gamma$}
\def \ly5  {Ly-5}
\def \ly6  {Ly-6}
\def \ly7  {Ly-7}
\def \nhi  {$N_{\rm HI}$}
\def \mnhi  {N_{\rm HI}}
\def \lnhi {$\log N_{HI}$}
\def \mlnhi {\log N_{HI}}
\def \etal {\textit{et al.}}
\def \lyaf {Lyman--$\alpha$ forest}
\def \mnmin {\mnhi^{\rm min}}
\def \nmin {$\mnhi^{\rm min}$}
\def \O {${\mathcal O}(N,X)$}
\newcommand{\cm}[1]{\, {\rm cm^{#1}}}
\def \snrlim {SNR$_{lim}$}

\title{Simple Wind Models}

\author{
J. Xavier Prochaska\altaffilmark{1}, 
Others
%John M. O'Meara\altaffilmark{2}, 
%Gabor Worseck\altaffilmark{1} 
%\& Scott Burles\altaffilmark{3}
}
\altaffiltext{1}{Department of Astronomy and Astrophysics, UCO/Lick Observatory, University of California, 1156 High Street, Santa Cruz, CA 95064}
%\altaffiltext{2}{Department of Chemistry and Physics, Saint Michael's College.
%One Winooski Park, Colchester, VT 05439}

\begin{abstract}
\begin{itemize}
\item Emission is a generic feature of (nearly) isotropic winds.  This
  emission fills in absorption at $v \approx 0$, significantly
  complicating the absorption-line analysis. 
\end{itemize}
\end{abstract}


\keywords{absorption lines -- intergalactic medium -- Lyman limit systems -- SDSS}

\section{Introduction}

\begin{itemize}
\item Usage of cool gas outflows to probe galactic-scale winds in
  star-forming galaxies is now a well-established observational
  technique
\item Simple anlaysis focusing on the central wavelength and EW of the
  lines
\item \lya, MgII, NaI and (now) FeII* show emission
\item Despite the wealth of observational diagnostics, little effor
  hast been taken to constrain the wind properties
\item These are some first steps by considering idealized wind
  profiles.  Algorithms can be extended to any wind output
\end{itemize}

\section{Methodology}
\label{sec:method}

\subsection{The Radiative Transitions}

[Table of $f,\lambda,A,\gamma$]

In this paper, we will focus on two sets of radiative transitions
associated with the Fe$^+$ and Mg$^+$ ions, a choice motivated by 
our own observations \citep{rubin10b}.  Although this is a limited
set, the two ions and their transitions have 
characteristics that resemble the properties for the majority of low-ions 
observed in cool-gas outflows. [wording still stinks] Therefore, many
of our results may be generalized to other studies.

The Mg$^+$ ion, with a single 3s electron in the ground-state,
exhibits an alkali doublet of transitions similar to the
\lya\ transitions of neutral hydrogen.  In Figure~\ref{fig:energy}
we present the energy level diagram for this 
\mgiid\ doublet.  In non-relativistic quantum
mechanics, the 2p$^6$3p level is said to be split by spin-orbit
coupling which produces the line doublet.  These are the only
\ion{Mg}{2} electric-dipole transitions 
with wavelengths near 2800\AA. Also, the transtion connecting
the $\rm {}^2P_{3/2}$ and $\rm {}^2P_{1/2}$ states is forbidden by several
selection rules.  Therefore, an absorption from
\maconfig~$\to$~\mbconfig\
is followed nearly 100$\%$ of the time by a spontaneous decay to the
ground state and our treatment will ignore any other possibilities
(e.g.\ absorption of a second photon).

The physics of the \mgiid\ doublet
is analagous to the physics of \ion{H}{1}
\lya, the \naid\ doublet, and many other doublets commonly
studied in the ISM and IGM.  These doublets differ only in
the transiton wavelengths and the energy of the doublet separation. 
For \ion{H}{1} \lya, the
separation is sufficiently small ($\Delta v = c \Delta E / E \approx
1.3 \mkms$) that most radiative transfer treatments actually ignore the fact that
there are two transitions.   This is generally a justifiable
assumption because 
most astrophysical processes have turbulent motions that
significantly exceed this velocity separation and effectively mix the
two transitions.  For \ion{Mg}{2} ($\Delta v \approx 770 \, \mkms$),  
\ion{Na}{1} ($\Delta v \approx 304 \, \mkms$), and most of the other doublets
commonly studied the separations are large and the lines
must be treated separately.  Regarding analysis of outflows, the line
separation plays an important role when the outflow speed $\mvr
\gtrsim \Delta v$.

Iron exhibits the most complex set of energy levels of the elements
frequently studied in astrophysics.  The Fe$^+$ ion alone has over XXX
energy levels recorded \citep{iron}, and even this is an
incomplete list.  
One reason for the complexity of iron is
that the majority of its configurations exhibit fine-structure splitting.
This includes the ground-state configuration (\aconfig) which is split
into 5 levels with 
excitation energies $T_{\rm ex} \equiv \Delta E / k$ ranging from
$T_{\rm ex} \approx 500-1500$\,K.  These levels are usually labelled by the total angular momentum
$J$, with the ground-state having $J=9/2$ (Figure~\ref{fig:energy}).  
Transitions between these fine-structure levels are 
forbidden (magnetic-dipole) with spontaneous decay times of several hours.  

In this paper, we will focus on transitions between the ground-state
configuration and the levels associated with the \zconfig\
configuration.  This set of transitions are named the
UV1 mutliplet with wavelengths near 2600\AA.
There are two resonance-line transitions (defined to originate in the
true ground-state) associated with this multiplet, \feiid,
corresponding to $\Delta J = 0, -1$.  The solid (green) downward
arrows in Figure~\ref{fig:energy} mark the transitions that connect to
the upper energy levels of the resonance lines.  These transitons can
occur following the absorption of a single photon by the ion in its
ground-state.  

The Figure also shows (as dashed downward arrows) two of the
transitions that connect to higher energy levels of the \zconfig\
configuration.  Radiatively, these can only occur after the absorption
of two photons: one to raise the electron from the ground-state to an
excited state and one from the excited state to one of the \zconfig\
levels with $J \le 5/2$.  The excitation of the Fe$^+$ ion by the
absorption of UV photons is termed indirect UV pumping
\citep[e.g][]{sv02,pcb06} and requires the ion to lie
near an intense source of UV photons (see $\S$~\ref{sec:pump}).  In the
following, we will assume that this does not occur and that only the 
true ground-state is populated by electrons. 

Our calculations also ignore collisional
processes\footnote{Recombination is also ignored.}, i.e.\ collisional
excitation and de-excitation of the various levels.  For the
fine-structure levels of the \aconfig\ configuration, the excitation
energies are modest ($T_{\rm ex} \sim 1000$\,K) but the critical
density $n_c$ is very large.  For the \aconfig$_{9/2} \to
$\aconfig$_{7/2}$ transition, the critical density $n_c \approx XX
\cm{-3}$.  At these densities, one would predict significant (and
detectable) quantities of \ion{Fe}{1} which has not yet been observed
\citep[see also][]{pcb06}.  The strongest arguement for neglecting
collisions is the general absence of {\it absorption} from the
fine-structure levels of the \aconfig\ configuration.  [Does Kate see
blue-shifted FeII* in absorption?]  This provides an empirical
demonstration that collsioinal excitation is insignificant, and by
inference one may neglect collisional de-excitation.  Meanwhile, the
critical density and excitation energies for \ion{Mg}{2} are $n_c =
??, T_{\rm ex} \approx 50,000$\,K.  

\subsection{The Source}

Nearly all of the absorption studies on cool-gas outflows 
have focused on intensely, star-forming galaxies.  The
intrinsic emisison of these galaxies is a complex combination of
stars and \ion{H}{2} regions that is then modulated by dust and gas
within the galaxy's ISM.  In terms of the spectral regions studied
here, most stars show a featureless continuum with a few spectral
types showing significant \ion{Mg}{2} and \ion{Fe}{2} absorption.
[Any emission??]  \ion{H}{2} regions, meanwhile, are observed to emit
at the \mgiid\ doublet, primarily due to recombinations in the outer
layers [How about FeII?].  It is beyond the scope of this paper to
properly model the 
stellar absorption and \ion{H}{2} region emission, but the reader
should be aware that they can complicate the observed spectrum,
independent of a galactic outflow.
In the following, we will simply assume a flat continuum 
normalized to unit value.  The size of the emitting
region $r_{\rm source}$, however, is a variable parameter.

[What does MgI tell us about winds?  Why no P-Cygni?]

\subsection{Monte Carlo Algorithm}

[Describe 1D and 3D algorithms]

[Put tests in an Appendix?]

\subsection{Dust}
\label{sec:dust_method}

\begin{itemize}
\item Define \taud
\item Dust scales with $n_H$
\item No wavelength dependence
\item Conversion to IR photons (not tracked)
\end{itemize}



\section{A Fiducial Wind Model}
\label{sec:fiducial}

In this section, we explore a simple yet illustrative wind model for
a galactic-scale outflow.  The properties of this wind were tuned, in
part, to yield a \ion{Mg}{2} absorption profile 
simlar to that observed toward $z<1$, star-forming galaxies
\citep{wcp+09,rubin10b}.  We emphasize, however, that we do not
favor this fiducial model over any other wind scenario nor are its
properties any more physically motivated. 
Its primary purpose is to establish a baseline for exploration.

The fiducial wind follows a density law

\begin{equation}
n_{\rm H} (r) = \frac{n^0_{\rm H}}{r^2}
\end{equation}
(mass-flux conserving) and a velocity law

\begin{equation}
v_r (r) = v_0 r
\end{equation}
which corresponds to a purely radial flow.  Turbulent motions are
characterized by a Doppler parameter $b_{\rm turb}$.  
The wind is isotropic, dust-free, and extends from an inner wind
radius $r_{\rm inner}$ to an outer wind radius $r_{\rm outer}$.  At
the center of the wind 
is a homogenous source of continuum photons with size $r_{\rm source}$.  Lastly,
we convert the hydrogen density $n_{\rm H}$ to the number density of
Mg$^+$ and Fe$^+$ ions by assuming solar relative abundances with an
absolute metallicity of 1/2 solar and depletion factors of 1/10 and
1/20 for Mg and Fe respectively, i.e.\  $\mnmg = 10^{-5.47} n_{\rm H}$ 
and \nfe=\nmg/2.   
These normalizations give a wind with a peak optical depth $\tau
>> 1$ to \mgiia\ that becomes optically thin over a velocity range
spanning $v_r \approx 100 - 1000 \, \mkms$.  
All of the parameters are summarized in Table~\ref{tab:fiducial}.  

In Figure~\ref{fig:fiducial_nvt} we plot the density and velocity
profiles against radius for the fiducial wind model;  
their single power-law expressions are evident.  The figure also
shows the optical depth profile for the \mgiia\ transition ($\tau_{2796}$).
This profile is what determines the observed absorption/emission lines
for the wind.  
Although plotted against radius, we first calculated $\tau_{2796}$ 
as a function of velocity by summing the opacity in a series of
discrete and small radial intervals
between $r_{\rm inner}$  and $r_{\rm outer}$.   We then mapped
$\tau_{2796}$ onto radius using the radial velocity profile and 
checked that this profile is in good agreement with
the Sobolev approximation.  
The $\tau_{2796}$ profile peaks at $\tau_{2796}^{\rm max} \approx 30$
at a velocity $v \approx 65 \mkms$ corresponding to $r \approx 1.3
r_{\rm inner}$.  The optical depth profile for the \mgiib\ transition
is scaled down by the ratio of $f\lambda$ but otherwise identical.  Similarly,
the optical depth profiles for the \feiid\ transitions are
scaled down by $f \lambda$ and the \nfe/\nmg\ ratio.  
The adopted Doppler parameter
(here $b_{\rm turb} = 15 \mkms$) also has a modest impact on the results,
specially at large optical depths.  A larger/smaller
$b$-value would imply ... [FILL IN]
In turn, this would affect the resulting emission profiles.

Because the results are dictated by the $\tau_{2796}$ profile and the
Doppler parameter, the actual dimensions and density of the wind are
essentially unimportant provided they scale together to give the same
$\tau_{2796}$ profile.  Therefore, one may consider our choices for
$r_{\rm inner}, r_{\rm outer}, n_{\rm H}^0$ as arbitrary.
Nevertheless, we attempted to adopt values for this fiducial model
that some astrophysical motivation.


Using the methodology described in $\S$~\ref{sec:method}, we
propogated photons from the source and through the outflow to an
`observer' at $r \gg r_{\rm outer}$ that views the entire wind+source
complex.  Figure~\ref{fig:fiducial_1d} presents the spectrum
that this observer would record with the ``unattenuated'' source flux
normalized to unit value.   The \ion{Mg}{2} profiles
show the canonical `P-Cygni profile' that characterize a continuum
source embedded within an outflow.  Strong absorption is evident at
$v < -50 \mkms$ in both transitions and each shows emission at
positive velocities.  For this isotropic and dust-free model, the
total equivalent width of the doublet must be zero
(Table~\ref{tab:fiducial_EW}), i.e.\ every photon
absorbed still escapes the system, primarily at lower
energies.  The wind simply shuffles photons in frequency space.

Focusing on the \ion{Mg}{2} absorption, one notes the profiles have
positive intensity yet similar depth even though their $f\lambda$
values differ by a factor of two.  In standard absorption line
analysis, this is generally 
the tell-tale signature of a `cloud' that has a high optical depth (i.e.\
saturated) which only partially covers the emitting source
\citep[e.g.][]{hamman10}.  Our fiducial wind model, however, 
{\it entirely} covers the source; the apparent partial covering must
be related to a different effect.
Figure~\ref{fig:noemiss}, which compares the true
absorption profiles against the case where absorbed photons are not
re-emitted, further illustrates this issue.   As expected from the
$\tau_{2796}$ profile (Figure~\ref{fig:fiducial_nvt}, the no-emission
case predicts a strong \mgiid\ doublet that absorbs all photons at
$v \approx -100 \mkms$, i.e.\ $I_{2796} = \exp(-\tau_{2796}) \approx 0$.
The actual profile, meanwhile, has been `filled in' at $v \approx -100
\mkms$ by photons scattered and re-emitted by the wind (see below).  An
absorption-line analysis that ignores these effects
\citep[e.g.][]{sato} would systematically underestimate the true optical
depth and falsely conclude partial covering.  We will find that this
behaviour is a generic results for models with a high degree of
isotropy.
[Dan's analytic calculation here? or Appendix?]

One also notes that the emission profiles of the \mgiid\ doublet
are very similar and have comparable equivalent widths.  In fact the
flux of the \mgiib\ transition even exceeds that for \mgiia.  The
emission profiles are similar because the absorption is saturated.
Furthermore, the flux from \mgiib\ is greater than that for
\mgiia\ because the wind speed exceeds the velocity separation
of the doublet, $|\mvr|_{\rm max} > (\Delta v)_{\rm MgII}$, such
that the \mgiib\ absorption profile partially absorbs the red wing of the
\mgiia\ emission profile.  The relative strengths of the emission
lines, therefore, is sensitive to both the optical depth and velocity
extent of the wind.

Now consider the \ion{Fe}{2} transitions.
The bottom left panel of Figure~\ref{fig:fiducial_1d} covers the
majority of the \ion{Fe}{2} UV1 transitions and several are
shown in the velocity plot.  The line
profile for \feiib\ is very similar to those for the \mgiid\ doublet;
one observes strong absorption to negative velocities and strong
emission at $v > 0 \mkms$ producing a P-Cygni profile.  Splitting
the profile at $v = -50 \mkms$, we measure an equivalent width
$W_{2600}^{\rm abs} = 1.16$\AA\ in absorption and $W_{2600}^{\rm em} =
-0.94$\AA\ in emission (Table~\ref{tab:fiducial_EW}) for a total
equivalent width of $W_{2600}^{\rm TOT} \approx 0.22$\AA.  
[Point out that if $\tau_{2600}$ were higher, than the photons would
likely be more trapped and more would come out as 2626.  Also a
v=0km/s component might make quite a difference.]
In contrast, the \feiia\ resonance line shows much weaker emission and
a much higher total equivalent width ($W_{2586}^{\rm TOT} = 0.49$\AA)
even though the line has a $2 \times$ lower $f\lambda$ value.
The difference comes the extra downward transition from the
\zconfig$_{7/2}$ level and (more importantly) that the Einstein A
coefficients  of the non-resonant lines match and even exceed that for
the resonant line.  In the case of \feiib, the
\ion{Fe}{2}~2626 transition 
has an approximately  $4\times$ smaller coefficient than the
resonance line.  Therefore, the majority of photons absorbed at
$\lambda 2600$ are re-emitted with the same energy and only scatter 
in frequency space around the line.  For \feiia, 
the majority of photons absorbed are re-emitted 
as longer wavelength \ion{Fe}{2}~$\lambda 2612$ or $\lambda 2632$
photons.  
The total equivalent width, however, of the lines connected to the
\zconfig$_{7/2}$ upper level does vainsh (i.e.\ photons are conserved
in this fiducial wind).

[Does the FeII 2600 P-Cygni go away if we increase $\tau_{2600}$?]


Our 3D Monte Carlo algorithm tracks the spatial emissivity of the source+wind
complex as a function of frequency.  From this output, we
constructed surface-brightness maps producing a
dataset analagous to intergral-field-unit (IFU) observations.  In
Figure~\ref{fig:fiducial_ifu_mgii}, we present the output for our
fiducial wind model at several velocities relative to the \mgiia\
transition. At $\mvr \le -200 \mkms$, where the wind has an optical
depth $\tau_{2796} < 1$ (Figure~\ref{fig:fig_fiducial_nvt}),
the source contributes significantly to the observed flux.  
At $\mvr=-100 \mkms$, however, the
wind absorbs all photons from the source and the observed emission is
entirely from photons scattered by the wind.  This scattered emission
actually exceeds the source+wind emission at 
$\mvr = -200\mkms$ such
that the absorption profile reaches peak depth at a velocity that is
negatively offset from
the peak in optical depth.  The net result of the scattered photons is
a weaker \ion{Mg}{2} profile that is offset fromt he true optical
depth profile.  Clearly, this will complicate efforts to estimate the
speed, covering fraction, and total column density of the wind.  At $v
= 0 \, \mkms$, the wind and source have comparable total flux and the
latter dominates at higher velocities.  
At $v_{\rm r} \ge 0
\mkms$,  both the source and wind contribute to the observed emission.

[Spatial cut? Yes]

Similar characteristics are observed for the \ion{Fe}{2} resonance
transitions (Figure~\ref{fig:fiducial_ifu_feii2600}).
For transitions to fine-structure levels of the \aconfig, the source
is unattenuated but there is also a significant contribution from the
wind (Figure~\ref{fig:fiducial_ifu_feii2612}).
[Show a cut and comment on the fraction of light in 1-3\,kpc, 1-10,
and 1-20; ignoring the source]

\section{Complexity}

\subsection{Anisotropic Winds}
\label{sec:anisotropic}

The fiducial model explored in the previous section assumed an
isotropic wind with variations in velocity and density only
along the radial dimension.  This is obviously an idealized case, but
one that is frequently adopted in the literature
\citep[e.g.][]{steidel+10}.   There are several reasons, however, why
the wind might not be isotropic.  For example, the sources driving the
wind (e.g.\ supernovae, an AGN) may be anisotropically distributed
within the galaxy.  It is also possible that gas within the galactic
halo preferentially suppresses the wind on one side.  [Any other
reason?]

With these considerations in mind, we reanalyzed the fiducial model
with the 3D algorithm after setting the density to zero for $2\pi$ steradians.
We then rotated our view of this system from $\theta = 0^\circ$ where
we view the source directly to $\theta = 180^\circ$ where the source
is covered by this anisotropic wind.  The resulting \ion{Mg}{2} and
\ion{Fe}{2} profiles are compared against the fiducial model
(isotropic wind) in Figure~\ref{fig:anisotropic}.  

Examining the \mgiid\ doublet first, 
there is no attenutation of the source by the wind for $\theta =
0^\circ$
but one does observe significant line emission from photons scattered
off the back side.  These, by definition, have $v \gtrsim 0 \mkms$
relative to line-center with a modest emission at $v<0 \mkms$ related
to turbulent motions within the wind ($b_{\rm turb} = 15 \mkms$).
When viewed from the opposite direction ($\theta = 180^\circ$), the
absorption lines dominate althrough there is still signficant
line-emission at $v \approx 0 \mkms$ (at at $v < 0 \mkms$ which fills
in the absorption) from photons that scatter through the wind.  The
key difference from the isotropic wind is the absence of photons
scatterd to $v > 100 \mkms$.  In this respect, the emission lines
provide a (robust?) diagnostic for wind isotropy.  [Also note the
greater depth of MgII 2803?]
For $\theta = 90^\circ$, the wind covers one half of the source and
one observes half the absorption and also emission at all velocities.

The results are very similar for the \feiid\ resonance lines.  The
\ion{Fe}{2}$^* \; \lambda 2612$ line, meanwhile, shows clearly the
separation in velocity for the line-emission.   [Anything else?]


[Does the emission extend to +1000\kms?  Is it a good tracer of the of
the dynamics (unlike \lya)? Good for Obs Section]

\subsection{Dust}

Essentially all astrophysical environments that contain both cool gas
and metals also show signatures of dust depletion and extinction.  This includes the
ISM of star-forming and \ion{H}{1}-selected galaxies
\citep[e.g.][]{ss96,pw01,pcd+07}, strong \ion{Mg}{2} metal-line
absorption systems \citep{york,menard}, and the galactic winds traced
by low-ion transitions \citep{cb58,naI}.  Although the galactic winds
traced by \ion{Mg}{2} and \ion{Fe}{2} transitions have not (yet) been
demonstrated to contain dust, it is reasonable to include at this
stage.  

Dust will affect our results for the line profiles in two respects.
First, it serves as an opacity for the photons
emitted by the source.  This will suppress the flux at all
wavelengths by $\approx \exp{-\mtaud}$, but because we are normalizing the profiles to the
continuum this is essentially ignored.  Second, photons that are
scattered by the wind (once ore multiple times) into our view will travel a greater
distance and suffer from greater extinction.  A photon that is
trapped for many scatterings may have a high probability of absorbed
by dust and leave the system as IR photons.  Indeed, this process is
often the explanation given for the weak (or absent) \lya\ emission
associated with star-forming galaxies \citep[e.g.][]{lya}.
Section~\ref{sec:dust_method} describes our treatment of dust; we only
remind the reader here that the opacity is assumed to scale with the
gas density and that we normalized the extinction by the total optical
depth $\tau_d$ that a photon would experience if it travelled from the
source to infinity without scattering. 

In Figure~\ref{fig:dust} we show the \ion{Mg}{2} and \ion{Fe}{2}
profiles of the fiducial model ($\mtaud = 0$) against a series of
models with $\mtaud > 0$.  For the \ion{Mg}{2} transitions, the
dominant effect is the suppression of the line emission at $v \ge 0
\mkms$.  These `red' photons, of course, were scattered off the
backside of the wind and must travel a longer path than the other
photons.  This differnetial reddening with increasing wavelength (and
velocity relative to line-center) is a natural consequence of dust
extinction and is most evident in the \ion{Fe}{2}$^* \; \lambda 2612$
emission profile.   Only for $\tau_d = 10$ does one observe a
significant difference in the absorption profiles, specifically a
greater degree of apparent absorption at $v \approx -150 \mkms$.  At
such high levels of extinction, even the photons off the frontside of the
wind are preferentially absorbed relative to the unabsorbed photons.
Such a high dust opacity, however, implies the source itself is
extinguished by 15\,magnitudes and it would never be observed. 
Even $\mtaud = 3$ would be much larger than typically inferred for the
star-forming galaxies that drive outflows \citep[e.g.][]{dust}.

To summarize,  dust is likely to have only a modest effect on
\ion{Mg}{2} and \ion{Fe}{2} transitions which have themselves only
modest optical depths.  [How big would $\tau_{\rm MgII}$ have to get?]
The dominant effect is a reduction in the flux of the emission lines
with greater extinction at higher velocities relative to line-center.
In these respects, this roughly mimics the behavior of the anistropic
wind described in Section~\ref{sec:anisotropic}. [Is there a
distinguishing factor, e.g. flux of 2612/2600??]

[Do again for fiducial but with a higher $n_H$?  Do for other density
laws, but maybe not show]


\section{Discussion}

[Are MgII and FeII* emission strict predictions of outflows?  People
have been so focused on absorption.]

\acknowledgments

J. X. P. and J.M.O. are supported by NASA grant
HST-GO-10878.05-A.  J.X.P and G.W. are partially supported
by an NSF CAREER grant (AST--0548180), and 
by NSF grant AST-0908910.


\bibliographystyle{/u/xavier/paper/Bibli/apj}
\bibliography{/u/xavier/paper/Bibli/allrefs}

 
 
\begin{deluxetable}{ccl}
\tablewidth{0pc}
\tablecaption{Wind Parameters: Fiducial Model\label{tab:fiducial}}
\tabletypesize{\footnotesize}
\tablehead{\colhead{Property} & \colhead{Parameter} & \colhead{Value} } 
\startdata
Density law  & $n(r)$ & $\propto r^{-2}$ \\
Velocity law  & $v_r$ & $ \propto r$ \\
Inner Radius & $r_{\rm inner}$ & 1\,kpc \\
Outer Radius & $r_{\rm outer}$ & 20\,kpc \\
Source size  & $r_{\rm source}$ & 0.5\,kpc \\
Density Normalization & $n^0_{\rm H}$ & $0.1\cm{-3}$ at $r_{\rm inner}$ \\
Velocity Normalization & $v^0$ & 50\kms at $r_{\rm inner}$ \\
Turbulence   & $b_{\rm turb}$  & 15 \kms \\
Mg$^+$ Normalization & \nmg\ & $10^{-5.47} n_{\rm H}$ \\
Fe$^+$ Normalization & \nfe\ & \nmg/2 \\
\enddata
%\tablecomments{Unless specified otherwise, all quantities refer to the \sna=2 threshold.  The cosmology assumed has $\Omega_\Lambda = 0.7, \Omega_m = 0.3$, and $H_0 = 72 \mkms \rm Mpc^{-1}$.}
%\tablenotetext{a}{Total redshift survey path for the \sna=2 criterion.}

\end{deluxetable}

\input{Tables/tab_fiducial_ew.tex}

\begin{figure}
\epsscale{0.8}
\plotone{Figures/energy_levels.ps}
\caption{
Energy level diagrams for the \mgiid\ doublet and the UV1
multiplet of \ion{Fe}{2} transitions   
(based on Figure~7 from \cite{hmt+99}).
Each transition shown is
labelled by its rest wavelength (\AA) and Einstein A-coefficient
(s$^{-1}$). Black updward arrows
indicate the resonance-line transitions connected to the ground
state of each ion.  The 2p$^6$3p configuration of Mg$^+$ is split into
two energy levels that give rise to the \mgiid\ doublet.  
Both the 3d$^6$4s ground state and 3d$^6$4p upper level of Fe$^+$
exhibit fine-structure splitting that gives rise to a series of
electric-dipole transitions. 
The downward (green) arrows show the transitions connected to the
resonance-line transitions (i.e.\ they share the same upper energy
levels).  We also show a pair of levels (\ion{Fe}{2}~$\lambda\lambda
2618,2631$) that arise from higher levels in the \zconfig\
configuration.  These transitions have not yet been observed in
galactic-scale outflows and are not considerd in our analysis.
}
\label{fig:energy}
\end{figure}

\begin{figure}
\epsscale{0.8}
\plotone{Figures/fig_nvtau_vs_r.ps}
\caption{
Density (solid; black), radial velocity (dotted; blue), and
\mgiia\ optical depth profiles (dashed; red) for the fiducial
wind model (see Table~\ref{tab:fiducial} for details).
The density and velocity profiles are simple $r^{-2}$ and $r$
power-laws.  The optical depth profile was calculated by summing
the opacity at small and discrete radial intervals.  One notes that
the wind is optically thick at the inner radius ($r_{\rm inner} =
1$\,kpc) and becomes optically thin at the outer radius ($r_{\rm
  outer} = 20$\,kpc).
Note that the density and velocity curves have been scaled for plotting
convenience.  
}
\label{fig:fiducial_nvt}
\end{figure}

\begin{figure}
\epsscale{0.8}
\plotone{Figures/fig_fiducial_1d.ps}
\caption{
{\it Left} -- (upper) \ion{Mg}{2} profiles for the fiducial wind model
described in Table~\ref{tab:fiducial} and
Figure~\ref{fig:fiducial_nvt}.  The doublet shows the P-Cygni profiles
characteritic of an outflow with significant absorption blueward of
line center (dashed vertical lines) extending to $v = -1000\mkms$
and signficiant emission redward of line center.  Note
that even though the peak optical depth of the \ion{Mg}{2} transitions
is nearly 30 at $v \approx -70 \mkms$, photons scattered off the outflow
fill in the absorption.
(lower) \ion{Fe}{2} absorption and emission profiles for the UV1
multiplet at $\lambda \approx 2600$\AA.  The \feiid\ resonance lines 
show weaker absorption due to the smaller Fe$^+$ number density and
lower $f\lambda$ values.  Each also shows a P-Cygni profile, although
the emission for \feiia\ is significantly weaker than that of the
\feiib\ and \mgiid\ transitions.  This is because a majority of the
absorbed \feiia\ photons are converted into
\ion{Fe}{2}~$\lambda\lambda 2612, 2632$ photons as observed.
[Mention the emission kinematics?]
}
\label{fig:fiducial_1d}
\end{figure}

\begin{figure}
\epsscale{0.8}
\plotone{Figures/fig_noemiss.ps}
\caption{
The black curves show the line profiles (absorption and emission) of
the \ion{Mg}{2} and \ion{Fe}{2} resonance lines for the fiducial wind
model.  These shows the canonical P-Cygni profiles of a source
embedded within an outflow.  Overplotted (red) on each transition is
the predicted absorption profile under the constraint that each
absorbed photon is lost from the system, i.e.\ no scattering or
re-emission.   It is evident that the true profiles have been `filled
in' significantly by light scattered in the wind.  Ignoring this
process, one would model the absorption lines with systematicall lower
optical depth and derive a partial covering. [word]
}
\label{fig:noemiss}
\end{figure}

\begin{figure}
\epsscale{0.8}
\plotone{Figures/fig_fiducial_ifu_mgii.ps}
\caption{
Surface-brightness emission maps around the \mgiia\ transition for the
source+wind complex.  The middel panel shows the 1D spectrum with $v=0
\mkms$ corresponding to $\lambda = 2796.35$\AA\ and the dotted vertical
curves indicating the velocity slices for the emission maps.  The
source has a size $r_{\rm source} = 0.5$\,kpc, traced by a few
pixels at the center.   At $\mvr=-200 \mkms$, the wind has an optical
depth of $\tau_{2796} \approx 1$ such that the source contributes
significantly to the observed flux.  At $\mvr=-100 \mkms$, however, the
wind absorbs all photons from the source and the observed emission is
entirely due to photons scattered by the wind.  Surprisingly, this
emission exceeds that observed at $\mvr = -200\mkms$. At $\mvr \ge 0
\mkms$,  both the source and wind contribute to the observed emission.
}
\label{fig:fiducial_ifu_mgii}
\end{figure}

\begin{figure}
\epsscale{0.8}
\plotone{Figures/fig_fiducial_ifu_feii2600.ps}
\caption{
Surface-brightness emission maps around the \feiib\ transition for the
source+wind complex.  The middle panel shows the 1D spectrum with $v=0
\mkms$ corresponding to $\lambda = 2600.173$\AA\ and the dotted vertical
curves indicating the velocity slices for the emission maps.  
The results are very similar to those observed for the \mgiid\ doublet
(Figure~\ref{fig:fiducial_ifu_mgii}).
}
\label{fig:fiducial_ifu_feii2600}
\end{figure}

\begin{figure}
\epsscale{0.8}
\plotone{Figures/fig_fiducial_ifu_feii2612.ps}
\caption{
Surface-brightness emission maps around the \ion{Fe}{2}~$\lambda 2612$ transition for the
source+wind complex.  The middle panel shows the 1D spectrum with $v=0
\mkms$ corresponding to $\lambda = 2612.654$\AA\ and the dotted vertical
curves indicating the velocity slices for the emission maps.  
In this case, the source is unattenuated yet scattered photons from
the wind also have a significant contribution. 
}
\label{fig:fiducial_ifu_feii2612}
\end{figure}

\begin{figure}
\epsscale{0.8}
\plotone{Figures/fig_asymm_spec.ps}
\caption{
Profiles of the \ion{Fe}{2} and \ion{Mg}{2} profiles for the fiducial
case (black lines) compared against an anisotropic wind blowing into
only $2\pi$ steradians viewed from $\theta = 0^\circ$ (source
uncovered) to $\theta = 180^\circ$ (source covered).  One detects
significant emission for all orientations of this wind with the
emission shifting from positive to negative velocities as $\theta$
increases.  
}
\label{fig:anisotropic}
\end{figure}

\begin{figure}
\epsscale{0.8}
\plotone{Figures/fig_dust_spec.ps}
\caption{
Profiles of the \ion{Fe}{2} and \ion{Mg}{2} profiles for the fiducial
model (black) against a series of models that include the effects of
dust extinction.  The primary difference is the suppression of line
emission relative to the continuum (again, normalized to unit value).
A more subtle but important effect is that the redder photons in the
emission lines (corresponding to higher velocity relative to
line-center) suffer greater extinction.  This is most evident in the
\feiic\ emission line and is due to the fact that these photons must
travel further by scattering off the backside of the wind.  Note that
the absorption lines are nearly unmodified until $\mtaud = 10$, a
level of extinction that would preclude observing the source
altogether.
}
\label{fig:dust}
\end{figure}


\end{document}
