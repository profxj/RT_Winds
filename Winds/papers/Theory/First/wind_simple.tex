\documentclass[12pt,preprint]{aastex}
\usepackage{natbib,amsmath}

\special{papersize=8.5in,11in}
\begin{document}

\newcommand{\bturb}{$b_{\rm turb}$}
\newcommand{\mbturb}{b_{\rm turb}}
\newcommand{\taud}{$\tau_{\rm dust}$}
\newcommand{\mtaud}{\tau_{\rm dust}}
\newcommand{\maconfig}{2p$^6$3s}
\newcommand{\mbconfig}{2p$^6$3p}
\newcommand{\aconfig}{a~$^6$D$^0$}
\newcommand{\zconfig}{z~$^6$D$^0$}
\newcommand{\mvr}{v_{\rm r}}
\newcommand{\naid}{\ion{Na}{1}~$\lambda\lambda 5891, 5897$}
\newcommand{\mgiid}{\ion{Mg}{2}~$\lambda\lambda 2796, 2803$}
\newcommand{\mgiia}{\ion{Mg}{2}~$\lambda 2796$}
\newcommand{\mgiib}{\ion{Mg}{2}~$\lambda 2803$}
\newcommand{\feiia}{\ion{Fe}{2}~$\lambda 2586$}
\newcommand{\feiib}{\ion{Fe}{2}~$\lambda 2600$}
\newcommand{\feiic}{\ion{Fe}{2}$^* \; \lambda 2612$}
\newcommand{\feiie}{\ion{Fe}{2}$^* \; \lambda 2626$}
\newcommand{\feiid}{\ion{Fe}{2}~$\lambda\lambda 2586, 2600$}
\newcommand{\nmg}{$n_{\rm Mg^+}$}
\newcommand{\mnmg}{n_{\rm Mg^+}}
\newcommand{\nfe}{$n_{\rm Fe^+}$}
\newcommand{\mnfe}{n_{\rm Fe^+}}
\def\hub{h_{72}^{-1}}
\def\umfp{{\hub \, \rm Mpc}}
\def\mzq{z_q}
\def\zabs{$z_{\rm abs}$}
\def\mzabs{z_{\rm abs}}
\def\intl{\int\limits}
\def\cmma{\;\;\; ,}
\def\perd{\;\;\; .}
\def\ltk{\left [ \,}
\def\ltp{\left ( \,}
\def\ltb{\left \{ \,}
\def\rtk{\, \right  ] }
\def\rtp{\, \right  ) }
\def\rtb{\, \right \} }
\def\sci#1{{\; \times \; 10^{#1}}}
\def \rAA {\rm \AA}
\def \zem {$z_{\rm em}$}
\def \mzem {z_{\rm em}}
\def\smm{\sum\limits}
\def \cmm  {cm$^{-2}$}
\def \cmmm {cm$^{-3}$}
\def \kms  {km~s$^{-1}$}
\def \mkms  {\, {\rm km~s^{-1}}}
\def \lyaf {Ly$\alpha$ forest}
\def \Lya  {Ly$\alpha$}
\def \lya  {Ly$\alpha$}
\def \mlya  {Ly\alpha}
\def \Lyb  {Ly$\beta$}
\def \lyb  {Ly$\beta$}
\def \lyg  {Ly$\gamma$}
\def \ly5  {Ly-5}
\def \ly6  {Ly-6}
\def \ly7  {Ly-7}
\def \nhi  {$N_{\rm HI}$}
\def \mnhi  {N_{\rm HI}}
\def \lnhi {$\log N_{HI}$}
\def \mlnhi {\log N_{HI}}
\def \etal {\textit{et al.}}
\def \lyaf {Lyman--$\alpha$ forest}
\def \mnmin {\mnhi^{\rm min}}
\def \nmin {$\mnhi^{\rm min}$}
\def \O {${\mathcal O}(N,X)$}
\newcommand{\cm}[1]{\, {\rm cm^{#1}}}
\def \snrlim {SNR$_{lim}$}

\title{Simple Wind Models}

\author{
J. Xavier Prochaska\altaffilmark{1}, 
Others
%John M. O'Meara\altaffilmark{2}, 
%Gabor Worseck\altaffilmark{1} 
%\& Scott Burles\altaffilmark{3}
}
\altaffiltext{1}{Department of Astronomy and Astrophysics, UCO/Lick Observatory, University of California, 1156 High Street, Santa Cruz, CA 95064}
%\altaffiltext{2}{Department of Chemistry and Physics, Saint Michael's College.
%One Winooski Park, Colchester, VT 05439}

\begin{abstract}
\begin{itemize}
\item Emission is a generic feature of (nearly) isotropic winds.  This
  emission fills in absorption at $v \approx 0$, significantly
  complicating the absorption-line analysis, especially of an ISM
  component. 
\item The relative strengths of the emission
lines, therefore, is sensitive to both the opacity and velocity
extent of the wind.
\end{itemize}
\end{abstract}

\keywords{absorption lines -- intergalactic medium -- Lyman limit systems -- SDSS}

\section{Introduction}

Nearly all gaseous objects in astrophysics that emit light are also
observed to drive gas away.  [word] This includes the jets of protostars, the
strong stellar winds of massive O and B stars, the gentle Solar wind
of our Sun, the associated absorption of bright quasars, and the
spectacular jets of radio-loud AGN.   These gaseous outflows moderate
the accretion of fresh material onto the object, inject energy and
momentum into gas on large scales, and ...
[Last line]

Star-burst galaxies, whose light is dominated by \ion{H}{2} regions
and massive stars, are also observed to drive gaseous outflows.  The
flows are generally expected (and sometimes observed) to have multiple
phases, with a hot and diffuse phase traced by X-ray emission
\citep[e.g.][]{strickland,martin+k} and a cool, denser phase traced by
H$\alpha$ emission \citep[e.g.][]{NGC891,M87}.  

Beyond the local universe, galactic outflows are predominatly
revealted by UV and optical absorption lines, e.g.\ \ion{Na}{1},
\ion{Mg}{2}, \ion{Si}{2} and \ion{C}{4} transitions.  With the galaxy
as a backlight, one almost universally observes predominantly
blue-shifted material indicating a galactic-scale flow of gas toward
Earth and away from the galaxy.  These transitions are sensitive to
the cool (\ion{Mg}{2}; $T \sim 10^4$K) and warm (\ion{C}{4}; $T \sim
10^5$K) phases of the flow.  The incidence of cool gas outflows is
nearly universal in star-forming galaxies;  this includes systems at $z \sim 0$
which exhibit \ion{Na}{1} and \ion{Mg}{2} absorption
\citep{rupke,martin,student}, $z \sim 1$ star-forming galaxies traced by
\ion{Fe}{2} and \ion{Mg}{2} transitions \citep{wcp+09,rubin+10}, and
$z>2$ Lyman break galaxies (LBGs) that show blue-shifted \ion{Si}{2},
\ion{C}{2}, and \ion{O}{1} transitions \citep{lowenthal,steidel94}.

Although the obsevation of metal-line absorption is a well established
means of identifying outflows at all epochs, little effort has been
made to compare these observations against (even idealized) wind
models.  The focus to date has been to gelan the limited information
affored by direct analysis of the absorption lines.  The data do yield
robust measurements of the speed of the gas yet poorly constrain the
optical depth, covering fraction, density, temperature, and distance
of the gas from the emitting regions.  Constraints related to the
mass, energetics, and momentum of the flow suffer from
orders of magnitude uncertainty.  Both the origin and impact of the
winds, therefore, remain open points of debate.

Recent studes of $z \sim 1$ star-forming galaixes have revealed that
the cool gas frequently exhibits significant \ion{Mg}{2} emission in
addition to the ubiquitous blue-shifted absorption
\citep{wcp+09,rubin+10b}.  \cite{rubin+10a} have also detected line
emission from non-resonance \ion{Fe}{2}~$^*$ transitions and spatially
extended \ion{Mg}{2} emission which they interpret as originating from
photons scattered off a galactic-scale wind.  They estimate that the
wind extends to at least 5\,kpc from the galaxy.  Line emission is
also observed for $z \sim 3$ LBGs via the resonant \lya\ transition
and fine-strcuture \ion{Si}{2}$^*$ transitions \citep{cb58,shapley}.
\cite{rubin+10a} emphasize that a comprehensive analysis of these data
(e.g.\ via integral-field-unit observations) could yield unique
diagnostics on the morphology and density of the outflow, eventually
yielding tigheter constrains on the energetics of the flow.  
The crucial piece missing from this path is a radiative transfer
treatment that properly ...

In this paper, we take the first steps toward modeling the absorption
and emission properties of cool gas outflows.  Using Monte Carlo
radiative transfer technqiues, we study the nature of \ion{Mg}{2} and
\ion{Fe}{2} absorption and emission for winds with a range of
properties.  Although the winds are idealized, the results frequently
contradict our initial intuition and caution one [word] on the
challenges of converting the observables to (even crude) physical
constraints.  [Add another line or two]

[Outline of paper]


[Point out IR emission from winds locally implies dust]

\section{Methodology}
\label{sec:method}

This section describes the methodology [employed] to generate
emission/absorption profiles from simple wind models.

\subsection{The Radiative Transitions}

In this paper, we focus on two sets of radiative transitions
arising from Fe$^+$ and Mg$^+$ ions
(Table~\ref{tab:atomic},Figure~\ref{fig:energy}).
Although this is a necessarily limited
set, the two ions and their transitions are characteristic
of the majority of low-ions 
observed in cool-gas outflows. Therefore, many
of the results may be generalized to observational studies that
consider other atoms and ions of cool gas.

The Mg$^+$ ion, with a single 3s electron in the ground-state,
exhibits an alkali doublet of transitions at $\lambda \approx
2800$\AA\ similar to the
\lya\ transitions of neutral hydrogen.  Figure~\ref{fig:energy}
presents the energy level diagram for this 
\mgiid\ doublet.  In non-relativistic quantum
mechanics, the 2p$^6$3p energy level is said to be split by spin-orbit
coupling which yields the observed line doublet.  These are the only
\ion{Mg}{2} electric-dipole transitions 
with wavelengths near 2800\AA\ and the transtion connecting
the $\rm {}^2P_{3/2}$ and $\rm {}^2P_{1/2}$ states is forbidden by several
selection rules.  Therefore, an absorption from
\maconfig~$\to$~\mbconfig\
is followed essentially 100$\%$ of the time by a spontaneous decay
($t_{\rm decay} \approx 4\sci{-9}$s to the
ground state. Our treatment will ignore any other possibilities
(e.g.\ absorption by a second photon when the electron is at the \mbconfig\ level).

The physics of the \mgiid\ doublet
is analagous to the physics of \ion{H}{1}
\lya, the \naid\ doublet, and many other doublets commonly
studied in the interstellar medium (ISM) of distant galaxies.  The
doublets differ only in 
their rest wavelengths and the energy of the doublet separation. 
For \ion{H}{1} \lya, the
separation is sufficiently small ($\Delta v = c \Delta E / E \approx
1.3 \mkms$) that most radiative transfer treatments actually ignore that
there are two transitions.   This is generally justifiable for \lya\ because 
most astrophysical processes have turbulent motions that
significantly exceed its velocity separation and effectively mix the
two transitions.  For \ion{Mg}{2} ($\Delta v \approx 770 \mkms$),  
\ion{Na}{1} ($\Delta v \approx 304 \mkms$), and most of the other doublets
commonly observed, the separation is large and the transitions
must be treated separately.  Regarding the analysis of outflows, the line
separation plays an important role when the outflow speed matches or
exceeds $\Delta v$.  [Explain?]

Iron exhibits the most complex set for energy levels of the elements
frequently studied in astrophysics.  The Fe$^+$ ion alone has over XXX
energy levels recorded \citep{iron}, and even this is an
incomplete list.  
One reason for iron's complexity is
that the majority of its configurations exhibit fine-structure splitting.
This includes the ground-state configuration (\aconfig) which is split
into 5 levels with 
excitation energies $T_{\rm ex} \equiv \Delta E / k$ ranging from
$T_{\rm ex} \approx 500-1500$\,K, labelled by the total angular momentum
$J$. The ground-state has $J=9/2$ (Figure~\ref{fig:energy}).  
Transitions between these fine-structure levels are 
forbidden (magnetic-dipole) with spontaneous decay times of several hours.  

In this paper, we examine transitions between the ground-state
configuration and the energy levels of the \zconfig\
configuration.  This set of transitions (named the
UV1 mutliplet) have wavelengths near 2600\AA.
There are two resonance-line transitions\footnote{We adopt the
  standard convention that a ``resonance line'' is an electric-dipole
  transition connected to the ground-state.} 
associated with this multiplet (\feiid)
corresponding to $\Delta J = 0, -1$.  The solid (green) downward
arrows in Figure~\ref{fig:energy} mark the transitions that connect to
the upper energy levels of the resonance lines.  These transitons can
occur following the absorption of a single photon by Fe$^+$ in its
ground-state.  

The Figure also shows (as dashed downward arrows) two of the
transitions that connect to higher energy levels of the \zconfig\
configuration.  Ignoring collisions and recombinations, these
transitions may only can only occur after the absorption
of two photons: one to raise the electron from the ground-state to an
excited state and one from the excited state to one of the \zconfig\
levels with $J \le 5/2$.  The excitation of these levels by
the absorption of UV photons is termed indirect UV pumping
\citep[e.g][]{sv02,pcb06} and requires the ion to lie
near an intense source of UV photons (see $\S$~\ref{sec:pump}).  In the
following, we will assume that this does not occur. 

Our calculations also ignore collisional
processes\footnote{Recombination is also ignored.}, i.e.\ collisional
excitation and de-excitation of the various levels.  For the
fine-structure levels of the \aconfig\ configuration, the excitation
energies are modest ($T_{\rm ex} \sim 1000$\,K) but the critical
density $n_c$ is very large.  For the \aconfig$_{9/2} \to
$\aconfig$_{7/2}$ transition, the critical density $n_e^C \approx 4
\sci{5} \cm{-3}$.  At these densities, one would predict  
detectable quantities of \ion{Fe}{1} which has not yet been observed
in galactic-scale outflows
\citep[see also][]{pcb06}.  Furthermore, observations rarely show
{\it absorption} from the
fine-structure levels of the \aconfig\ configuration and that material
is not significantly blue-shifted \citep{rubin+10}.
Empirically, collisional excitation is insignificant and, by
inference, one may neglect collisional de-excitation.  
For Mg$^+$,$T_{\rm ex} \approx 50,000$\,K  
$n_e^C \approx 3 \sci{14} \cm{-3}$  [Maybe critical density is
not so interesting here...]
implying negligible collisional excitation for the cool gas.  
This implies the gas has zero opacity to the non-resonant lines.

\subsection{The Source}

Nearly all the absorption studies of cool-gas outflows 
have focused on intensely, star-forming galaxies.  The
intrinsic emisison of these galaxies is a complex combination of
stars and \ion{H}{2} regions that is then modulated by dust and gas
within the ISM.  For the spectral regions studied
here, most stars show a featureless continuum but a few spectral
types do show significant \ion{Mg}{2} and \ion{Fe}{2} absorption.
[Any emission??]  \ion{H}{2} regions, meanwhile, are observed to emit
at the \mgiid\ doublet, primarily due to recombinations in the outer
layers [How about FeII?].  It is beyond the scope of this paper to
properly model the 
stellar absorption and \ion{H}{2} region emission, but the reader
should be aware that they can complicate the observed spectrum,
independent of the galactic outflow, especially at velocities $v \approx 0 \mkms$.
In the following, we will simply assume a flat continuum 
normalized to unit value.  The size of the emitting
region $r_{\rm source}$ is a free parameter, although we tend to
restrict its value to be smaller than the minimum radial extent of any
gaseous component.

[What does MgI tell us about winds?  Why no P-Cygni?]

\subsection{Monte Carlo Algorithm}

[Describe 1D and 3D algorithms]

[Put tests in an Appendix?]

\subsection{Dust}
\label{sec:dust_method}

For the majority of models studied in this paper, we assume no dust.
This is an invalid assumption, espeically for material associated with
the ISM of the galaxy.  Because absorption line analysis is performed
on normalized spectra, the effects of dust are largely minimized.  For
scattered and resonantly trapped photons, however, the effects of dust
extinction can be significant.  Indeed, dust is frequently invoked to
explain the weak (or absent) \lya\ emission from star-forming galaxies
\citep[e.g.][]{shapley03}.  Although the transitions studied here have
significantly lower opacity than \lya, dust may play an important role
in the predicted profiles.

In a few models, we include the absorption by dust under the following
assumptions:
(i) the dust opacity scales with the density of the gas (i.e.\ we
adopt a fixed dust-to-gas ratio);
(ii) the opacity is independent of wavelength, a reasonable
approximation given the small spectral range analyzed;
(iii) the photons absorbed by dust are re-emitted at IR wavelengths
and `lost' from the system.  The dust absorption is normalized by \taud, 
the integrated opacity of dust from the center of the system to
infinity.  The ISM of a star-forming galaxy may be expected to exhibit
\taud\ of a few \citep[e.g.][]{steidel98}.



\section{The Fiducial Wind Model}
\label{sec:fiducial}

In this section, we study a simple yet illustrative wind model for
a galactic-scale outflow.  The properties of this wind were tuned, in
part, to yield a \ion{Mg}{2} absorption profile 
similar to that observed toward $z<1$, star-forming galaxies
\citep{wcp+09,rubin10b}.  We emphasize, however, that we do not
favor this fiducial model over any other wind scenario nor do its
properties have special physical motivation.
Its primary purpose is to establish a baseline
for discussion.

The fiducial wind follows a (mass-fluxing conserving) density law,

\begin{equation}
n_{\rm H} (r) = \frac{n^0_{\rm H}}{r^2} \;\;\; , 
\label{eqn:density}
\end{equation}
and a velocity 

\begin{equation}
\vec v = v_r (r) \hat r = v_0 r \hat r \;\;\; , 
\label{eqn:vel}
\end{equation}
i.e., a purely radial flow.  Turbulent motions are
characterized by a Doppler parameter $b_{\rm turb}$.  
The wind is isotropic, dust-free, and extends from an inner wind
radius $r_{\rm inner}$ to an outer wind radius $r_{\rm outer}$.  
We scale the hydrogen density $n_{\rm H}$ to the number densities of
Mg$^+$ and Fe$^+$ ions by assuming solar relative abundances with an
absolute metallicity of 1/2 solar and depletion factors of 1/10 and
1/20 for Mg and Fe respectively, i.e.\  $\mnmg = 10^{-5.47} n_{\rm H}$ 
and \nfe=\nmg/2. At the center of the wind is a homogenous source of
continuum photons with a size $r_{\rm source}$. These parameters are
summarized in Table~\ref{tab:fiducial}.   

In Figure~\ref{fig:fiducial_nvt} we plot the density and velocity
profiles against radius for the fiducial wind model;  
their single power-law expressions are evident.  The figure also
shows the optical depth profile for the \mgiia\ transition ($\tau_{2796}$).
Although plotted against radius, we first calculated $\tau_{2796}$ 
as a function of velocity by summing the opacity in a series of
discrete and small radial intervals
between $r_{\rm inner}$  and $r_{\rm outer}$.   We then mapped
$\tau_{2796}$ onto radius using the velocity law
(Equation~\ref{eqn:vel}). 
The $\tau_{2796}$ profile peaks at $\tau_{2796}^{\rm max} \approx 30$
with a velocity $v \approx 65 \mkms$ corresponding to $r \approx 1.3
r_{\rm inner}$.  The optical depth profile for the \mgiib\ transition
is scaled down by the $f\lambda$ ratio but otherwise identical.  Similarly,
the optical depth profiles for the \feiid\ transitions are
scaled down by $f \lambda$ and the \nfe/\nmg\ ratio.  
The adopted Doppler parameter
($\mbturb = 15 \mkms$) has only a modest impact on the results.
%%%%%%%%%%%%%%%%%%%%%%%%%%%%%%
The absorption profiles are insensitive to \bturb\ and this parameter 
only tends to modify the widths and modestly
shift the centroids of emission lines.

Because the results are dictated by the $\tau_{2796}$ profile in
velocity space, the actual dimensions and density of the wind are
essentially unimportant provided they scale together to give nearly the same
$\tau_{2796}$ velocity profile.  Therefore, one may consider the choices for
$r_{\rm inner}, r_{\rm outer}, n_{\rm H}^0$ as arbitrary.
Nevertheless, we adopted values for this fiducial model with
some astrophysical motivation,  e.g., values that correspond to
galactic dimensions and a normalization that gives $\tau_{2796}^{\rm
  max} \sim 10$.


Using the methodology described in $\S$~\ref{sec:method}, we
propogated photons from the source and through the outflow to an
`observer' at $r \gg r_{\rm outer}$ that views the entire wind+source
complex.  Figure~\ref{fig:fiducial_1d} presents the 1D spectrum
that this observer would record, with the unattenuated flux
normalized to unit value.   The \ion{Mg}{2} doublet
shows the canonical `P-Cygni profile' that characterizes a continuum
source embedded within an outflow.  Strong absorption is evident at
$v < -50 \mkms$ in both transitions (equivalent widths $W_{2796} =
2.96$\AA\ and $W_{2803} = 1.29$\AA) and each shows emission at
positive velocities.  For this isotropic and dust-free model, the
total equivalent width of the doublet must be zero
(Table~\ref{tab:fiducial_EW}), i.e.\ every photon
absorbed eventually escapes the system, typically at lower
energy.  The wind simply shuffles the photons in frequency space.

Focusing further on the \ion{Mg}{2} absorption, one notes the profiles lie
well above zero intensity and have similar depth even though their $f\lambda$
values differ by a factor of two.  In standard absorption line
analysis, this is 
the tell-tale signature of a `cloud' that has a high optical depth (i.e.\
saturated) which only partially covers the emitting source
\citep[e.g.][]{hamman10}.  Our fiducial wind model, however, 
{\it entirely covers the source}; the apparent partial covering must
be related to an alternate effect.
Figure~\ref{fig:noemiss} further emphasizes this point by comparing the 
absorption profiles from Figure~\ref{fig:fiducial_1d} against an
artificial model where no absorbed photons are not
re-emitted.   As expected from the
$\tau_{2796}$ profile (Figure~\ref{fig:fiducial_nvt}), the
`no-emission' model
produces a strong \mgiid\ doublet that absorbs all photons at
$v \approx -100 \mkms$, i.e.\ $I_{2796}^{\rm min} = \exp(-\tau^{\rm
  max}_{2796}) \approx 0$.
The fiducial model, in contrast, has been `filled in' at $v \approx -100
\mkms$ by photons scattered and re-emitted by the wind.  An
absorption-line analysis that ignores these effects
\citep[e.g.][]{sato} would (i) systematically underestimate the true optical
depth and (ii) falsely conclude that the wind partially covers the
source.  We will find that this
behaviour is a generic result of the wind models considered, even for cases that are
not fully isotropic.

Turning to the emission profiles of the \mgiid\ doublet, one notes
that they are also similar and have comparable equivalent widths
[word].  The
flux of the \mgiib\ transition even exceeds that for \mgiia\ giving a
line ratio that is far below the $2:1$ ratio that one may have naively
expected. 
[Does recombination predict 2:1??]
The emission profiles are very similar because the gas is optically
thick for a significant portion of the profile. 
The ultimate outcome is a conversion of some \mgiia\ photons into
\mgiib\ photons.
Furthermore, the flux from \mgiib\ exceeds the
\mgiia\ emission because the wind speed is greater than the velocity separation
of the doublet, $|\mvr|_{\rm max} > (\Delta v)_{\rm MgII}$.
Therefore,
the \mgiib\ absorption profile partially absorbs the red wing of the
\mgiia\ emission profile.  The relative strengths of the emission
lines, therefore, is sensitive to both the opacity and velocity
extent of the wind.  

Now consider the \ion{Fe}{2} transitions.
The bottom left panel of Figure~\ref{fig:fiducial_1d} covers the
majority of the \ion{Fe}{2} UV1 transitions and several are
shown in the velocity plot.  The line
profile for \feiib\ is very similar to the \mgiid\ doublet;
one observes strong absorption to negative velocities and strong
emission at $v > 0 \mkms$ producing the characteristic P-Cygni profile of
an outflow.  Splitting
the profile at $v = -50 \mkms$, we measure an equivalent width
$W_{2600}^{\rm abs} = 1.16$\AA\ in absorption and $W_{2600}^{\rm em} =
-0.94$\AA\ in emission (Table~\ref{tab:fiducial_EW}) for a total
equivalent width of $W_{2600}^{\rm TOT} \approx 0.22$\AA.  
In contrast, the \feiia\ resonance line shows much weaker emission and
a much higher total equivalent width ($W_{2586}^{\rm TOT} = 0.49$\AA),
even though the line has a $2 \times$ lower $f\lambda$ value.
These differences between the \ion{Fe}{2} resonance lines and with the
\ion{Mg}{2} doublet occur because of the complex of non-resonant
\ion{Fe}{2} transitions (Figure~\ref{fig:energy}).  Specfically,
resonance photons absorbed at \feiid\ have a finite probability of
being scattered as a non-resonant photon which escapes the system
without further interaction.  The principal effects are to reduce the
line emission of \feiid\ and produce non-resonant line-emission (e.g.\
\feiic).

The reduced \feiia\ emission relative to \feiib\ results from the
\ion{Fe}{2}$^*$ emission.
The \feiia\ flux is reduced by two factors:
(i) there is an additional downward transition from the
\zconfig$_{7/2}$ level and 
(ii) the Einstein A
coefficients of the non-resonant lines are comparable to and even
exceed the Einstein A coefficient of
the resonant transition.  In contrast, 
the \ion{Fe}{2}~2626 transition (associated with \feiib)
has an approximately  $4\times$ smaller A coefficient than the
resonance line.  Therefore, the majority of photons absorbed at
$\lambda 2600$ are re-emitted as \feiib\ photons whereas 
the majority of photons absorbed as \feiia\ are re-emitted 
at longer wavelengths (\ion{Fe}{2}$^* \; \lambda 2612$ or $\lambda
2632$).
If we were to increase $\tau_{2600}$ (and especially if we include
gas with $\mvr \approx 0 \mkms$) then the emission at \feiib\ is
significantly suppressed (e.g.\ $\S$~\ref{sec:ISM}).
The total equivalent width, however, of the lines connected to the
\zconfig$_{7/2}$ upper level must still vanish (photons are conserved
in this fiducial wind model).


The preceding discussion emphasizes the filling-in of resonance absorption at $v
\lesssim -50 \mkms$ and the generation of emission lines at $v \approx
0 \mkms$ by photons scattered in the wind.  To study the spatial
extent of this emission, we perfomed 3D calculations with the fiducial
model.  The output is a set of surface-brightness maps in a series of
frequency channels; the
dataset is analagous to integral-field-unit (IFU) observations.  In
Figure~\ref{fig:fiducial_ifu_mgii}, we present the output 
at several velocities relative to the \mgiia\
transition. At $v = -250 \mkms$, where the wind has an optical
depth $\tau_{2796} < 1$ (Figure~\ref{fig:fiducial_nvt}),
the source contributes roughly half of the observed flux.  
At $v=-100 \mkms$, however, the
wind absorbs all photons from the source and the observed emission is
entirely from photons scattered by the wind.  This scattered emission
actually exceeds the source+wind emission at 
$v = -250 \mkms$ such
that the absorption profile is
negatively offset from the velocity where $\tau_{2796}$ is maximal
(Figure~\ref{fig:fiducial_nvt}).
The net result of the scattered photons is
a weaker \ion{Mg}{2} profile that is offset from the true optical
depth profile.  Clearly, the effects complicate estimates for the
speed, covering fraction, and total column density of the wind.  At $v
= 0 \mkms$, the wind and source have comparable total flux with the
latter dominating at higher velocities.  

Similar results are observed for the \ion{Fe}{2} resonance
transitions (Figure~\ref{fig:fiducial_ifu_feii2600}).
For transitions to fine-structure levels of the \aconfig, the source
is unattenuated but there is a significant contribution from photons
generated in the wind. 

At all velocities, the majority of emission occurs at $r \le 5$\,kpc.
A better assessment of the radial profiles is provided by
Figure~\ref{fig:fiducial_cuts}, where we collapse the 
azimuthally symmetric surface-brightness maps in
one spatial dimension.  The majority of
\ion{Mg}{2} emission occurs within the inner few kpc, e.g.\ $50-60\%$
of the light at $v=-100$ to $+100 \mkms$
comes from $|r| < 3$\,kpc.
The emission
is even more centrally concentrated for the \ion{Fe}{2} transitions.
These distributions are [critically important] for observations
acquired through a slit, i.e.\ where the aperture has a limited extent
in one or more dimensions.  A standard longslit on 10m-class
telescopes, for examples, subtends $\approx 1''$ corresponding to
$5-10$\,kpc for $z \sim 1$.    We return to this issue in
$\S$~\ref{sec:obs}. 

[Comment that an observation like constrains the radial extent +
density/velocity profiles of the wind]


%%%%%%%%%%%%%%%%%%%%%%%%%%%%%%%%%%%%%%%%%%%%%%%%%%%%%%%%%%
\section{Variations to the Fiducial Model}

In this section, we investigate a series of more complex wind
scenarios
through modifications to the fiducial model.  These include relaxing
the assumption of isotropy, introducing dust, and adding an ISM
component within $r_{\rm inner}$.

\subsection{Anisotropic Winds}
\label{sec:anisotropic}

The fiducial model assumes an
isotropic wind with only radial variations in velocity and density. 
Angular isotropy is obviously an idealized case, but
it is frequently assumed in studies of galactic-scale outflows
\citep[e.g.][]{steidel+10}.   There are several reasons, however, to
consider anisotropic winds.  Firstly, galaxies are not spherically
symmetric;  the sources driving the
wind (e.g.\ supernovae, AGN) are very unlikely to be anisotropically distributed
within the galaxy.  
Secondly, the galactic ISM frequently has a disk-like morphology
which will suppress the wind preferntially at low galactic latitudes,
perhaps yielding a bi-conic morphology \citep[e.g.][]{M87}.
Lastly, the galaxy may be surrounded by an
anisotropic gaseous halo that would produce an irregularly shaped
wind.

With these considerations in mind, we reanalyzed the fiducial model
with the 3D algorithm after departing from isotopy.  It is beyond the
scope of this paper to explore a full suite of anisotropic profiles;
the following simply assumes half of the fiducial wind, i.e.,  
the density is set to zero for $2\pi$ steradians.
We have viewed this system from $\theta = 0^\circ$ where
the source is observed directly to $\theta = 180^\circ$ where the source
is covered by this anisotropic wind.  The resulting \ion{Mg}{2} and
\ion{Fe}{2} profiles are compared against the fiducial model
(isotropic wind) in Figure~\ref{fig:anisotropic}.  

Examining the \mgiid\ doublet, 
there is no attenutation of the source by the wind for $\theta =
0^\circ$
but one does observe significant line emission from photons scattered
off the back side.  These photons, by definition, have $v \gtrsim 0 \mkms$
relative to line-center (a subset have $v \lesssim 0 \mkms$ because
of turbulent motions in the wind). 
When viewed from the opposite direction ($\theta = 180^\circ$), the
absorption lines dominate but there is still significant
line-emission -- at $v \approx 0 \mkms$ and at $v < 0 \mkms$ which fills
in the absorption -- from photons that scatter through the wind.  The
key difference from the isotropic wind is the absence of photons
scattered to $v > 100 \mkms$;  this also implies a deeper 
\mgiib\ absorption profile at $v \approx -100 \mkms$. The asymmetry
shift in velocity centroid of the emission lines
may provide a diagnostic of wind isotropy, especially when combined
with the absorption profile [word].

The results are very similar for the \feiid\ resonance lines.  The
\ion{Fe}{2}$^* \; \lambda 2612$ line, meanwhile, shows most clearly the
offset in velocity between the source unobscured ($\theta = 0^\circ$)
and source covered ($\theta = 180^\circ$) cases.  This line [shift]
produces the most robust constrain on wind isotropy. [Observable: dust
is a trade-off]

%%%%%%%%%%%%%%%%%%%%%%%%%%%%%
\subsection{Dust}
\label{sec:dust}

Essentially all astrophysical environments that contain both cool gas
and metals also show signatures of dust depletion and extinction.  This includes the
ISM of star-forming and \ion{H}{1}-selected galaxies
\citep[e.g.][]{ss96,pw01,pcd+07}, strong \ion{Mg}{2} metal-line
absorption systems \citep{york,menard}, and the galactic winds traced
by low-ion transitions \citep{cb58,naI}.  Although the galactic winds
traced specifically by \ion{Mg}{2} and \ion{Fe}{2} transitions have not (yet) been
demonstrated to contain dust, it is reasonable to consider its
effects.

Dust modifies the observed flux in two manners. 
First, it is a source of opacity for all of 
the photons.  This suppresses the flux at all
wavelengths by $\approx \exp(-\mtaud)$, but because we normalize the profiles 
this effect is essentially ignored.  Second, photons that are
scattered by the wind will travel a greater
distance and suffer from greater extinction.  A photon that is
trapped for many scatterings may have a high probability of being absorbed
by dust.  This process is
often the explanation given for the weak (or absent) \lya\ emission
associated with star-forming galaxies \citep[e.g.][]{lya}.
Section~\ref{sec:dust_method} describes our treatment of dust; we only
remind the reader here that we assume a constant dust-to-gas ratio 
normalized by the total optical
depth \taud\ that a photon would experience if it travelled from the
source to infinity without scattering. 

In Figure~\ref{fig:dust}, we show the \ion{Mg}{2} and \ion{Fe}{2}
profiles of the fiducial model ($\mtaud = 0$) against a series of
models with $\mtaud > 0$.  For the \ion{Mg}{2} transitions, the
dominant effect is the suppression of line emission at $v \ge 0
\mkms$.  These `red' photons have scattered off the
backside of the wind and must travel a longer path than other
photons.  This creates a differential reddening that increases with wavelength (and
velocity relative to line-center). It is a natural consequence of dust
extinction;  the effect is most evident in the \ion{Fe}{2}$^* \; \lambda 2612$
emission profile which is otherwise symmetrically distributed around
$v \approx 0 \mkms$.   In terms of absorption, the profiles are
nearly identical for $\mtaud \le 3$.  One requires a very high
extinction to see an effect at $v \approx -100 \mkms$.

% The following could go in the Discussion section
We conclude that dust has only a modest influence on this fiducial model and,
by inference, models with moderate peak optical depth and
significant velocity gradients with radius (i.e.\ scenarios where the
photons scatter one to a few times before exiting).
To achieve qualitative changes, one requires an extreme level of
extinction ($\mtaud = 10$).  In this case, the source would be
extinguished by 15\,magnitudes and could never be observed. 
Even $\mtaud = 3$ is larger than typically inferred for the
star-forming galaxies that drive outflows \citep[e.g.][]{dust}.
For the emission lines,
the dominant effect is a reduction in the flux 
with a greater extinction at higher velocities relative to line-center.
In these respects, dust extinction crudely mimics the behavior of the anistropic
wind described in Section~\ref{sec:anisotropic}. [Is there a
distinguishing factor, e.g. flux of 2612/2600??]


\subsection{ISM}
\label{sec:ISM}

The fiducial model does not include gas associated with
the interstellar medium of the galaxy, i.e.\ material at $r \sim
0$\,kpc with $v_{\rm r} \approx 0 \mkms$.  This allowed us to focus on
results related solely to the wind.  The decision to ignore the ISM
was also motivated by the general absence of significant absorption at
$v \approx 0 \mkms$ in galaxies that exhibit outflows 
\citep[e.g.][]{wcp+09,steidel+10,rubin+10}.
On the other hand, the stars and \ion{H}{2} regions that comprise our
background sources are very likely embedded within and fueled by gas
of the galactic ISM.  
Consider, then, a modification to the fiducial model that has the
same wind but also includes an ISM component. 
Specifically, the ISM component has $n_{\rm H} = 1 \cm{-3}$ for
$r_{\rm ISM} \le r < r_{\rm inner}$ with $r_{\rm ISM} = 0.5$\,kpc, 
an average velocity of $\mvr = 0 \mkms$, and a larger turbulent velocity $b_{\rm ISM} = 40 \mkms$.
Figure~\ref{fig:ISM} summarizes the model.
The resultant optical depth profile $\tau_{2796}$ is identical to the
fiducial model for $r > 2$\,kpc, a slightly higher opacity at
$r=1-2$\,kpc, and a large opacity at $r = 0.5-1$\,kpc.

In Figure~\ref{fig:ISM_spec}, the solid curves show the \ion{Mg}{2} and
\ion{Fe}{2} profiles for the ISM+wind (red) and fiducial (black)
models. In comparison, the
dotted curve shows the absorption profile for the ISM+wind in the (unphysical) case
where none of the absorbed photons are scattered or re-emitted.   Focus first on the
\ion{Mg}{2} doublet.  As expected, the dotted curve shows strong
absorption at $v \approx 0 \mkms$ and blueward.  The full models,
in contrast, show non-zero flux at these velocities and even a
normalized flux exceeding unity at $v \approx 0 \mkms$.  In
fact, the ISM+wind model is nearly identical to the fiducial model;
the only quantitiative difference is that the velocity centroid of
the emission lines are shifted redward by $\approx +100 \mkms$.

There are, however, several qualitiative differences 
for the \ion{Fe}{2} transitions between the ISM+wind and fiducial
models.
First,
the \feiia\ transition shows much stronger absorption at $v \approx 0$
to $-100 \mkms$.  In contrast to the \ion{Mg}{2} doublet,
the profile is not filled in by scattered photons. Instead, 
the majority of \feiia\ photons absorbed are re-emitted as
\ion{Fe}{2}~$^* \lambda\lambda 2612, 2632$ photons.  In fact, the
ISM+wind \feiia\ profile nearly matches the profile without re-emission 
(compare to the dotted lines); this transition provides a
very good description of the ISM+wind opacity profile.  
We conclude that resonant transitions tha are coupled to (multiple)
non-resonant, electric dipole transitions may offer the best
diagnostic of ISM absorption.

The differences in the \ion{Fe}{2} absorption profiles are reflected
in the much higher strengths ($3-10\times$) of emission from
transitions to the excited states of \aconfig.   This occurs because
of : (1) the greater absorption in the \feiid\
resonance lines; (2) the high opacity of the ISM component leads to
an enhanced conversion of resonance photons with $v \approx 0\mkms$
into the non-resonance lines.  This is especially notable for the
\feiib\ transition whose `partner' shows an equivalent width nearly
$10\times$ stronger than for the fiducial model.  The relative
strengths of the \feiib\ and \feiie\ lines provide a direct
diagnostic on the degree to which the resonance lines are trapped.
[Comment:  this is a product of $\tau_{2600}$ and the velocity
gradient]

[3D discussion]

[Fix the bug in the FeII* emission]

\subsection{Summary Table}

Table~\ref{tab:line_diag} presents a series of quantitative measures
for the \ion{Mg}{2} and \ion{Fe}{2} absorption and emission lines for
the fiducial model and a set of the variations.  Listed are the
equivalent wdiths (absorption and emission), the peak optical depth
for the absorption $\tau_{\rm pk} \equiv -\ln(I_{\rm min})$, the
velocity where the optical depth peaks $v_\tau$, the optical
depth-weighted velocity centroid $v_{\bar \tau} \equiv \int dv v
\ln[I(v)] / \int dv \ln[I(v)]$, the peak flux $f_{\rm pk}$ in
emission, the velocity where the flux peaks $v_f$, and the
flux-weighted velocity centroid of the emission line $v_{\bar f}$.

One observes a wide range in these measures.  This emphasize the
challenge in converting the absorption/emission data to physical
constraints on wind schenarios.  We return to further consider these
results in $\S$~\ref{sec:obs}.

%%%%%%%%%%%%%%%%%%%%%%%%%%%%%%%%%%%%%%%%%%%%%%%%%%%%%%%%%%%%%%%%%
\section{Alternate Wind Models}

\subsection{Power-Law Models}
\label{sec:power}

The models investigated in the previous two sections
assumed power-law descriptions for both the density and velocity
expressions of the wind (Equations~\ref{eqn:density},\ref{eqn:vel}).
The power-law exponents were arbitrarily chosen, with minimal physical
motivation.  In this sub-section, we explore the results for a series
of other power-law expressions.

\subsection{Radiation Pressure}

\begin{equation}
\mvr(r) = 2\sigma \sqrt{R_g \ltp \frac{1}{R_0} - \frac{1}{r} \rtp
   + \ln\ltp R_0/r \rtp }
\end{equation}

\begin{equation}
n(r) = \frac{dM_{\rm wind}/dt}{r^2 v(r)}
\end{equation}

\subsection{The Lyman Break Galaxy Model}
\label{sec:lbg}

The Lyman break galaxies (LBGs), UV color-selected galaxies at $z \sim 3$,
exhibit cool gas outflows in \ion{Si}{2}, \ion{C}{2},
etc.\ transitions with speeds up to 1000\kms\
\citep[e.g.][]{lowenthal,pettini,steidel+10}.
Researchers have invoked these winds to explain enrichment of
the intergalactic medium \citep[e.g.][]{aguirre,evan}, the origin of the
damped \lya\ systems \citep{nelson,joop98}, and XXX.  Although the
presence of these outflows were established over a decade ago
\citep{lowenthal,pettini98}, the processes that drive them remain
unidentified.  Similarly,  current estimates of the mass and energetics of the
outflow suffer from orders of magnitude uncertainty.

Recently, \cite[][; hereafter S10]{steidel+10} introduced a model to
explain jointly the average absorption they observed
toward a set of several hundred LBGs and the average absorption in cool gas
observed transverse to these galaxies.  
Their wind model is defined by two
expressions: (i) a radial velocity law $\mvr(r)$ and (ii) the covering
fraction of optically thick gas $f_c(r)$.  For the latter, S10
envision an 
ensemble of small, optically thick clouds $(\tau \gg 1)$ that only
partially cover the galaxy.
For the velocity law, they adopted the following functional
form

\begin{equation}
\mvr = \ltp \frac{A_{\rm LBG}}{1-\alpha} \rtp^{1/2} \ltk r_{\rm
  inner}^{1-\alpha} - r^{1-\alpha} \rtk^{1/2}
\label{eqn:LBG_vlaw}
\end{equation}
with $A_{\rm LBG}$ a constant that sets the terminal speed,
$r_{inner}$ is the inner radius of the wind (taken to be 1\,kpc), and
$\alpha$ describes how steeply the velocity curve rises.  Their
analysis of the LBG absorption profiles implied
a very steeply rising curve with $\alpha \approx 1.3$.
This velocity expression is shown as a dotted line in 
Figure~\ref{fig:LBG_Sobolev}a.  

The covering fraction of optically thick cold gas, meanwhile, was assumed to have
the functional form

\begin{equation}
f_c(r) = f_{c,max} \ltp \frac{r}{r_{\rm inner}} \rtp^{-\gamma} \cmma
\label{eqn:covering}
\end{equation}
with $\gamma \approx 0.5$ and $f_{c,max}$ the maximum covering
fraction.  From this expression and the velocity law, one can recover
an absorption profile $I_{\rm LBG}(v) = 1 - f_c(r[v])$, written
explicitly as

\begin{equation}
I_{\rm LBG}(v) = 1 - f_{c,max} \ltk r_{\rm inner}^{1-\alpha} - \ltp
\frac{1-\alpha}{A_{\rm LBG}} \rtp v^2 \rtk^{\gamma/(\alpha-1)}
\perd
\label{eqn:LBG_I}
\end{equation}
The resulting profile for $f_{c,max} = 0.6$, $\gamma=0.5$,
$\alpha=1.3$, and $A_{\rm LBG} = -192,000 \, \rm km^2 s^{-2} kpc^{-2}$ 
is displayed in Figure~\ref{fig:LBG_Sobolev}b.  

In the following, we consider two methods to analyze the LBG wind.
Both approaches assume isotropy and adopt the velocity law given by
Equation~\ref{eqn:LBG_vlaw}.  In one model, we treat the cool gas as a
diffuse medium with unit convering fraction and a radial density
profile determined from the Sobolev approximation.  We then apply 
the methodology used for the other wind models.  In the other model,
we modify our algorithms to more precisely mimic the concept of an
ensemble of optically thick clouds with a partial covering fraction.

%%%%%%%%%%%%%%%%%%%%%%%%%%%%%%%%%%%%%%%%%%%%%%%%%%
\subsubsection{Sobolev approximation}
\label{sec:Sobolev}

As demonstrated in Figure~\ref{fig:LBG_Sobolev}, the wind velocity for
the LBG model rises very steeply with increasing radius before
flattening at large radii.  Under these conditions, the Sobolev
approximation provies an accurate relation between the optical depth
of the flowing medium to the
density and velocity gradient ($d\mvr/dr$) of the gas.  In our case, we 
generate the optical depth profile from Equation~\ref{eqn:LBG_I}, 

\begin{equation}
\tau_{\rm LBG}(v) = -\ln \ltk I_{\rm LBG}(v) \rtk \cmma
\label{eqn:tauLBG}
\end{equation}
and invert the Sobolev approximation to derive the density profile.
Following \cite{lamers+c99}, we have for a purely radial flow

\begin{equation}
n_{\rm LBG}(r) = \frac{\tau_{\rm LBG}(r) \; d\mvr/dr}{\kappa_\ell
  \lambda} \cmma
\label{eqn:Sobolev}
\end{equation}
with $\lambda$ the rest wavelength and $\kappa_\ell \equiv f\pi^2
e^2/m_e c$.  

The solid curve in Figure~\ref{fig:LBG_Sobolev}a shows the
resultant density profile for Mg$^+$ assuming that the \mgiia\ line
follows the intensity profile drawn in Figure~\ref{fig:LBG_Sobolev}b. 
This is a relatively extreme density profile.  From the inner radius
of 1\,kpc to 2\,kpc, the density drops by over 2 orders of
magnitude including nearly one order of magnitude over the first
10\,pc.  Beyond 2\,kpc, the density drops even more rapidly, falling
orders of magnitude from 2 to 100\,kpc.

We verified that the density profile shown in
Figure~\ref{fig:LBG_Sobolev}a
reproduces the proper
absorption profile by discretizing the gas into a series of layers
and calculating the integrated absorption profile.  This
calculation is shown as a dotted red curve in
Figure~\ref{fig:LBG_Sobolev}b which is an excellent
match to the desired profile (black curve).
To calculate the optical depth profiles for the other transitions, we
assume $\mnfe = \mnmg/2$ and scale $\tau$ by $f\lambda$.

We generated \ion{Mg}{2} and \ion{Fe}{2} profiles for this wind
using the 1D algorithm with no dust extinction; these are shown as
red curves in Figure~\ref{fig:LBG_spec}.   For this analysis, one
should focus on the \mgiia\ transition.  The dotted line in the Figure
shows the intensity profile for the model when one ignores re-emission
of absorbed photons.  By construction, it follows\footnote{Note that
  one should not make this comparison for the other transitions in
  this Sobolev model because those are scaled down by $f\lambda$ and
  for \ion{Fe}{2} the reduced Fe$^+$ abundance.} the profile
described by Equation~\ref{eqn:LBG_I} as plotted in
Figure~\ref{fig:LBG_Sobolev}b.   In comparison, the full model (solid,
red curve)
shows much weaker absorption, especially at $v = 0$ to $-300 \mkms$
due to scattered photons.
In this respect, our LBG-Sobolev model is an
inaccurate description of the observations. The model
also predicts significant emission in the \ion{Mg}{2} lines and several of
the \ion{Fe}{2}$^*$ transitions.   Emission associated with cool gas
has been observed for \ion{Si}{2}$^*$
transitions in LBGs \citep{cb58,shapley}, but
the \ion{Fe}{2}$^*$ transitions
modeled here lie in the near-IR and have not yet been investigated.
On the other hand, there have been no 
reported detections of significant line-emission related to resonance
transitions (e.g.\ \ion {Mg}{2}) in LBGs, only $z \le 1$ star-forming galaxies
\citep{wcp+09,rubin09}.  
The principal result is that the scattering and re-emission of
absorbed photons signficantly alters the predicted absorption profiles
for the inputted model.  This is, of course, an unavoidable
consequence of an isotropic, dust-free model with unit covering
fraction.
[Transition?]

[Do CIV scattered photons bugger up the P-Cygni analysis of Pettini
et al. ?]
 
%%%%%%%%%%%%%%%%%%%%%%%%%%%%%%%%%%%%%%%%%%%%%%%%%%
\subsubsection{Partial Covering Fraction}
\label{sec:Covering}

In the previous sub-section, we described a Sobolev solution that
reproduces the average absorption profile of LBGs in cool gas
transitions where scattered photons are ignored.  A proper analysis of
this wind model, however, predicts line profiles that are qualitatively
different from the ones observed because scattered photons fill-in
absorption and generate significant line emission (similar to the
fiducial wind model; $\S$~\ref{sec:fiducial}).
The LBG-Sobolev model, however, differs from
the one proposed by S10:  they proposed an ensemble of optically
thick clouds with the (parital) covering fraction described by
Equation~\ref{eqn:covering} whereas the Sobolev model assumes
a diffuse medium with a declining density profile but a unit covering
fraction.  At face value, one questions whether this difference leads to
the contradictory results of the model. 

To more properly model the LBG wind described in S10, we 
performed the following Monte Carlo calculation.  First, we propogated
a photon from the source until its velocity relative to line-center
resonates with the wind (the photon escapes if this never occurs).
The photon then has a probability $P = f_c(r)$ of scattering.  If it
scatters, we track the photon until it comes into
resonance again or escapes the system.  In this model, all of the
resonance transitions are assumed to have identical (high) optical
depth. 

The results of the full calculation (absorption plus scattering) are
shown as the black curve in Figure~\ref{fig:LBG_spec}.  The results
are very similar to the Sobolev calculation; scattered photons fill-in
the absorption profiles at $v \approx 0 \mkms$ and yield significant
emission lines at $v \gtrsim 0 \mkms$ in the resonance lines and at
all velocities for the \ion{Fe}{2}$^*$ transitions.  For \feiia,
the majority of absorbed photons have been converted to its
fine-structure counterparts lending to very weak emission at this
transition. The absorption profile very nearly matches the
model without re-emission indicating that this transition well
reproduces the opacity of the wind.  
The equivalent width of \feiia\ even exceeds that for \feiib, an
inversion that if observed would strongly support this model.
[Could SiII or OI do the same?]

Comparing the LBG Sobolev and parital covering models in
Figure~\ref{fig:LBG_spec} (via \mgiia), we note nearly identical
results.  The differencs that occur for the other transitions,
meanwhile, are only because their opacities are scaled downward in the
Sobolev model (as required).  [Do the data prefer one to the other?
And what if we modified the density away from Sobolev?]
In principle, they provide a means to distinguish between the two LBG
models.  

[Does most of the scattering occur within the inner few kpc? Explore
this with outflow.cc]

Given the essentially perfect correspondence between the Sobolev and
partical covering models (for \mgiia), we are motivated to consider
further the characteristics of the Sobolev model which offers a
density profile.  It is straightforward to convert the density and
velocity profiles into distributions of mass, energy, and momentum in
the wind.  These are shown in cumulative form in
Figure~\ref{fig:LBG_cumul}.  Before discussing the results, we offer
two cautionary comments: (i) the conversion of \nmg\ to $n_{\rm H}$
assumes a very poorly constrained scalar factor of $10^{5.7}$.  One
should give minimal weight to the absolute values for any of the
quantities;
(ii) the Sobolev approximation is not a proper description of the S10
LBG wind model.

The primary result expressed by Figure~\ref{fig:LBG_cumul} is that the
majority of energy, mass, and momentum in the wind is transported by
the outer layers ($r > 20$\,kpc).  This is surprising given that the
density is $\approx 5$~orders of magnitude lower at these radii than
at $r = 1$\,pc.  [Is this unphysical?  Can we conflate it to then kill
the S10 model?]

\section{Observables}
\label{sec:obs}

Although the focus of this paper is to explore the absorption/emission
profiles for idealized wind models, we ultimately intend to compare 
profiles like these against observations to constrain characteristics
of the wind (e.g.\ raidal extent, energetics).  In this section we
consider a few observables and their dependence on the wind properties.

\subsection{Absorption profiles}

 \begin{enumerate}
   \item Peak optical depth
     \begin{itemize}
       \item Not a robust measure of $\tau_{wind}$
       \item Not even a good relative predictor for $\tau$
       \item Not a robust predictor of $f_c$
     \end{itemize}
   \item Kinematics
     \begin{itemize}
       \item Nearly robust for $v<v_{min}$
       \item But what sets $v_{min}$?
       \item Generally get the full extent of the wind speed
     \end{itemize}
   \item EW
     \begin{itemize}
       \item Primarily set by $v_{min}(\tau \sim 1)$
       \item $f\lambda$ matters, but so do emission channels
     \end{itemize}
   \item Fine-structure
     \begin{itemize}
       \item Distance to the source
     \end{itemize}
 \end{enumerate}
     
\subsection{Emission profiles}

 \begin{enumerate}
   \item Resonant lines
     \begin{itemize}
       \item Kinematics of backside
       \item Flux + relative
    \end{itemize}
   \item Fine-structure
     \begin{itemize}
       \item Flux $\propto \tau_{wind}$ + relative
       \item Kinematics (anisotropy)
    \end{itemize}
 \end{enumerate}
     

\section{Discussion}

[Are MgII and FeII* emission strict predictions of outflows?  People
have been so focused on absorption.]

[Does recombination predict 2:1 for MgII??]
[We also emphasize that departures from a 2:1
ratio indicate that processes other than simple recombination are
active. ]

[Profiles are modified by {\it many} factors => Difficult to derive
robust constraints from absorption alone.]

[Is one safe if there is no emission detected?]

\acknowledgments

J. X. P. and J.M.O. are supported by NASA grant
HST-GO-10878.05-A.  J.X.P and G.W. are partially supported
by an NSF CAREER grant (AST--0548180), and 
by NSF grant AST-0908910.

\clearpage

%\bibliographystyle{/u/xavier/NSF/SASIR/SASIR-ATI/prop2009/Text/nsfati}
%\bibliography{/u/xavier/NSF/SASIR/SASIR-ATI/prop2009/Text/nsfati09}
\bibliographystyle{/u/xavier/paper/Bibli/apj}
\bibliography{/u/xavier/paper/Bibli/allrefs}

\clearpage

\begin{deluxetable}{lcccccc}
\tabletypesize{\footnotesize}
\tablecolumns{11}
\tablecaption{Observed Transitions and Limits \label{tab:atomic}}
\tablewidth{0pt}
\tablehead{\colhead{} & \colhead{$\rm E_{high}$} & \colhead{$\rm E_{low}$} & \colhead{$J_{\rm high}$} & \colhead{$J_{\rm low}$} & \colhead{$\lambda$} & \colhead{$A$} \\
 & \colhead{($\rm cm^{-1}$)} & \colhead{($\rm cm^{-1}$)} &&& \colhead{(\AA)} & \colhead{($\rm s^{-1}$)} } 
\startdata
\ion{Fe}{2} UV1 & 38458.98 &     0.00 &   9/2 & 9/2 & 2600.173 & 2.36E08  \\
           & 38458.98 &   384.79 &   9/2 & 7/2 & 2626.451 & 3.41E+07 \\
           & 38660.04 &     0.00 &   7/2 & 9/2 & 2586.650 & 8.61E+07 \\
           & 38660.04 &   384.79 &   7/2 & 7/2 & 2612.654 & 1.23E+08 \\
           & 38660.04 &   667.68 &   7/2 & 5/2 & 2632.108 & 6.21E+07 \\
           & 38858.96 &   667.68 &   5/2 & 5/2 & 2618.399 & 4.91E+07 \\
           & 38858.96 &   862.62 &   5/2 & 3/2 & 2631.832 & 8.39E+07 \\
           & 39013.21 &   667.68 &   3/2 & 5/2 & 2607.866 & 1.74E+08 \\
           & 39013.21 &   862.61 &   3/2 & 3/2 & 2621.191 & 3.81E+06 \\
           & 39013.21 &   977.05 &   3/2 & 1/2 & 2629.078 & 8.35E+07 \\
           & 39109.31 &   862.61 &   1/2 & 3/2 & 2614.605 & 2.11E+08 \\
           & 39109.31 &   977.05 &   1/2 & 1/2 & 2622.452 & 5.43E+07 \\
\tableline \\ [-1.5ex]
\ion{Mg}{2}& 35760.89 &     0.00 &   3/2 &   0 & 2796.351 & 2.63E+08\\
           & 35669.34 &     0.00 &   1/2 &   0 & 2803.528 & 2.60E+08\\
\enddata
\tablecomments{Atomic data was obtained from \citet{Morton2003} unless otherwise indicated.}
\end{deluxetable}

 
 
\begin{deluxetable}{ccl}
\tablewidth{0pc}
\tablecaption{Wind Parameters: Fiducial Model\label{tab:fiducial}}
\tabletypesize{\footnotesize}
\tablehead{\colhead{Property} & \colhead{Parameter} & \colhead{Value} } 
\startdata
Density law  & $n(r)$ & $\propto r^{-2}$ \\
Velocity law  & $v_r$ & $ \propto r$ \\
Inner Radius & $r_{\rm inner}$ & 1\,kpc \\
Outer Radius & $r_{\rm outer}$ & 20\,kpc \\
Source size  & $r_{\rm source}$ & 0.5\,kpc \\
Density Normalization & $n^0_{\rm H}$ & $0.1\cm{-3}$ at $r_{\rm inner}$ \\
Velocity Normalization & $v^0$ & 50\kms at $r_{\rm inner}$ \\
Turbulence   & $b_{\rm turb}$  & 15 \kms \\
Mg$^+$ Normalization & \nmg\ & $10^{-5.47} n_{\rm H}$ \\
Fe$^+$ Normalization & \nfe\ & \nmg/2 \\
\enddata
%\tablecomments{Unless specified otherwise, all quantities refer to the \sna=2 threshold.  The cosmology assumed has $\Omega_\Lambda = 0.7, \Omega_m = 0.3$, and $H_0 = 72 \mkms \rm Mpc^{-1}$.}
%\tablenotetext{a}{Total redshift survey path for the \sna=2 criterion.}

\end{deluxetable}

\input{Tables/tab_fiducial_ew.tex}
 
 
\begin{deluxetable}{ccrccccccccccc}
\rotate
\tablewidth{0pc}
\tablecaption{Line Diagnostics for the Fiducial Model and Variants
\label{tab:line_diag}}
\tabletypesize{\footnotesize}
\tablehead{\colhead{Transition} & \colhead{Model} & \colhead{$v_{\rm int}^a$} & \colhead{$W_{\rm i}$} & \colhead{$W_{\rm a}$} & \colhead{$\tau_{\rm pk}$} & \colhead{$v_\tau$} 
& \colhead{$v_{\bar \tau}$}
& \colhead{$v_{\rm int}^b$} & \colhead{$W_{\rm e}$} & \colhead{$f_{\rm pk}$} & \colhead{$v_f$} 
& \colhead{$v_{\bar f}$}
\\
&& (\kms) & (\AA) & (\AA) && (\kms) & (\kms) & (\kms) & (\AA) & & (\kms) & (\kms)}
\startdata
  MgII 2796  \\
&Fiducial&[$-1009,-43$]& 4.78& 2.83&0.94&$ -215$&$ -372$&[$-32,311$]&$-1.77$& 2.48&$   32$&$  117$\\
&$\phi=0^\circ$&$\dots$&$\dots$&$\dots$&$\dots$&$\dots$&$\dots$&$\dots$&$\dots$&$\dots$&$\dots$&$\dots$&$\dots$\\
&$\phi=180^\circ$&[$-1030,-65$]& 4.78& 2.98&1.03&$ -199$&$ -369$&[$-65,70$]&$-0.40$& 1.78&$  -11$&$    6$\\
&\taud=1&[$-1009,-32$]& 4.77& 2.94&0.95&$ -215$&$ -360$&[$-32,257$]&$-0.92$& 1.90&$   32$&$  101$\\
&\taud=3&[$-998,-32$]& 4.78& 3.07&1.03&$ -193$&$ -342$&[$-22,182$]&$-0.39$& 1.48&$   43$&$   76$\\
&ISM&[$-1009,-65$]& 6.36& 2.67&0.90&$ -204$&$ -390$&[$-54,311$]&$-1.60$& 2.36&$  118$&$  125$\\
&ISM+dust&[$-998,150$]& 6.42& 4.06&1.06&$ -193$&$ -270$&[$590,601$]&$ 0.12$& 0.40&$  590$&$  595$\\
  MgII 2803  \\
&Fiducial&[$-437,-41$]& 3.29& 1.19&0.76&$ -148$&$ -193$&[$-41,676$]&$-2.22$& 2.55&$   34$&$  269$\\
&$\phi=0^\circ$&$\dots$&$\dots$&$\dots$&$\dots$&$\dots$&$\dots$&$\dots$&$\dots$&$\dots$&$\dots$&$\dots$&$\dots$\\
&$\phi=180^\circ$&[$-699,-57$]& 3.29& 1.98&0.97&$ -137$&$ -257$&[$-57,104$]&$-0.50$& 1.81&$   -3$&$   22$\\
&\taud=1&[$-479,-41$]& 3.28& 1.41&0.83&$ -137$&$ -195$&[$-41,591$]&$-1.27$& 1.95&$   34$&$  247$\\
&\taud=3&[$-533,-30$]& 3.26& 1.67&0.94&$ -116$&$ -195$&[$-30,484$]&$-0.64$& 1.50&$   34$&$  213$\\
&ISM&[$-426,-62$]& 6.49& 1.03&0.67&$ -169$&$ -206$&[$-51,655$]&$-2.09$& 2.57&$   98$&$  265$\\
&ISM+dust&[$-1774,130$]& 6.51& 6.73&1.06&$ -961$&$ -686$&[$173,195$]&$-0.01$& 1.03&$  184$&$  184$\\
  FeII 2586  \\
&Fiducial&[$-348,-35$]& 0.82& 0.61&1.01&$  -70$&$ -119$&[$-35,128$]&$-0.08$& 1.11&$   35$&$   47$\\
&$\phi=0^\circ$&$\dots$&$\dots$&$\dots$&$\dots$&$\dots$&$\dots$&$\dots$&$\dots$&$\dots$&$\dots$&$\dots$&$\dots$\\
&$\phi=180^\circ$&[$-365,-46$]& 0.82& 0.60&1.01&$  -75$&$ -133$&[$-46,70$]&$-0.01$& 1.06&$  -17$&$   12$\\
&\taud=1&[$-348,-35$]& 0.82& 0.61&1.04&$  -70$&$ -118$&[$-35,116$]&$-0.04$& 1.06&$   23$&$   41$\\
&\taud=3&[$-313,-35$]& 0.94& 0.60&1.06&$  -70$&$ -113$&[$-23,46$]&$-0.02$& 1.04&$  -12$&$   12$\\
&ISM&[$-348,104$]& 1.90& 1.60&2.75&$    0$&$  -29$&[$1670,1948$]&$-0.19$& 1.15&$ 1682$&$ 1806$\\
&ISM+dust&[$-313,116$]& 1.84& 1.63&2.96&$   23$&$  -25$&[$278,290$]&$-0.00$& 1.03&$  278$&$  284$\\
  FeII 2600  \\
&Fiducial&[$-580,-37$]& 1.87& 1.18&1.08&$  -83$&$ -181$&[$-37,459$]&$-0.82$& 1.70&$   32$&$  191$\\
&$\phi=0^\circ$&$\dots$&$\dots$&$\dots$&$\dots$&$\dots$&$\dots$&$\dots$&$\dots$&$\dots$&$\dots$&$\dots$&$\dots$\\
&$\phi=180^\circ$&[$-597,-49$]& 1.87& 1.18&1.31&$  -78$&$ -188$&[$-49,95$]&$-0.23$& 1.39&$  -20$&$   22$\\
&\taud=1&[$-591,-37$]& 1.87& 1.20&1.14&$  -83$&$ -176$&[$-37,332$]&$-0.48$& 1.45&$   32$&$  138$\\
&\taud=3&[$-580,-37$]& 1.95& 1.25&1.31&$  -72$&$ -169$&[$-37,228$]&$-0.20$& 1.22&$   44$&$   93$\\
&ISM&[$-568,90$]& 3.12& 1.88&1.19&$  -72$&$  -99$&[$101,378$]&$-0.19$& 1.15&$  113$&$  237$\\
&ISM+dust&[$-603,101$]& 3.06& 2.12&1.40&$   44$&$  -93$&[$182,194$]&$-0.00$& 1.02&$  182$&$  188$\\
  FeII* 2612 \\
&Fiducial&&&&&&&$\dots$&$\dots$&$\dots$&$\dots$&$\dots$&$\dots$\\
&$\phi=0^\circ$&&&&&&&$\dots$&$\dots$&$\dots$&$\dots$&$\dots$&$\dots$\\
&$\phi=180^\circ$&&&&&&&$\dots$&$\dots$&$\dots$&$\dots$&$\dots$&$\dots$\\
&\taud=1&&&&&&&$\dots$&$\dots$&$\dots$&$\dots$&$\dots$&$\dots$\\
&\taud=3&&&&&&&$\dots$&$\dots$&$\dots$&$\dots$&$\dots$&$\dots$\\
&ISM&&&&&&&$\dots$&$\dots$&$\dots$&$\dots$&$\dots$&$\dots$\\
&ISM+dust&&&&&&&$\dots$&$\dots$&$\dots$&$\dots$&$\dots$&$\dots$\\
  FeII* 2626 \\
&Fiducial&&&&&&&$\dots$&$\dots$&$\dots$&$\dots$&$\dots$&$\dots$\\
&$\phi=0^\circ$&&&&&&&$\dots$&$\dots$&$\dots$&$\dots$&$\dots$&$\dots$\\
&$\phi=180^\circ$&&&&&&&$\dots$&$\dots$&$\dots$&$\dots$&$\dots$&$\dots$\\
&\taud=1&&&&&&&$\dots$&$\dots$&$\dots$&$\dots$&$\dots$&$\dots$\\
&\taud=3&&&&&&&$\dots$&$\dots$&$\dots$&$\dots$&$\dots$&$\dots$\\
&ISM&&&&&&&$\dots$&$\dots$&$\dots$&$\dots$&$\dots$&$\dots$\\
&ISM+dust&&&&&&&$\dots$&$\dots$&$\dots$&$\dots$&$\dots$&$\dots$\\
  FeII* 2632 \\
&Fiducial&&&&&&&$\dots$&$\dots$&$\dots$&$\dots$&$\dots$&$\dots$\\
&$\phi=0^\circ$&&&&&&&$\dots$&$\dots$&$\dots$&$\dots$&$\dots$&$\dots$\\
&$\phi=180^\circ$&&&&&&&$\dots$&$\dots$&$\dots$&$\dots$&$\dots$&$\dots$\\
&\taud=1&&&&&&&$\dots$&$\dots$&$\dots$&$\dots$&$\dots$&$\dots$\\
&\taud=3&&&&&&&$\dots$&$\dots$&$\dots$&$\dots$&$\dots$&$\dots$\\
&ISM&&&&&&&$\dots$&$\dots$&$\dots$&$\dots$&$\dots$&$\dots$\\
&ISM+dust&&&&&&&$\dots$&$\dots$&$\dots$&$\dots$&$\dots$&$\dots$\\
\enddata
\tablecomments{{L}isted are the equivalent widths (intrinsic, absorption, and emission), the peak optical depth for the absorption
$\tau_{\rm pk} \equiv -\ln(I_{\rm min})$, the velocity where the optical depth peaks $v_\tau$, the optical depth-weighted velocity centroid 
$v_{\bar \tau} \equiv \int dv \, v \ln[I(v)] / \int dv \ln[I(v)]$, the peak flux $f_{\rm pk}$ in emission, the velocity where the flux peaks 
$v_f$, and the flux-weighted velocity centroid of the emission line $v_{\bar f}$.}
\end{deluxetable}

\input{Tables/tab_meas_plaws.tex}

\begin{figure}
\epsscale{0.8}
\plotone{Figures/energy_levels.ps}
\caption{
Energy level diagrams for the \mgiid\ doublet and the UV1
multiplet of \ion{Fe}{2} transitions   
(based on Figure~7 from \cite{hmt+99}).
Each transition shown is
labelled by its rest wavelength (\AA) and Einstein A-coefficient
(s$^{-1}$). Black updward arrows
indicate the resonance-line transitions connected to the ground
state of each ion.  The 2p$^6$3p configuration of Mg$^+$ is split into
two energy levels that give rise to the \mgiid\ doublet.  
Both the 3d$^6$4s ground state and 3d$^6$4p upper level of Fe$^+$
exhibit fine-structure splitting that gives rise to a series of
electric-dipole transitions. 
The downward (green) arrows show the transitions connected to the
resonance-line transitions (i.e.\ they share the same upper energy
levels).  We also show a pair of levels (\ion{Fe}{2}~$\lambda\lambda
2618,2631$) that arise from higher levels in the \zconfig\
configuration.  These transitions have not yet been observed in
galactic-scale outflows and are not considerd in our analysis.
}
\label{fig:energy}
\end{figure}

\begin{figure}
\epsscale{0.8}
\plotone{Figures/fig_nvtau_vs_r.ps}
\caption{
Density (solid; black), radial velocity (dotted; blue), and
\mgiia\ optical depth profiles (dashed; red) for the fiducial
wind model (see Table~\ref{tab:fiducial} for details).
The density and velocity profiles are simple $r^{-2}$ and $r$
power-laws.  The optical depth profile was calculated by summing
the opacity at small and discrete radial intervals.  One notes that
the wind is optically thick at the inner radius ($r_{\rm inner} =
1$\,kpc) and becomes optically thin at the outer radius ($r_{\rm
  outer} = 20$\,kpc).
Note that the density and velocity curves have been scaled for plotting
convenience.  
}
\label{fig:fiducial_nvt}
\end{figure}

\begin{figure}
\epsscale{0.8}
\plotone{Figures/fig_fiducial_1d.ps}
\caption{
{\it Left} -- (upper) \ion{Mg}{2} profiles for the fiducial wind model
described in Table~\ref{tab:fiducial} and
Figure~\ref{fig:fiducial_nvt}.  The doublet shows the P-Cygni profiles
characteritic of an outflow with significant absorption blueward of
line center (dashed vertical lines) extending to $v = -1000\mkms$
and signficiant emission redward of line center.  Note
that even though the peak optical depth of the \ion{Mg}{2} transitions
is nearly 30 at $v \approx -70 \mkms$, photons scattered off the outflow
fill in the absorption.
(lower) \ion{Fe}{2} absorption and emission profiles for the UV1
multiplet at $\lambda \approx 2600$\AA.  The \feiid\ resonance lines 
show weaker absorption due to the smaller Fe$^+$ number density and
lower $f\lambda$ values.  Each also shows a P-Cygni profile, although
the emission for \feiia\ is significantly weaker than that of the
\feiib\ and \mgiid\ transitions.  This is because a majority of the
absorbed \feiia\ photons are converted into
\ion{Fe}{2}~$\lambda\lambda 2612, 2632$ photons as observed.
[Mention the emission kinematics?]
}
\label{fig:fiducial_1d}
\end{figure}

\begin{figure}
\epsscale{0.8}
\plotone{Figures/fig_noemiss.ps}
\caption{
The black curves show the line profiles (absorption and emission) of
the \ion{Mg}{2} and \ion{Fe}{2} resonance lines for the fiducial wind
model.  These shows the canonical P-Cygni profiles of a source
embedded within an outflow.  Overplotted (red) on each transition is
the predicted absorption profile under the constraint that each
absorbed photon is lost from the system, i.e.\ no scattering or
re-emission.   It is evident that the true profiles have been `filled
in' significantly by light scattered in the wind.  Ignoring this
process, one would model the absorption lines with systematicall lower
optical depth and derive a partial covering. [word]
}
\label{fig:noemiss}
\end{figure}

\begin{figure}
\epsscale{0.8}
\plotone{Figures/fig_fiducial_ifu_mgii.ps}
\caption{
Surface-brightness emission maps around the \mgiia\ transition for the
source+wind complex.  The middel panel shows the 1D spectrum with $v=0
\mkms$ corresponding to $\lambda = 2796.35$\AA\ and the dotted vertical
curves indicating the velocity slices for the emission maps.  The
source has a size $r_{\rm source} = 0.5$\,kpc, traced by a few
pixels at the center.   At $\mvr=-200 \mkms$, the wind has an optical
depth of $\tau_{2796} \approx 1$ such that the source contributes
significantly to the observed flux.  At $\mvr=-100 \mkms$, however, the
wind absorbs all photons from the source and the observed emission is
entirely due to photons scattered by the wind.  Surprisingly, this
emission exceeds that observed at $\mvr = -200\mkms$. At $\mvr \ge 0
\mkms$,  both the source and wind contribute to the observed emission.
}
\label{fig:fiducial_ifu_mgii}
\end{figure}

\begin{figure}
\epsscale{0.8}
\plotone{Figures/fig_fiducial_ifu_feii2600.ps}
\caption{
(a) Surface-brightness emission maps around the \feiib\ transition for the
source+wind complex.  The middle panel shows the 1D spectrum with $v=0
\mkms$ corresponding to $\lambda = 2600.173$\AA\ and the dotted vertical
curves indicating the velocity slices for the emission maps.  
The results are very similar to those observed for the \mgiid\ doublet
(Figure~\ref{fig:fiducial_ifu_mgii}).
}
\label{fig:fiducial_ifu_feii2600}
\end{figure}

\begin{figure*}
\epsscale{0.8}
\plotone{Figures/fig_fiducial_ifu_feii2612.ps}
\caption{
(b) Surface-brightness emission maps around the \ion{Fe}{2}~$\lambda 2612$ transition for the
source+wind complex.  The middle panel shows the 1D spectrum with $v=0
\mkms$ corresponding to $\lambda = 2612.654$\AA\ and the dotted vertical
curves indicating the velocity slices for the emission maps.  
In this case, the source is unattenuated yet scattered photons from
the wind also have a significant contribution. 
}
\label{fig:fiducial_ifu_feii2612}
\end{figure*}

\begin{figure}
\epsscale{0.8}
\plotone{Figures/fig_fiducial_cut.ps}
\caption{
Spatial cuts
}
\label{fig:fiducial_cuts}
\end{figure}

\begin{figure}
\epsscale{0.8}
\plotone{Figures/fig_asymm_spec.ps}
\caption{
Profiles of the \ion{Fe}{2} and \ion{Mg}{2} profiles for the fiducial
case (black lines) compared against an anisotropic wind blowing into
only $2\pi$ steradians as viewed from $\theta = 0^\circ$ (source
uncovered) to $\theta = 180^\circ$ (source covered).  One detects
significant emission for all orientations of this wind. The velocity
centroid of this 
emission shifts from positive to negative velocities as $\theta$
increases.  The absorption reaches a maximum at $\theta > 100^\circ$
and disappears for $\theta = 0^\circ$.
}
\label{fig:anisotropic}
\end{figure}

\begin{figure}
\epsscale{0.8}
\plotone{Figures/fig_dust_spec.ps}
\caption{
Profiles of the \ion{Fe}{2} and \ion{Mg}{2} profiles for the fiducial
model (black) against a series of models that include the effects of
dust extinction parameterized by \taud.  The primary difference is the suppression of line
emission relative to the continuum (again, normalized to unit value).
A more subtle but important effect is that the redder photons in the
emission lines (corresponding to higher velocity relative to
line-center) suffer greater extinction.  This is most evident in the
\feiic\ emission line and occurs because these photons must
travel further to scatter off the backside of the wind.  Note that
the absorption lines are nearly unmodified until $\mtaud = 10$, a
level of extinction that would preclude observing the source
altogether.
}
\label{fig:dust}
\end{figure}

\begin{figure}
\epsscale{0.8}
\plotone{Figures/fig_ism_diagn.ps}
\caption{
Density (dashed; red), radial velocity (dotted; blue), and
\mgiia\ optical depth profiles (solid; black) for the ISM+wind
model.  The wind is identical to the fiducial model
(Figure~\ref{fig:fiducial_nvt}) but this model also includes an ISM component
with $n_{\rm H} = 1 \cm{-3}$ that has an average velocity of $\mvr = 0
\mkms$ and a large turbulent velocity $b_{\rm ISM} = 40 \mkms$.
Note that the density and velocity curves have been scaled for plotting
convenience.  
}
\label{fig:ISM}
\end{figure}

\begin{figure}
\epsscale{0.7}
\plotone{Figures/fig_ism_spec.ps}
\caption{
Profiles of the \ion{Fe}{2} and \ion{Mg}{2} profiles for the ISM+wind
model (red) compared against the fiducial wind model (black). 
The dotted line traces the predicted absorption profile in the absence
of any re-emission of absorbed photons.
Regarding the \ion{Mg}{2}~doublet, the primary difference between the
ISM+wind and the fiducial models is the shift of $\approx 100 \mkms$
in the emission lines from $\mvr \approx 0 \mkms$ for the fiducial
model to $\mvr \approx +100\mkms$ for the ISM+wind model. 
The \ion{Fe}{2} profiles, however, show qualitative differences from
the ISM model.  The \feiid\ resonance transitions each exhibit much
greater absorption at $v \approx 0 \mkms$ than the fiducial model.
This emission associated with these lines is also substantially
reduced, implying much higher fluxes for the non-resonant lines (e.g.\
\feiic).  The lack of scattered \feiia\ photons results in an
absorption profile that very nearly matches the input opacity profile.
We conclude that fine-structure transitions may offer the best
characterization of an ISM component.
}
\label{fig:ISM_spec}
\end{figure}

\begin{figure}
\epsscale{0.8}
\plotone{Figures/fig_lbg_sobolev.ps}
\caption{
{\it Upper:} The dotted curve shows the velocity law for the LBG model
of S10.  Note how rapidly the velocity rises from $r =
1$ to 2\,kpc.  The solid curve shows the Sobolev solution for the gas
density derived from 
the average absorption profile of LBG galaxies (lower panel).
The density drops off as $(r/{\rm kpc}-1)^{-0.6}$ initially and then
steepends to $(r/{\rm kpc}-1)^{-1.8}$.
{\it Lower:} Average absorption profile for LBG cool gas absorption as
measured and defined by S10
(black solid curve).  Overplotted on this curve is a (red) dotted line that
shows the absorption profile dervied from the density (and velocity)
law shown in the upper panel.  The excellent agreement confrims the
validity of the Sobolev approximation.
}
\label{fig:LBG_Sobolev}
\end{figure}

\clearpage

\begin{figure}
\epsscale{0.8}
\plotone{Figures/fig_lbg_spec.ps}
\caption{
Profiles of the \ion{Fe}{2} and \ion{Mg}{2} profiles for the two
LBG models considered in the text: (red) the profiles for the density
law derived from the Sobolev approximation using the average LBG
absorption profile and (black) a model constructed to faithfully
correspond to the wind model advocated by S10.  The dotted line,
meanwhile, shows the absorption profiles of the latter model where one
ignores scattered and re-emitted photons.  Similar to the fiducial
model (Figure~\ref{fig:fiducial_1d}), scattered and re-emitted photons
significantly modify the absorption profiles (especially \ion{Mg}{2})
and produce significant emission lines.  There is remarkable agreement
between the \mgiia\ profile for the two models. [Cast doubt on S10
here?]
}
\label{fig:LBG_spec}
\end{figure}

\begin{figure}
\epsscale{0.8}
\plotone{Figures/fig_lbg_cumul.ps}
\caption{
Profiles of the \ion{Fe}{2} and \ion{Mg}{2} profiles for the fiducial
}
\label{fig:LBG_cumul}
\end{figure}



\end{document}
