\documentclass[12pt,preprint]{aastex}
\usepackage{natbib,amsmath}

\special{papersize=8.5in,11in}
\begin{document}

\newcommand{\bturb}{$b_{\rm turb}$}
\newcommand{\mbturb}{b_{\rm turb}}
\newcommand{\taud}{$\tau_{\rm dust}$}
\newcommand{\mtaud}{\tau_{\rm dust}}
\newcommand{\maconfig}{2p$^6$3s}
\newcommand{\mbconfig}{2p$^6$3p}
\newcommand{\aconfig}{a~$^6$D$^0$}
\newcommand{\zconfig}{z~$^6$D$^0$}
\newcommand{\mvr}{v_{\rm r}}
\newcommand{\naid}{\ion{Na}{1}~$\lambda\lambda 5891, 5897$}
\newcommand{\mgiid}{\ion{Mg}{2}~$\lambda\lambda 2796, 2803$}
\newcommand{\mgiia}{\ion{Mg}{2}~$\lambda 2796$}
\newcommand{\mgiib}{\ion{Mg}{2}~$\lambda 2803$}
\newcommand{\feiia}{\ion{Fe}{2}~$\lambda 2586$}
\newcommand{\feiib}{\ion{Fe}{2}~$\lambda 2600$}
\newcommand{\feiic}{\ion{Fe}{2}$^* \; \lambda 2612$}
\newcommand{\feiie}{\ion{Fe}{2}$^* \; \lambda 2626$}
\newcommand{\feiid}{\ion{Fe}{2}~$\lambda\lambda 2586, 2600$}
\newcommand{\feiis}{\ion{Fe}{2}$^*$}
\newcommand{\nmg}{$n_{\rm Mg^+}$}
\newcommand{\mnmg}{n_{\rm Mg^+}}
\newcommand{\nfe}{$n_{\rm Fe^+}$}
\newcommand{\mnfe}{n_{\rm Fe^+}}
\def\hub{h_{72}^{-1}}
\def\umfp{{\hub \, \rm Mpc}}
\def\mzq{z_q}
\def\zabs{$z_{\rm abs}$}
\def\mzabs{z_{\rm abs}}
\def\intl{\int\limits}
\def\cmma{\;\;\; ,}
\def\perd{\;\;\; .}
\def\ltk{\left [ \,}
\def\ltp{\left ( \,}
\def\ltb{\left \{ \,}
\def\rtk{\, \right  ] }
\def\rtp{\, \right  ) }
\def\rtb{\, \right \} }
\def\sci#1{{\; \times \; 10^{#1}}}
\def \rAA {\rm \AA}
\def \zem {$z_{\rm em}$}
\def \mzem {z_{\rm em}}
\def\smm{\sum\limits}
\def \cmm  {cm$^{-2}$}
\def \cmmm {cm$^{-3}$}
\def \kms  {km~s$^{-1}$}
\def \mkms  {\, {\rm km~s^{-1}}}
\def \lyaf {Ly$\alpha$ forest}
\def \Lya  {Ly$\alpha$}
\def \lya  {Ly$\alpha$}
\def \mlya  {Ly\alpha}
\def \Lyb  {Ly$\beta$}
\def \lyb  {Ly$\beta$}
\def \lyg  {Ly$\gamma$}
\def \ly5  {Ly-5}
\def \ly6  {Ly-6}
\def \ly7  {Ly-7}
\def \nhi  {$N_{\rm HI}$}
\def \mnhi  {N_{\rm HI}}
\def \lnhi {$\log N_{HI}$}
\def \mlnhi {\log N_{HI}}
\def \etal {\textit{et al.}}
\def \lyaf {Lyman--$\alpha$ forest}
\def \mnmin {\mnhi^{\rm min}}
\def \nmin {$\mnhi^{\rm min}$}
\def \O {${\mathcal O}(N,X)$}
\newcommand{\cm}[1]{\, {\rm cm^{#1}}}
\def \snrlim {SNR$_{lim}$}

\title{Simple Wind Models}

\author{
J. Xavier Prochaska\altaffilmark{1}, 
Others
%John M. O'Meara\altaffilmark{2}, 
%Gabor Worseck\altaffilmark{1} 
%\& Scott Burles\altaffilmark{3}
}
\altaffiltext{1}{Department of Astronomy and Astrophysics, UCO/Lick Observatory, University of California, 1156 High Street, Santa Cruz, CA 95064}
%\altaffiltext{2}{Department of Chemistry and Physics, Saint Michael's College.
%One Winooski Park, Colchester, VT 05439}

\begin{abstract}
\begin{itemize}
\item Emission is a generic feature of (nearly) isotropic winds.  This
  emission fills in absorption at $v \approx 0$, significantly
  complicating the absorption-line analysis, especially of an ISM
  component. 
\item The relative strengths of the emission
lines, therefore, is sensitive to both the opacity and velocity
extent of the wind.
\end{itemize}
\end{abstract}

\keywords{absorption lines -- intergalactic medium -- Lyman limit systems -- SDSS}

\section{Introduction}

Nearly all gaseous objects that shine are also
observed to generate gaseous flows.  This includes the jets of protostars, the
stellar winds of massive O and B stars, the gentle Solar wind
of our Sun, the associated absorption of bright quasars, and the
spectacular jets of radio-loud AGN.   These gaseous outflows moderate
the accretion of new material onto the object, inject energy and
momentum into gas on large scales, and ...
Developing a comprehensive model for these flows is critical to
understanding the evolution of the source and its impact on the
surrounding environment.

Star-burst galaxies, whose luminosity is dominated by \ion{H}{2} regions
and massive stars, are also observed to drive gaseous outflows.  These
flows are generally expected (and sometimes observed) to have multiple
phases, for example a hot and diffuse phase traced by X-ray emission
together with a cool, denser phase traced by H$\alpha$ emission 
\citep[e.g.][]{shc+04,km10}. 
Several spectacular
examples in the local universe demonstrate that flows can extend to
tens of kpc from the galaxy \citep{M87} carrying significant speed
to escape from the gravitational potential well of the galaxy's dark
matter halo \citep{wind_escape}.

Beyond $z \sim 0$, galactic outflows are 
revealed by UV and optical absorption lines, e.g.\ \ion{Na}{1},
\ion{Mg}{2}, \ion{Si}{2} and \ion{C}{4} transitions.  With the galaxy
as a backlight, one observes gas that is predominantly
blue-shifted which indicates a flow toward
Earth and away from the galaxy.  These transitions are sensitive to
the cool (\ion{Mg}{2}; $T \sim 10^4$K) and warm (\ion{C}{4}; $T \sim
10^5$K) phases of the flow.  The incidence of cool gas outflows is
nearly universal in star-forming galaxies;  this includes systems at $z \sim 0$
which exhibit \ion{Na}{1} and \ion{Mg}{2} absorption
\citep{rvs05a,martin05,smn+09}, $z \sim 1$ star-forming galaxies traced by
\ion{Fe}{2} and \ion{Mg}{2} transitions \citep{wcp+09,rubin09}, and
$z>2$ Lyman break galaxies (LBGs) that show blue-shifted \ion{Si}{2},
\ion{C}{2}, and \ion{O}{1} transitions \citep{sgp+96,lkg+97,shapley03}.

The observation of metal-line absorption is now a well established
means of identifying outflows. Very little research,
however,  has been directed
to comparing the observations against (even idealized) wind
models.  Instead, researchers have gleaned what limited information
is affored by direct analysis of the absorption lines.  The data do yield
robust measurements of the speed of the gas yet they poorly constrain the
optical depth, covering fraction, density, temperature, and distance
of the flow from the galaxy.   In turn, constraints related to the
mass, energetics, and momentum of the flow suffer from
orders of magnitude uncertainty.  Both the origin and impact of
galactic-scale winds, therefore, remain open matters of debate
\citep{debate}.

Recent studes of $z \sim 1$ star-forming galaixes have revealed that
the cool gas often exhibits significant resonant-line emission (e.g.\
\ion{Mg}{2}) in
tandem with the nearly ubiquitous blue-shifted absorption
\citep{wcp+09,rubin09,rubin+10b}.  The resultant spectra resemble the P-Cygni
profile characteristic of stellar winds.
\cite{rubin+10a} have also detected line
emission from non-resonant \ion{Fe}{2}$^*$ transitions 
Their interpretation is that the emission is the result of photons
scattered off the wind into our sightline, analagous to P-Cygni.
\cite{rubin+10a} also observed spatially extended \ion{Mg}{2} emission
which they use to estimate the size of the outflow.  They infer that
the wind extends to at least 5\,kpc from the galaxy.  Line emission is
also observed for $z \sim 3$ LBGs in the resonant \lya\ transition
and non-resonant \ion{Si}{2}$^*$ transitions \citep{prs+02,shapley03}.
A comprehensive analysis of the emission lines
related to galactic-scale outflows
(e.g.\ via deep integral-field-unit observations) may offer unique
diagnostics on the morphology and density of the outflow, eventually
setting tighter constraints on the energetics of the flow.  

Although astronomers are rapidly producing a wealth of observational
datasets on galactic-scale winds, a key ingredient to a proper
analysis is absent.
[Draw analaogy to exploring SN lightcurves and spectra]

In this paper, we take the first steps toward modeling the absorption
and emission properties of cool gas outflows.  Using Monte Carlo
radiative transfer technqiues, we study the nature of \ion{Mg}{2} and
\ion{Fe}{2} absorption and emission for winds with a range of
properties.  Although the winds are idealized, the results frequently
contradict our intuition and 
challenge the straightforward conversion of observables to (even crude) physical
constraints.  [Add another line or two]

The paper is organized as follows.  In $\S$~\ref{sec:method}, we
describe the methodology of our radiative transfer algorithms.  These
are applied to a fiducial wind model in $\S$~\ref{sec:fiducial} and
variations of this model in $\S$~\ref{sec:variants}.  In
$\S$~\ref{sec:alternate}, we explore wind models with a broader range
of density and velocity laws.  Consideration of key observables is
given in $\S$~\ref{sec:obs} and we discuss the principal results
in $\S$~\ref{sec:discuss}.  A brief summary is given in
$\S$~\ref{sec:summary}.

\section{Methodology}
\label{sec:method}

This section describes our methodology for generating
emission/absorption profiles from simple wind models.

\subsection{The Radiative Transitions}

In this paper, we focus on two sets of radiative transitions
arising from Fe$^+$ and Mg$^+$ ions
(Table~\ref{tab:atomic}, Figure~\ref{fig:energy}).
This is a necessarily limited
set, but the two ions and their transitions do have characteristics
shared by the majority of low-ion transitions
observed in cool-gas outflows. Therefore, many
of the results that follow may be generalized to observational studies that
consider other atoms and ions of cool gas.

The Mg$^+$ ion, with a single 3s electron in the ground-state,
exhibits an alkali doublet of transitions at $\lambda \approx
2800$\AA\ analagous to the
\lya\ doublet of neutral hydrogen.  Figure~\ref{fig:energy}
presents the energy level diagram for this 
\mgiid\ doublet.  In non-relativistic quantum
mechanics, the 2p$^6$3p energy level is said to be split by spin-orbit
coupling giving the observed line doublet.  These are the only
\ion{Mg}{2} electric-dipole transitions 
with wavelengths near 2800\AA\ and the transtion connecting
the $\rm {}^2P_{3/2}$ and $\rm {}^2P_{1/2}$ states is forbidden by several
selection rules.  Therefore, an absorption from
\maconfig~$\to$~\mbconfig\
is followed $\approx 100\%$ of the time by a spontaneous decay
($t_{\rm decay} \approx 4\sci{-9}$s) to the
ground state. Our treatment will ignore any other possibilities
(e.g.\ absorption by a second photon when the electron is at the \mbconfig\ level).

In terms of radiative transfer, the 
\mgiid\ doublet is very similar to that for \ion{H}{1}
\lya, the \naid\ doublet, and many other doublets commonly
studied in the interstellar medium (ISM) of distant galaxies.  
Each of these has the ground-state connected to a pair of electric
dipole transitions with nearly identical energy.
The doublets differ only in 
their rest wavelengths and the energy of the doublet separation. 
For \ion{H}{1} \lya, the
separation is sufficiently small ($\Delta v = c \Delta E / E \approx
1.3 \mkms$) that most radiative transfer treatments actually ignore it
is a doublet.
This is generally justifiable for \lya\ because 
most astrophysical processes have turbulent motions that
significantly exceed the doublet's velocity separation and effectively mix the
two transitions.  For \ion{Mg}{2} ($\Delta v \approx 770 \mkms$),  
\ion{Na}{1} ($\Delta v \approx 304 \mkms$), and most of the other doublets
commonly observed, the separation is large and the transitions
must be treated separately.  

Iron exhibits the most complex set of energy levels for elements
frequently studied in astrophysics.  The Fe$^+$ ion alone has over XXX
energy levels recorded \citep{iron}, and even this is an
incomplete list.  
One reason for iron's complexity is
that the majority of its configurations exhibit fine-structure splitting.
This includes the ground-state configuration (\aconfig) which is split
into 5 levels, 
labelled by the total angular momentum $J$, 
with excitation energies $T_{\rm ex} \equiv \Delta E / k$ ranging from
$T_{\rm ex} \approx 500-1500$\,K (Figure~\ref{fig:energy}).  
Transitions between these fine-structure levels are 
forbidden (magnetic-dipole) and have spontaneous decay times of several hours.  

In this paper, we examine transitions between the ground-state
configuration and the energy levels of the \zconfig\
configuration.  This set of transitions (named the
UV1 mutliplet) have wavelengths near 2600\AA.
There are two resonance-line transitions\footnote{We adopt the
  standard convention that a ``resonance line'' is an electric-dipole
  transition connected to the ground-state.  We also label
  non-resonsant transitions with an asterisk, e.g.\ \feiic.} 
associated with this multiplet (\feiid)
corresponding to $\Delta J = 0, -1$; these are indicated by upward (black) arrows
in Figure~\ref{fig:energy}. The solid (green) downward
arrows in Figure~\ref{fig:energy} mark the transitions that are
connected to
the upper energy levels of the resonance lines.  These transitons may
occur following the absorption of a single photon by Fe$^+$ in its
ground-state.  

The Figure also shows (as dashed, downward arrows) two of the
\feiis\ transitions that connect to higher energy levels of the \zconfig\
configuration.  Ignoring collisions and recombinations, these
transitions may only can only occur after the absorption
of two photons: one to raise the electron from the ground-state to an
excited state and another to raise the electron from the excited state
to one of the \zconfig\ levels with $J \le 5/2$.  The excitation of
fine-structure levels by 
the absorption of UV photons is termed indirect UV pumping
\citep[e.g][]{silva02,pcb06} and requires the ion to lie
near an intense source of UV photons (see $\S$~\ref{sec:pump}).  In the
following, we will assume that this process does not occur. 

Our calculations also ignore collisional
processes\footnote{Recombination is also ignored.}, i.e.\ collisional
excitation and de-excitation of the various levels.  For the
fine-structure levels of the \aconfig\ configuration, the excitation
energies are modest ($T_{\rm ex} \sim 1000$\,K) but the critical
density $n_c$ is large.  For the \aconfig$_{9/2} \to
$\aconfig$_{7/2}$ transition, the critical density $n_e^C \approx 4
\sci{5} \cm{-3}$.  At these densities, one would predict  
detectable quantities of \ion{Fe}{1} which has not yet been observed
in galactic-scale outflows
\citep[see also][]{pcb06}.  Furthermore, observations rarely show
{\it absorption} from the
fine-structure levels of the \aconfig\ configuration and even that material
is not significantly blue-shifted \citep{rubin+10}.
In the following, we assume that electrons only occupy the
ground-state, i.e.\ the gas has zero opacity to the non-resonant
lines.
Empirically, because collisional excitation is insignificant and
then one may also neglect collisional de-excitation.  
The excitation energy for Mg$^+$ ($T_{\rm ex} \approx 50,000$\,K),  
meanwhile, is significantly higher implying 
negligible collisional processes.

\subsection{The Source}

Nearly all of the absorption studies of galactic-scale outflows 
have focused on intensely, star-forming galaxies.  The
intrinsic emisison of these galaxies is a complex combination of
stars and \ion{H}{2} regions that is then modulated by dust and gas
within the interstellar medium (ISM).  For the spectral regions studied
here, most stars show a featureless continuum but a few spectral
types do show significant \ion{Mg}{2} and \ion{Fe}{2} absorption.
[Any emission??]  \ion{H}{2} regions, meanwhile, are observed to emit
at the \mgiid\ doublet, primarily due to recombinations in the outer
layers [How about FeII?].  It is beyond the scope of this paper to
properly model the 
stellar absorption and \ion{H}{2} region emission, but the reader
should be aware that they can complicate the observed spectrum,
independent of the outflow, especially at velocities $v \approx 0 \mkms$.
In the following, we will simply assume a flat continuum 
normalized to unit value.  The size of the emitting
region $r_{\rm source}$ is a free parameter, but we tend to
restrict its value to be smaller than the minimum radial extent of any
gaseous component.


\subsection{Monte Carlo Algorithm}

[Describe 1D and 3D algorithms]

[Put tests in an Appendix?]

\subsection{Dust}
\label{sec:dust_method}

For the majority of models studied in this paper, we assume no dust.
This is an invalid assumption, especially for material associated with
the ISM of a galaxy.  
There is also empirical evidence for dust emission from galactic-scale
outflows \citep[e.g.][]{M87_dust}.
Because absorption line analysis is performed
on normalized spectra, these effects of dust are largely minimized; 
the effects are corrected by simply rescaling the normalized flux.  For
scattered and resonantly trapped photons, however, the relative effect of dust
extinction can be much greater.  Indeed, dust is frequently invoked to
explain the weak (or absent) \lya\ emission from star-forming galaxies
\citep[e.g.][]{shapley03}.  Although the transitions studied here have
much lower opacity than \lya, dust could play an important role
in the predicted profiles.

In a few models, we include absorption by dust under the following
assumptions:
(i) the dust opacity scales with the density of the gas (i.e.\ we
adopt a fixed dust-to-gas ratio);
(ii) the opacity is independent of wavelength, a reasonable
approximation given the small spectral range analyzed;
(iii) the photons absorbed by dust are re-emitted at IR wavelengths
and are `lost' from the system.  The dust absorption is normalized by \taud, 
the integrated opacity of dust from the center of the system to
infinity.  The ISM of a star-forming galaxy may be expected to exhibit
\taud\ values of one to a few \citep[e.g.][]{dust_again}.



\section{The Fiducial Wind Model}
\label{sec:fiducial}

In this section, we study a simple yet illustrative wind model for
a galactic-scale outflow.  The properties of this wind were tuned, in
part, to yield a \ion{Mg}{2} absorption profile 
similar to those observed for $z \sim 1$, star-forming galaxies
\citep{wcp+09,rubin+10b}.  We emphasize, however, that we do not
favor this fiducial model over any other wind scenario nor do its
properties have special physical motivation.
Its primary purpose is to establish a baseline
for discussion.

The fiducial wind follows a (mass-flux conserving) density law,

\begin{equation}
n_{\rm H} (r) = \frac{n^0_{\rm H}}{r^2} \;\;\; , 
\label{eqn:density}
\end{equation}
and a velocity law with a purely radial flow

\begin{equation}
\vec v = v_r (r) \hat r = v_0 r \hat r  \perd
\label{eqn:vel}
\end{equation}
Turbulent motions are
characterized by a Doppler parameter $b_{\rm turb}$.  
The wind is isotropic, dust-free, and extends from an inner wind
radius $r_{\rm inner}$ to an outer wind radius $r_{\rm outer}$.  
We convert the hydrogen density $n_{\rm H}$ to the number densities of
Mg$^+$ and Fe$^+$ ions by assuming solar relative abundances with an
absolute metallicity of 1/2 solar and depletion factors of 1/10 and
1/20 for Mg and Fe respectively, i.e.\  $\mnmg = 10^{-5.47} n_{\rm H}$ 
and \nfe=\nmg/2. At the center of the wind is a homogenous source of
continuum photons with size $r_{\rm source}$. These parameters are
summarized in Table~\ref{tab:fiducial}.   

In Figure~\ref{fig:fiducial_nvt} we plot the density and velocity
laws against radius;  
their single power-law expressions are evident.  The figure also
shows the optical depth profile for the \mgiia\ transition ($\tau_{2796}$).
Although plotted against radius, we calculated $\tau_{2796}$ first
as a function of velocity by summing the opacity for a series of
discrete and small radial intervals
between $r_{\rm inner}$  and $r_{\rm outer}$.   We then mapped
$\tau_{2796}$ onto radius using the velocity law
(Equation~\ref{eqn:vel}). 
The $\tau_{2796}$ profile peaks with $\tau_{2796}^{\rm max} \approx 30$
at a velocity $v \approx 65 \mkms$ corresponding to $r \approx 1.3
r_{\rm inner}$.  The optical depth profile for the \mgiib\ transition
(not plotted) is scaled down by the $f\lambda$ ratio but is otherwise identical.  Similarly,
the optical depth profiles for the \feiid\ transitions are
scaled down by $f \lambda$ and the \nfe/\nmg\ ratio.  
The adopted Doppler parameter
($\mbturb = 15 \mkms$) has a minor impact on the results.
The absorption profiles are insensitive to its value and varying \bturb\
only tends to modify the widths and modestly
shift the centroids of emission lines.

Because the results are dictated by the $\tau_{2796}$ profile in
velocity space, the actual dimensions and density of the wind are
essentially unimportant provided they scale together to give nearly the same
$\tau_{2796}$ profile.  Therefore, one may consider the choices for
$r_{\rm inner}, r_{\rm outer}, n_{\rm H}^0$ as arbitrary.
Nevertheless, we adopted values for this fiducial model with
some astrophysical motivation,  e.g., values that correspond to
galactic dimensions and a normalization that gives $\tau_{2796}^{\rm
  max} \sim 10$.


Using the methodology described in $\S$~\ref{sec:method}, we
propogated photons from the source and through the outflow to an
`observer' at $r \gg r_{\rm outer}$ that views the entire wind+source
complex.  Figure~\ref{fig:fiducial_1d} presents the 1D spectrum
that this observer would record, with the unattenuated flux
normalized to unit value.   The \ion{Mg}{2} doublet
shows the canonical `P-Cygni profile' that characterizes a continuum
source embedded within an outflow.  Strong absorption is evident at
$v < -50 \mkms$ in both transitions (equivalent widths, $W_{2796} =
3.0$\AA\ and $W_{2803} = 1.3$\AA) and each shows emission at
positive velocities.  For this isotropic and dust-free model, the
total equivalent width of the doublet must be zero
(Table~\ref{tab:fiducial_EW}), i.e.\ every photon
absorbed eventually escapes the system, typically at lower
energy.  The wind simply shuffles the photons in frequency space.

Focusing further on the \ion{Mg}{2} absorption, one notes that the profiles lie
well above zero intensity and have similar depth even though their $f\lambda$
values differ by a factor of two.  In standard absorption line
analysis, this is 
the tell-tale signature of a `cloud' that has a high optical depth (i.e.\
saturated) which only partially covers the emitting source
\citep[e.g.][]{hamann+10}.  Our fiducial wind model, however, 
{\it entirely covers the source}; the apparent partial covering must
be related to an alternate effect.
Figure~\ref{fig:noemiss} further emphasizes this point by comparing the 
absorption profiles from Figure~\ref{fig:fiducial_1d} against an
artificial model where no absorbed photons are 
re-emitted.   As expected from the
$\tau_{2796}$ profile (Figure~\ref{fig:fiducial_nvt}), the
`no-emission' model
produces a strong \mgiid\ doublet that absorbs all photons at
$v \approx -100 \mkms$, i.e.\ $I_{2796}^{\rm min} = \exp(-\tau^{\rm
  max}_{2796}) \approx 0$.
The fiducial model, in contrast, has been `filled in' at $v \approx -100
\mkms$ by photons scattered and re-emitted by the wind.  An
absorption-line analysis that ignores these effects
would (i) systematically underestimate the true optical
depth and (ii) falsely conclude that the wind partially covers the
source.  We will find that this
behaviour is a generic result of the wind models considered, even for cases that are
not fully isotropic.

Turning to the emission profiles of the \mgiid\ doublet, one notes
that they are also similar with comparable equivalent widths The
flux of the \mgiib\ transition even exceeds that for \mgiia\ giving a
line ratio that is far below the $2:1$ ratio that one may have naively
expected. 
[Does recombination predict 2:1??]
The emission profiles are very similar because the gas is optically
thick for a significant portion of the profile. 
Furthermore, the flux from \mgiib\ exceeds the
\mgiia\ emission because the wind speed is greater than the velocity separation
of the doublet, $|\mvr|_{\rm max} > (\Delta v)_{\rm MgII}$.
Therefore,
the \mgiib\ absorption profile partially absorbs the red wing of the
\mgiia\ emission profile.  The relative strengths of the emission
lines is sensitive to both the opacity and velocity
extent of the wind.  

Now consider the \ion{Fe}{2} transitions.
The bottom left panel of Figure~\ref{fig:fiducial_1d} covers the
majority of the \ion{Fe}{2} UV1 transitions and several are
shown in the velocity plot.  The line
profile for \feiib\ is very similar to the \mgiid\ doublet;
one observes strong absorption to negative velocities and strong
emission at $v > 0 \mkms$ producing a characteristic P-Cygni profile. 
Splitting
the profile at $v = -50 \mkms$, we measure an equivalent width
$W_{2600}^{\rm abs} = 1.16$\AA\ in absorption and $W_{2600}^{\rm em} =
-0.94$\AA\ in emission (Table~\ref{tab:fiducial_EW}) for a total
equivalent width of $W_{2600}^{\rm TOT} \approx 0.22$\AA.  
In contrast, the \feiia\ resonance line shows much weaker emission and
a much higher total equivalent width ($W_{2586}^{\rm TOT} = 0.49$\AA),
even though the line has a $2 \times$ lower $f\lambda$ value.
These differences between the \ion{Fe}{2} resonance lines and with the
\ion{Mg}{2} doublet occur because of the complex of non-resonant
\ion{Fe}{2} transitions (Figure~\ref{fig:energy}).  Specifically,
resonance photons absorbed at \feiid\ have a finite probability of
being re-emitted as a non-resonant photon which escapes the system
without further interaction.  The principal effects are to reduce the
line emission of \feiid\ and produce non-resonant line-emission (e.g.\
\feiic).

The reduced \feiia\ emission relative to \feiib\ is related to
two factors:
(i) there is an additional downward transition from the
\zconfig$_{7/2}$ level and 
(ii) the Einstein A
coefficients of the non-resonant lines are comparable to and even
exceed the Einstein A coefficient of
the resonant transition.  In contrast, 
the \ion{Fe}{2}~2626 transition (associated with \feiib)
has an approximately  $4\times$ smaller A coefficient than the
resonance line.  Therefore, the majority of photons absorbed at
$\lambda \approx 2600$ are re-emitted as \feiib\ photons whereas 
the majority of photons absorbed at $\lambda \approx 2586$ are re-emitted 
at longer wavelengths (\ion{Fe}{2}$^* \; \lambda 2612$ or $\lambda
2632$).
If we were to increase $\tau_{2600}$ (and especially if we include
gas with $\mvr \approx 0 \mkms$) then the emission at \feiib\ is
significantly suppressed (e.g.\ $\S$~\ref{sec:ISM}).
The total equivalent width, however, of the three lines connected to the
\zconfig$_{7/2}$ upper level must still vanish (photons are conserved
in this fiducial wind model).


The preceding discussion emphasizes the filling-in of resonance absorption at $v
\lesssim -50 \mkms$ and the generation of emission lines at $v \approx
0 \mkms$ by photons scattered in the wind.  To study the spatial
extent of this emission, we perfomed 3D calculations with the fiducial
model.  The output is a set of surface-brightness maps in a series of
frequency channels yielding a
dataset analagous to integral-field-unit (IFU) observations.  In
Figure~\ref{fig:fiducial_ifu_mgii}, we present the output 
at several velocities relative to the \mgiia\
transition. At $v = -250 \mkms$, where the wind has an optical
depth $\tau_{2796} < 1$ (Figure~\ref{fig:fiducial_nvt}),
the source contributes roughly half of the observed flux.  
At $v=-100 \mkms$, however, the
wind absorbs all photons from the source and the observed emission is
entirely from photons scattered by the wind.  This scattered emission
actually exceeds the source+wind emission at 
$v = -250 \mkms$ so that the absorption profile is
negatively offset from the velocity where $\tau_{2796}$ is maximal
(Figure~\ref{fig:fiducial_nvt}).
The net result is
a weaker \ion{Mg}{2} absorption line peaks blueward of the peak in the
optical depth profile.  Clearly, these effects complicate estimates for the
speed, covering fraction, and total column density of the wind.  At $v
= 0 \mkms$, the wind and source have comparable total flux with the
latter dominating at higher velocities.  

Similar results are observed for the \ion{Fe}{2} resonance
transitions (Figure~\ref{fig:fiducial_ifu_feii}).
For transitions to fine-structure levels of the \aconfig, the source
is unattenuated but there is a significant contribution from photons
generated in the wind. 
The results presented in Figures~\ref{fig:fiducial_ifu_mgii} and
\ref{fig:fiducial_ifu_feii} are sensitive to the radial extent,
morphology, density and velocity profiles of this galactic-scale
wind.  Consequenetly, IFU observations of line emission from low-ion
transitions may offer the most direct constraints on galactic-scale
wind properties. 

At all velocities, the majority of light comes from the inner regions
of the wind. 
Figure~\ref{fig:fiducial_cuts} presents the relative flux for the 
azimuthally symmetric surface-brightness maps collapsed along
one spatial dimension.  The majority of
\ion{Mg}{2} emission occurs within the inner few kpc, e.g.\ $50-60\%$
of the light at $v=-100$ to $+100 \mkms$
comes from $|r| < 3$\,kpc.
The emission
is even more centrally concentrated for the \ion{Fe}{2} transitions.
A proper treatment of these
distributions is critical to interpret observations
acquired through a slit, i.e.\ where the aperture has a limited extent
in one or more dimensions.  A standard longslit on 10m-class
telescopes, for example, subtends $\approx 1''$ corresponding to
$5-10$\,kpc for $z \sim 1$.    We return to this issue in
$\S$~\ref{sec:obs}. 


%%%%%%%%%%%%%%%%%%%%%%%%%%%%%%%%%%%%%%%%%%%%%%%%%%%%%%%%%%
\section{Variations to the Fiducial Model}
\label{sec:variants}

In this section, we investigate a series of more complex wind
scenarios
through modifications to the fiducial model.  These include relaxing
the assumption of isotropy, introducing dust, adding an ISM
component within $r_{\rm inner}$, and varying the normalization of the
optical depth profiles.

\subsection{Anisotropic Winds}
\label{sec:anisotropic}

The fiducial model assumes an
isotropic wind with only radial variations in velocity and density. 
Angular isotropy is obviously an idealized case, but
it is frequently assumed in studies of galactic-scale outflows
\citep[e.g.][]{steidel+10}.   There are several reasons, however, to
consider anisotropic winds.  Firstly, galaxies are not spherically
symmetric;  the sources driving the
wind (e.g.\ supernovae, AGN) are very unlikely to be isotropically distributed
within the galaxy.  
Secondly, the galactic ISM frequently has a disk-like morphology
which will suppress the wind preferentially at low galactic latitudes,
perhaps yielding a bi-conic morphology \citep[e.g.][]{M87}.
Lastly, a galaxy could be surrounded by an
anisotropic gaseous halo that would produce an irregularly shaped
wind.

With these considerations in mind, we reanalyzed the fiducial model
with the 3D algorithm after departing from isotropy.  It is beyond the
scope of this paper to explore a full suite of anisotropic profiles;
the following simply assumes half of the fiducial wind by setting
the density to zero for $2\pi$ steradians.
We have viewed this system from $\theta = 0^\circ$ where
the source is observed directly to $\theta = 180^\circ$ where the source
is covered by this anisotropic wind.  The resulting \ion{Mg}{2} and
\ion{Fe}{2} profiles are compared against the fiducial model
(isotropic wind) in Figure~\ref{fig:anisotropic}.  

Examining the \mgiid\ doublet, 
there is no attenutation of the source by the wind for $\theta =
0^\circ$
but one does observe significant line emission from photons scattered
off the back side.  These photons, by definition, have $v \gtrsim 0 \mkms$
relative to line-center (a subset have $v \lesssim 0 \mkms$ because
of turbulent motions in the wind). 
When viewed from the opposite direction ($\theta = 180^\circ$), the
absorption lines dominate but there is still significant
line-emission -- at $v \approx 0 \mkms$ and at $v < 0 \mkms$ which fills
in the absorption -- from photons that scatter through the wind.  The
key difference from the isotropic wind is the absence of photons
scattered to $v > 100 \mkms$;  this also implies deeper 
\mgiib\ absorption at $v \approx -100 \mkms$. The 
shift in velocity centroid and asymmetry of the emission lines
serve to diagnose the degree of wind isotropy, especially in
conjunction with analysis of the absorption profiles. 

The results are similar for the \feiid\ resonance lines.  The
\ion{Fe}{2}$^* \; \lambda 2612$ line, meanwhile, shows most clearly the
offset in velocity between the source unobscured ($\theta = 0^\circ$)
and source covered ($\theta = 180^\circ$) cases.  The offset of the
\feiis\ lines is the most significant 
difference from the fully isotropic wind.

%%%%%%%%%%%%%%%%%%%%%%%%%%%%%
\subsection{Dust}
\label{sec:dust}

Essentially all astrophysical environments that contain both cool gas
and metals also show signatures of dust depletion and extinction.  This includes the
ISM of star-forming and \ion{H}{1}-selected galaxies
\citep[e.g.][]{ss96,pw01,pcd+07}, strong \ion{Mg}{2} metal-line
absorption systems \citep{ykv+06,mnt+08}, and the galactic winds traced
by low-ion transitions \citep{prs+02,rvs05b}.  Although the galactic winds
traced specifically by \ion{Mg}{2} and \ion{Fe}{2} transitions have not (yet) been
demonstrated to contain dust, it is reasonable to consider its
effects.

Dust modifies the observed flux in two manners. 
First, it is a source of opacity for all of 
the photons.  This suppresses the flux at all
wavelengths by $\approx \exp(-\mtaud)$ but because we re-normalize the
profiles, this effect is essentially ignored.  Second, photons that are
scattered by the wind will travel a greater
distance and therefore suffer from greater extinction.  A photon that is
trapped for many scatterings may have a very high probability of being absorbed
by dust.  This process is
often the explanation given for the weak (or absent) \lya\ emission
associated with star-forming galaxies. 
Section~\ref{sec:dust_method} describes our treatment of dust; we 
remind the reader that we assume a constant dust-to-gas ratio 
normalized by the total optical
depth \taud\ that a photon would experience if it travelled from the
source to infinity without scattering. 

In Figure~\ref{fig:dust}, we show the \ion{Mg}{2} and \ion{Fe}{2}
profiles of the fiducial model ($\mtaud = 0$) against a series of
models with $\mtaud > 0$.  For the \ion{Mg}{2} transitions, the
dominant effect is the suppression of line emission at $v \ge 0
\mkms$.  These `red' photons have scattered off the
backside of the wind and must travel a longer path than other
photons.  Dust leads to a differential reddening that increases with wavelength (and
velocity) relative to line-center. This is a natural consequence of dust
extinction and is most evident in the \ion{Fe}{2}$^* \; \lambda 2612$
emission profile which is symmetrically distributed around
$v=0 0 \mkms$ in the $\mtaud=0$ model.   In terms of absorption, the profiles are
nearly identical for $\mtaud \le 3$.  One requiresvery high
extinction to see a deepening of the profiles at $v \approx -100 \mkms$.

% The following could go in the Discussion section
We conclude that dust has only a modest influence on this fiducial model and,
by inference, models with moderate peak optical depth and
significant velocity gradients with radius (i.e.\ scenarios where the
photons scatter one to a few times before exiting).
For qualitative changes, one requires an extreme level of
extinction ($\mtaud = 10$).  In this case, the source would be
extinguished by 15\,magnitudes and could never be observed. 
Even $\mtaud = 3$ is larger than typically inferred for the
star-forming galaxies that drive outflows \citep[e.g.][]{dust}.
For the emission lines,
the dominant effect is a reduction in the flux 
with a greater extinction at higher velocities relative to line-center.
In these respects, dust extinction crudely mimics the behavior of the anistropic
wind described in Section~\ref{sec:anisotropic}. [Is there a
distinguishing factor, e.g. flux of 2612/2600??]


\subsection{ISM}
\label{sec:ISM}

The fiducial model does not include gas associated with
the interstellar medium of the galaxy, i.e.\ material at $r \sim
0$\,kpc with $v_{\rm r} \approx 0 \mkms$.  This allowed us to focus on
results related solely to the wind.  The decision to ignore the ISM
was also motivated by the general absence of significant absorption at
$v \approx 0 \mkms$ in galaxies that exhibit outflows 
\citep[e.g.][]{wcp+09,rubin09,steidel+10}.
On the other hand, the stars and \ion{H}{2} regions that comprise our
background sources are very likely embedded within and fueled by gas
of the galactic ISM.  
Consider, then, a modification to the fiducial model that has the
same wind but also includes an ISM component. 
Specifically, the ISM component has $n_{\rm H} = 1 \cm{-3}$ for
$r_{\rm ISM} \le r < r_{\rm inner}$ with $r_{\rm ISM} = 0.5$\,kpc, 
an average velocity of $\mvr = 0 \mkms$, and a larger turbulent velocity $b_{\rm ISM} = 40 \mkms$.
Figure~\ref{fig:ISM} summarizes the model.
The resultant optical depth profile $\tau_{2796}$ is identical to the
fiducial model for $r > 2$\,kpc, a slightly higher opacity at
$r=1-2$\,kpc, and a large opacity at $r = 0.5-1$\,kpc.

In Figure~\ref{fig:ISM_spec}, the solid curves show the \ion{Mg}{2} and
\ion{Fe}{2} profiles for the ISM+wind and fiducial 
models. In comparison, the
dotted curve shows the absorption profile for the ISM+wind in the (unphysical) case
where none of the absorbed photons are scattered or re-emitted.   Focus first on the
\ion{Mg}{2} doublet.  As expected, the dotted curve shows strong
absorption at $v \approx 0 \mkms$ and blueward.  The full models,
in contrast, show non-zero flux at these velocities and even a
normalized flux exceeding unity at $v \approx 0 \mkms$.  In
fact, the ISM+wind model is nearly identical to the fiducial model;
the only quantitiative difference is that the velocity centroid of
the emission lines are shifted redward by $\approx +100 \mkms$.

There are, however, several qualitiative differences 
for the \ion{Fe}{2} transitions. 
First, the \feiia\ transition in the ISM+wind model
shows much stronger absorption at $v \approx 0$
to $-100 \mkms$.  In contrast to the \ion{Mg}{2} doublet,
the profile is not filled in by scattered photons. Instead, 
the majority of \feiia\ photons that are absorbed are re-emitted as
\ion{Fe}{2}~$^* \lambda\lambda 2612, 2632$ photons.  In fact, the
ISM+wind \feiia\ profile nearly matches the profile without re-emission 
(compare to the dotted lines); this transition provides a
very good description of the intrinsic ISM+wind optical depth profile.  
We conclude that resonant transitions tha are coupled to (multiple)
non-resonant, electric dipole transitions offer the best
diagnostic of ISM absorption.

The differences in the \ion{Fe}{2} absorption profiles are reflected
in the much higher strengths ($3-10\times$) of emission from
transitions to the excited states of \aconfig.   This occurs because:
(1) there is greater absorption by the \feiid\
resonance lines; (2) the high opacity of the ISM component leads to
an enhanced conversion of resonance photons with $v \approx 0\mkms$
into \feiis\ photons.  This is especially notable for the
\feiib\ transition whose `partner' shows an equivalent width nearly
$10\times$ stronger than for the fiducial model.  The relative
strengths of the \feiib\ and \feiie\ lines provide a direct
diagnostic on the degree to which the resonance lines are trapped.
[Comment:  this is a product of $\tau_{2600}$ and the velocity
gradient]

[3D discussion; Figure~\ref{fig:ISM_ifu}]

[Fix the bug in the FeII* emission]

[Do dust too (as a green curve in Figure~\ref{fig:ISM_spec})]

\subsection{Varying $n_{\rm H}^0$}

[Consider a series of models with the optical depth varying]

\subsection{Summary Table}

Table~\ref{tab:line_diag} presents a series of quantitative measures
for the \ion{Mg}{2} and \ion{Fe}{2} absorption and emission lines for
the fiducial model ($\S$~\ref{sec:fiducial}) and a set of the models
presented in this section.  Listed are the
equivalent widths (absorption and emission), the peak optical depth
for the absorption $\tau_{\rm pk} \equiv -\ln(I_{\rm min})$, the
velocity where the optical depth peaks $v_\tau$, the optical
depth-weighted velocity centroid $v_{\bar \tau} \equiv \int dv \, v
\ln[I(v)] / \int dv \ln[I(v)]$, the peak flux $f_{\rm pk}$ in
emission, the velocity where the flux peaks $v_f$, and the
flux-weighted velocity centroid of the emission line $v_{\bar f}$.

One observes a wide range in these measures.  This emphasizes the
challenges to convert 1D spectra to physical
constraints on wind scenarios.  We consider these
results further in $\S$~\ref{sec:obs}.

%%%%%%%%%%%%%%%%%%%%%%%%%%%%%%%%%%%%%%%%%%%%%%%%%%%%%%%%%%%%%%%%%
\section{Alternate Wind Models}
\label{sec:alternate}

\subsection{Power-Law Models}
\label{sec:power}

The models investigated in the previous two sections
assumed power-law descriptions for both the density and velocity
expressions of the wind (Equations~\ref{eqn:density},\ref{eqn:vel}).
The power-law exponents were arbitrarily chosen, with minimal physical
motivation.  In this sub-section, we explore the results for a series
of other power-law expressions.
We consider three different density laws ($n_{\rm H}(r) \propto
r^{-3}, r^{0}, r^2$) and three different velocity laws ($\mvr(r)
\propto r^{-2}, r^{-1}, r^{0.5}$) for 9 new wind models
(Table~\ref{tab:plaws}, Figure~\ref{fig:plaws}a).  [Comment in the Table how $\mvr$ is
different when the velocity law is positive]  
The density and velocity normalizations $n_{\rm H}^0, v^0$ have been
modified to yield an \mgiia\ optical depth profile that peaks at
$\tau_{2796}^{\rm max} \approx 10-10^3$ that usually decreases to
$\tau_{2796} < 1$ (Figure~\ref{fig:plaws}b).

[Discuss models]

[Refer to Table~\ref{tab:plaw_diag}]

\subsection{Radiation Pressure}

[Kate: Write the details of velocity + density law]

\begin{equation}
\mvr(r) = 2\sigma \sqrt{R_g \ltp \frac{1}{R_0} - \frac{1}{r} \rtp
   + \ln\ltp R_0/r \rtp }
\end{equation}

\begin{equation}
n(r) = \frac{dM_{\rm wind}/dt}{r^2 v(r)}
\end{equation}

Figure~\ref{fig:rad_nvt}, \ref{fig:rad_spec}

\subsection{The Lyman Break Galaxy Model}
\label{sec:lbg}

The Lyman break galaxies (LBGs), UV color-selected galaxies at $z \sim 3$,
exhibit cool gas outflows in \ion{Si}{2}, \ion{C}{2},
etc.\ transitions with speeds up to 1000\,\kms\
\citep[e.g.][]{lkg+97,pks+98}.
Researchers have invoked these winds to explain enrichment of
the intergalactic medium \citep[e.g.][]{aguirre,spa+06}, the origin of the
damped \lya\ systems \citep{nbf98,schaye01a}, and XXX.  Although the
presence of these outflows were established over a decade ago,
the processes that drive them remain
unidentified.  Similarly,  current estimates of the mass and energetics of the
outflow suffer from orders of magnitude uncertainty.

Recently, \cite[][; hereafter S10]{steidel+10} introduced a model to
explain jointly the average absorption they observed
toward a set of several hundred LBGs and the average absorption in gas
observed transverse to these galaxies.  
Their wind model is defined by two
expressions: (i) a radial velocity law $\mvr(r)$ and (ii) the covering
fraction of optically thick `clouds' $f_c(r)$.  For the latter, S10
envision an 
ensemble of small, optically thick $(\tau \gg 1)$ clouds that only
partially cover a galactic-scale source (i.e.\ a few to 10\,kpc).
For the velocity law, they adopted the following functional
form

\begin{equation}
\mvr = \ltp \frac{A_{\rm LBG}}{1-\alpha} \rtp^{1/2} \ltk r_{\rm
  inner}^{1-\alpha} - r^{1-\alpha} \rtk^{1/2}
\label{eqn:LBG_vlaw}
\end{equation}
with $A_{\rm LBG}$ a constant that sets the terminal speed,
$r_{inner}$ is the inner radius of the wind (taken to be 1\,kpc), and
$\alpha$ describes how steeply the velocity curve rises.  Their
analysis of the LBG absorption profiles implied
a very steeply rising curve with $\alpha \approx 1.3$.
This velocity expression is shown as a dotted line in 
Figure~\ref{fig:LBG_Sobolev}a.  

The covering fraction of optically thick clouds, meanwhile, was assumed to have
the functional form

\begin{equation}
f_c(r) = f_{c,max} \ltp \frac{r}{r_{\rm inner}} \rtp^{-\gamma} \cmma
\label{eqn:covering}
\end{equation}
with $\gamma \approx 0.5$ and $f_{c,max}$ the maximum covering
fraction.  From this expression and the velocity law, one can recover
an absorption profile $I_{\rm LBG}(v) = 1 - f_c(r[v])$, written
explicitly as

\begin{equation}
I_{\rm LBG}(v) = 1 - f_{c,max} \ltk r_{\rm inner}^{1-\alpha} - \ltp
\frac{1-\alpha}{A_{\rm LBG}} \rtp v^2 \rtk^{\gamma/(\alpha-1)}
\perd
\label{eqn:LBG_I}
\end{equation}
The resulting profile for $f_{c,max} = 0.6$, $\gamma=0.5$,
$\alpha=1.3$, and $A_{\rm LBG} = -192,000 \, \rm km^2 s^{-2} kpc^{-2}$ 
is displayed in Figure~\ref{fig:LBG_Sobolev}b.  

In the following, we consider two methods to analyze the LBG wind.
Both approaches assume isotropy and adopt the velocity law given by
Equation~\ref{eqn:LBG_vlaw}.  In one model, we treat the cool gas as a
diffuse medium with unit convering fraction and a radial density
profile determined from the Sobolev approximation.  We then apply 
the Monte Carlo methodology used for the other wind models to predict
\ion{Mg}{2} and \ion{Fe}{2} line profiles.  In the other model,
we modify our algorithms to more precisely mimic the concept of an
ensemble of optically thick clouds with a partial covering fraction on
galactic scales.

%%%%%%%%%%%%%%%%%%%%%%%%%%%%%%%%%%%%%%%%%%%%%%%%%%
\subsubsection{Sobolev approximation}
\label{sec:Sobolev}

As demonstrated in Figure~\ref{fig:LBG_Sobolev}, the wind velocity for
the LBG model rises very steeply with increasing radius before
flattening at large radii.  Under these conditions, the Sobolev
approximation provies an accurate relation between the optical depth
of the flowing medium to the
density and velocity gradient ($d\mvr/dr$) of the gas.  In our case, we 
generate the optical depth profile from Equation~\ref{eqn:LBG_I}, 

\begin{equation}
\tau_{\rm LBG}(v) = -\ln \ltk I_{\rm LBG}(v) \rtk \cmma
\label{eqn:tauLBG}
\end{equation}
and invert the Sobolev approximation to derive the density profile.
Following \cite{lc99}, we have for a purely radial flow

\begin{equation}
n_{\rm LBG}(r) = \frac{\tau_{\rm LBG}(r) \; d\mvr/dr}{\kappa_\ell
  \lambda} \cmma
\label{eqn:Sobolev}
\end{equation}
with $\lambda$ the rest wavelength and $\kappa_\ell \equiv f\pi^2
e^2/m_e c$.  

The solid curve in Figure~\ref{fig:LBG_Sobolev}a shows the
resultant density profile for Mg$^+$ assuming that the \mgiia\ line
follows the intensity profile drawn in Figure~\ref{fig:LBG_Sobolev}b. 
This is a relatively extreme density profile.  From the inner radius
of 1\,kpc to 2\,kpc, the density drops by over 2 orders of
magnitude including nearly one order of magnitude over the first
10\,pc.  Beyond 2\,kpc, the density drops even more rapidly, falling
orders of magnitude from 2 to 100\,kpc.

We verified that the density profile shown in
Figure~\ref{fig:LBG_Sobolev}a
reproduces the proper
absorption profile by discretizing the gas into a series of layers
and calculating the integrated absorption profile.  This
calculation is shown as a dotted red curve in
Figure~\ref{fig:LBG_Sobolev}b; it is an excellent
match to the desired profile (black curve).
To calculate the optical depth profiles for the other transitions, we
assume $\mnfe = \mnmg/2$ and scale $\tau$ by the $f\lambda$ product.

We generated \ion{Mg}{2} and \ion{Fe}{2} profiles for this wind
using the 1D algorithm with no dust extinction; these are shown as
red curves in Figure~\ref{fig:LBG_spec}.   For this analysis, one
should focus on the \mgiia\ transition.  The dotted line in the Figure
shows the intensity profile when one ignores re-emission
of absorbed photons.  By construction, it follows\footnote{Note that
  one should not make this comparison for the other transitions in
  the Sobolev model because those are scaled down by $f\lambda$ and
  for \ion{Fe}{2} the reduced Fe$^+$ abundance.} the profile
described by Equation~\ref{eqn:LBG_I} as plotted in
Figure~\ref{fig:LBG_Sobolev}b.   In comparison, the full model (solid,
red curve)
shows much weaker absorption, especially at $v = 0$ to $-300 \mkms$
due to scattered photons.
In this respect, our LBG-Sobolev model is an
inaccurate description of the observations. The model
also predicts significant emission in the \ion{Mg}{2} lines and several of
the \ion{Fe}{2}$^*$ transitions.   Emission associated with cool gas
has been observed for \ion{Si}{2}$^*$
transitions in LBGs \citep{prs+02,shapley03}, but
the \ion{Fe}{2}$^*$ transitions
modeled here lie in the near-IR and have not yet been investigated.
On the other hand, there have been no 
reported detections of significant line-emission related to resonance
transitions (e.g.\ \ion {Mg}{2}) in LBGs, only $z \le 1$ star-forming galaxies
\citep{wcp+09,rubin09}.  
The principal result is that the scattering and re-emission of
absorbed photons signficantly alters the predicted absorption profiles
for the inputted model.  This is, of course, an unavoidable
consequence of an isotropic, dust-free model with unit covering
fraction.
[Transition?]

[Do CIV scattered photons bugger up the P-Cygni analysis of Pettini
et al. ? I bet they do!]
 
%%%%%%%%%%%%%%%%%%%%%%%%%%%%%%%%%%%%%%%%%%%%%%%%%%
\subsubsection{Partial Covering Fraction}
\label{sec:Covering}

In the previous sub-section, we described a Sobolev solution that
reproduces the average absorption profile of LBGs in cool gas
transitions where scattered photons are ignored.  A proper analysis of
this wind model, however, predicts line profiles that are qualitatively
different from the ones observed because scattered photons fill-in
absorption and generate significant line emission (similar to the
fiducial wind model; $\S$~\ref{sec:fiducial}).
The LBG-Sobolev model, however, differs from
the one proposed by S10:  these authors proposed an ensemble of optically
thick clouds with the (parital) covering fraction described by
Equation~\ref{eqn:covering} whereas the Sobolev model assumes
a diffuse medium with a declining density profile but a unit covering
fraction.  At face value, one questions whether this difference leads to
the contradictory results of the model. 

To more properly model the LBG wind described in S10, we 
performed the following Monte Carlo calculation.  First, we propogated
a photon from the source until its velocity relative to line-center
resonates with the wind (the photon escapes if this never occurs).
The photon then has a probability $P = f_c(r)$ of scattering.  If it
scatters, we track the photon until it comes into
resonance again or escapes the system.  In this model, all of the
resonance transitions are assumed to have identical (high) optical
depth. 

The results of the full calculation (absorption plus scattering) are
shown as the black curve in Figure~\ref{fig:LBG_spec}.  The results
are very similar to the Sobolev calculation; scattered photons fill-in
the absorption profiles at $v \approx 0 \mkms$ and yield significant
emission lines at $v \gtrsim 0 \mkms$ in the resonance
lines. Furthermore, significant emission is observed at
all velocities for the \feiis\ transitions.  For \feiia,
the majority of absorbed photons have been converted to its
fine-structure counterparts lending to very weak emission at this
transition. The absorption profile very nearly matches the
model without re-emission indicating that this transition well
reproduces the opacity of the wind.  
The equivalent width of \feiia\ even exceeds that for \feiib, an
inversion that, if observed, would strongly support this model.
[Could SiII or OI do the same?]

Comparing the LBG Sobolev and parital covering models in
Figure~\ref{fig:LBG_spec} (via \mgiia), we note nearly identical
results.  The differences that occur for the other transitions,
meanwhile, are only because the opacities are scaled downward in the
Sobolev model (as required).  [Do the data prefer one to the other?
And what if we modified the density away from Sobolev?]
In principle, they provide a means to distinguish between the two LBG
models.  

[Does most of the scattering occur within the inner few kpc? Explore
this with outflow.cc]

Given the essentially perfect correspondence between the Sobolev and
partical covering models (for \mgiia), we are motivated to consider
further the characteristics of the Sobolev model which has the
distinct advantage of producing a 
density profile.  It is straightforward, then, to convert the density and
velocity profiles into distributions of mass, energy, and momentum
within
the wind.  These are shown in cumulative form in
Figure~\ref{fig:LBG_cumul}.  Before discussing the results, we offer
two cautionary comments: (i) the conversion of \nmg\ to $n_{\rm H}$
assumes a very poorly constrained scalar factor of $10^{5.47}$.  One
should give minimal weight to the absolute values for any of the
quantities;
(ii) the Sobolev approximation is not a proper description of the S10
LBG wind model.

The primary result expressed by Figure~\ref{fig:LBG_cumul} is that the
majority of energy, mass, and momentum in the wind is transported by
the outer layers ($r > 20$\,kpc).  This is surprising given that the
density is $\approx 5$~orders of magnitude lower at these radii than
at $r = 1$\,pc.  [Is this unphysical?  Can we conflate it to then kill
the S10 model?]

\section{Observables}
\label{sec:obs}

Although the focus of this paper is to explore the absorption/emission
profiles for idealized wind models, we ultimately intend to compare 
profiles like these against observations to constrain characteristics
of the wind (e.g.\ raidal extent, energetics).  In this section we
consider a few observables and their dependence on the wind properties.

\subsection{Absorption profiles}

 \begin{enumerate}
   \item Peak optical depth
     \begin{itemize}
       \item Not a robust measure of $\tau_{wind}$
       \item Not even a good relative predictor for $\tau$
       \item Not a robust predictor of $f_c$
     \end{itemize}
   \item Kinematics
     \begin{itemize}
       \item Nearly robust for $v<v_{min}$
       \item But what sets $v_{min}$?
       \item Generally get the full extent of the wind speed
     \end{itemize}
   \item EW
     \begin{itemize}
       \item Primarily set by $v_{min}(\tau \sim 1)$
       \item $f\lambda$ matters, but so do emission channels
     \end{itemize}
   \item Fine-structure
     \begin{itemize}
       \item Distance to the source
     \end{itemize}
 \end{enumerate}
     
\subsection{Emission profiles}

 \begin{enumerate}
   \item Resonant lines
     \begin{itemize}
       \item Kinematics of backside
       \item Flux + relative
    \end{itemize}
   \item Fine-structure
     \begin{itemize}
       \item Flux $\propto \tau_{wind}$ + relative
       \item Kinematics (anisotropy)
    \end{itemize}
 \end{enumerate}
     

\section{Discussion}
\label{sec:discuss}

[Are MgII and FeII* emission strict predictions of outflows?  People
have been so focused on absorption.]

[Does recombination predict 2:1 for MgII??]
[We also emphasize that departures from a 2:1
ratio indicate that processes other than simple recombination are
active. ]

[Profiles are modified by {\it many} factors => Difficult to derive
robust constraints from absorption alone.]

[Is one safe if there is no emission detected?]

[What does MgI tell us about winds?  Why no P-Cygni?]

\section{Summary}
\label{sec:summary}

\acknowledgments

J.X.P and K.R. are partially supported
by an NSF CAREER grant (AST--0548180), and 
by NSF grant AST-0908910.

\clearpage

%\bibliographystyle{/u/xavier/NSF/SASIR/SASIR-ATI/prop2009/Text/nsfati}
%\bibliography{/u/xavier/NSF/SASIR/SASIR-ATI/prop2009/Text/nsfati09}
\bibliographystyle{/u/xavier/paper/Bibli/apj}
\bibliography{/u/xavier/paper/Bibli/allrefs}

\clearpage

\begin{deluxetable}{lcccccc}
\tabletypesize{\footnotesize}
\tablecolumns{11}
\tablecaption{Observed Transitions and Limits \label{tab:atomic}}
\tablewidth{0pt}
\tablehead{\colhead{} & \colhead{$\rm E_{high}$} & \colhead{$\rm E_{low}$} & \colhead{$J_{\rm high}$} & \colhead{$J_{\rm low}$} & \colhead{$\lambda$} & \colhead{$A$} \\
 & \colhead{($\rm cm^{-1}$)} & \colhead{($\rm cm^{-1}$)} &&& \colhead{(\AA)} & \colhead{($\rm s^{-1}$)} } 
\startdata
\ion{Fe}{2} UV1 & 38458.98 &     0.00 &   9/2 & 9/2 & 2600.173 & 2.36E08  \\
           & 38458.98 &   384.79 &   9/2 & 7/2 & 2626.451 & 3.41E+07 \\
           & 38660.04 &     0.00 &   7/2 & 9/2 & 2586.650 & 8.61E+07 \\
           & 38660.04 &   384.79 &   7/2 & 7/2 & 2612.654 & 1.23E+08 \\
           & 38660.04 &   667.68 &   7/2 & 5/2 & 2632.108 & 6.21E+07 \\
           & 38858.96 &   667.68 &   5/2 & 5/2 & 2618.399 & 4.91E+07 \\
           & 38858.96 &   862.62 &   5/2 & 3/2 & 2631.832 & 8.39E+07 \\
           & 39013.21 &   667.68 &   3/2 & 5/2 & 2607.866 & 1.74E+08 \\
           & 39013.21 &   862.61 &   3/2 & 3/2 & 2621.191 & 3.81E+06 \\
           & 39013.21 &   977.05 &   3/2 & 1/2 & 2629.078 & 8.35E+07 \\
           & 39109.31 &   862.61 &   1/2 & 3/2 & 2614.605 & 2.11E+08 \\
           & 39109.31 &   977.05 &   1/2 & 1/2 & 2622.452 & 5.43E+07 \\
\tableline \\ [-1.5ex]
\ion{Mg}{2}& 35760.89 &     0.00 &   3/2 &   0 & 2796.351 & 2.63E+08\\
           & 35669.34 &     0.00 &   1/2 &   0 & 2803.528 & 2.60E+08\\
\enddata
\tablecomments{Atomic data was obtained from \citet{Morton2003} unless otherwise indicated.}
\end{deluxetable}

 
 
\begin{deluxetable}{ccl}
\tablewidth{0pc}
\tablecaption{Wind Parameters: Fiducial Model\label{tab:fiducial}}
\tabletypesize{\footnotesize}
\tablehead{\colhead{Property} & \colhead{Parameter} & \colhead{Value} } 
\startdata
Density law  & $n(r)$ & $\propto r^{-2}$ \\
Velocity law  & $v_r$ & $ \propto r$ \\
Inner Radius & $r_{\rm inner}$ & 1\,kpc \\
Outer Radius & $r_{\rm outer}$ & 20\,kpc \\
Source size  & $r_{\rm source}$ & 0.5\,kpc \\
Density Normalization & $n^0_{\rm H}$ & $0.1\cm{-3}$ at $r_{\rm inner}$ \\
Velocity Normalization & $v^0$ & 50\kms at $r_{\rm inner}$ \\
Turbulence   & $b_{\rm turb}$  & 15 \kms \\
Mg$^+$ Normalization & \nmg\ & $10^{-5.47} n_{\rm H}$ \\
Fe$^+$ Normalization & \nfe\ & \nmg/2 \\
\enddata
%\tablecomments{Unless specified otherwise, all quantities refer to the \sna=2 threshold.  The cosmology assumed has $\Omega_\Lambda = 0.7, \Omega_m = 0.3$, and $H_0 = 72 \mkms \rm Mpc^{-1}$.}
%\tablenotetext{a}{Total redshift survey path for the \sna=2 criterion.}

\end{deluxetable}

\input{Tables/tab_fiducial_ew.tex}
 
 
\begin{deluxetable}{ccrccccccccccc}
\rotate
\tablewidth{0pc}
\tablecaption{Line Diagnostics for the Fiducial Model and Variants
\label{tab:line_diag}}
\tabletypesize{\footnotesize}
\tablehead{\colhead{Transition} & \colhead{Model} & \colhead{$v_{\rm int}^a$} & \colhead{$W_{\rm i}$} & \colhead{$W_{\rm a}$} & \colhead{$\tau_{\rm pk}$} & \colhead{$v_\tau$} 
& \colhead{$v_{\bar \tau}$}
& \colhead{$v_{\rm int}^b$} & \colhead{$W_{\rm e}$} & \colhead{$f_{\rm pk}$} & \colhead{$v_f$} 
& \colhead{$v_{\bar f}$}
\\
&& (\kms) & (\AA) & (\AA) && (\kms) & (\kms) & (\kms) & (\AA) & & (\kms) & (\kms)}
\startdata
  MgII 2796  \\
&Fiducial&[$-1009,-43$]& 4.78& 2.83&0.94&$ -215$&$ -372$&[$-32,311$]&$-1.77$& 2.48&$   32$&$  117$\\
&$\phi=0^\circ$&$\dots$&$\dots$&$\dots$&$\dots$&$\dots$&$\dots$&$\dots$&$\dots$&$\dots$&$\dots$&$\dots$&$\dots$\\
&$\phi=180^\circ$&[$-1030,-65$]& 4.78& 2.98&1.03&$ -199$&$ -369$&[$-65,70$]&$-0.40$& 1.78&$  -11$&$    6$\\
&\taud=1&[$-1009,-32$]& 4.77& 2.94&0.95&$ -215$&$ -360$&[$-32,257$]&$-0.92$& 1.90&$   32$&$  101$\\
&\taud=3&[$-998,-32$]& 4.78& 3.07&1.03&$ -193$&$ -342$&[$-22,182$]&$-0.39$& 1.48&$   43$&$   76$\\
&ISM&[$-1009,-65$]& 6.36& 2.67&0.90&$ -204$&$ -390$&[$-54,311$]&$-1.60$& 2.36&$  118$&$  125$\\
&ISM+dust&[$-998,150$]& 6.42& 4.06&1.06&$ -193$&$ -270$&[$590,601$]&$ 0.12$& 0.40&$  590$&$  595$\\
  MgII 2803  \\
&Fiducial&[$-437,-41$]& 3.29& 1.19&0.76&$ -148$&$ -193$&[$-41,676$]&$-2.22$& 2.55&$   34$&$  269$\\
&$\phi=0^\circ$&$\dots$&$\dots$&$\dots$&$\dots$&$\dots$&$\dots$&$\dots$&$\dots$&$\dots$&$\dots$&$\dots$&$\dots$\\
&$\phi=180^\circ$&[$-699,-57$]& 3.29& 1.98&0.97&$ -137$&$ -257$&[$-57,104$]&$-0.50$& 1.81&$   -3$&$   22$\\
&\taud=1&[$-479,-41$]& 3.28& 1.41&0.83&$ -137$&$ -195$&[$-41,591$]&$-1.27$& 1.95&$   34$&$  247$\\
&\taud=3&[$-533,-30$]& 3.26& 1.67&0.94&$ -116$&$ -195$&[$-30,484$]&$-0.64$& 1.50&$   34$&$  213$\\
&ISM&[$-426,-62$]& 6.49& 1.03&0.67&$ -169$&$ -206$&[$-51,655$]&$-2.09$& 2.57&$   98$&$  265$\\
&ISM+dust&[$-1774,130$]& 6.51& 6.73&1.06&$ -961$&$ -686$&[$173,195$]&$-0.01$& 1.03&$  184$&$  184$\\
  FeII 2586  \\
&Fiducial&[$-348,-35$]& 0.82& 0.61&1.01&$  -70$&$ -119$&[$-35,128$]&$-0.08$& 1.11&$   35$&$   47$\\
&$\phi=0^\circ$&$\dots$&$\dots$&$\dots$&$\dots$&$\dots$&$\dots$&$\dots$&$\dots$&$\dots$&$\dots$&$\dots$&$\dots$\\
&$\phi=180^\circ$&[$-365,-46$]& 0.82& 0.60&1.01&$  -75$&$ -133$&[$-46,70$]&$-0.01$& 1.06&$  -17$&$   12$\\
&\taud=1&[$-348,-35$]& 0.82& 0.61&1.04&$  -70$&$ -118$&[$-35,116$]&$-0.04$& 1.06&$   23$&$   41$\\
&\taud=3&[$-313,-35$]& 0.94& 0.60&1.06&$  -70$&$ -113$&[$-23,46$]&$-0.02$& 1.04&$  -12$&$   12$\\
&ISM&[$-348,104$]& 1.90& 1.60&2.75&$    0$&$  -29$&[$1670,1948$]&$-0.19$& 1.15&$ 1682$&$ 1806$\\
&ISM+dust&[$-313,116$]& 1.84& 1.63&2.96&$   23$&$  -25$&[$278,290$]&$-0.00$& 1.03&$  278$&$  284$\\
  FeII 2600  \\
&Fiducial&[$-580,-37$]& 1.87& 1.18&1.08&$  -83$&$ -181$&[$-37,459$]&$-0.82$& 1.70&$   32$&$  191$\\
&$\phi=0^\circ$&$\dots$&$\dots$&$\dots$&$\dots$&$\dots$&$\dots$&$\dots$&$\dots$&$\dots$&$\dots$&$\dots$&$\dots$\\
&$\phi=180^\circ$&[$-597,-49$]& 1.87& 1.18&1.31&$  -78$&$ -188$&[$-49,95$]&$-0.23$& 1.39&$  -20$&$   22$\\
&\taud=1&[$-591,-37$]& 1.87& 1.20&1.14&$  -83$&$ -176$&[$-37,332$]&$-0.48$& 1.45&$   32$&$  138$\\
&\taud=3&[$-580,-37$]& 1.95& 1.25&1.31&$  -72$&$ -169$&[$-37,228$]&$-0.20$& 1.22&$   44$&$   93$\\
&ISM&[$-568,90$]& 3.12& 1.88&1.19&$  -72$&$  -99$&[$101,378$]&$-0.19$& 1.15&$  113$&$  237$\\
&ISM+dust&[$-603,101$]& 3.06& 2.12&1.40&$   44$&$  -93$&[$182,194$]&$-0.00$& 1.02&$  182$&$  188$\\
  FeII* 2612 \\
&Fiducial&&&&&&&$\dots$&$\dots$&$\dots$&$\dots$&$\dots$&$\dots$\\
&$\phi=0^\circ$&&&&&&&$\dots$&$\dots$&$\dots$&$\dots$&$\dots$&$\dots$\\
&$\phi=180^\circ$&&&&&&&$\dots$&$\dots$&$\dots$&$\dots$&$\dots$&$\dots$\\
&\taud=1&&&&&&&$\dots$&$\dots$&$\dots$&$\dots$&$\dots$&$\dots$\\
&\taud=3&&&&&&&$\dots$&$\dots$&$\dots$&$\dots$&$\dots$&$\dots$\\
&ISM&&&&&&&$\dots$&$\dots$&$\dots$&$\dots$&$\dots$&$\dots$\\
&ISM+dust&&&&&&&$\dots$&$\dots$&$\dots$&$\dots$&$\dots$&$\dots$\\
  FeII* 2626 \\
&Fiducial&&&&&&&$\dots$&$\dots$&$\dots$&$\dots$&$\dots$&$\dots$\\
&$\phi=0^\circ$&&&&&&&$\dots$&$\dots$&$\dots$&$\dots$&$\dots$&$\dots$\\
&$\phi=180^\circ$&&&&&&&$\dots$&$\dots$&$\dots$&$\dots$&$\dots$&$\dots$\\
&\taud=1&&&&&&&$\dots$&$\dots$&$\dots$&$\dots$&$\dots$&$\dots$\\
&\taud=3&&&&&&&$\dots$&$\dots$&$\dots$&$\dots$&$\dots$&$\dots$\\
&ISM&&&&&&&$\dots$&$\dots$&$\dots$&$\dots$&$\dots$&$\dots$\\
&ISM+dust&&&&&&&$\dots$&$\dots$&$\dots$&$\dots$&$\dots$&$\dots$\\
  FeII* 2632 \\
&Fiducial&&&&&&&$\dots$&$\dots$&$\dots$&$\dots$&$\dots$&$\dots$\\
&$\phi=0^\circ$&&&&&&&$\dots$&$\dots$&$\dots$&$\dots$&$\dots$&$\dots$\\
&$\phi=180^\circ$&&&&&&&$\dots$&$\dots$&$\dots$&$\dots$&$\dots$&$\dots$\\
&\taud=1&&&&&&&$\dots$&$\dots$&$\dots$&$\dots$&$\dots$&$\dots$\\
&\taud=3&&&&&&&$\dots$&$\dots$&$\dots$&$\dots$&$\dots$&$\dots$\\
&ISM&&&&&&&$\dots$&$\dots$&$\dots$&$\dots$&$\dots$&$\dots$\\
&ISM+dust&&&&&&&$\dots$&$\dots$&$\dots$&$\dots$&$\dots$&$\dots$\\
\enddata
\tablecomments{{L}isted are the equivalent widths (intrinsic, absorption, and emission), the peak optical depth for the absorption
$\tau_{\rm pk} \equiv -\ln(I_{\rm min})$, the velocity where the optical depth peaks $v_\tau$, the optical depth-weighted velocity centroid 
$v_{\bar \tau} \equiv \int dv \, v \ln[I(v)] / \int dv \ln[I(v)]$, the peak flux $f_{\rm pk}$ in emission, the velocity where the flux peaks 
$v_f$, and the flux-weighted velocity centroid of the emission line $v_{\bar f}$.}
\end{deluxetable}

\input{Tables/tab_meas_plaws.tex}

\begin{figure}
\epsscale{0.95}
\plotone{Figures/energy_levels.ps}
\caption{
Energy level diagrams for the \mgiid\ doublet and the UV1
multiplet of \ion{Fe}{2} transitions   
(based on Figure~7 from \cite{hmt+99}).
Each transition shown is
labelled by its rest wavelength (\AA) and Einstein A-coefficient
(s$^{-1}$). Black upward arrows
indicate the resonance-line transitions, i.e.\ those connected to the ground
state.  The 2p$^6$3p configuration of Mg$^+$ is split into
two energy levels that give rise to the \mgiid\ doublet.  
Both the 3d$^6$4s ground state and 3d$^6$4p upper level of Fe$^+$
exhibit fine-structure splitting that gives rise to a series of
electric-dipole transitions. 
The downward (green) arrows show the \feiis\ transitions that are connected to the
resonance-line transitions (i.e.\ they share the same upper energy
levels).  We also show a pair of levels (\ion{Fe}{2}$^* \lambda\lambda
2618,2631$) that arise from higher levels in the \zconfig\
configuration.  These transitions have not yet been observed in
galactic-scale outflows and are not considered in our analysis.
}
\label{fig:energy}
\end{figure}

\begin{figure}
\epsscale{0.8}
\plotone{Figures/fig_nvtau_vs_r.ps}
\caption{
Density (dashed; red), radial velocity (dotted; blue), and
\mgiia\ optical depth profiles (solid; black) for the fiducial
wind model (see Table~\ref{tab:fiducial} for details).
The density and velocity laws are simple $r^{-2}$ and $r^1$
power-laws.  The optical depth profile was calculated by summing
the opacity at small and discrete radial intervals in velocity space
and was then converted to radius with the velocity law.  One notes that
the wind is optically thick at the inner radius ($r_{\rm inner} =
1$\,kpc) and becomes optically thin at the outer radius ($r_{\rm
  outer} = 20$\,kpc).
Note that the density and velocity curves have been scaled for plotting
convenience.  
}
\label{fig:fiducial_nvt}
\end{figure}

\begin{figure}
\includegraphics[scale=0.6,angle=90]{Figures/fig_fiducial_1d.ps}
\caption{
{\it Left} -- (Upper) \mgiid\ profiles for the fiducial wind model
described in Table~\ref{tab:fiducial} and
Figure~\ref{fig:fiducial_nvt}.  The doublet shows the P-Cygni profiles
characteritic of an outflow with significant absorption blueward of
line-center (dashed vertical lines) extending to $v = -1000\mkms$
and significant emission redward of line-center.  Note
that even though the peak optical depth of the \ion{Mg}{2} transitions
is $\tau_{2796}^{\rm max} \approx 30$ at $v \approx -70 \mkms$,
photons scattered off the outflow fill in the absorption.
(Lower) \ion{Fe}{2} absorption and emission profiles for the UV1
multiplet at $\lambda \approx 2600$\AA.  The \feiid\ resonance lines 
show weaker absorption due to the smaller Fe$^+$ number density and
lower $f\lambda$ values.  Each also shows a P-Cygni profile, although
the emission for \feiia\ is significantly weaker than that of the
\feiib\ and \mgiid\ transitions.  This is because a majority of the
absorbed \feiia\ photons are converted into
\ion{Fe}{2}$^*~\lambda\lambda 2612, 2632$ photons.
{\it Right} -- A subset of the transitions displayed in a velocity
plot.
}
\label{fig:fiducial_1d}
\end{figure}

\begin{figure}
\epsscale{0.8}
\plotone{Figures/fig_noemiss.ps}
\caption{
The solid curves show the line profiles (absorption and emission) of
the \ion{Mg}{2} and \ion{Fe}{2} resonance lines for the fiducial wind
model.  These include the effects of scattered photons and show the
canonical P-Cygni profiles of a source 
embedded within an outflow.  Overplotted (dotted line) on each transition is
the predicted absorption profile under the constraint that every
absorbed photon is lost from the system, i.e.\ no scattering or
re-emission occurs.   The true profiles have been `filled
in' significantly by light scattered in the wind.  Ignoring this
process, one would model the absorption lines with a systematically lower
optical depth and (incorrectly) conclude that the source is partly covered by the
gas.  
}
\label{fig:noemiss}
\end{figure}

\begin{figure}
\epsscale{0.8}
\plotone{Figures/fig_fiducial_ifu_mgii.ps}
\caption{
Surface-brightness emission maps around the \mgiia\ transition for the
source+wind complex of the fiducial model.  The middel panel shows the 1D spectrum with $v=0
\mkms$ corresponding to $\lambda = 2796.35$\AA\ and the dotted vertical
curves indicate the velocity slices for the emission maps.  The
source has a size $r_{\rm source} = 0.5$\,kpc, traced by a few
pixels at the center.   At $v=-250 \mkms$, the wind has an optical
depth of $\tau_{2796} \approx 1$ and the source contributes
roughly half of the observed flux.  At $v=-100 \mkms$ the
wind absorbs all photons from the source and the observed emission is
entirely due to photons scattered by the wind.  Impressively, this
emission exceeds the integrated flux at $v = -250\mkms$ such that the
line center is offset from the peak optical depth.  At $v \ge 0
\mkms$,  both the source and wind contribute to the observed emission.
At all velocities, the majority of emission comes from the inner
$\approx 5$\,kpc.
}
\label{fig:fiducial_ifu_mgii}
\end{figure}

\begin{figure}
\epsscale{0.8}
\plotone{Figures/fig_feii_ifu.ps}
\caption{
(Upper) Surface-brightness emission maps around the \feiib\ transition for the
source+wind complex.  The middle panel shows the 1D spectrum with $v=0
\mkms$ corresponding to $\lambda = 2600.173$\AA\ and the dotted vertical
curves indicating the velocity slices for the emission maps.  
The results are very similar to those observed for the \mgiid\ doublet
(Figure~\ref{fig:fiducial_ifu_mgii}).
(Lower) Surface-brightness emission maps around the \ion{Fe}{2}~$\lambda
2612$ transition for the 
source+wind complex.  The middle panel shows the 1D spectrum with $v=0
\mkms$ corresponding to $\lambda = 2612.654$\AA\ and the dotted vertical
curves indicating the velocity slices for the emission maps.  
In this case, the source is unattenuated yet scattered photons from
the wind also have a significant contribution. 
}
\label{fig:fiducial_ifu_feii}
\end{figure}

\begin{figure}
\epsscale{0.8}
\plotone{Figures/fig_fiducial_cut.ps}
\caption{
The panels show the relative emission from the wind as a function of
radius for (a) \mgiia\ and (b) \feiib\ in a series of velocity
channels.  With the exception of $v = -100 \mkms$ for \mgiia, the wind
has finite transmission such that one observes the source at $r =
0$\,kpc.  In all cases,  the majority of emission occurs at $r
\lesssim 2.5$\,kpc, even if one ignores the source emission.
}
\label{fig:fiducial_cuts}
\end{figure}

\begin{figure}
\epsscale{0.8}
\plotone{Figures/fig_asymm_spec.ps}
\caption{
Profiles of the \ion{Fe}{2} and \ion{Mg}{2} profiles for the fiducial
case (black lines) compared against an anisotropic wind blowing into
only $2\pi$ steradians as viewed from $\theta = 0^\circ$ (source
uncovered) to $\theta = 180^\circ$ (source covered).  One detects
significant emission for all orientations of this wind but significant
absorption only for $\theta \ge 90^\circ$.
The velocity centroid of the 
emission shifts from positive to negative velocities as $\theta$
increases and one transitions from viewing the wind to be behind the
source to in front.  The velocity centroid of emission, therefore, may
give a robust measure of wind isotropy.
}
\label{fig:anisotropic}
\end{figure}

\begin{figure}
\epsscale{0.8}
\plotone{Figures/fig_dust_spec.ps}
\caption{
Profiles of the \ion{Fe}{2} and \ion{Mg}{2} profiles for the fiducial
model (black) against a series of models that include 
dust extinction.  The primary effect is suppression of
the line-emission relative to the continuum. 
A more subtle but important effect is that the redder photons in the
emission lines (corresponding to higher velocity relative to
line-center) suffer greater extinction.  This is most evident in the
\feiic\ emission line; it occurs because these photons must
travel farther to scatter off the backside of the wind.  Note that
the absorption lines are nearly unmodified until $\mtaud = 10$, a
level of extinction that would preclude observing the source
altogether.
}
\label{fig:dust}
\end{figure}

\begin{figure}
\epsscale{0.8}
\plotone{Figures/fig_ism_diagn.ps}
\caption{
Density (dashed; red), radial velocity (dotted; blue), and
\mgiia\ optical depth profiles (solid; black) for the ISM+wind
model.  The wind is identical to that from the fiducial model
(Figure~\ref{fig:fiducial_nvt}) but this ISM+wind model also includes an ISM component
with $n_{\rm H} = 1 \cm{-3}$ that has an average velocity of $\mvr = 0
\mkms$ and a large turbulent velocity $b_{\rm ISM} = 40 \mkms$.
Note that the density and velocity curves have been scaled for plotting
convenience.  
}
\label{fig:ISM}
\end{figure}

\begin{figure}
\epsscale{0.7}
\plotone{Figures/fig_ism_spec.ps}
\caption{
Profiles of the \ion{Fe}{2} and \ion{Mg}{2} profiles for the ISM+wind
model (red) compared against the fiducial wind model (black). 
The dotted line traces the predicted absorption profile in the absence
of re-emission and scattering.
Regarding the \ion{Mg}{2}~doublet, the primary difference between the
ISM+wind and the fiducial models is the shift of $\approx 100 \mkms$
in the emission lines from $v \approx 0 \mkms$ for the fiducial
model to $v \approx +100\mkms$ for the ISM+wind model. 
For the ISM+wind model, 
the \ion{Fe}{2} profiles show qualitative differences. 
the ISM model.  The \feiid\ resonance transitions each exhibit much
greater absorption at $v \approx 0 \mkms$ than the fiducial model.
The resonant line-emission is also substantially
reduced, implying much higher fluxes for the non-resonant lines (e.g.\
\feiic).  The lack of scattered \feiia\ photons results in an
absorption profile that very nearly matches the input opacity profile.
We conclude that resonance transitions that are strongly coupled to
non-resonant lines offer the best
characterization of an ISM component.
}
\label{fig:ISM_spec}
\end{figure}

\begin{figure}
\epsscale{0.8}
\plotone{Figures/fig_ism_ifu.ps}
\caption{
Surface-brightness emission maps for ISM+wind model
}
\label{fig:ISM_ifu}
\end{figure}

\begin{figure}
\epsscale{0.8}
\plotone{Figures/fig_vary_profiles.ps}
\caption{
Power-law profiles
}
\label{fig:plaws}
\end{figure}

\begin{figure}
\includegraphics[scale=0.6,angle=90]{Figures/fig_plaw_spec.ps}
\caption{
Power-law profiles spectral lines
}
\label{fig:plaws_spec}
\end{figure}

\begin{figure}
\epsscale{0.8}
\plotone{Figures/fig_radiation_nvt.ps}
\caption{
Radiation diagnosed
}
\label{fig:rad_nvt}
\end{figure}

\begin{figure}
\epsscale{0.8}
\plotone{Figures/fig_rad_spec.ps}
\caption{
Radiation spec
}
\label{fig:rad_spec}
\end{figure}


\begin{figure}
\epsscale{0.8}
\plotone{Figures/fig_lbg_sobolev.ps}
\caption{
(a) The dotted curve shows the velocity law (values are labelled on the
right side) for the LBG model
of S10.  Note how rapidly the velocity rises from $r =
1$ to 2\,kpc.  The solid curve shows the Sobolev solution for the
Mg$^+$ gas
density derived from 
the average absorption profile of LBG galaxies (see below).
The density drops off as $(r/{\rm kpc}-1)^{-0.6}$ initially and then
steepens to $(r/{\rm kpc}-1)^{-1.8}$.
(b) Average absorption profile for LBG cool gas absorption as
measured and defined by S10
(black solid curve).  Overplotted on this curve is a (red) dotted line that
shows the \mgiia\ absorption profile dervied from the density (and velocity)
law shown in the upper panel.  The excellent agreement confirms the
validity of the Sobolev approximation.
}
\label{fig:LBG_Sobolev}
\end{figure}

\clearpage

\begin{figure}
\epsscale{0.8}
\plotone{Figures/fig_lbg_spec.ps}
\caption{
Profiles of the \ion{Fe}{2} and \ion{Mg}{2} profiles for the two
LBG models considered: (red) the profiles from the 
Sobolev approximation 
and (black) a model constructed to faithfully
correspond to the wind model advocated by S10.  The dotted line
shows the absorption profiles of the latter model if one were
to neglect scattered and re-emitted photons.  Similar to the fiducial
model (Figure~\ref{fig:fiducial_1d}), scattered and re-emitted photons
significantly modify the absorption profiles (especially \ion{Mg}{2})
and produce significant emission lines.  
The result, for \ion{Mg}{2} especially, is a set of profiles that do
not match the average observed LBG absorption profile.
Note the excellent agreement
between the \mgiia\ profile for the two models. [Cast doubt on S10
here?]
}
\label{fig:LBG_spec}
\end{figure}

\begin{figure}
\includegraphics[scale=0.6,angle=90]{Figures/fig_lbg_cumul.ps}
\caption{
Profiles of the \ion{Fe}{2} and \ion{Mg}{2} profiles for the fiducial
}
\label{fig:LBG_cumul}
\end{figure}



\end{document}
