\documentclass[12pt,preprint]{aastex}
\usepackage{natbib,amsmath}
\special{papersize=8.5in,11in}
\begin{document}


\def\hub{h_{72}^{-1}}
\def\umfp{{\hub \, \rm Mpc}}
\def\mzq{z_q}
\def\zabs{$z_{\rm abs}$}
\def\mzabs{z_{\rm abs}}
\def\msna{{\rm S/N}^{\rm A}_{912}}
\def\sna{S/N$^{\rm A}_{912}$}
\def\mnull{\nu_{\rm 912}}
\def\nnull{$\nu_{\rm 912}$}
\def\intl{\int\limits}
\def\nstatqso{193}
\def\maxoff{0.4}
\def\clls{1.9 \pm 0.2}
\def\alls{5.2 \pm 1.5}
\def\blls{-0.9^{+0.4}_{-0.05}}
\def\cmma{\;\;\; ,}
\def\perd{\;\;\; .}
\def\ltk{\left [ \,}
\def\ltp{\left ( \,}
\def\ltb{\left \{ \,}
\def\rtk{\, \right  ] }
\def\rtp{\, \right  ) }
\def\rtb{\, \right \} }
\def\sci#1{{\; \times \; 10^{#1}}}
\def \rAA {\rm \AA}
\def \zem {$z_{\rm em}$}
\def \mzem {z_{\rm em}}
\def \mzlls {z_{\rm LLS}}
\def \zlls {$z_{\rm LLS}$}
\def \mzend {z_{\rm end}}
\def \zend {$z_{\rm end}$}
\def \mzstrtt {z_{\rm start}^{\rm S/N^A=2}}
\def \zstrtt {$z_{\rm start}^{\rm S/N^A=2}$}
\def \zstrto {$z_{\rm start}^{\rm S/N^A=1}$}
\def \mzstrto {z_{\rm start}^{\rm S/N^A=1}}
\def \zstrt {$z_{\rm start}$}
\def \mzstrt {z_{\rm start}}
\def\smm{\sum\limits}
\def \lll  {$\lambda_{\rm 912}$}
\def \mlll  {\lambda_{\rm 912}}
\def \mtll  {\tau_{\rm 912}}
\def \tll  {$\tau_{\rm 912}$}
\def \mavtll  {<\mtll>}
\def \tigm  {$\tau_{\rm IGM}$}
\def \mtigm  {\tau_{\rm IGM}}
\def \cmm  {cm$^{-2}$}
\def \cmmm {cm$^{-3}$}
\def \kms  {km~s$^{-1}$}
\def \mkms  {{\rm km~s^{-1}}}
\def \lyaf {Ly$\alpha$ forest}
\def \Lya  {Ly$\alpha$}
\def \lya  {Ly$\alpha$}
\def \mlya  {Ly\alpha}
\def \Lyb  {Ly$\beta$}
\def \lyb  {Ly$\beta$}
\def \lyg  {Ly$\gamma$}
\def \ly5  {Ly-5}
\def \ly6  {Ly-6}
\def \ly7  {Ly-7}
\def \nhi  {$N_{\rm HI}$}
\def \mnhi  {N_{\rm HI}}
\def \lnhi {$\log N_{HI}$}
\def \mlnhi {\log N_{HI}}
\def \etal {\textit{et al.}}
\def \ob {$\Omega_b$}
\def \obh {$\Omega_bh^{-2}$}
\def \om {$\Omega_m$}
\def \ol {$\Omega_{\Lambda}$}
\def \gz {$g(z)$}
\def \mgz {g(z)}
\def \lyaf {Lyman--$\alpha$ forest}
\def \fnhi {$f(\mnhi,X)$}
\def \mfnhi {f(\mnhi,X)}
\def \myfnhi {f_{\rm Ly\alpha}(\mnhi,X)}
\def \yfnhi {$f_{\rm Ly\alpha}(\mnhi,X)$}
\def \lfnhi {$f_{\rm LLS}(\mnhi,X)$}
\def \dfnhi {$f_{\rm DLA}(\mnhi,X)$}
\def \sfnhi {$f_{\rm SLLS}(\mnhi,X)$}
\def \mlfnhi {f_{\rm LLS}(\mnhi,X)}
\def \mdfnhi {f_{\rm DLA}(\mnhi,X)}
\def \msfnhi {f_{\rm SLLS}(\mnhi,X)}
\def \ztot {$\Delta z_{\rm TOT}$}
\def \mztot {\Delta z_{\rm TOT}}
\def \mlplls {\ell_{\rm{PLLS}}(z)}
\def \lplls {$\ell_{\rm{PLLS}}(z)$}
\def \mlzlls {\ell_{\rm{LLS}}(z)}
\def \lzlls {$\ell_{\rm{LLS}}(z)$}
\def \mllls {\ell_{\rm{LLS}}(X)}
\def \llls {$\ell_{\rm{LLS}}(X)$}
\def \ldla {$\ell_{\rm{DLA}}(X)$}
\def \lslls{$\ell_{\rm{SLLS}}(X)$}
\def \mlslls{\ell_{\rm{SLLS}}(X)}
\def \nlls {$n_{\rm LLS}$}
\def \slls {$\sigma_{\rm LLS}$}
\def \mnlls {n_{\rm LLS}}
\def \mslls {\sigma_{\rm LLS}}
\def \drlls {$\Delta r_{\rm LLS}$}
\def \mdrlls {\Delta r_{\rm LLS}}
\def \mlmfp {\lambda_{\rm mfp}^{912}}
\def \lmfp {$\lambda_{\rm mfp}^{912}$}
\def \mbplls {\beta_{\rm pLLS}}
\def \bplls {$\beta_{\rm pLLS}$}
\def \btlls {$\beta_{\rm LLS}$}
\def \mbtlls {\beta_{\rm LLS}}
\def \lteff {$\tau_{\rm eff,LL}$}
\def \teff {$\tau_{\rm eff,LL}$}
\def \tlya {$\tau_{\rm eff}^{\rm Ly\alpha}$}
\def \mtlya {\tau_{\rm eff}^{\rm Ly\alpha}}
\def \mteff {\tau_{\rm eff}}
\def \mllteff {\mteff^{912}}
\def \llteff {$\mteff^{912}$}
%\def \llsteff {$\tilde\mtll$}
%\def \mllsteff {\tilde\mtll}
\def \mnmin {\mnhi^{\rm min}}
\def \nmin {$\mnhi^{\rm min}$}
\def \O {${\mathcal O}(N,X)$}
\newcommand{\cm}[1]{\, {\rm cm^{#1}}}
\def \snrlim {SNR$_{lim}$}
\def\mglls {\gamma_{\rm LLS}}
\def\mavgt {<\mtll>}

\title{The SDSS-DR7 Survey for Lyman Limit Systems}

\author{
J. Xavier Prochaska\altaffilmark{1}, 
John M. O'Meara\altaffilmark{2}, 
Gabor Worseck\altaffilmark{1} 
%\& Scott Burles\altaffilmark{3}
}
\altaffiltext{1}{Department of Astronomy and Astrophysics, UCO/Lick Observatory, University of California, 1156 High Street, Santa Cruz, CA 95064}
\altaffiltext{2}{Department of Chemistry and Physics, Saint Michael's College.
One Winooski Park, Colchester, VT 05439}

\begin{abstract}
We present the results from a survey for Lyman limit system absorption in the Sloan Digital Sky Survey, Data release 5.  I like abstracts, they are fun!
\end{abstract}


\keywords{absorption lines -- intergalactic medium -- Lyman limit systems -- SDSS}

\section{Introduction}
% JO: Introduce LLS
% JO: Mention previous surveys.
% JO:  Leads into survey size \& bias.
Studies of hydrogen  absorption in the lines of sight towards 
distant quasars have served to both define, and in recent years bring
precision to, our cosmological models for the universe.  
The low density, highly ionized  \lyaf\ lines (a.k.a.\ the intergalactic
medium, IGM), with 
\ion{H}{1} column densities $\mnhi < 10^{17.2} \cm{-2}$,
through their aggregate statistical properties (e.g.\ their
flux power spectrum, mean flux, and column density distributions)
constrain cosmological parameters such as the primordial power spectrum
and the baryonic mass density
\citep[e.g.][]{rau98,cwb+03,mcdonald05,tytler04,fpl+08}.
The high-density, predominantly neutral damped \lya\ systems (DLAs),
with $\mnhi \ge 10^{20.3} \cm{-2}$, trace the gas which forms stars, and
likely represent the progenitors of modern-day galaxies 
\citep[e.g.][]{wlf+95,wgp05,pw09}.

The majority of \lyaf\ lines and the DLAs have, through
analysis of their \lya\ lines, precisely measured \nhi\ values
that permit detailed study of their physical propertiees (e.g.\ metallicity).
For systems with intermediate \nhi\ values ($\approx 10^{18} \cm{-2}$),
however, \lya\ and most of the Lyman series lines lie on the
flat portion of the curve-of-growth and the \nhi\ value is difficult
to cosntrain.
On the other hand, these systems are optically thick to ionizing
radiation and impose a readily identify spectral signature in a quasar
spectrum at the Lyman limit.
These so-called Lyman limit systems (LLS), currently the least-well
studied of \ion{H}{1} absorption systems at high redshift,
are the focus of this manuscript.

Historically, the LLS were among the first class of quasar
absorption line (QAL) systems to be surveyed \citep{tytler84}.
This is because their spectral signature is easy to identify
even with low-resolution, low S/N spectra.  The principal challenge
is that the Lyman limit occurs redward of the atmospheric
cutoff only for systems (hence quasars) with redshifts $z>2.7$.
For lower redshifts, one requires spectrometers on space-bourn
ultraviolet sattelites.
By the mid 1990's, samples of several tens of LLS were generated
spanning redshifts $z 0 - 4$ \citep{ssb89,lzt9X,stengler95,storrie96}.
These results were derived from heterogeneous sets of quasars discovered
from a combination of color-selection, radio detection, and
spectroscopic surveys.    The spectra, too, was acquired with a 
diverse set of instrumentation and therefore varying S/N and spectral
resolution mitigating differing sensitivity to the precise optical
depth at the Lyman limit.  Although the results were not fully
consistent with one another, the general picture that resulted
was a rapidly evolving population of absorption systems reasonably
described by a $(1+z)^{1.5}$ power-law.

Cosmologically, the LLS contribute much if not most of the universe's
opacity to ionizing radiation.  Until recently, the observed incidence
of the LLS provided the only direct means of estimating the mean
free path \lmfp\ at any redshift \citep[e.g.][]{meiksin,madau99,fg08b}.
In a companion paper \citep{pwo09}, we have presented a new
technique to measure the \lmfp\ that circumvents any knowledge
of the LLS.  A more precise census of the LLS will serve as a 
consistency check for this \lmfp\ calculation, but is unlikely to
ever again be a competitive approach.  Instead, the incidence of LLS
can be used in combination with estimates of \lmfp\ to assess the
\nhi\ frequency distribution for gas 
with $\mnhi \approx 10^{16-18} \cm{-2}$, 
a regime that is very difficult to explore by studying individual
absorption systems.   Surveys of the LLS are also likely to 
place tight constraints on $z \sim 3$
cosmological simulations that include radiative transfer.

Physically, the nature of systems that give rise to LLS remains
an open question.  The systems with the largest \nhi\ values
(i.e.\ $\mnhi \ge 10^{19} \cm{-2}$, the so-called the super-LLS and
DLAs) are likely associated with the interstellar medium and
outer regions of high $z$ galaxies.  These high \nhi\ systems, however,
are only a subset of the LLS population.  Unfortunately, a proper
modeling of the LLS almost certainly requires careful modeling of 
radiative transfer which has thus far been beyond the scope of
modern computations.  Indeed, the few studies to date have tended
to severely underestimate the incidence of LLS \citep{katz94,gardnerXX,gnedin}.
In recent simulations of high $z$ galaxy formation, however, 
theorists have placed great attention on `streams' of cold gas that
carry fresh material from the IGM to star-forming galaxies 
\citep{keres99,dekel08}.  These cold streams have relatively large
surface densities ($N_{\rm H} \sim 10^{20} \cm{-2}$) and could therefore
produce Lyman limit absorption provided the material has a
non-negligible neutral fraction.  Consequently, an accurate census
of the LLS with redshift may directly constrain the nature and
prevalence of cold streams in the young universe.

A final, yet perhaps most important, motivation for studying the LLS
is that these systems may dominated the census of metals at all epochs.
The majority of LLS are metal-bearing, showing metal-line transitions of 
common low and high-ions \citep[e.g.][]{p99,pks2000}.
Because the estimated ionization corrections for LLS with 
$\mnhi \approx 10^{18} \cm{-2}$ is large, the observations likely
track only a trace amount of the metals.  [last comment]
 
[Add shot in the dark(s)?]

In this paper, we survey the homogeneous dataset of quasar
spectra from the Sloan Digital Sky Survey (SDSS), using all 
7 public data releases.  Our observational analysis
aims to produce the most precise measurement of 
the LLS incidence paying careful attention to systematic
biases.  The wavelength coverage and data quality of the SDSS
quasar spectra focusing the survey to $z \approx 3.5$.  Future
work will depend on follow-up obseravtions of well-define
quasar samples at other wavelengths.

The paper is organized as follows.  In $\S$~\ref{sec:def},
we present a set of LLS defintions used throughout the manuscript.
We adopt a cosmology with 
$H_0 = 72 \, h_{72} \, \mkms \, \rm Mpc^{-1}$, 
$\Omega_{\rm m} = 0.3$, and $\Omega_\Lambda = 0.7$
and report proper lengths unless specified.

\section{Lyman Limit System Definitions}
\label{sec:def}

The cross-section of \ion{H}{1} at energies above the Lyman limit
may be approximated by:

\begin{equation}
\sigma_{\rm LL}(\nu \ge \mnull) \approx  
  6.35 \sci{-18} \ltp \frac{\nu}{\mnull} \rtp^{-3} \; \cm{2} \cmma
\end{equation}

\begin{equation}
\mnull = E_{912}/h = c/\mlll \cmma 
\end{equation}

\noindent $E_{912} = 1$\,Ryd.  Specifically, 
$\nu_{\rm LL} = 3.29 \sci{15}$Hz and 
$\mlll = 911.7641$\AA.
This implies a continuum opacity at wavelengths $\lambda \le \mlll$,

\begin{equation}
\mtll(\lambda \le \mlll) \approx \frac{\mnhi}{10^{17.2} \cm{-2}} \ltp 
   \frac{\lambda}{\mlll} \rtp^{-3} \cmma
\end{equation}
where \nhi\ is the \ion{H}{1} column density. 
For a gas `cloud' intersecting a background source with intrinsic flux
$F_{\rm int}(\lambda)$, the observed flux $F_{\rm obs}(\lambda)$
blueward of the Lyman limit is 

\begin{equation}
F_{\rm obs}(\lambda \le \mlll) = 
  F_{\rm int}(\lambda) \exp \ltk -\mtll(\lambda) \rtk
\label{eqn:flux}
\end{equation}

In what follows, we will define a `standard' Lyman limit system (LLS) to be
one where $\mtll \ge 2$, i.e.\ $\mnhi \ge 10^{17.5} \cm{-2}$.
This corresponds to greater than 85\%\ attenuation of an 
incident ionizing radiation field. 
By this definition, the class of LLS includes systems with
$10^{20.3} \cm{-2} \le \mnhi \le 10^{19} \cm{-2}$ 
\citep[the so-called super-LLS or sub-DLAs; e.g.][]{opb+07} and systems with
$\mnhi \ge 10^{20.3} \cm{-2}$ \citep[the damped \lya\ systems; e.g.][]{wgp05}.
In a few cases, we will distinguish between these `strong' LLS from 
those with lower \nhi, referring to the latter as LLS.
We also note that our $\mtll \ge 2$ definition for LLS differs
from other works which adopted $\mtll \ge 1$ or $\mtll \ge 1.5$.
Unfortunately, there is no well-established criterion.


Observationally, the absorption of a background source
by a $\mtll \ge 2$ LLS is 
readily apparent, even in low S/N spectra.
We define absorbers with $\mnhi < 10^{17.5} \cm{-2}$ as the 
partial Lyman limit systems (pLLSs).  To survey these systems,
one requires higher quality spectra or an alternate approach to the analysis.

We define the redshift of an LLS as

\begin{equation}
\mzlls \equiv \frac{\mlll^{\rm LLS}}{\mlll} - 1
\end{equation}
where $\mlll^{\rm LLS}$ marks the observed onset of LL absorption.
In practice, this is is often estimated from strong Lyman series lines
(e.g.\ \lya, \lyb) that accompany the Lyman limit opacity.

We define the sub-set of LLS that occur within 3000\kms\
of the emission redshift of the background source as proximate LLS
(PLLS).  We separate the analysis of these systems from the rest
to investigate changes in the incidence of optically thick gas
near high $z$ quasars due to, e.g.\ the quasar's radiation field
and local environment $\S$~\ref{sec:prox}.

Finally, we define the observable \lzlls\ as the average
number of LLS detected per unit redshift at a given redshift.
In the previous literature, this quantity is also expressed
as $n(z)$, $dN/dz$ and $dn/dz$.

\section{SDSS Quasar Sample and Spectroscopy}

One of the primary objectives of the Sloan Digital Sky Survey (SDSS)
was to discover $\sim 100,000$ new quasars across the northern sky
\citep{yaa+00}.  The strategy of the SDSS team to achieve this
ambitious goal was a four-fold process:
(i) obtain deep, multi-band images across a large area of the sky;
(ii) select quasar candidates by demanding a point-like point-spread-function
and imposing color criteria that separate the candidates from the
Galactic stellar locus;
(iii) obtain follow-up spectra for a magnitude-limited sample
with a fiber-fed spectrograph.
The details of target selection and quasar completeness with
redshift is described at length in a series of SDSS papers
\citep{rfn+02}, but see \citep{wp09} for a new and more accurate analysis;  
and
(iv) automatically identify quasars and estimate their redshifts
(\zem) through template fitting to the optical spectroscopy.

Of these steps, the second has the greatest 
impact on a survey for high $z$ Lyman limit systems. 
The key issue for our survey is whether the presence 
of an intervening LLS biases the targeting of the
background quasar for follow-up spectroscopy.  
In effect, a high $z$ LLS severely `reddens' the quasar
at the bluest optical wavelengths of the SDSS imaging.  
With this effect in mind, the SDSS team imposed cuts
on the $(u-g)$ color which better separated the quasar
locus in color pace from the stellar locus.  The net effect,
however, is to bias the spectroscopic follow-up against 
quasar sightlines {\it without} foreground LLS
\citep[][hereafter PWO09]{pwo09}.  Our analysis indicates
an important bias for quasars with $\mzem < 3.6$. 
For this reason, we will limit the statistical analysis
to quasars with $\mzem \ge 3.6$, but we also explore this bias
by considering the incidence of LLS toward quasars with $\mzem = 3.4 - 3.6$.


The quasar spectra analyzed in this paper were taken from the Sloan
Digital Sky Survey, Data Release 7 \citep{sdssdr7}.  
We retrieved
the `best' 1D spectrum for every source flagged as a QSO or HIZ\_QSO.
This totalled 102,418 unique spectra\footnote{  
The spectra were processed through our automated algorithms for
finding absorption-line features and damped \lya\ candidates
\citep{phw05,shf06}.  
Approximately 10 of the spectra failed 
to be processed (primarily because the SDSS-repoted emission redshift
is erroneous) and were removed from any subsequent analysis.}.
The SDSS survey employs a 
fiber-fed, dual-camera spectrometer that provides continuous wavelength coverage
from $\lambda \approx 3800-9200$\AA\ at a spectral resolution of 
FWHM~$\approx 150 \mkms$.
The SDSS team employs a custom, data-reduction pipeline that performs
sky subtraction using emprical measurements from fibers placed
to avoid objects detected in the SDSS images.
The majority of data suffer from excessive sky noise at long
wavelengths ($\lambda > 8000$\AA)
and the instrument throughput and atmospheric absorption
limits the sensitivity at the shortest wavelengths ($\lambda < 4200$\AA).


A survey for \ion{H}{1} Lyman limit absorption in quasar spectra
invovles two principle steps.  First, one must assess
the flux at wavelengths near the Lyman limit
in the quasar's rest-frame, $\lambda \lesssim \mlll(1+\mzem)$.  
We discuss our procedure for this step in the following section.  
The second step is to 
estimate the flux at wavelengths blueward of LLS candidates.
There are several characteristics of the SDSS spectroscopy which
negatively affect this estimate.  A generic concern is the poorer instrument
response at the bluest wavelengths.  At the bluest wavelengths,
many of the spectra exhibit a very low signal-to-noise ratio 
and yield flux estimates consistent with zero, even without an
intervening LLS.
Therefore, we have limited our survey to redshifts $\mzlls \ge 3.3$
corresponding to $\lambda > 3920$\AA.
In practice, we restrict the quasar sample to objects with
$\mzem \ge 3.6$ and perform a search for LLS
at all redshifts, but then only analyze absorption systems with
$\mzlls \ge 3.3$.  We also limit the survey to quasars with
$\mzem \le 5$ because the SDSS spectra of higher redshift objects
are generally too low S/N to permit a robust analysis.
Figure~\ref{fig:snz}a presents histograms of the emission redshifts
for the statistical sample and the signal-to-noise of the
absorbed continuum at the Lyman limit
(\sna; see below for the definition).


Another difficulty with the SDSS spectra at blue wavelengths
is that the two-dimensional spectra of faint sources may be improperly traced.
On occasion, the 1D extractions
include flux from a neighboring object and yield a systematic 
overestimate of the flux.  This effect reduces the estimated opacity
for a LLS.  
Another issue, especially with a fiber-fed spectrometer, is that
the sky model is estimated from nearby fibers that are intentionally
placed on `object-free' regions of the sky.  Although the SDSS
project has worked carefully to mitigate the effects of variable
fiber efficiency and XXX \cite{skysub},  significant misestimates 
of the sky occur.  
We have identified tens of objects where the extracted
flux is significantly negative, indicating an overestimate
in the sky model.  
This may convert a partial LLS (with $\mtll < 1$) into
an LLS.  By a similar token, a proper $\mtll \ge 2$ LLS may
appear as a pLLS if the sky is underestimated.
We proceed under the expectation that this effect
is nearly random, i.e.\ for every underestimate of the sky there
is a corresponding overestimate, but this has not been rigorously
established.  
The SDSS fibers are sufficiently wide
(diameter of $3''$) that they will occasionally include flux
from a projected neighbor.  These coincident objects may be
much fainter than the quasar at redder wavelengths, but they
could contribute all of the flux blueward of a strong LLS
and lead to an underestimate of the LL opacity\footnote{An amusing
(and plausible) systematic effect related to this is contamination
by the light reflected from terestrial satellites crossing the night sky.  
Even a brief `exposure' through the 3$''$ fiber could dominate the
flux at the bluest wavelengths, although this should generally be
mitigated by the fact that the SDSS team aquires 
3 unique exposures per target.}.


\section{The Absorbed Quasar Continuum}
\label{sec:continuum}

Absent any other sources of opacity, one can trivially
estimate \tll\ from
the quasar spectrum through measurements of the flux
both redward and blueward of the observed Lyman limit (Equation~\ref{eqn:flux}).  
In practice, however,
the quasar flux is also attenuated by line opacity from the so-called
\lya\ forest (a.k.a., the intergalactic medium; IGM).
For example, consider a Lyman limit system at $\mzlls = 3.5$
intervening a $\mzem = 4$ quasar.  The LLS attenuates the quasar flux
blueward of $\mlll^{\rm LLS} = 4103$\AA.  At this wavelength,
the quasar spectrum recorded on Earth will also include opacity
from the \lya\ forest at 
$z_{\rm \mlya} = (1+\mzlls)(\mlll/\lambda_{\rm Ly\alpha}) - 1$,
\lyb\ absorption from the IGM at 
$z_{\rm Ly\beta} = (1+\mzlls)(\mlll/\lambda_{\rm Ly\beta}) - 1$, etc.
It is necessary, therefore, to account for these additional sources
of opacity when estimating \tll.

We can express
the observed quasar flux $F_{\rm obs}$ in terms of
the intrinsic flux (just) redward of the Lyman limit $F_{\rm int}$ as

\begin{equation}
F_{\rm obs} (\lambda \gtrsim \mlll) = F_{\rm int} \exp [-\mtigm(\lambda)] \cmma
\end{equation}
where \tigm\ is the effective opacity of the IGM from
Lyman series line-opacity\footnote{In the following, we do not
explicitly derive the 
opacity from metals in the IGM, but these may be considered included in \tigm.}.
The LLS introduces an additional, continuous opacity: 

\begin{equation}
F_{\rm obs} (\lambda \le \mlll) = 
  F_{\rm int} \exp [-\mtigm(\lambda) -\mtll(\lambda)] \perd
\end{equation}
A precise estimate of \tll, therefore, requires an estimation
of the absorbed quasar flux not its intrinsic flux.  Conveniently,
this quantity is the observed flux recorded in the spectrum at
$\lambda \gtrsim \mlll$.  
There are still signficant challenges because
the IGM opacity is stochastic on both small (individual Lyman lines)
and modest scales (many 10\AA) 
and 
the intrinsic quasar spectrum (both shape and normalization) varies
from source to source.
We now describe an automated procedure
used to infer the absorbed continuum from each quasar spectrum.


The traditional method of estimating the
quasar continuum is to first identify
regions of unabsorbed quasar flux and then to interpolate a continuum level
between these regions.  For quasars at
high redshift, this method is particularly error prone, because at
wavelengths below \lya\ emission we expect few (and at very
high redshift, none) of the pixels to be free
of absorption from the IGM.   Moreover, this
traditional method frequently requires by-hand modification,
which is time-intensive and subjective to individual biases.
Methods do exist to automatically generate a quasar
continuum from emission-line characteristics \citep[e.g.][]{suzuki+05},
but these are designed to infer the intrinsic quasar spectrum
not the IGM-absorbed continuum.
Our approach is to match a template model of the average absorbed
continuum to each spectrum,
allowing for a large-scale tilt (i.e. a unique underlying power-law slope)
and arbitrary normalization.

Our first step is to derive the templates for the average absorbed
quasar continuum.  Because the line-density of the IGM and therefore \tigm\
increase with redshift, we perform this analysis in small redshift intervals
($\delta z = 0.3$ to 0.6; Figure~\ref{fig:template}). 
For every quasar within the redshift interval,
we shift the spectrum to the quasar rest-frame 
using the SDSS reported \zem\ value.  
Next, we "detilt" the spectrum by removing a power-law shape.
This power-law is determined as follows.  We have
constructed from the SDSS-DR3 dataset an average template spectrum,
archived in XIDL\footnote{http://www.ucolick.org/$\sim$xavier/IDL}
as ``full\_SDSS\_LLS.fits''.
For each individual quasar spectrum,
we sample quasar pixels with wavelengths greater 
than \lya\ emission, divide by the template spectrum,
and measure the spectral slope of the resulting spectrum.

After aligning the spectra in the quasar rest-frame
(nearest pixel), we median-combine the data in each redshift 
interval\footnote{This stack is not optimal for deriving \lya\
forest statistics \citep{dlw08}.}.
The resultant template
represents the median intrinsic quasar continuum modulated by the median
flux decrement of the IGM.  Due to the presence of
Lyman limit systems, the quasar templates will not be useful at
wavelengths near and below the rest-frame Lyman limit;
this portion of the spectrum, however, can be used to 
constrain the mean free path to ionizing radiation \citep{pwo09}.
For wavelengths blueward of 920\AA, therefore,
we set the template to have the value recorded 920\AA.
Conveniently, there are no strong emission features in the
quasar SED at these wavelengths \citep[e.g.][]{telfer02}.
The resultant template spectra are shown in Figure \ref{fig:template}.

With the templates constucted, a model of the absorbed continuum
for each quasar is determined as follows.
First, we shift the observed quasar spectrum to the rest-frame
and divide by the appropriate template spectrum (i.e.\ according to \zem).
Second, we sample the quasar in the wavelength range 
$950 \rm \AA < \lambda < 1800 \rm \AA$ and fit a power-law 
(p$(\lambda)$ = A + B log[$\lambda$/\AA]) to the observed 
flux, weighting by the inverse variance array.
The emission lines in this spectral range may bias the fit,
but we do review and modify these fits (see below).
The product of this power-law with the template, when
shifted to the observed frame, provies our model for the
absorbed quasar continuum.  The power-law parameters derived
in this fashion are listed in Table~\ref{tab:quasars}.
Sample fits are shown in Figure~\ref{fig:ex_conti}.
With these models, we can calculate the ratio of the absorbed
continuum to the $1\sigma$-error array at the Lyman limit,
which we denote as \sna.
Figure~\ref{fig:snz}b shows the distribution of \sna\ values
for the statistical survey.

\section{Lyman Limit System Search and Characterization}
\label{sec:lls}

In the following, we parameterize a LLS by two quantities:
(1) the absorption redshift \zlls\ and
(2) the total \ion{H}{1} column density \nhi.
Although the \ion{H}{1} Lyman series lines are sensitive to the component
structure and the Doppler parameters (also known as $b$-values)
of the `clouds' comprising an LLS,
the opacity blueward of \lll\ is insensitive to the details.
Furthermore, the SDSS spectra are generally of too poor quality
to constrain such structure using the observed Lyman series lines.
Therefore, our model of an LLS assumes a single cloud with a 
Doppler parameter of $b=30 \,\mkms$.
An implication of this parameterization is that two systems with
small redshift separation are modeled as a single system with
the total of the \nhi\ values.  Our tests with mock spectra ($\S$~\ref{sec:mock})
indicate that two absorbers with $\delta z < 0.1$ are often
indistinguishable from a single LLS.  The survey presented here,
therefore, refers to LL absorption smoothed over a redshift
interval of $\delta z \approx 0.1$.  We return to this point
in our presentation and discussion of the survey results.

We have developed an algorithm ({\it sdss\_findlls})
to automatically search for
and characterize LLS absorption in quasar spectra.
In brief, the code generates a set of model spectra
for the line and continuum opacity of
a single LLS with redshifts covering 
$z=z_0=(\lambda_0/\mlll-1)$ to (\zem+0.2) where $\lambda_0$
is the starting redshift of the SDSS spectrum and \zem\ is the
quasar emission redshift reported in the DR7\footnote{Modified by
the analysis of \cite{phh08} where applicable for quasars from 
the DR5 release.}.
The grid of models assumes \nhi\ column densities 
$\log \mnhi = 16.0, 16.2, 16.4, ..., 19.8$
and a Doppler parameter $b = 30 \mkms$.  
We implement a grid with 0.2\,dex spacing in \nhi\ because very
few of the spectra have sufficient S/N to provide a more precise
estimate.  Furthermore, we estimate systematic uncertainties
(e.g.\ continuum placement, sky subtraction) to be of this order.
These models are convolved with the SDSS instrumental
resolution and then applied to the absorbed quasar continuum 
($\S$~\ref{sec:continuum}).
Finally, the code constructs a $\chi^2$ grid in \zlls\ and \nhi\
space, identifies the minimum $\chi^2$, and records the `best-fit' values.  
In general, the spectra provide very little constraint
on \nhi\ for values exceeding $10^{17.5} \cm{-2}$ until the \lya\
profile becomes damped \citep[e.g.][]{phw05}.
Therefore, we report lower limits to \nhi\ for any LLS with 
$\mtll \ge 2$.  This approach differs from previous methods
which focused solely on the Lyman limit \citep[e.g.][]{sl94} or
relied on `by-eye' analysis \citep{lzt9X}.

For sightlines with a single LLS having $\mnhi > 10^{17.2} \cm{-2}$
and good \sna\ (i.e.\ greater than 5 pix$^{-1}$ 
at $\mlll^{em} \equiv \mlll \times [1+\mzem]$),
we find that the automated
algorithm is highly successful on its own.
In practice, however, there are several aspects of the data and
analysis that require visual inspection of the spectra
and interactive modification to the model:
First, many spectra have 
such low S/N that \zlls\ and \nhi\ are poorly determined.
In these cases, the a local minimum in $\chi^2$ can occur which
gives a misestimate of these quantities.
Second, the presence of multiple
absorbers along the sightline (e.g.\ one or more pLLS with a lower
redshift LLS) gives a spectrum that cannot be well
modeled by a single LLS.
Third, a non-negligible number of the spectra
retrieved from the SDSS database purported to be high $z$
quasars are either at a lower redshift or some other astronomical
object altogether.  
Fourth, we found that half of the absorbed continuum models required
scaling to higher or lower value by greater than $10\%$.  
Finally,  we prefer to avoid quasars with 
with strong broad absorption line (BAL) or associated systems
to focus the analysis on intervening LLS.

Given the above complications to an automated analysis, we
built a graphical user interface (GUI) within the IDL software
package ({\it sdss\_chklls}; bundled within XIDL)
that inputs the data and best-fit LLS model for each object.
Two of the authors (JXP and JMO) used this GUI to validate
and/or modify all of the models.  These authors 
flagged erroneous spectra (159 examples), 
strong BAL or associated absorption (quasars showing very strong
\ion{C}{4}, \ion{N}{5}, and \ion{O}{6} absorption; 290 quasars), 
or data with such low
S/N that any analysis was deemed impossible (114 spectra).
For the remainder of sightlines, the authors
could modify the continuum (via a multiplicative scalar; Table~\ref{tab:quasars})
and/or change the model of LLS absorption (i.e.\ \zlls, \nhi).
This includes absorption due to candidate pLLS.  
In many cases, \zlls\ was modified to correspond to the strongest,
local \lya\ absorption line at $\lambda = (1+\mzlls) \times 1215.67$\AA,
especially for those systems that also showed absorption at the
expected wavelength for \lyb.

After every sightline was analyzed in this manner,
the results from the two authors were compared to assess
consistency.  Roughly half of the spectra were reviewed
because of conflicts in the models.
The majority of these were associated 
with the absorbed continuum placement which implied 
differences in the search path of $|\Delta z| > 0.1$
(see $\S$~\ref{sec:path}).
In the majority of these cases, we simply averaged the
two estimations of the continuum.  
The second most frequent conflict was on the definition
of strong BAL absorption, primarily because we did not adopt
uniform or strict criteria.  In most cases, we 
conservatively excluded the sightline.
There were also $\approx 100$ cases where one author estimated
$\log \mnhi = 17.4$ when the other estimated   
$\log \mnhi = 17.6$, i.e., straddling the $\mtll=2$ boundary
that defines our LLS search. 
These were especially scrutinized for the presence of
higher-order Lyman series lines. 
Where necessary, the final best-estimate for \nhi\ was 
deferred to the third author (GW).

Tables~\ref{tab:survey} and \ref{tab:llscand} list the set of 
Lyman limit systems identified in the SDSS-DR7 for systems
that (respectively) influence our statistical analysis and otherwise. 
For each system, we list our best estimate for \zlls\ and the \nhi\ value.
Because the data offer minimal constraint on \nhi\
for values exceeding 10$^{17.6} \cm{-2}$,
in the statistical sample we only report a lower
limit to \nhi\ for all systems with best-estimates of
$\mnhi \ge 10^{17.6} \cm{-2}$. 
For the pLLS, the typical $1\sigma$ uncertainty
is approximately 0.2\,dex for $\mnhi \ge 10^{17} \cm{-2}$
and is dominated by systematic error in
the continuum placement and the stochastic nature of the IGM opacity. 
As noted above,
\zlls\ was frequently modified
in the interactive analysis to correspond to strong \lya\ and
\lyb\ lines.
Our analysis of mock spectra indicate typical redshift
uncertainies of $\sigma(z) < 0.02$ with occasional
`catastrophic failures' due to line-blending or spurious
spectra (Appendix~\ref{appx:mock}).
No attempt was made to improve this
estimate by searching for the presence of metal-line
absorption (e.g.\ \ion{C}{4}) outside the \lya\ forest.
There is a tendency, both in the automated algorithm and in 
interactive analysis, to set \zlls\ at the highest value possible
that can be accomodated by the data.  To this extent, we suspect
that there is a modest bias in our \zlls\ values to slightly
higher redshifts (less than 0.01 on average, but with
the occasional large offset).

Figure~\ref{fig:ex_conti} shows a representative sample of four quasar
spectra, zoomed into the region blueward of \lyb, with the
absorbed continuum and LLS models indicated. 
We provide snapshots of the 
LL region for all quasar spectra in the statistical sample 
online\footnote{http://www.ucolick.org/$\sim$xavier/SDSSLLS}.


%%%%%%%%%%%%%%%%%%%%%%%%%%%%%%%%%%%%%%%%%%%%%%%%%%%%%%%%%%%%%%%%
%%%%%%%%%%%%%%%%%%%%%%%%%%%%%%%%%%%%%%%%%%%%%%%%%%%%%%%%%%%%%%%%
\section{Survey Path}
\label{sec:path}

Analagous to galaxy surveys where one defines a search
volume based on the depth of imaging and spectroscopic
follow-up, measurements of the incidence of quasar absorption line systems
requires an estimate of the total spectral path sensitive 
to a robust search.  This is generally referred to as the redshift
path covered (by translating observed wavelength into redshift).
For the survey of LLS, we have adopted the following criteria
for including spectral regions in the search.  These are based on our 
automated and interactive analysis of the SDSS spectra, 
our modelling of Keck/LRIS follow-up spectra, and our analysis
of simulated spectra ($\S$~\ref{sec:mock}, Appendix~\ref{appx:keck}):

\begin{enumerate}
\item The search path will begin at a minimum redshift of
  $z_{\rm start} \ge 3.3$
\item For the intervening LLS sample, the search path ends at the redshift
  $z_{\rm end}$
  corresponding to 3000~\kms\ (relativistic) blueward of \zem.
\item The absorbed continuum flux must exceed twice the estimated error array, 
 i.e.\ \sna.
\item The search path begins at a maximum offset of 
   $\delta z = \mzem - z_{\rm start} \le 0.4$.
\end{enumerate}
The first criterion is motivated by the starting wavelength 
of the SDSS spectra ($\lambda_0 \approx 3800$\AA) and the
poorer quality of the data at the bluest wavelengths.
For redshifts less than 3.3, there is insufficient spectral
coverage and/or data quality to confidently assess the
presence of an LLS.  The second criterion minimizes
the influence of the quasar and its environment on
the analysis.  This criterion is relaxed in the study of PLLS. 


The third criterion is the most subjective, yet important,
for setting the redshift survey path of each quasar.
Algorithmically,  we impose this constraint by identifying
the first pixel blueward of $\mlll^{\rm em} \equiv \mlll(1+\mzem)$
where our model of the absorbed quasar continuum falls
below twice the median-smoothed (15 pixels) $1\sigma$ error array. 
This pixel defines the starting redshift \zstrtt\ corresponding
to a \sna=2 limit.  
If the first pixel blueward of $\mlll^{\rm em}$ does not satisfy
the S/N threshold, the quasar has zero redshift path, 
i.e.\ $z_{\rm start} = z_{\rm end}$.
One can, of course, define 
starting redshifts
corresponding to higher (or lower) \sna\ limits;  indeed, our
fiducial choice of \sna=2 should be considered arbitrary, although 
it is guided by our analysis of real and simulated spectra.
And to avoid a systematic bias associated with pLLS (see $\S$~\ref{sec:pLLS}),
one must choose the \sna\ criterion to be sufficiently high
to discover $\mtll \ge 2$ LLS even in the presence of a 
pLLS\footnote{Contrary to some of the earliest work 
on LLS \citep[e.g.][]{tytler84},
we do not use surival statistics to estimate the number of LLS
at redshifts below any observed LLS.}.
We investigate the impact of this choice on our results
later in the manuscript.

The starting redshift is further modified by the presence 
of Lyman limit absorption.  In the case of a $\mtll \ge 2$ absorber,
the quasar flux is severely depressed below $\lambda = \mlll (1+\mzlls)$
and we terminate the search path at this wavelength.  
Specifically, this implies $z_{\rm start} \ge \mzlls$ for all
sightlines with an $\mtll \ge 2$ LLS.
For sightlines where one or more pLLS are identified\footnote{Again,
all pLLS should be considered candidates.  Some of the spectral
features modelled as pLLS are instead due to unusual variations
in the absorbed quasar continuum (e.g.\ highly reddened quasars).}, 
we had originally intended to terminate the search once the absorbed 
quasar continuum convolved with the pLLS absorption failed to satisfy
the S/N threshold.  In our analysis of mock spectra, 
however,  we found that this introduces a ``pLLS-bias'' where the
incidence of $\mtll \ge 2$ LLS is overestimated.
In part, our \sna=2 threshold is chosen so that one can 
robustly search for $\mtll \ge 2$ LLS even along sightlines where
one or more pLLS are present.  

Table~\ref{tab:survey} presents the list of quasars in SDSS-DR7
that 
(i) have $5 \ge \mzem \ge 3.6$, 
(ii) were not identified to
exhibit strong BAL signatures, and 
(iii) have $3.3 \le \mzstrtt < \mzem$.
There are \nstatqso\ quasars satisfying these criteria.
For each sightline, we list the starting redshifts for
\sna=2 and 3 limits.
We report results for two values of the S/N threshold to
search for data-quality biases. 
Table~\ref{tab:survey} also lists all Lyman limit systems
with $\mzlls \ge \mzstrtt$.  
We do list these quantities for quasars with $\mzem < 3.6$,
but these were not included in our final statistical analysis
because of the bias related to SDSS targeting criteria
previously mentioned by \citep{pwo09}.

Using the values presented in Table~\ref{tab:survey}, it is
straightforward to calculate \zend\ and the redshift search
path for each quasar: $\Delta z_i = \mzend - \mzstrt$.  For
a given \sna\ limit, the total search path for the full dataset is 

\begin{equation}
\Delta z_{\rm TOT} = \smm \Delta z_i  \perd
\end{equation}
We calculate $\mztot^{S/N=2} = 96$.
%of XX increase over the combined search paths of all previous
%$z>2$ surveys for LLS \citep{peroux05}.
To maintain a homogeneous sample and set of search criteria,
we do not include previous studies in our analysis.
We compare against previous results in $\S$~\ref{sec:discuss}.

In Figure~\ref{fig:goz}, we present the sensitivity function
\gz\ which expresses the number of SDSS quasars at redshift $z$
where a robust search for LLS is possible.
The several solid curves represent differing \sna\ limits for the survey.
We also present the sensitivity function for proximate LLS (PLLS)
which corresponds to the sample of quasar spectra that satisfy
the $\msna \ge 2$ criterion at 3000~\kms\ blueward of \zem.

With the definition of the search path 
and the identification of the Lyman limit systems along each
sightline ($\S$~\ref{sec:lls}),
it is straightforward to calculate the incidence of intervening
LLS per redshift interval, \lzlls.  
The standard estimator is to compare the total number of LLS
against the total survey path in discrete redshift intervals.
We will return to evaluate \lzlls\ and discuss the values
after exploring several sources of systematic error.


%%%%%%%%%%%%%%%%%%%%%%%%%%%%%%%%%%%%%%%%%%%%%%%%%%%%%%%%%%%
%%%%%%%%%%%%%%%%%%%%%%%%%%%%%%%%%%%%%%%%%%%%%%%%%%%%%%%%%%%
\section{Mock Spectra and Systematic Errors}
\label{appx:mock}

With a survey the size of SDSS, one can quickly reduces the statistical
noise in measurements to very small levels.  In this
regime, one must carefully assess all sources of systematic
error as these may dominate the measurement uncertainty.
To this end, we have conducted a range of tests with mock
spectra as described in this section.

\subsection{Mock Spectra Construction}


We generated a set of 800 mock SDSS quasar 
spectra and analyzed them in the same way as the real data in 
order to assess bias and completeness in our LLS survey. 
The \ion{H}{1} forest absorption spectra 
were generated via a Monte Carlo routine similar 
to the one described in \cite{dww08}, 
assuming that the \lya\ forest is well characterized by 
three independent distributions: 
(i) the \lya\
line-incidence $\ell_\alpha (z) \propto (1+z)^\gamma$, 
(ii) the \ion{H}{1} column density distribution $f(\mnhi) \propto \mnhi^\beta$, 
and (iii) the Doppler parameter distribution parameterized as 
$f(b) \propto b^{-5} \exp{-b_\sigma^4/b^4}$ \citep{hr99}. 
Each simulated line of sight was filled with \ion{H}{1} \lya\ absorption lines at 
$2<z<4.6$ until the \ion{H}{1} \lya\ effective optical depth was consistent with 
\cite{fpl+08}, both in normalization and redshift evolution. We did not model the 
$z\sim 3.2$ dip in the effective optical depth 
measured by Faucher-Giguere et al., 
and instead adopted a simple power-law 
$\tau_{\rm eff}^{\alpha} =0.0011(1+z)^{4.23}$. 
If the number of lines in a given patch of the forest is 
Poisson-distributed, a power-law line density evolution 
$\ell_\alpha (z) \propto (1+z)^\gamma$ yields a power-law 
effective optical depth evolution $\tau_{\rm eff}^\alpha
\propto (1+z)^{\gamma+1}$ \citep{zuo93}. 
The column density distribution was modeled with a single power-law index 
$\beta=-1.5$ for $12<\log(\mnhi)<19$, but with a 0.5\,dex break at 
$\log(\mnhi)=14.5$ in order to account for the dearth of high column density lines, 
consistent with observations \citep[e.g.][]{hkc+95,kim02}. 
For the Doppler parameter distribution we set 
$b_\sigma=24 \, \mkms$ \citep{kim+01}. 

Because SLLSs $(19 \le \log(\mnhi)<20.3)$ and DLAs $(\log(\mnhi) \ge 20.3$) have 
different column density distributions and are usually excluded in 
measurements of $\tau_{\rm eff}^\alpha$,
these were added after the $\log(\mnhi)<19$ 
line forest converged to the chosen $\tau_{\rm eff}^\alpha (z)$. 
To constrain the redshift evolution of SLLSs, we combined the sample by 
\cite{opb+07} and the lower limit given in \cite{rtn06}, 
yielding $\ell_{\rm SLLS} (z) \sim 0.066(1+z)^{1.70}$. 
For the SLLS column density distribution we adopted $\beta=-1.4$ 
\citep{opb+07}. 
The DLAs were modeled via $\ell_{\rm DLA} (z) =0.044(1+z)^{1.27}$ \citep{rtn06} 
and $\beta=-2$ \citep{phw05}, 
ignoring deviations in \fnhi\ from a single power law. 
The Doppler parameter distribution was left unchanged.

With the overall opacity of the modeled \lya\ forest consistent with observations, 
and the high column density systems incorporated, we used the generated line lists to compute HI Lyman series (up to Ly30) and Lyman continuum absorption spectra. In total we computed 800 different lines of sight for quasars in the redshift range of our sample, 
160 each at $z=3.4, z=3.6, z=3.8, z=4.0$ and $z=4.2$, respectively. 
From these we then generated mock SDSS spectra. First, the resolved \ion{H}{1} forest spectra were multiplied onto synthetic quasar SEDs generated from principal component spectra \citep{suzuki+05,suzuki06}. 
We then degraded the resolution of the mock spectra to $R=2000$ by convolving them with a Gaussian, and rebinned them to $\delta v=69 \mkms$, matching the approximate 
resolution and pixel size of the SDSS spectra. Finally, we added Gaussian noise to the mock SDSS spectra. In each mock spectrum the S/N was normalized in the quasar continuum 
at 1450\AA\ and varied as a function of flux and wavelength according to the throughput of the SDSS spectrograph. The sky level was approximated as a constant and readout noise was also incorporated. 
Finally, for a subset of the mocks
we imposed a sky subtraction error implemented by subtracting/adding
a constant to the spectrum.
These were generated to mimic such systematic errors
that occasionally occur in the SDSS spectra.

Four representative examples are shown in Figure~\ref{fig:mockex}. 

\subsection{LLS Recovery and Sky Subtraction Bias}

Two of the authors (JXP,JMO) analyzed 500 of the mock spectra using
the identical tools and procedures applied to the SDSS spectra;
these steps were done without knowledge 
of the mock line distribution and column densities (constructed by author GW).
The results for the two authors were nearly identical; the following
discussion and figures refer to the results related to JXP.
Figure~\ref{fig:mockdz} summarizes the completeness and several biases
uncovered by our analysis.  In the top panel, all cases where a mock
$\mtll \ge 2$ LLS exists with $\mzabs > 3.2$ and an LLS was `observed'
are presented; these correspond to $> 80\%$ of the cases.  
Specifically, we plot the offset $\delta z$
between the true and
observed LLS absorption redshifts as a function of the \sna\ of the
spectrum.  
We find excellent agreement (small $\delta z$), nearly independent
of the \sna\ of the data.
There are, however, a number of cases with $\delta z > 0.01$, primarily
related to the blending of absorption lines (see below).

The lower panel in Figure~\ref{fig:mockdz} presents those cases
that are false negatives (diamonds) and false positives (triangles).
The latter are very rare. [seem too rare..]
The former, however, did occur quite often in our analysis and can
be divided into two classes:  
(i) cases with $|\delta z| < 0.05$ which are proper false negatives, i.e.\
true $\mtll \ge 2$ LLS that were modeled as pLLS;
(ii) cases with $\delta z > 0.1$ which are sightlines where a
higher $z$ pLLS precluded the detection of a lower $z$ LLS.
The majority of the latter cases 
are due to bona-fide pLLS at higher $z$ which greatly diminish
the S/N of the spectra at shorter wavelengths and `obscure'
the presence of a $\mtll \ge 2$ LLS.
Almost none of these cases, however, satisfy the selection
criteria established in $\S$~\ref{sec:survey};  either the
data have too low \sna\ or the absorption redshift of the LLS
gives $\mzem - \mzabs \ge 0.4$.  

The first class of false negatives, meanwhile,
are almost exclusively associated with spectra
that had systematically low estimates for the sky background.
In these cases, a $\mtll \ge 2$ LLS has an apparent flux 
at $\lambda < \mlll$ and therefore was modeled as a pLLS.
This is the dominant effect of a sky subtraction bias.
Although our mock spectra had even numbers of over and under-subtracted
sky backgrounds, only the former
are relatively easy to identify (large
regions of spectra are significantly negative) and ignore.
The net effect of a random sky subtraction error is a 
systematic underestimate in the incidence of LLS.
We stress, however, that the magnitude and frequency of poor
sky subtraction in the mock spectra was intentionally elevated
so that we could explore these effects.  The incidence of
such effects within the SDSS spectra is much lower, an
assertion supported by our follow-up spectra with Keck/LRIS
(Appendix~\ref{sec:keck}).
Therefore, we are confident that the sky subtraction bias 
gives rise to a $<10\%$ systematic error for \lzlls.




\subsection{The pLLS Bias}
\label{sec:pLLS}

Originally, we inteded to perform a search for LLS in spectra
with \sna=1 to maximize the pathlength of the survey 
(a nearly 4$\times$ increase over \sna=2).  Our
tests with mock spectra and follow-up observations with 
Keck/LRIS (Appendix~\ref{appx:lris}) indicated that we 
could robustly identify $\mtll \ge 2$ LLS in such data.
We also noted that many of the spectra showed
pLLS candidates which reduced \sna\ to below 1 and made the
search for $\mtll \ge 2$ LLS more challenging.  Our
reaction was to redefine the search path 
by attenuating the absorbed continuum due to
any identified pLLS candidates and then reapply
the \sna\ criterion for the remaining $z<z_{\rm pLLS}$
spectral range.

In princple, this modification should provide an unbiased
search for $\mtll \ge 2$ LLS.  Our trials with mock spectra,
however, revealed an insidiuous bias associated with this
redefinition of the search path.  Specifically,  it is very
difficult to identify pLLSs when they occur at a small
redshift offset ($\delta z \lesssim 0.1$) redward
from a $\mtll \ge 2$ LLS.
In these cases, instead of the search being terminated at
the redshift of the pLLS with the lower $z$ LLS ignored,
the pLLS is ignored and the LLS is included within the
statistical sample.   Furthermore, the redshift
of the recovered LLS is biased to a higher value which 
reduces the survey pathlength by a small but non-negligible
quantity.  Together, these two effects lead to an overestimate
of \lzlls\ by values ranging from $30-50$\%.  
It is very difficult to precisely estimate the magnitude of this
systematic bias for it depends sensitively on \lzlls, the 
incidence of pLLS, and the quasar \zem\ distribution.
In our opinion, one cannot robustly correct for 
this systematic pLLS bias and we caution against performing
any analysis that would be subject to it.
For these reasons, we {\it ignore} pLLS when defining
the survey path and performing the search for $\mtll \ge 2$ LLS.


\subsection{The Blending Bias}


As noted in the previous sub-section, LLS and pLLS
that lie close to one another in redshift are very difficult
to distinguish as individual systems.
This is even true in the limit where one has spectra with
exquisite S/N and resolution when $\delta z < 0.1$.
With SDSS spectra, the limited information provided by the Lyman
limit and the strongest Lyman series lines is insufficient
to robustly distinguish multiple LLS from a single system.
This leads to a ``Blending Bias'' that manifests itself in several
ways.

First, the blending bias increases the number of LLS
observed because pairs of pLLSs blend together to give a single
system with $\mtll \ge 2$.
Second, the absorption redshifts of the LLSs are shifted to higher
redshifts because one generally adopts \zlls\ from the higher
of the pair of systems.  
This leads to a smaller survey path and possibly a
higher inferred incidence of LLS. 
More importantly (see below), many LLS are shifted into the
proximate region of the quasar. 
This causes an underestimate of \lzlls\
for intervening LLS and an overestimate of PLLS.

We explored the quantitative effects of the blending bias with
the following analysis.  We constructed a set of mock absorption
lines for each quasar in the statistical survey (Table~\ref{tab:survey})
with an incidence set to match our measurements ($\S$~\ref{sec:loz}).
Specifically, we adopted an \nhi\ frequency distribution

\begin{equation}
f(\mnhi,z) = C \mnhi^\beta \, \ltp \frac{1+z}{1+z_*} \rtp^\gamma
\label{eqn:fnmock}
\end{equation}
with $\beta = -1.3$ for $10^{16.5} \cm{-2} \le \mnhi \le 10^{19.5} \cm{-2}$
and $\beta = -2$ for $\mnhi \ge 10^{19.5}$, $z_* = 3.7$,
$\gamma = 5.1$ and $C=1.244 \sci{5}$.
From the mock absorber list we identified all LLS with $\mnhi \ge 10^{17.5} \cm{-2}$.
This formed the control sample.
Then, we blended together all systems with $|\delta z| \le \delta z_j$
where $\delta z_j = [0.01, 0.05, 0.1, 0.2]$ and reidentified systems
satisfying $\mnhi \ge 10^{17.5} \cm{-2}$.  When blending two or more 
systems together, we set \zabs\ to the maximum of all the lines.
Finally, we then calcualted the incidence of LLS using the survey
path and LLS for each $\delta z_j$.

Figure~\ref{fig:bbias}a presents the results of the blending bias in
terms of the enhancement/decrement of the incidence of $\mtll \ge 2$ LLS
relative to the perfect model ($\delta z_j = 0$).
For $\delta z_j < 0.1$, there is only a small and ignoreable effect.
For $\delta z_j \ge 0.1$, however, we predict a systematic underestimate
for the incidence of intervening LLS, especially at $z>4$ where
the absolute incidence is highest.  
This deficit in \lzlls\ runs contrary to expectation and is entirely
due to the redshift bias where the blended LLS end up with \zlls\
within the proximate region.  
In turn, we predict a systematic over-estimate of \lzlls\ 
for PLLS (Figure~\ref{fig:bbias}b).
[JMO: Consider whether this affects any previous work]

Our experiments with mock spectra indicate that we commonly blend
together systems with $\delta z_j = 0.05$ to 0.1, with a weak dependence
on redshift or S/N.  
We conclude, therefore, that the effects of the blending bias
on our SDSS analysis are $< 10\%$ for measurements of \lzlls\
for intervening LLS.
This is below the current level of the statistical
error and we will ignore it in the presentation of the results and
discussion.  For the PLLS, the effect for $\delta z_j = 0.1$ 
ranges from $20-90\%$ and we will report our results on \lzlls\ 
for these absorbers as upper limits, especially for $\mzabs > 4$.

Before closing this section, we stress that the blending bias
affects all previous and future LLS surveys.  In particular,
we caution that the incidence of LLS cannot trivially be
used to constrain the \ion{H}{1} frequency distribution \fnhi,
because the latter assumes that every absorption system is
a unique, identifiable line.  For the LLS in particular (but
this is also true for the \lya\ forest), blending smears these
lines over non-negligible redshift intervals ($\delta z \approx 0.1$)
and this affect must be considered when comparing against 
theoretical line densities.
{\bf [Does this affect our f(N) estimates!!]}




%%%%%%%%%%%%%%%%%%%%%%%%%%%%%%%%%%%%%%%%%%%%%%%%%%%%%%%%%%%%%%%%%%%%
%%%%%%%%%%%%%%%%%%%%%%%%%%%%%%%%%%%%%%%%%%%%%%%%%%%%%%%%%%%%%%%%%%%%
\section{Results}
\label{sec:results}

In this section, we present the results of our survey.  We
defer extended discussion of previos work and the implications
of our analysis to the following section.

%%%%%%%%%%%%%%%%%%%%%%%%%%%%%%
\subsection{\lzlls: The Incidence of 
Intervening $\mtll \ge 2$ LLS per Redshift Interval}
\label{sec:loz}

The incidence of LLS per redshift interval \lzlls\ 
is a purely observational quantity, i.e.\ its value 
is independent of any assumed cosmology.
Consequently, \lzlls\ has limited physical meaning yet it 
it is the proper starting point for a discussion
of our observational survey. 

Following standard
practice, we estimate \lzlls\ from the 
ratio of the number of LLS $(N_{\rm LLS})$
detected in a redshift interval to the total search path
for that redshift interval (\ztot):

\begin{equation}
\mlzlls = \frac{N_{\rm LLS}}{\mztot}  \perd
\label{eqn:loz}
\end{equation}
The error for \lzlls\ from this estimator is assumed to be dominated
by the Poission uncertainty in $N_{\rm LLS}$.  Figure~\ref{fig:loz_lls}
presents the estimates of \lzlls\ for the \sna=2 criterion in
a set of arbitrary redshift intervals chosen to give roughly 
constant $N_{\rm LLS}$ per bin (at least for $z<4$).
Table~\ref{tab:llssumm} lists these values for \sna\ thresholds
of 2 and 3, and we note no dependence with this threshold.

Figure~\ref{fig:loz_lls} reveals that the incidence of $\mtll \ge 2$ LLS
increases monotonically for $z>3.5$.
Following previous work, we have modeled the redshift
evolution in \lzlls\ as a power-law with the functional form:

\begin{equation}
\mlzlls = C_{\rm LLS} \ltk \frac{1+z}{1+z_*} \rtk^{\gamma_{\rm LLS}}
\label{eqn:powlaw}
\end{equation}
setting $z_* \equiv 3.7$.
Using standard maximum likelihood techniques
\citep[e.g.][]{sim96}, we find best-fit values 
to the data at $z \ge 3.5$ of
%These are grabbed from running fig_loz
$C_{\rm LLS} = \clls$ and 
$\mglls = \alls$ (68\%\ c.l.).  The best-fit model
is overplotted on the data in Figure~\ref{fig:loz_lls}.
The relatively large uncertainty in $\gamma_{\rm LLS}$
is due to the small redshift interval covered by our survey.
Nevertheless, we conclude at high confidence ($>95\%$)
that \lzlls\ is increasing at least as steeply as $\mglls = 2$
at $z>3.5$.

For $z <3.5$, Figure~\ref{fig:loz_lls} shows two evaluations
of \lzlls.  The light, solid points show the \lzlls\ 
values derived from our statistical quasar sample with
the restriction that $\mzem \ge 3.6$.  
These values are consistent with an extrapolation
of the best-fit power-law. 
The dotted points in the figure, meanwhile,
show the values of \lzlls\ when one  also surveys
quasars with $3.4 \le \mzem \le 3.6$.  
In this case, we find systematically higher \lzlls\
values which would indicate a non-physical, non-monotonic 
evolution in \lzlls.  These results confirm the findings
of PWO09 that the SDSS targeting criteria for quasar 
spectroscopy biases the sample against sightlines without LLS.
This leads to a systematically high and erroneous \lzlls.

The sub-panel of Figure~\ref{fig:loz_lls} compares the cumulative number of LLS
detected in the survey against redshift both as observed (solid)
and as predicted (dotted) by the best-fit power-law model.  For the
latter, we adopt the $g(z)$ curves for \sna=2 from Figure~\ref{fig:goz}.
A one-sided Kolmogorov-Smirnov test yields a probability $P_{\rm KS} = 0.95$
that the observed distribution is drawn from the adopted power-law
expression.  We conclude that the power-law model is a good
description of the observations.
%We return to a discussion of the relatively large,
%albeit poorly constrained, exponent in $\S$~\ref{sec:discuss}.  

%%%%%%%%%%%%%%%%%%%%%%%%%%%%%%%%%%
\subsection{The Incidence of LLS in $\Lambda$CDM}
\label{sec:lox}

If one introduces a cosmological model, the
observed incidence of LLS with redshift \lzlls\ may be translated
into physical
quantities.  Consider first the average distance a photon travels
before encountering an LLS with $\mtll \ge 2$, \drlls.  
Specifically, we define

\begin{equation}
\mdrlls \equiv \mlzlls^{-1} \frac{dr}{dz} \cmma
\end{equation}
where 
\begin{equation}
\frac{dr}{dz}  = \frac{c}{(1+z) H(z)}
\end{equation}
where
\begin{equation}
H(z) = H_0 \ltk \Omega_\Lambda + (1+z)^3 \Omega_m \rtk^{1/2} \perd
\end{equation}
Assuming $\Omega_\Lambda = 0.7, \Omega_m = 0.3$, and  $H_0=72 \mkms \rm Mpc^{-1}$
\citep[e.g.][]{wmap05},
we find \drlls\ ranges from $\approx 100$ to 40\,Mpc proper distance
from $z=3.5$ to $z=4.4$ (Table~\ref{tab:summ}).
This is substantially larger than the seperation of bright quasars 
\citep[several Mpc 
for $L_B \ge 10^{40} \rm \, erg \, s^{-1} \, Hz^{-1}$;][]{fg08b} 
and therefore a simple restatement that 
the universe was reionized before $z=4$.


Another approach toward exploring the origin of absorption systems, 
i.e.\ associating
these systems to structures in the Universe (e.g.\ galaxies, filaments),
is to compute
the number of systems per absorption length \llls \citep[][]{bp69},
where $\ell(X) dX = \ell(z) dz$ and 

\begin{equation}
dX = \frac{H_0}{H(z)}(1+z)^2 dz  \perd
\end{equation}
The quantity \llls\ is defined to remain constant if \nlls\ the
comoving number density
of structures giving rise to $\mtll \ge 2$ LLS
times \slls\ the average physical size of the structure 
remains constant, i.e.\ $\mllls \propto \mnlls \mslls$.

Figure~\ref{fig:complox} presents the evolution of \llls\
for our cosmology as a function of redshift (see also
Table~\ref{tab:llssumm}).  
We observe a rise in \llls\ with redshift of roughly two times
over the $\approx 1\,$Gyr from $z=3.4$ to 4.4.
At 99\% confidence, we infer an increase in \llls\ over
this redshift interval.
This follows, of course, from the very steep
redshift evolution observed for \lzlls\ ($\S$~\ref{sec:loz});
in the $\Lambda$CDM cosmology, an \lzlls\ evolution steeper
than $(1+z)^{1/2}$ at $z>3$ implies \llls\ is also increasing.
We conclude
that \nlls\ and/or \slls\ are increasing with redshift
at $z \approx 3.5$.  

%%%%%%%%%%%%%%%%%%%%%%%%%%%%%%%%%%%%%%%%%%%%%%%%%%%%%%%
\subsection{\lfnhi\ at $z \approx 3.7$}

Although our observations and LLS analysis are insensitive
to the \ion{H}{1} column densities of the LLS, 
they do provide an integral
constraint on the frequency distribution of \nhi\ values
when combined with other surveys.
Specifically, we can constrain the
\nhi\ frequency distribution of LLS per absorption length
\lfnhi\ at column densities
$\mnhi = 10^{17.5} - 10^{19} \cm{-2}$ as follows.
Previous surveys of $z>3$ DLAs and SLLSs 
have measured \fnhi\ at $z>3$ covering column densities 
$\mnhi \ge 10^{19} \cm{-2}$ \citep{phw05,opb+07,pw09}.
These authors have parameterized the
distribution functions as single (SLLSs) and double (DLAs) power-laws
of the following form:

\begin{equation}
\msfnhi = C_{\rm SLLS} \mnhi^{\beta_{\rm SLLS}} 
\label{eqn:fslls}
\end{equation}

and

\begin{equation}
\mdfnhi = C_{\rm DLA} \ltp \frac{N}{N_d} \rtp^{\beta_{\rm DLA}}
  \; {\rm where} \; \beta_{\rm DLA} =  
\begin{cases}
\beta_3:  N < N_d ; \quad \\ 
\beta_4:  N \geq N_d \\
\end{cases}
\label{eqn:fdla}
\end{equation}
In Figure~\ref{fig:fn}, we present these frequency distributions
as bands which represent the approximate $1 \sigma$ uncertainties.
For the SLLS we have taken $\beta_{\rm SLLS} = -1.3 \pm 0.2$ and
$C_{\rm SLLS} = 0.25 \pm 0.05$.
These values are in reasonable agreement with other estimates
\citep{petitjean}.
For the DLAs, we have evaluated \fnhi\ from the SDSS-DR5 \citep{pw09}
over the redshift interval $z=[3.4,4.0]$, giving 
$N_d = ?$, $\beta_3 = ?$ and $C_{\rm DLA} = ?$ {\bf [FILL IN]}.
Together, these evaluations give \fnhi\ for $\mnhi> 10^{19} \cm{-2}$
at $z \approx 3.7$.

We can estimate \fnhi\ for the interval $\mnhi = [10^{17.5}, 10^{19}]\cm{-2}$,
which we will refer to as \lfnhi,
under the following assumptions:
(i) \lfnhi\ monotonically increases with decreasing \nhi\ 
and
(ii) \lfnhi\ has a power-law form 

\begin{equation}
\mlfnhi = C_{\rm LLS} \mnhi^{\beta_{\rm LLS}}.
\end{equation}
We then impose the integral constraint that 
the incidence of LLS with $\mtll \ge 2$ from our survey equals:

\begin{equation}
\ell_{\rm LLS} (X) = \intl_{10^{17.5} \cm{-2}}^\infty \mfnhi \, d\mnhi \perd
\end{equation}
At $z \approx 3.7$, we estimate $\mlzlls = 0.5 \pm 0.1$ (Figure~\ref{fig:complox})
and find $\mbtlls = \blls$ and $C_{\rm LLS} = \clls$. 

A plot of \fnhi\ for $z \approx 3.7$ and $\mnhi \ge 10^{17.5} \cm{-2}$
is shown in Figure~\ref{fig:fn}.
Our analysis reveals that \fnhi\ becomes increasingly shallow
with decreasing \nhi. This was suggested by previous authors
based on similar analysis but with much poorer observational
constraints on \llls\ \citep{phw05,opb+07}.
Our results actually indicate that $\mbtlls > -1$ (95$\%$ c.l.)
which means that the universe has a {\it higher} cross-section to 
sightlines with $\mnhi = 10^{19} \cm{-2}$ than those 
with $\mnhi = 10^{18} \cm{-2}$.

We can extend the analysis of \fnhi\ one step further by adopting
recent results on the mean free path to ionizing radiation
(\lmfp; PWO09).  At $z=3.7$, PWO09 estimate 
$\mlmfp = 47 h^{-1}_{72} \, \rm Mpc$ proper distance.
This implies that the effective optical depth \teff\ is unity 
for a photon travelling a redshift path
$\delta z = \mlmfp (dz/dr) = 0.30$ (i.e., \teff=1 at $z=3.5$ for photons
from a quasar with $\mzem = 3.8$).
The evaluation of \teff\ follows from


\begin{equation}
\mteff(z_{912},\mzq) = \intl_{z_{912}}^{z_q} \intl_{\mnmin}^\infty f(\mnhi,z')
   \lbrace 1 - \exp \ltk - \mnhi \sigma_{\rm ph}(z') \rtk \rbrace d\mnhi dz' 
\label{eqn:teff}
\end{equation}
where $\sigma_{\rm ph}$ is the photoionization cross-section evaluated
at the photon frequency and the results are insensitive to the
minimum \nhi\ column density for any value 
$\mnmin \le 10^{12} \cm{-2}$.

Figure~\ref{fig:mfp} shows the \lmfp\ value at $z=3.7$ from PWO09
as a solid horizontal band, assuming a 15\%\ error. 
The solid curve, meanwhile, corresponds to the evaluation of
Equation~\ref{eqn:teff} using our estimation of \fnhi\ integrated
from $z=3.5$ to $z=3.8$ as a cumulative function of \nmin.
At the limiting \nhi\ value of our survey ($10^{17.5} \cm{-2}$), 
we estimate that 70\%\ of the opacity to ionizing radiation is
contributed by $\mtll \ge 2$ LLS.  We estimate a 20\%\
error in this evaluation (overplotted in the Figure).

It is evident from Figure~\ref{fig:mfp} that systems with 
$\mtll \le 2$ must contribute to \lmfp.
For $\mnmin < 10^{17.5} \cm{-2}$ we continue the calculation
by assuming that \fnhi\ follows a power-law


\begin{equation}
f_{\rm pLLS}(\mnhi < 10^{17.5} \cm{-2}, X) = 
  C_{\rm pLLS} \mnhi^{\beta_{\rm pLLS}} 
\label{eqn:fplls}
\end{equation}
constrained to match \lfnhi\ at $\mnhi = 10^{17.5} \cm{-2}$.
We find that models with $\mbplls \ge -1$ cannot reproduce
the \lmfp\ results.  In fact, the data favor $\mbplls \le -1.5$,
i.e.\ a steeper power-law than inferred for the LLS and than
that commonly observed for the \lya\ forest.  These conclusions
do rely on our
estimations of \llls; slopes as shallow as $-1.5$ are possible
if one ignores constraints from the \lya\ forest.


In Figure~\ref{fig:splinefN}, we make a first attempt at deriving
the full \fnhi\ function for $\mnhi \ge 10^{12} \cm{-2}$ by integrating
all of the constrains discussed thus far and also including the
\lya\ forest.  
For the latter, we assume $\myfnhi \propto \mnhi^{-1.5}$
and we normalize this power-law at $z=3.7$ by matching to the
effective \lya\ optical depth \lteff\ measured by \cite{fpl+08}
assuming a $b$-value distribution
$f(b) \propto b^{-5} \exp{-b_\sigma^4/b^4}$ \citep{hr99}. 
Absent a physical model for \fnhi\ we take an empirical approach.
We will evaluate \fnhi\ as a cubic spline \citep{press92}
defined by the values of 12 spline points at 
$\log \mnhi = [12.0, 14.0, 16.0, 17.5, 19.0, 20.3, 21.7, 22.5]$.
The first, third, and last of these are arbitrarily defined
while the others are taken to correspond to the following
observational constraints:

\begin{enumerate}
\item We minimize $\chi^2$ for the binned evaluations of \fnhi\
of the DLAs \citep{pw09} at $z=3.5$ over the interval 
$\mnhi = [10^{20.3}, 10^{21.7}] \cm{-2}$, where the latter
point corresponds to the `break' in \fnhi\ at high column
densities \citep{phw05}.  This establishes the $y$-values
of the spline points at $\log \mnhi = 20.3$ and 21.7 (Table~\ref{tab:spline}).
For the spline point at $\log \mnhi = 22.5$, we force
$\mfnhi \propto \mnhi^{-4}$ for $\mnhi \ge 10^{21.7} \cm{-2}$.
This spline point has no effect on the analysis.

\item The spline point at $\log \mnhi = 19$ is constrained to give
$\mlslls = 0.2$ \citep{opb+07}.
\item The spline point at $\log \mnhi = 17.5$ is constrained
to give $\mllls = 0.5$ (this paper);
\item The spline point at $\log \mnhi = 16$, whose x-value is 
the most arbitrary, is constrained to give $\mlmfp = 47 \umfp$ (PWO09);
\item The spline points at $\log \mnhi = 12 \; {\rm and} \; 14$
are constrained to give $\mfnhi \propto \mnhi^{-1.5}$ and to
give a \lya\ opacity $\mtlya =$ {\bf XX} \citep{fpl+08}.
\end{enumerate}
This approach has several advantages over our previous \fnhi\ evaluations
(Figures~\ref{fig:fn}, \ref{fig:mfp}): 
(i) introducing the \lya\ forest
places a constraint at low \nhi\ values which implies more of an
interpolative analysis then extrapolative;
(ii) the resulting function is smooth in the first derivative
and continuous in the second derivative.

Table~\ref{tab:spline} lists the spline point values and
Figure~\ref{fig:splinefN} plots the points and the resulting spline
evaluation for \fnhi.  The slope $\beta \equiv d\ln[\mfnhi]/d\ln\mnhi$
at each of the spline points is also given.
Finally, we also show the \fnhi\ evaluations of the SLLS $(z>3$)
and DLAs ($z=3.5$) from \cite{opb+07} and \cite{pw09}.

We will defer a full discussion of the results to the following
section, but offer the following observations.
First, we again find that \fnhi\ is very shallow in the LLS
region with $\beta > -1$ at $\mnhi = 10^{19} \cm{-2}$.
Second, we observed that \fnhi\ becomes steeper than
$\beta = -1.5$ (perhaps steeper than $\beta = -2$) for
$\mnhi \approx 10^{16} \cm{-2}$.  This was first
suggested by \cite{petit93} and subsequently argued for
by \cite{rauch} and \cite{misawa}?

[Comment on the uncertainties in all of this. Juggle the values
about, e.g. bootstrap.  Try the shallower Petitjean slope for the SLLS
if the dN/dX is much different]





\subsection{The Incidence of Proximate LLS}


The results presented thus far all refer to intervening LLS, systems
restricted to have \zlls\ blueward of 3000\kms\ from the quasar
emission redshift.  This restriction was imposed to isolate the
`ambient' IGM and avoid biases related to having performed
the analysis with bright, background quasars.
In the space surrounding a bright quasar, one predicts at least two
such biases:
(1) bright, high $z$ quasars are known to cluster strongly
\citep[$r_0 \approx ? \, h^{-1} \rm Mpc$;][]{shen07}
suggesting these objects trace massive structures in the young universe.
The local environment of bright quasars, therefore, has an uncommonly
high density (at least in dark matter) which may give a higher
incidence of LLS;
(2) the radiation field of the quasar will ionize gas to large
distances, reducing the incidence of LLS.
For damped \lya\ systems, the first effect dominates
as one observes an enhanced rate of proximate DLAs (PDLAs)
relative to the intervening systems \citep{reb06,phh08}.

Following the formalism presented in \cite{phh08}, we have estimated
the incidence of proximate LLS (PLLS) in a series of redshift intervals.
First, we re-measured the quasar emission redshifts for all
systems with $\msna \ge 2$ at the Lyman limit and with 
a $\mtll \ge 2$ LLS within 5000\kms\ of \zem.
{\bf [Do this!]}
The SDSS quasar redshifts reported in the standard DR7 data release
are known to have significant systematic errors. 
Following the prescriptions described in \cite{shen07},
J. Hennawi has kindly remeasured the redshifts for all of the quasars.
Second, we analyzed the quasars whose absorbed continuum at the wavelength
of the Lyman limit corresponding (3000\kms\ blueward of \zem) is
twice the median-smoothed, $1\sigma$ error-array.  This establishes
the survey path.  All PLLS identified redward of this 3000\kms\ offset
form the statistical sample (Table~\ref{tab:PLLS}).
The incidence, \lplls, is then estimated in arbitrary redshift
intervals assuming the same estimator for intervening 
LLS (Equation~\ref{eqn:loz}).
These results are presented in Figure~\ref{fig:prox} and compared
against the incidence of interevening LLS, \lzlls.


Ignoring the data at $z < 3.6$ (which we suspect to be biased high
by the SDSS targeting criteria; PWO09), the
incidence of PLLS roughly tracks that of intervening LLS.
At $z<4$, however, the PLLS show an $\approx 25\%$ lower
incidence than the intervening systems.  The inset figure
shows the observed cumulative number of PLLS (dark curve) versus
the predicted number (light curve) assuming the best-fit power-law
for \lzlls\ of intervening LLS.
A one-sided KS test yields only a 1\%\ probability that the
two distributions are drawn from the same parent population.
The principal implication is that $\mtll \ge 2$ LLS toward
$i<20$\,mag quasars at $z>3.5$ suffer from a proximity effect
of lower incidence presumably due to the ionizing radiation
field of the quasar itself.
Given the observed enhancement of strong LLS at $z \approx \mzem$
along sightlines {\it transverse} to such quasars
\citep{hpb+06}, our results lend further evidence that quasar emission
is anisotropic \citep{hp07}.

Before concluding this section, we comment that the PLLS
analysis is subject to a potentially large systematic error.
In performing our LLS survey, we have identified and removed
all quasars with very strong associated absorption,
e.g.\ BALs.  The intent of this procedure was to remove the
signatures of absorption from gas very local 
($<1$\,kpc) to the quasar from the analysis.  It is possible, however,
that the associated absorption in some of these removed
quasars is due to a PLLS at distances $\gg 1$\,kpc and not very
local gas.
This would lead to an underestimate of \lplls.
Alternatively, we may not have identified all of the
local absorbers and therefore might have overestiamted \lplls.
In either case, we caution that a systematic error of the
order of $10-20\%$ should be attributed to this effect.


%%%%%%%%%%%%%%%%%%%%%%%%%%%%%%%%%%%%%%%%%%%%%%%%%%%%%%%%%%%%%%%%%%%%%%%

\section{Discussion}
\label{sec:discuss}

\begin{itemize}
\item  Compare to previous work; include Misawa
\item  Stack spectrum:  Compare $\tilde\mtll$
\item  What gives rise to LLS?  Why is it only a 9x higher incidence than DLAs?
  What is the connection to $z=0$?
[What is dN/dX today?  How about in simulations?  Is the radius to
17.6 only 3x larger?]
\item  Taueff
\end{itemize}

With an estimation of \lfnhi\ in hand, we can draw several inferences
about the opacity of the Universe to ionizing radiation.  First,
we can estimate the mean opacity at 1\,Ryd for all LLS having $\mtll \ge 2$
by taking a weighted mean,

%NOTE:  See fig_avgtaueff [actually fig_fn] for the exact f(N_HI) actually used

\subsection{Comparisons with Previous Work}

Surveys for Lyman limit systems have been carried out for several
decades now \citep{tytler84,ssb89,lzt93,sl95,sl96,peroux03}.
These have been  performed primarily at optical wavelengths on 
heterogeneous quasar samples drawn from a diverse set of
survey approaches: color-selection, radio detection, slitless
spectroscopy, etc.  The authors also had differing completeness
limits for \tll\ (ranging from 1 to 3) and differing approaches
to establishing the pathlength that establishes \lzlls.  Almost
no attention was given to assessing systematic error, and several
of the effects described in $\S$~\ref{sec:mock} assuredly apply
to the previous works.  It comes with little surprise, therefore,
that many of the previous estiamtes of \lzlls\ are in disagreement
with our results.

In Figure~\ref{fig:literature} we present estimates of \lzlls\
from several previous studies, parameterized as in Equation~\ref{eqn:powlaw}
(Table~\ref{tab:param} lists the parameters).  
Tge dotted lines show estimates from \cite{ssb89} and \cite{lzt91},
which included very few observations at $z>3.5$ and not surprisingly 
had rather discrepant estimates for the high $z$ universe.
The solid curves show the results from \cite{sl94} and \cite{peroux03}
who performed surveys of LLS at $z \sim 4$ using the color-selected
APM quasars.  All of these analyses were claimed to correspond to 
LLS with $\mtll \ge 1$, although a careful review of the literature
raises doubts to this assertion in several cases.  To compare against
our $\mtll \ge 2$ results, therefore, we have boosted each of
our \lzlls\ estimations by $8\%$ which is the correction factor
imlied by the \fnhi\ function derived at $z = 3.7$ (Figure~\ref{fig:fnhi}).

Our results on \lzlls\ are somewhat lower than the values derived 
from the APM surveys, but the values lie within
1.5$\sigma$ of concordance.  The more important difference lies between
the estimations at $z<4$.  All of the previous work, which relied
almost entirely on the surveys of \cite{ssb89} and \cite{lzt93}, 
[assumed] $\mzlls \approx 2$ at $z = 3$ giving a more shallow
evolution for \lzlls\ than observed in the SDSS data.
Our results suggest that the previous work at $z \sim 3$ overestimated
the incidence of LLS.  Actually, the original survey by \cite{ssb89}
is in fair agreement with an extrapolation of our power-law form
for \lzlls\ to $z=3$, but the reanalysis by \cite{sl95} of still
unpublished spectray by Steidel \& Sargent led to a higher estimate
at this redshift.  We suspect that this later work gave \lzlls\ values
that were too high either because of sample variance (i.e.\ small
number statistics) or selection bias in the quasar sample 
\citep[e.g.\ the color-selection bias 
described in][and $\S$~\ref{sec:loz}]{pwo09}.
On this point, we note that if we adopt the \cite{peroux03} estimation
of $\mlzlls = 3$ at $z=3.7$, then the LLS alone predict a mean free path
to ionizing radiation $\mlmfp = 45 \umfp$ that matches the measured
value \citep{pwo09}.  This would mean that gas with $\mtll < 1$
cannot contribute any to \lmfp, a physically impossible scenario.
[It gets worse when you realize this gives a steep LLS f(N) which *must*
monotonically hook onto the LLS.]

[Does the MFP evolution require strong LLS evolution?]

[Calculate MFP from Peroux LLS stats alone]


\subsection{Redshift Evolution}

The redshit evolution in the incidence of LLS \lzlls\ impacts
the evolution of the EUVB at low $z$ and the history of reionization
at high $z$.  Previous work has debated whether \lzlls evolves
as a singe power-law $(1+z)^\mglls$ from $z \approx 0 - 4$.
In $\S$~\ref{sec:loz}, we modeled our observations with a 
single power-law having $\mglls = \alls$.
As noted in the previous sub-section, measurements of \lmfp\ also
suggest a steep evolution in the incidence of LLS at $z<4$. 
We now consider whether a single power-law extrapolation is
a good description of \lzlls\ for $z < 3$.

Survey of other \ion{H}{1} absorption systems have demonstrated
that a single $(1+z)^\gamma$ power-law is a poor description
of the \lya\ forest \citep{weymann} and the damped \lya\ systems
\citep{phh08}.  In the former case, one observes a flattening
in the \lya\ line-density at $z \sim 1$ which has been interpreted
to result from a corresponding decline in the intensity of 
the EUVB \citep{weymann,dave}.  It is plausible that a similar
effect would influence the LLS.

In Figure~\ref{fig:zloz}, we present our estimates of \lzlls\
at $z \sim 3.5$ and the estimate of \lzlls\ at $z \sim 1$
from \cite[][see Howk et al., in prep for a new estimate]{sl95}.
Overplotted on the data is a solid curve that shows the best-fit
power law to the SDSS results.  The dashed curves show $2\sigma$
departures from this model.  It is evident that none of these
curves intersect the low redshift observation.
[Did I remember to boost the \lzlls\ values for $\mtll 2 \to 1$?]
We conclude, with high confidenc, that a strict $(1+z)^\mglls$
power-law does not describe the evolution of \lzlls\ across
all of cosmic time.   [Plot log (1+z) on the x-axis?]
Instead, the data suggest a break in the high $z$ power-law
at $z \sim 2$, similar to that observed for the \lya\ forest
albeit at somewhat higher redshift.  

The dash-dot curve, is an attempt to model this break.  
For $z<2.3$, we adopt the power-law form that matches
the \lya\ forest at low redshift ($\mglls = 0.26$) and
demand that it intersect the central value of the \cite{sl95}
measurement.  For $z\ge 2.3$, the model breaks to a $\mglls = 2.78$
power-law, again consistent with the high $z$ evolution of the
\lya\ forest \citep[e.g.][]{kim02}.  This is a reasonably
good description of the data and we conclude that a break
in the power-law description of \lzlls\ likely occurs at $z \approx 2$.
We will test this prediction with an (ongoing) survey for LLS
at $z \sim 2$ in HST/ACS and WFC3 slitless spectra of 
$z \approx 2.3$ quasars (PI: O'Meara).

\subsection{The Average LLS Spectrum}

To gain additional insight into the absorption properties of the
LLS, we have constructed an average (stacked) spectrum 
by (i) shifting each quasar spectrum containing a $\mtll \ge 2$ LLS
to the LLS rest-frame and (ii) straight averaging the fluxed data.
This stacked spectrum is primarily illustrative and has little 
quantitative value.  The resulting spectrum is shown in 
Figure~\ref{fig:stack}.  The peak in emission at 
$\lambda_{\rm r} \approx 1280$\AA\ is from the \lya\ emission
peak of the background quasar.  It is significantly
offset from 1215\AA\ because the stack only includes intervening
LLS, i.e.\ those that are offset by at least 3000 \kms from the 
quasar emission redshift.  

The second strongest feature in the
spectrum is the Lyman limit at the expected wavelength of 912\AA.
Shortward of the Lyman limit, one observes a non-zero flux that
extends down to $\approx 770$\AA.  In these relative
flux units, we measure a flux at 900\AA\ of $f({\rm 900 \AA}) = 24$.  
Estimating the absorbed continuum at 960\AA\ to be 
$f({\rm 980 \AA}) = 810$, we calculate an average optical depth
at the Lyman limit of $\mavtll = 3.5$.  This value may be compared
to the average optical depth dervied from our \fnhi\ estimation:

\begin{equation}
\mavtll = - \ln \ltb
\frac{\int \exp[-\mnhi \sigma_{LL}] \mfnhi \, d\mnhi}{\int \mfnhi \, d\mnhi}
\rtb
\label{eqn:avgtau}
\end{equation}
where the integrals are evaluated over the interval
$\mnhi = [10^{17.5},10^{22}] \cm{-2}$.
We find $\mavtll \approx 5$ for our assumed \fnhi\ distribution function.
We compare this directly against the mean spectrum of our LLS sample in 
$\S$~\ref{sec:XX}.



\subsection{The Physical Nature of the LLS}

Viewing the situation in the opposite sense (from high $z$ to low $z$),
we speculate that the decrease in \llls\ results from an increasing
number of ionizing sources that photoevaporate the structures associated
with LLS.   
Indeed, the number density of 
bright quasars is observed to increase from $z=4$ to 3 \citep{qsoevol}.
We postulate that these sources are 
reducing \nlls\ and/or \slls.

To explore this point further,  we
decompose \llls\ into various absorption classes (each with $\mtll \ge 2$)
using constraints from the literature.
In Figure~\ref{fig:complox}, we present the \llls\ values (in cumulative
form) for DLAs \citep{pw09} and super Lyman limit systems 
\citep[SLLS, absorbers with $\mnhi = 10^{19} - 10^{20.3} \cm{-2}$;][]{opb+07}.
For the latter, we assume $\ell_{\rm SLLS}(X) = 0.20$ at all redshifts because
only a single estimate exists at $z>3$. 
We have adopted a 20$\%$ lower (1$\sigma$) value than reported
by \cite{opb+07} to crudely correct for the SDSS quasar targeting bias
that will affect their measurement (PWO09). 
The figure demonstrates that the DLAs (especially) and 
the SLLSs have modest contributions to \llls.
At $z=3.4$, they contribute roughly half of the observed incidence
of $\mtll \ge 2$ LLS but this fraction likely decreases to
$\approx 33\%$ by $z=4$.  This latter conclusion hinges on our asssumptions for 
\lslls, in particular at $z \approx 4$ where the value is not
well constrained.  But, we expect the SLLS to behave similarly
to the DLAs, whose incidence is not increasing significantly at these
redshifts.  We conclude that the observed evolution in \llls\
is driven instead by systems with 
$\mnhi = 10^{17.5} - 10^{19} \cm{-2}$.
By the same token, we infer that the decrease in \llls\ with 
decreasing redshift is related to the photoionization 
of these lower \nhi\ systems by the increasing number
of ionizing sources.  The systems giving rise to DLAs (i.e. galaxies)
are too optically thick to show significant effects 
from an increasing radiation field.
[THIS STUFF MIGHT BE BETTER LEFT TO THE DISCUSSION SECTION]

It is difficult, if not impossible, to produce such an effect
for simple cloud geometries in the influence of an ionizing
radiation field \citep{zm0?}.
We spectulate instead that the LLS with $\mnhi \sim 10^{19} \cm{-2}$
arise in extended, partially collapsed structres, e.g. filaments.
We further hypothesize that the flattening of \fnhi\ may be
associated with the transition from
the optically thick regime to being optically thin.
Finally, we note that the rise in \llls\ with redshift
(Figure~\ref{fig:complox}) implies that \btlls\ likely decreases
(becomes steeper) with increasing redshift.



\section{Summary}

\acknowledgments

Funding for the SDSS and SDSS-II has been provided by the Alfred P. Sloan 
Foundation, the Participating Institutions, the National Science Foundation, 
the U.S. Department of Energy, the National Aeronautics and Space 
Administration, the Japanese Monbukagakusho, the Max Planck Society, 
and the Higher Education Funding Council for England. The SDSS Web Site 
is http://www.sdss.org/.

The SDSS is managed by the Astrophysical Research Consortium for the Participating Institutions. The Participating Institutions are the American Museum of Natural History, Astrophysical Institute Potsdam, University of Basel, University of Cambridge, Case Western Reserve University, University of Chicago, Drexel University, Fermilab, the Institute for Advanced Study, the Japan Participation Group, Johns Hopkins University, the Joint Institute for Nuclear Astrophysics, the Kavli Institute for Particle Astrophysics and Cosmology, the Korean Scientist Group, the Chinese Academy of Sciences (LAMOST), Los Alamos National Laboratory, the Max-Planck-Institute for Astronomy (MPIA), the Max-Planck-Institute for Astrophysics (MPA), New Mexico State University, Ohio State University, University of Pittsburgh, University of Portsmouth, Princeton University, the United States Naval Observatory, and the University of Washington.

J. X. P. and J.M.O. are supported by NASA grant
HST-GO-10878.05-A.  J.X.P and G.W. are partially supported
by an NSF CAREER grant (AST--0548180), and 
by NSF grant AST-0908910.

\appendix

\section{Comparisons with Keck+LRIS Spectra}
\label{appx:keck}

To assess uncertainties (statistical and systematic)
of surveying LLS in the SDSS quasar spectra, we
obtained independent, higher quality spectra using the LRIS spectrometer
\citep{lris} on the Keck I telescope.  
LRIS employs a dichroic to split the data into two
spectral channels, each with its own camera.
For our observations, we employed the d560 dichroic which splits the 
light at $\approx 5600$\AA.
For the blue channel, we used the 640/4000 grism
which provides a dispersion of 0.63 \AA\ per unbinned pixel and has a nominal
wavelength coverage of $3100 \rAA < \lambda < 5600$ \AA.  The blue channel
data was binned by 2 in both the spatial and spectral dimensions. For the red
channel, we used the 600/7500 grating which provides a dispersion of
1.28 \AA\ per unbinned pixel, and which was tilted to provide a wavelength
coverage of $5600 \rAA < \lambda < 8200$\AA.  The red channel data was unbinned.
All observations were obtained using a 1 arcsecond slit which provides
an $\approx 4$\,pixel FWHM corresponding to 
$\approx 290\mkms$ and $\approx 220 \mkms$ for the blue and red
data respectively.
The data were obtained in good sky conditions during a 4 night
run in October~2008 and had
exposure times ranging from 300 to 500 seconds.
The data were reduced using the LowRedux
pipeline\footnote{http://www.ucolick.org/$\sim$xavier/LowRedux/} which
bias subtracts, flat fields, optimally extracts, wavelength and flux
calibrates the data to produce a final 1D spectrum.  

The SDSS targets for LRIS observations were chosen to sample a range
of LLS, e.g., LLS with $\mtll \ge 2$, pLLS
candidates, PLLS and spectra without apparent LLS.  Furthermore, 
an emphaisis was placed on quasars with lower S/N SDSS spectra 
to assess the completeness of recovering LLS. 
In all cases, the LRIS spectra have sufficient S/N 
to unambigously detect the presence of absorbers with $\mtll > 1$ 
over the full SDSS wavelength range (i.e.\ $\lambda > 3800$ \AA) for
the intervening LLS survey. 
Particular emphasis was given 
to determine what absorbed continuum S/N cutoff should be applied to
the SDSS sample.  To this end, two of the authors (JXP and JMO)
independently modeled LLS absorption in the LRIS data and 
compared the results to similar analysis of the 
the SDSS spectra.  The same
codes were used to model LLS absorption.  In Table
\ref{tab:lriscomp}, we present the results of these comparisons.  In
Figure \ref{fig:lris} we show a representative sample of the
LRIS data alongside their SDSS counterparts.  For the SDSS data in
Figure \ref{fig:lris}, we also show the continuum level assigned
to each spectrum.  

The LRIS/SDSS comparison illustrates that with a choice of absorbed
continuum S/N $>1$ 
we recover nearly 100\%\ of the LLS with  $\mtll > 2$ in the SDSS.  We see this
explicitly in Table \ref{tab:lriscomp}, where we give the values for
\zstrt\ in the SDSS search, where \zstrt\ is the redshift at which
the absorbed continuum S/N crosses the value of 1.  In some cases, we
identify LLS at redshifts lower than \zstrt\ in both the LRIS and
SDSS data.  Although these LLS will not contribute to our results,
they lend additional confidence in our absorbed continuum S/N cutoff.
The only exceptions to the identification of LLS in the LRIS and 
SDSS spectra at S/N~$\ge 1$ are for systemss with $\mtll \approx 2$.
At this optical depth, JXP and JMO did not always
agree, even in the LRIS results. 
(WE NEED TO CHECK ALL THE TABLE RESULTS FOR SILLY MISTAKES).  
This highlights the fact that our results have an inherent uncertainty
in $\log \mnhi$ of $\approx 0.2$\,dex. 
Most importantly, this
disagreement does not appear to depend on the S/N of the spectrum for
the range of S/N we could expect from the SDSS data, and thus does not
effect our choice for the S/N threshold.


\bibliographystyle{/u/xavier/paper/Bibli/apj}
\bibliography{/u/xavier/paper/Bibli/allrefs}

\input{Tables/tab_subqso.tex}
\input{Tables/tab_subsurvey.tex}
\input{Tables/tab_subprox.tex}
\input{Tables/tab_sub_candlls.tex}
\input{Tables/tab_spline.tex}
\input{Tables/tab_summ.tex}
\input{Tables/tab_sdss_lris.tex}
%\input{Tables/tab_lls.tex}
%\input{Tables/tab_lls_nostat.tex}
%\input{Tables/tab_lls_prox.tex}

\begin{figure}
\epsscale{0.8}
\plotone{Figures/fig_qsohist.ps}
\caption{Quasar historgrams
}
\label{fig:snz}
\end{figure}


\begin{figure}
\epsscale{0.9}
\plotone{Figures/fig_templates.ps}
\caption{Mean (median?) observed (not intrinsic)
quasar spectra (detilted?) for 
the full SDSS-DR7 (clipping?) in redshift intervals from
top to bottom of $z=(3.4,3.7); (3.7,4.0); (4.0,4.4)$ and
$(4.4, 5.0)$.    
The increasing impact of the \lya\ forest is readily apparent
at wavelengths $\lambda_{\rm rest} < 1200$\AA.
}
\label{fig:template}
\end{figure}


\begin{figure}
\epsscale{0.8}
\plotone{Figures/fig_exmpl_conti.ps}
\caption{Representative spectra of the quasars in the statistical sample.
Each example shows the absorbed continuum model (dark blue, dotted)
and the green curve indicates the model that includes LL absorption. 
For this presentation, we have not shown the Lyman series in the model.
The vertical dotted line indicates the Lyman limit wavelength
in the quasar rest-frame, i.e.\ $\lambda = \lambda_{LL} (1+z_{em})$.
}
\label{fig:ex_conti}
\end{figure}

\begin{figure}
\epsscale{0.8}
\plotone{Figures/fig_goz.ps}
\caption{Summary of the redshift path surveyed in the SDSS-DR7
for $\mtll \ge 2$ LLS absorption assuming a range of S/N thresholds
(solid curves).  Specifically, $g(z)$ represents the number of 
unique quasars in the SDSS that provide a search for LLS over
the interval $dz$ at redshift $z$.  The dotted curve, meanwhile,
represents the same quantity but for proximate LLS (PLLS; systems
within $\delta v = 3000 \mkms$ of the quasar) and corresponds
to a S/N=2 threshold.  This curves fall rapidly for $z<3.4$
because we limit the analysis to quasars with $\mzem \ge 3.4$.
The non-monotonic nature of $g(z)$ is due primarily to the presence
of LLS along the quasar sightlines which abruptly truncate the search path.
}
\label{fig:goz}
\end{figure}

\begin{figure}
\epsscale{0.7}
\plotone{Figures/mockspec_bw.eps}
\caption{
(a): A spectrum representative of the typical data quality 
(S/N$\sim 2$ at the Lyman limit of the quasar). 
There is a LLS ($z=3.88059, \log\mnhi=18.27$) that can be easily identified 
at this S/N. 
Two $16< \log\mnhi <18.27$ systems at higher $z$ do not yield a 
high \tll\ before the light hits the optically thick system.
(b): A high-S/N spectrum (S/N$\sim 6$ at the Lyman limit of the quasar), 
rendering its partial LLS ($z=3.98387, \log\mnhi=17.24$) 
easily visible. An additional SLLS at lower redshift 
$(z=3.76609, \log\mnhi=19.28)$ sets the flux to zero at the Lyman limit.
(c): A roll-off spectrum. There is a LLS $(z=3.68084, \log\mnhi=17.87)$, 
but there are five $16 < \log \mnhi <17$ systems at higher $z$ that result 
in a roll-off. The S/N at the Lyman limit of the quasar is quite low 
(S/N$\sim 1.5$), the intervening systems further decrease 
the S/N, rendering the LLS invisible.
(d): A spectrum with strongly undersubtracted sky background. The 
first strong system encountered is a LLS 
$(z=3.87495, \log\mnhi=18.34)$, but the flux is above zero even 
after hitting a DLA $(z=3.57678, \log\mnhi=20.43)$. 
You can easily see that the sky subtraction is poor, because 
Lyman alpha of the DLA does not saturate. 
}
\label{fig:mockex}
\end{figure}

\begin{figure}
\epsscale{0.8}
\plotone{Figures/fig_mock_dz.ps}
\caption{Offset in redshift $\delta z$ for the `observed' LLS and pLLS
from the \zabs\ value of the nearest true $\mtll \ge 2$ LLS
in our mock spectra.  The analysis is restricted to the highest
\zabs\ LLS along each sightline.
In the upper panel, we show the $\delta z$ value for sightlines
where a LLS was `observed' and actually exists.  These are the majority
of cases $(>80\%$)
and we find small $\delta z$ values with a small, but important
bias to $\delta z > 0.01$.
In the lower panel, we show false negatives (diamonds) and false
positive (triangles) detections.  The former correspond to a true LLS
that was observed as ($\delta z < 0.1$) or hidden by a pLLS ($\delta z > 0.1$).
The misidentifications ($\delta z < 0.1$) are dominated by the mock
spectra with large underestimates in the sky background.  The dominant
effect of a sky subtraction bias is an underestimate in in the incidence
of LLS.
}
\label{fig:mockdz}
\end{figure}

\begin{figure}
\epsscale{0.8}
\plotone{Figures/fig_both_bbias.ps}
\caption{Blending bias (enhancement/decrement of \lzlls\ relative
to no bias) for mock absorption line systems for 
(upper) intervening $\mtll \ge 2$ LLS and
(lower) proximate $\mtll \ge 2$ LLS.
The absorption line statistics were set to roughly match the observed
incidence. 
Even for the blending over redshifts $\delta z_j = 0.2$, the bias
for intervening systems is relatively small.
In contrast, the blending bias systematically elevates \lzlls\ for PLLS,
especially at $z>4$.
}
\label{fig:bbias}
\end{figure}


\begin{figure}
\epsscale{0.8}
\plotone{Figures/fig_loz.ps}
\caption{Incidence of intervening LLS with $\mtll \ge 2$ \lzlls\
as a function of redshift.  
The binning at $z<4$ is arbitrarily chosen to give roughly the
same number of LLS per sample.  
The solid (blue) line shows the best-fit power-law
to the data at $z > 3.5$:  
$\mlzlls = C_{\rm LLS} [(1+z)/(1+z_*)]^\alpha_{\rm LLS}$, 
with $z_* \equiv 3.7$, $C_{\rm LLS} = \clls$, and
$\beta_{\rm LLS} = \alls$ (68\%\ c.l.).
The gray and solid points at $z < 3.5$ correspond to estimates
of \lzlls\ where one restricts the analysis to quasars
with $\mzem \ge 3.6$.  The dotted points, meanwhile, show
\lzlls\ when one includes quasars with $3.4 \le \mzem \le 3.6$.
These measurements are significantly biased to higher values by the
SDSS quasar-targeting criteria (PWO09).
In the subpanel, 
the black (solid) curve shows the cumulative number of
$\mtll \ge 2$ LLS detected in our survey of the SDSS-DR7
database adopting the \sna=2 threshold.
The blue (dotted) curve shows the predicted number of LLS assuming
the best-fit power-law from Figure~\ref{fig:loz_lls} and adopting the
\sna=2 $g(z)$ function from Figure~\ref{fig:goz}.  
A one-sided KS-test does not rule out the null hypothesis that
the model distribution is statistically different from the observations.
}
\label{fig:loz_lls}
\end{figure}


%\begin{figure}
%\epsscale{0.8}
%\plotone{Figures/fig_cumlz.ps}
%\caption{The black (solid) curve shows the cumulative number of
%$\mtll \ge 2$ LLS detected in our survey of the SDSS-DR7
%database adopting the \sna=2 threshold.
%The blue (dotted) curve shows the predicted number of LLS assuming
%the best-fit power-law from Figure~\ref{fig:loz_lls} and adopting the
%\sna=2 $g(z)$ function from Figure~\ref{fig:goz}.  
%A one-sided KS-test does not rule out the null hypothesis that
%the model distribution is statistically different from the observations.
%}
%\label{fig:cumlz}
%\end{figure}

\begin{figure}
\epsscale{0.8}
\plotone{Figures/fig_complox.ps}
\caption{
The dashed lines bound our estimate of \llls\ for $\mtll \ge 2$ LLS
as a function of redshift. The widths of the
two lower bands (blue, red) indicate estimates of $\ell(X)$ for
DLAs and SLLS \citep{opb+07,pw09}; see the text for details. 
These bands are plotted on top of one another to indicate their total
contribution to \llls\ (as indicated by the vertical arrows on the right-hand
side of the figure).  The (top) blue band, therefore, represents the 
estimate to \llls\ from LLS
with $\mnhi = 10^{17.5} - 10^{19} \cm{-2}$. 
[COMMENT MORE]
}
\label{fig:complox}
\end{figure}

\begin{figure}
\epsscale{0.8}
\plotone{Figures/fig_fn.ps}
\caption{\nhi\ frequency distribution \fnhi\ observed for the 
SLLS \citep[green, $\mnhi = 10^{19} - 10^{20.3} \cm{-2}$;][]{opb+07}
and DLAs \citep[red, $\mnhi \ge 10^{20.3} \cm{-2}$;][]{pw09}.
The blue band is an estiamte of \fnhi\ for LLS having 
$\mnhi = 10^{17.5} - 10^{19} \cm{-2}$ under the assumptions
of a power-law form ($f(\mnhi,X) \propto \mnhi^{\beta_{\rm LLS}}$)
and a monotonic increase from the SLLS, 
and constrained by the incidence of observed $\tau \ge 2$ LLS (this paper).
We find $\mbtlls = -1.0 \pm 0.3$ (90\%\ c.l.) at $z \approx 3.5$.
}
\label{fig:fN}
\end{figure}

\begin{figure}
\epsscale{0.9}
\plotone{Figures/fig_mfp.ps}
\caption{This figure shows the mean free path contributed
by absorbers with $\mnhi \ge \mnmin$. For $\mnmin \ge 10^{17.5} \cm{-2}$,
(solid curve) which corresponds to our LLS survey, we have adopted
the estimate of \fnhi\ from Figure~\ref{fig:fN} in the calculation.
We estimate a 20\%\ uncertainy in the contribution of LLS to
\lmfp, as shown in the figure.  For $\mnmin < 10^{17.5} \cm{-2}$,
we assume \fnhi\ follows a simple power-law with exponent \bplls,
and show a series of extrapolations (dash and dotted curves).
The solid (red) horizontal band centered at $47 \umfp$
indicates the measurement of \lmfp\ at $z \approx 3.7$ from PWO09.
These results imply that LLS contribute $\approx 70\%$ of the 
mean free path and that \bplls\ must be steeper than $-1$
and likely steeper than $-1.5$.
}
\label{fig:mfp}
\end{figure}

\begin{figure}
\epsscale{0.8}
\plotone{Figures/fig_splinefn.ps}
\caption{The solid black curve shows our full estimation of
\fnhi\ as a spline to the 7 black crosses (plus an 8th cross
at $\log \mnhi = 22.5$ that is not shown; Table~\ref{tab:spline}).
At each spline point, we calculate the local power-law slope
$\beta = dlnf/dln\mnhi$.  The y-values of the
spline points are constrained as follows: 
(1) the points at $\log \mnhi = 20.3, 21.7$ are set to minimize
$\chi^2$ from the binned evaluations of \fnhi\ of the DLAs
at $z=3.5$ \citep{pw09}.  The spline point at $\log \mnhi = 22.5$ is then
constrained to give $\mfnhi \propto \mnhi^{-4}$ for
$\log \mnhi \ge 21.7$;
(2) the spline point at $\log \mnhi = 19$ is placed to give
$\mlslls = 0.2$ \citep{opb+07};
(3) the spline point at $\log \mnhi = 17.5$ is constrained
to give $\mllls = 0.5$ (this paper);
(4) the spline point at $\log \mnhi = 16$, whose x-value is 
the most arbitrary, is constrained to give $\mlmfp = 47 \umfp$ (PWO09);
(5) the spline points at $\log \mnhi = 12 \; {\rm and} \; 14$
are constrained to give $\mfnhi \propto \mnhi^{-1.5}$ and to
give a \lya\ opacity $\mtlya = XX$ \citep{fpl+08}.
}
\label{fig:splinefN}
\end{figure}



\begin{figure}
\epsscale{0.8}
\plotone{Figures/fig_prox_loz.ps}
\caption{The solid and dark points show the incidence of
proximate LLS per unit redshift (PLLS; LLS with $\mtll \ge 2$ that 
occur within 3000\kms\ of the background quasar) against redshift.
These are compared against the same quantity for intervening LLS
(gray points).  The data point at $z<3.6$ corresponds to PLLS
and has been dotted out because it is expected to be biased
by the SDSS targetting criteria (PWO09).  Ignoring that last
point, we find that the incidence of PLLS roughly follows that
of intervening systems but is systematically lower by $\approx 25\%$
at $z<4$.  The inset figure shows the cumulative number of 
PLLS observed (black curve) against the predicted number using
the power-law model for \lzlls\ ($\S$~\ref{sec:loz}) and
the $g(z)$ curve for PLLS (Figure~\ref{fig:goz}).
}
\label{fig:prox}
\end{figure}


\begin{figure}
\epsscale{0.9}
\plotone{Figures/fig_literature.ps}
\caption{Comparisons with the literature
}
\label{fig:literature}
\end{figure}

\begin{figure}
\epsscale{0.9}
\plotone{Figures/fig_allloz.ps}
\caption{Redshift evolution in \lzlls
}
\label{fig:zloz}
\end{figure}


\begin{figure}
\epsscale{0.9}
\plotone{Figures/stack_lls.ps}
\caption{Stack of LLS spectra
}
\label{fig:stack}
\end{figure}

\begin{figure}
\epsscale{0.9}
\plotone{Figures/fig_lris_sdss.ps}
\caption{LRIS spectra
}
\label{fig:lris}
\end{figure}

%\begin{figure}
%\epsscale{0.9}
%\plotone{Figures/fig_mock_loz.ps}
%\caption{l(z) for the Mocks
%}
%\label{fig:mockloz}
%\end{figure}

%\begin{figure}
%\epsscale{0.9}
%\plotone{Figures/fig_loz_prox.ps}
%\caption{$\ell(z)$ for proximate LLS with $\mtll \ge 2$
%}
%\label{fig:loz-lls}
%\end{figure}


\end{document}
