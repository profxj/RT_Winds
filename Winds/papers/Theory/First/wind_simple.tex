\documentclass[12pt,preprint]{aastex}
\usepackage{natbib,amsmath}
\special{papersize=8.5in,11in}
\begin{document}


\newcommand{\naid}{\ion{Na}{I}~$\lambda\lambda 5891, 5897$}
\newcommand{\mgiid}{\ion{Mg}{2}~$\lambda\lambda 2796, 2803$}
\newcommand{\mgiia}{\ion{Mg}{2}~$\lambda 2796$}
\newcommand{\mgiib}{\ion{Mg}{2}~$\lambda 2803$}
\def\hub{h_{72}^{-1}}
\def\umfp{{\hub \, \rm Mpc}}
\def\mzq{z_q}
\def\zabs{$z_{\rm abs}$}
\def\mzabs{z_{\rm abs}}
\def\intl{\int\limits}
\def\cmma{\;\;\; ,}
\def\perd{\;\;\; .}
\def\ltk{\left [ \,}
\def\ltp{\left ( \,}
\def\ltb{\left \{ \,}
\def\rtk{\, \right  ] }
\def\rtp{\, \right  ) }
\def\rtb{\, \right \} }
\def\sci#1{{\; \times \; 10^{#1}}}
\def \rAA {\rm \AA}
\def \zem {$z_{\rm em}$}
\def \mzem {z_{\rm em}}
\def\smm{\sum\limits}
\def \cmm  {cm$^{-2}$}
\def \cmmm {cm$^{-3}$}
\def \kms  {km~s$^{-1}$}
\def \mkms  {{\rm km~s^{-1}}}
\def \lyaf {Ly$\alpha$ forest}
\def \Lya  {Ly$\alpha$}
\def \lya  {Ly$\alpha$}
\def \mlya  {Ly\alpha}
\def \Lyb  {Ly$\beta$}
\def \lyb  {Ly$\beta$}
\def \lyg  {Ly$\gamma$}
\def \ly5  {Ly-5}
\def \ly6  {Ly-6}
\def \ly7  {Ly-7}
\def \nhi  {$N_{\rm HI}$}
\def \mnhi  {N_{\rm HI}}
\def \lnhi {$\log N_{HI}$}
\def \mlnhi {\log N_{HI}}
\def \etal {\textit{et al.}}
\def \lyaf {Lyman--$\alpha$ forest}
\def \mnmin {\mnhi^{\rm min}}
\def \nmin {$\mnhi^{\rm min}$}
\def \O {${\mathcal O}(N,X)$}
\newcommand{\cm}[1]{\, {\rm cm^{#1}}}
\def \snrlim {SNR$_{lim}$}

\title{Simple Wind Models}

\author{
J. Xavier Prochaska\altaffilmark{1}, 
Others
%John M. O'Meara\altaffilmark{2}, 
%Gabor Worseck\altaffilmark{1} 
%\& Scott Burles\altaffilmark{3}
}
\altaffiltext{1}{Department of Astronomy and Astrophysics, UCO/Lick Observatory, University of California, 1156 High Street, Santa Cruz, CA 95064}
%\altaffiltext{2}{Department of Chemistry and Physics, Saint Michael's College.
%One Winooski Park, Colchester, VT 05439}

\begin{abstract}
Blah
\end{abstract}


\keywords{absorption lines -- intergalactic medium -- Lyman limit systems -- SDSS}

\section{Introduction}

\begin{itemize}
\item Usage of cool gas outflows to probe galactic-scale winds in
  star-forming galaxies is now a well-established observational
  technique
\item Simple anlaysis focusing on the central wavelength and EW of the
  lines
\item \lya, MgII, NaI and (now) FeII* show emission
\item Despite the wealth of observational diagnostics, little effor
  hast been taken to constrain the wind properties
\item These are some first steps by considering idealized wind
  profiles.  Algorithms can be extended to any wind output
\end{itemize}

\section{Methodology}

\subsection{The Radiative Transitions}

[Neglect collisional excitation and de-excitation.  These would
require $n_e > 10^3 \cm{-3}$.  Non-detection of FeI argues against
this.]

In this paper, we will focus on two sets of radiative transitions
associated with the Fe$^+$ and Mg$^+$ ions.  The choice of these
specific low-ions is motivated by our own observations
\citep{rubin10b} but (as described below) these two ions exhibit
the dominant characteristics of the majoriy of low-ions used to study
cool-gas outflows. [word] Therefore, the majority of our results can
be generalized to other transitions that are freuqenly used.

The Mg$^+$ ion, with a single 3s electron in the ground-state,
exhibits an alkali doublet of transitions that are analagous to the
\lya\ transitions of neutral hydrogen.  In Figure~\ref{fig:energy}
we present the energy level diagram for the Mg$^+$ ion associated with
the commonly observed \mgiid\ doublet.  In non-relativistic quantum
mechanics, the 2p$^6$3p level is said to be split by spin-orbit
coupling.  This doublet marks the only \ion{Mg}{2} electric-dipole transitions
with wavelengths near 2800\AA.  Furthermore, the transtion connecting
the $\rm {}^2P_{3/2}$ and $\rm {}^2P_{1/2}$ is forbidden by several
selection rules.  Therefore, an absorption to either of the upper
states is followed essentially 100$\%$ of the time by a spontaneous decay to the
ground state.  

The physics of the \mgiid\ doublet
is analagous [repeating myself] to the physics of \ion{H}{1}
\lya, and also the \naid\ doublet and many other doublets commonly
studied in the ISM and IGM.  The key quantitative differences are in
the wavelengths of the transitions and, for the calculations of interest
in ths paper, their energy separations.  For \ion{H}{1} \lya, the
separation is sufficiently small $\Delta v = c \Delta E / E \approx 1.3
\mkms$ that most radiative transfer treatments ignore the fact that
\lya\ is a pair of transitions.   This is justified by the fact that
most astrophysical processes have turbulent processes that
significnatly exceed this velocity separation and effectively mix the
two transitions.  For \ion{Mg}{2} ($\Delta v \approx 770 \mkms$),  
\ion{Na}{1} ($\Delta v \approx 304 \mkms$), and most other doublets
commonly studies the separations are large and demand that the lines
be treated separately.  In terms of the analysis of outflows, the line
separation is especially relevant when the outflow speed matches the
line separation.

Iron exhibits the most complex set of energy levels of the elements
frequently studied in astrophysics.  The Fe$^+$ ion alone has over XXX
energy levels recorded \citep{iron} and even this is a necessarily
incomplete list.  In this paper, we will focus on 9 of the levels
related to the UV1 mutliplet whose wavelengths lie near 2600\AA\
(Figure~\ref{fig:energy}).  One reason for the complexity of iron is
that the majoriyt of energy levels exhibit fine-structure splitting.
This includes the ground-state which has four fine-structure levels with
excitation energies $T_{\rm ex} = \Delta E / k$ ranging from $\approx
500-1500$\,K.  We label these according to the total angular momentum
of each level $J$, with the ground-state having $J=9/2$.
Transitions between these fine-structure levels are magnetic-dipole
transitions with spontaneous decay times of several hours.  In
principle, these levels can be populated by indirect UV pumping
\citep[e.g][]{sv02,pcb06}, but this would require that the gas lie
close to the source of UV photons (see $\S$~\ref{sec:pump}).  In the
following, we will assume that only the ground-state is populated by
electrons and therefore capable of absorbing photons.

[Comment on UV1 a bit more]

\subsection{The Source}

The majority (all?) of observational studies of cool gas outflows in
absorption have focused on intervening, star-forming galaxies.  The
unobscured emisison of these galaxies is a complex combination of
stars and \ion{H}{2} regions that is then modulated by dust and gas
within its ISM.  In terms of the spectral regions studied in this
paper, most stars show a featureless continuum with a few spectral
types showing significant \ion{Mg}{2} and \ion{Fe}{2} absorption.
[Any emission??]  Furthermore, \ion{H}{2} regions are observed to emit
at the \mgiid\ doublet, primarily due to recombinations in the outer
layers.  It is beyond the scope of this paper to properly model this
stellar absorption and \ion{H}{2} region emission, but the reader
should be aeware that they can complicate the observed spectrum,
independent of the effects of a galactic outflow.
In the following, we will simply assume a flat continuum that will
generally be normalized to unit value.

[What does MgI tell us about winds?  Why no P-Cygni?]

\acknowledgments

J. X. P. and J.M.O. are supported by NASA grant
HST-GO-10878.05-A.  J.X.P and G.W. are partially supported
by an NSF CAREER grant (AST--0548180), and 
by NSF grant AST-0908910.


\bibliographystyle{/u/xavier/paper/Bibli/apj}
\bibliography{/u/xavier/paper/Bibli/allrefs}

%\input{Tables/tab_subqso.tex}

\begin{figure}
\epsscale{0.8}
\plotone{Figures/energy_levels.ps}
\caption{
Energy level diagrams for the \mgiid\ doublet and the UV1
multiplet of \ion{Fe}{2} transitions.   Each transition shown is
labelled by its rest wavelength (\AA) and Einstein A-coefficient
(s$^{-1}$). Black updward arrows
indicate the resonance-line transitions connected to the ground
state of each ion.  The 2p$^6$3p configuration of Mg$^+$ is split into
two energy levels that give rise to the \mgiid\ doublet.  
Both the 3d$^6$4s ground state and 3d$^6$4p upper level of Fe$^+$
are split into multiple energy levels giving rise to a series of
electric-dipole transitions between these fine-structure levels.   
The downward (green) arrows show the transitions connected to the
resonance-line transitions (i.e.\ they share the same upper energy
levels).  We also show a pair of levels (\ion{Fe}{2}~$\lamabda\lambda
2618,2631$) that arise from even higher levels [Should we bother?]
Based on Figure~7 from \cite{hartagan9X}.
}
\label{fig:energy}
\end{figure}


\end{document}
